\documentclass[a4paper,10pt]{article}
\usepackage[utf8]{inputenc}

\usepackage[margin=2cm,headheight=26pt,includeheadfoot]{geometry}

\usepackage[german]{babel}
\usepackage[german=quotes]{csquotes}

\usepackage{nameref}
\usepackage{microtype}
\usepackage{float}
\usepackage{siunitx}
\sisetup{
    locale = DE,
    binary-units,
    detect-all, 
    per-mode = symbol %enables m/s instead of ms^-1
}
\AtBeginDocument{\DeclareSIUnit{\kWh}{kWh}}

\usepackage{caption} %for \caption*
\usepackage{hhline}
\usepackage{tabularx}
\usepackage{array}
\usepackage{calc}
\usepackage{multicol}
\usepackage{multirow}
\usepackage{parskip}
\usepackage{booktabs}

\usepackage{fancyhdr}
\pagestyle{fancy}
\setlength{\headheight}{48pt}
\renewcommand{\headrulewidth}{0pt}

\usepackage{xcolor,colortbl}
\usepackage{makecell}

\usepackage[symbol*]{footmisc}
\renewcommand{\thefootnote}{\fnsymbol{footnote}}
\renewcommand{\thempfootnote}{\fnsymbol{mpfootnote}}

\usepackage{color} 
\usepackage{enumitem}

\usepackage{pdfpages}

% Word-Stealth-Modus, kann aber kein Omega
%\usepackage[scaled]{helvet}

\usepackage[inline,nomargin]{fixme}
\fxsetup{
    author=,
    layout=inline,
    theme=color
}

\definecolor{fxnote}{rgb}{0.8000,0.0000,0.0000}
\colorlet{fxnotebg}{yellow}

\definecolor{boxgray}{rgb}{0.33,0.33,0.33}
\usepackage{tcolorbox}
\title{}
\author{}

\renewcommand{\familydefault}{\sfdefault}

\newcommand{\hint}[1]{\begin{tcolorbox}[colback=boxgray,colframe=black,coltext=
white,title=Hinweis]#1\end{tcolorbox}}

\newcommand{\gfx}[1]{\includegraphics[width=\linewidth]{#1}}

\newcommand*{\fullref}[1]{\hyperref[{#1}]{\ref*{#1}~\nameref*{#1}}}

\fancyhf{}
\fancyhead{\colorbox{boxgray}{
    \makebox[\dimexpr\linewidth-2\fboxsep][l]{
        \includegraphics[height=1cm]{./img/resized/logo}\hfill\color{white}\Huge\raisebox{.5ex}{\thepage}
        }
    }
}

\usepackage{hyperref}
\usepackage{qrcode}

\appto\UrlNoBreaks{\do\.\do\:\do\/\do\_}

\usepackage{hyphenat}

\begin{document}
\pagestyle{empty}
\begin{titlepage}
	\vspace*{-3.08cm}
	\colorbox{boxgray}{\makebox[\dimexpr\linewidth-2\fboxsep][c]{\includegraphics[width=0.6\textwidth]{./img/resized/logo}}}
	\vfill
	\begin{center}
		\Huge
		WARP Charger Betriebsanleitung\\\vspace{1cm}
		\large
		Version 1.2.5\\\vspace{0.25cm}
		\today
	\end{center}
	\vfill \gfx{./img/resized/warp_perspective_blue_ready}
\end{titlepage}
\newpage
\null
\newpage
\pagestyle{fancy}
\begin{multicols*}{2}
	\tableofcontents \section{Einführung}
	\subsection{Vorwort} Vielen Dank, dass du
	dich für einen WARP Charger von Tinkerforge entschieden hast!

	\enquote{WARP} steht
	für \textbf{W}all \textbf{A}ttached
	\textbf{R}echarge \textbf{P}oint. Mit dem WARP Charger
	erhältst du eine hochqualitative und langlebige Wallbox, mit der du dein
	Elektrofahrzeug laden kannst. Die Wallbox ist modular aufgebaut, sodass
	einzelne Komponenten einfach ausgetauscht werden können. Sowohl Hardware als
	auch Software sind Open Source. Die nachfolgende Betriebsanleitung gibt dir
	alle notwendigen Informationen zu Sicherheit, Montage, Installation, Betrieb
	und Wartung der Wallbox.

	\subsection{Funktionsweise}
	Den WARP Charger bieten wir aktuell in drei Varianten: Basic, Smart und Pro.
	Mit allen kannst du dein Elektrofahrzeug nach DIN EN 61851‐1 Mode 3 mit Strom
	laden. Jedes Modell ermöglicht einphasiges und dreiphasiges Laden (je nach
	Anschlussart) und ist als \SI{11}{\kilo\watt}- und
	\SI{22}{\kilo\watt}-Variante erhältlich. Bei der \SI{11}{\kilo\watt}- und
	der \SI{22}{\kilo\watt}-Variante unterscheiden sich unter anderem die
	Leitungsquerschnitte der Fahrzeug-Ladekabel der Wallboxen. Der maximale Ladestrom
	kann von \SI{16}{\ampere}
	(dreiphasig \SI{11}{\kilo\watt})~/~\SI{32}{\ampere} (dreiphasig \SI{22}{\kilo\watt}) über
	Schaltereinstellungen in der Wallbox reduziert werden. Minimal sind
	\SI{6}{\ampere} möglich. Nach dem Einstecken des Typ-2-Ladesteckers in
	dein Fahrzeug zeigt dir eine blaue LED auf der Frontblende der Wallbox den
	Ladezustand an. Über einen Schlüsselschalter kannst du die Lademöglichkeit der
	Wallbox deaktivieren. Innerhalb der Front-LED befindet sich ein Taster, mit dem
	du sofort einen aktiven Ladevorgang abbrechen kannst.

	Die Modellreihe WARP Charger Smart ist zusätzlich mit einem WLAN-Controller
	ausgestattet. Dieser kann als \nohyphens{Access} Point ein eigenes WLAN eröffnen oder in
	ein vorhandenes Netz eingebunden werden. Die WARP Charger Wallbox verfügt über
	eine eigene Webseite, auf der der aktuelle Ladezustand angezeigt wird und du
	Einstellungen vornehmen kannst. Du kannst über die Webseite zum Beispiel die
	Ladeleistung konfigurieren. Über die MQTT-Schnittstelle der Wallbox kannst du
	die Wallbox auch fernsteuern.

	Die Modellreihe WARP Charger Pro bietet dir alles, was der WARP Charger Smart
	bietet. Zusätzlich ist diese Wallbox aber mit einem MID-geeichten Zähler
	ausgestattet. Dieser misst, wie viel Energie (\SI{}{\kWh}) geladen
	wurde und bietet dir Statistiken mit denen du einen Überblick über deine
	Stromkosten erhältst.

	\gfx{./img/resized/type_2_connector_ready}

	Alle Wallboxen werden mit einem fest angeschlossenen
	\SI{5}{\meter}- oder \SI{7,5}{\meter}-Ladekabel mit Typ-2-Stecker geliefert. Die Modelle WARP
	Charger Basic und Smart sind ohne Anschlusskabel (Anschluss der Zuleitung)
	oder mit einem \SI{2}{\meter}-Anschlusskabel mit CEE-Stecker
	erhältlich. Das Modell WARP Charger Pro wird immer mit einem
	\SI{2}{\meter}-Anschlusskabel ausgeliefert, da auf Grund des zusätzlich
	verbauten Stromzählers nicht genug Platz zur Verfügung steht, um eine Zuleitung
	direkt anzuschließen. Es gibt daher die Varianten mit
	\SI{2}{\meter}-Kabel und offenen Kabelenden und
	\SI{2}{\meter}-Kabel mit CEE-Stecker.

	Folgende Anschlusskabel werden verwendet (je nach gewählten Optionen):

	\begin{description}[leftmargin=!,labelwidth=\widthof{\textbf{\SI{22}{\kilo\watt}}}]
		\item[\SI{11}{\kilo\watt}]Gummianschlussleitung H07RN-F 5G4
		      (\SI{4}{\square\milli\meter}
		      Querschnitt) + \SI{16}{\ampere}-CEE-Stecker
		\item[\SI{22}{\kilo\watt}]Gummianschlussleitung H07RN-F 5G6
		      (\SI{6}{\square\milli\meter}
		      Querschnitt) + \SI{32}{\ampere}-CEE-Stecker
	\end{description}

	\section{Sicherheitshinweise}
	\subsection{Allgemein}
	Die Wallbox ist so konstruiert, dass ein sicherer Betrieb gewährleistet ist,
	wenn sie korrekt installiert wurde, in einem einwandfreien technischen Zustand
	ist und diese Betriebsanleitung befolgt wird. \hint{Die Wallbox darf nur von einer ausgewiesenen Elektrofachkraft installiert
		werden.}

	\subsection{Bestimmungsgemäße Verwendung}
	Mit der WARP Wallbox können Elektrofahrzeuge gemäß DIN EN 61851-1 geladen
	werden. Für andere Anwendungen ist die Wallbox nicht geeignet. Eine Verwendung
	an Orten, an denen explosionsfähige oder brennbare Substanzen lagern, ist nicht
	zulässig. Jegliche Modifikation des Ladesystems und auch der Betrieb mit
	Verlängerungskabeln, Mehrfach-Steckdosen oder Ähnlichem ist verboten. Der
	Ladestecker ist vor Beschädigungen, Feuchtigkeit und Verschmutzungen zu
	schützen und darf nicht genutzt werden, wenn kein sicherer Betrieb
	gewährleistet werden kann. \hint{Mit einem beschädigten, verschmutzten oder feuchten Ladestecker darf kein Ladevorgang durchgeführt
		werden.}

	\subsection{Gerätestörung / Technischer Defekt}
	Sollte es Anzeichen für einen technischen Defekt geben, trenne sofort die
	Stromversorgung zur Wallbox, indem du die Wallbox-Sicherung in der
	Hausinstallation abschaltest. Markiere die Sicherung mit dem Hinweis, dass die
	Sicherung nicht wieder eingeschaltet werden darf und informiere umgehend den
	Installateur.

	\subsection{Schutzeinrichtungen der Wallbox}
	Der AC-Fehlerstromschutz wird über den hausseitig verbauten
	Typ-A AC-Fehlerstromschutzschalter (RCCB) oder einem eigens dafür installierten
	Typ-A \SI{30}{\milli\ampere}-Fehlerstromschutzschalter gewährleistet. Die Wallbox ist
	mit einem integrierten DC-Fehlerstromwächter der Firma Alcona ausgestattet
	(ALC-DC6-CO30). Bei einem DC-Fehlerstrom $\geq \SI{6}{\milli\ampere}$ generiert dieser
	einen AC-Fehlerstrom, der den hausseitig verbauten
	Typ-A-Fehlerstromschutzschalter sicher auslöst (AC Auslösestrom
	$\geq \SI{70}{\milli\ampere}$. Somit wird sichergestellt, dass bei Auftreten eines
	DC-Fehlerstroms die Stromversorgung unterbrochen wird.

	Darüber hinaus bietet die Wallbox weitere Schutzeinrichtungen: Dazu zählt eine
	permanente Erdungsüberwachung (PE). Ist die Erdung unterbrochen, so geht die
	Wallbox in einen Fehlerzustand. Als Letztes prüft die Box bei jedem
	Schaltvorgang, ob das verbaute Schütz korrekt schaltet. Sollte das
	Schütz defekt sein (schaltet nicht an oder ab), geht die Wallbox
	ebenfalls in einen Fehlerzustand.

	\section{Montage und Installation}
	\subsection{Montage}
	\subsubsection{Lieferumfang}
	Im Lieferumfang der Wallbox befinden sich:
	\begin{itemize}
		\item Vormontierte Wallbox inkl. Deckel
		\item Bohrschablone
		\item Diese Betriebsanleitung
		\item Testprotokoll der Wallbox
	\end{itemize}

	\subsubsection{Montageort}
	Nach Möglichkeit sollte die Wallbox vor Witterungseinflüssen geschützt
	installiert werden. Direkte Sonneneinstrahlung ist zu vermeiden, um ein
	unnötiges Aufheizen der Wallbox zu verhindern. Auf eine ausreichende Belüftung
	ist zu achten.

	\subsubsection{Wandmontage}\label{wandmontage}
	Zur Montage der Wallbox muss der Deckel entfernt werden. Dazu müssen die
	vier Kreuzschlitzschrauben gelöst werden.

	\gfx{./img/resized/warp_screw_points_ready}
	Nach Lösen der Schrauben des Deckels kann dieser von der Box herunter genommen
	werden.

	\gfx{./img/resized/warp_button_connect_arrow_ready}

	\hint{Achtung! Der Taster im Deckel ist über ein Anschlusskabel verbunden und muss
		durch Drücken der Raste vom Kabel gelöst werden.}
	Zusätzlich muss der Erdungsstecker von der Frontblende abgesteckt werden.
	Erst danach kann der Deckel vollständig zur Seite gelegt werden.

	Nach Entfernen des Deckels kann das Gehäuse an die Wand montiert werden. Zum
	Bohren der Befestigungslöcher kann die mitgelieferte Bohrschablone genutzt
	werden. Auf einen ausreichend stabilen Untergrund ist bei der Montage zu
	achten.

	\subsubsection{Anforderungen an die Elektroinstallation}
	Die Wahl des Leitungsquerschnitts und der Leitungsabsicherung der
	Wallboxzuleitung muss in Übereinstimmung mit den nationalen Vorschriften
	erfolgen. Es sollte ein 3-poliger Leitungsschutzschalter mit C-Charakteristik
	verwendet werden.
	Die Wallbox verfügt über eine interne DC-Fehlerstromerkennung, welche
	bei einem DC-Fehlerstrom $\geq \SI{6}{\milli\ampere}$ einen
	$\SI{70}{\milli\ampere}$-AC-Fehlerstrom erzeugt, der dazu gedacht, ist einen
	vorgeschalteten AC-Fehler-stromschutzschalter (RCCB) auszulösen.
	\hint{Um im Fehlerfall eine Abschaltung zu garantieren, ist daher ein vorgeschalteter
		\SI{30}{\milli\ampere}-Fehlerstromschutzschalter (RCCB) Typ-A notwendig.}
	Die Wallbox darf nur in einem TN / TT-Netz angeschlossen werden.

	\newpage
	\subsection{Elektrischer Anschluss}
	\hint{Die in diesem Kapitel beschriebenen Arbeiten dürfen nur von einer ausgewiesenen
		Elektrofachkraft durchgeführt werden.}
	Der elektrische Anschluss unterscheidet sich bei den Varianten Basic / Smart
	(ohne Zähler) und Pro (mit Zähler).

	% Force non-floating figure. Floating envs are not allowed in multicol.
	\begin{figure}[H]
		\gfx{./img/resized/warp_basic_inlay_ready}
		\caption*{WARP Charger Basic.}
	\end{figure}

	\begin{figure}[H]
		\gfx{./img/resized/warp_pro_inlay_ready}
		\caption*{WARP Charger Pro.}
	\end{figure}

	\subsubsection{Variante Basic / Smart}
	Nachdem die Wallbox montiert wurde, kann diese nun angeschlossen werden. Dazu
	ist zusätzlich zum Deckel (siehe \fullref{wandmontage}) auch der interne
	Berührungsschutz zu entfernen. Dieser wird durch Lösen der vier
	Schlitzschrauben entfernt.

	Bei der Variante Smart ist auf dem Berührungsschutz ein WLAN-Controller (ESP32 Brick)
	angeschlossen, zu dem zwei Kabel führen: Ein zweipoliges
	Stromversorgungskabel und ein siebenpoliges Bricklet-Kabel, über das die Verbindung mit dem
	Ladecontroller (EVSE Bricklet) auf der Hutschiene hergestellt wird. Eigentlich reicht es aus,
	den Berührungsschutz ohne Trennung dieser Leitungen einfach an die Seite zu
	legen. Soll dennoch der Berührungsschutz vollständig entfernt werden, so
	müssen diese Kabel getrennt werden. Beide Kabel werden am besten direkt
	an der linken, bzw Unterseite des ESP32 Brick ausgesteckt.
	Das Bricklet-Kabel verfügt über eine kleine Verriegelungstaste,
	diese muss zum Entfernen gedrückt werden.

	\gfx{./img/resized/warp_cable_cut_ready}

	Bei den Wallbox-Varianten Basic und Smart
	wird die Zuleitung an einen internen Phoenix Contact PT 6 Klemmenblock
	angeschlossen. Um bei starren Leitern maximalen Bewegungsspielraum zu bieten,
	werden die Adern um den Klemmenblock geführt und von der Rückseite
	angeschlossen. Wir empfehlen, das Kabel dafür auf einer Länge von
	\SI{20}{\centi\meter} abzumanteln. Für die PT 6 Klemmen wird eine
	Abisolierlänge von 10 bis \SI{12}{\milli\meter} vom Hersteller vorgegeben.

	Die Adern werden anhand der Reihenfolge und Klemmenbeschriftungen in die
	Klemmen gesteckt. Anschließend sollten die Adern vorsichtig nach unten gedrückt
	werden, so dass später die Frontblende wieder über dem Klemmenblock angebracht
	werden kann. Als letztes muss die Kabelverschraubung festgezogen werden. Die Verschraubung
	hat einen Klemmebereich von \SI{11}{\milli\meter} bis \SI{22}{\milli\meter} und soll laut Hersteller mit
	\SI{10}{\newton{}\meter} angezogen werden.

	Der korrekte Sitz der Adern und die Phasenzugehörigkeit ist nach der
	Installation zu prüfen! Fortfahren mit \fullref{ladestrom_schalter}!

	\gfx{./img/resized/warp_wire_ready}

	\subsubsection{Variante Pro}
	Die Variante Pro wird direkt mit einer \SI{2}{\meter}-Gummileitung vom
	Typ H07RN-F 5G ausgeliefert (\SI{4}{\milli\meter\squared} bei \SI{11}{\kilo\watt}, \SI{6}{\milli\meter\squared} bei \SI{22}{\kilo\watt}).
	Diese wird extern mit der Zuleitung verbunden, zum
	Beispiel über eine Verteilerdose.

	Der korrekte Sitz der Adern und die Phasenzugehörigkeit ist nach der
	Installation zu prüfen! Fortfahren mit \fullref{ladestrom_schalter}!

	\subsubsection{Einstellen des Ladestroms (Alle Varianten)}\label{ladestrom_schalter}
	Der maximal erlaubte Ladestrom muss abhängig von der gebäudeseitigen
	Leitungsabsicherung eingestellt werden. Der Ladestrom darf nicht höher gewählt
	werden, als die Leitungsabsicherung es zulässt.

	Zum Einstellen des Ladestroms muss der Deckel (siehe \fullref{wandmontage})
	und der interne Berührungsschutz geöffnet werden. Der Berührungsschutz wird
	durch Lösen der vier Schlitzschrauben entfernt. Achtung! Vom
	Berührungsschutz gehen mehrere Kabel in die Wallbox. Zum Einstellen des
	Ladestroms muss der Berührungsschutz nicht vollständig entfernt werden,
	so dass die Kabel angeschlossen bleiben dürfen.

	\gfx{./img/resized/warp_evse_switch_cut_ready_small}

	Über zwei Schiebeschalter auf dem internen Ladecontroller (EVSE) wird der
	maximale Ladestrom eingestellt. Die verschiedenen Schalterstellungen sind neben
	den Schaltern dokumentiert. Der weiße Block stellt dabei jeweils die Position
	des Schalters dar. Im Auslieferungszustand sind die Schalter so eingestellt,
	dass die Wallbox inaktiv ist. Im obigen Foto sind exemplarisch beide
	Schalter auf die mittlere Position gestellt worden. Damit wurde eine
	maximale Ladeleistung von \SI{11}{\kilo\watt} (\SI{16}{\ampere}) vorgegeben.
	\hint{Die Schalterstellung und der damit verbundene maximale Ladestrom dürfen nach der
		Installation nur von einer ausgewiesenen Elektrofachkraft unter
		Berücksichtigung der genannten Bedingungen geändert werden!}

	\subsection{Prüfungen}\label{tests}
	Im Werk wurde die Wallbox nach IEC 60364-6 sowie den entsprechenden gültigen
	nationalen Vorschriften geprüft, das Messprotokoll liegt bei.
	Vor der ersten Inbetriebnahme ist dennoch eine Prüfung der Gesamtinstallation
	nach den selben Vorschriften notwendig.

	Die Wallbox führt in den ersten ca. 12 Sekunden nach dem Herstellen der Stromversorgung
	eine DC-Fehlerstromerkennungskalibrierung durch (\enquote{Wallbox-LED blinkt sehr schnell}).
	Ein Ladevorgang kann erst nach dieser Kalibrierung beginnen.

	Bei der Messung des Isolationswiderstands wird für L1 ein niedrigerer Wert
	gemessen (ca. \SI{249}{\kilo\ohm}). Dies hat den Hintergrund, dass
	der verbaute Ladecontroller über je einen Optokoppler mit
	\SI{249}{\kilo\ohm} Vorwiderstand, vor und nach dem Schütz, zwischen L1 und
	PE verfügt (Erdungsüberwachung, Schützüberwachung).

	Der DC-Fehlerstromschutz kann getestet werden, indem der orangene oder schwarze Taster (siehe
	nachfolgendes Foto) auf dem DC-Fehlerstromschutzmodul gedrückt wird. In diesem
	Fall wird ein AC-Fehlerstrom erzeugt, welcher den vorgeschalteten
	AC-Fehlerstromschutzschalter auslöst. Der Taster muss bis zu 10 Sekunden lang gedrückt werden,
	damit ein AC-Fehlerstrom erzeugt wird.
	\vfill
	\gfx{./img/resized/warp_hole_button_ready}
	\section{Bedienung / Erstinbetriebnahme}

	\gfx{./img/resized/warp_button_key_ready}

	Nachdem die Wallbox installiert
	und die korrekte elektrische Installation überprüft wurde, kann die Wallbox in
	Betrieb genommen werden.
	Im ersten Schritt wird die Stromversorgung zur Wallbox eingeschaltet. Die
	blaue LED (1) der Wallbox blinkt anschließend sehr schnell. Die Wallbox führt
	für die ersten 12 Sekunden eine Kalibrierung der
	DC-Fehlerstrom-Schutzeinrichtung durch. Nach Abschluss dieser Kalibrierung
	leuchtet die LED dauerhaft. Die Wallbox ist nun betriebsbereit. Sollte die LED
	nicht permanent leuchten, ist die Wallbox entweder über den Schlüsselschalter (2) deaktiviert,
	oder es wurde ein Fehler erkannt (siehe \fullref{fehlerbehebung}).

	Als Nächstes kann ein Elektrofahrzeug zum Laden mit der Wallbox verbunden
	werden. Dazu wird die Schutzkappe vom Ladestecker entfernt und den Stecker in die
	Ladebuchse des Elektrofahrzeugs gesteckt. Nach einer kurzen Zeit sollte hörbar
	das Schütz in der Wallbox schalten und das Elektrofahrzeug sollte den Beginn
	der Ladung anzeigen. Die Wallbox-LED \enquote{atmet} während des
	Ladevorgangs. Ist die Ladung beendet, so leuchtet die LED permanent. Nach ca.
	15 Minuten Inaktivität schaltet sich die LED aus.

	\subsection{Bedienelemente}\label{lockswitch}
	Das Drücken des Tasters (1) auf der Frontseite unterbricht einen aktiven Ladevorgang
	sofort. Alternativ kann das Ladekabel vom Elektrofahrzeug entriegelt werden,
	wodurch der Ladevorgang ebenfalls unterbrochen wird. Um den Ladevorgang erneut
	zu starten, muss in beiden Fällen die Verbindung zum Fahrzeug getrennt und
	anschließend erneut hergestellt werden (Kabel aus- und wieder einstecken).

	Über den Schlüsselschalter (2) kann die Ladefunktion der Wallbox permanent deaktiviert
	werden (Stellung AUS). Laden ist dann erst wieder möglich, nachdem der Schlüsselschalter
	auf die Stellung AN gebracht wurde.

	\section{Webinterface}
	Das Webinterface der Wallbox ist nur bei den Varianten Smart und Pro verfügbar.

	Über das Webinterface kannst du unter anderem das Laden steuern und überwachen.
	Die Wallbox kann sich sowohl als Client mit einem bestehenden WLAN verbinden,
	als auch als Access Point einen eigenes WLAN eröffnen. Der Betrieb als Client
	und Access Point ist parallel möglich.

	\subsection{Ersteinrichtung}
	Im Auslieferungszustand öffnet die Wallbox einen WLAN-Access-Point, über den
	die Konfiguration des Webinterfaces vorgenommen werden kann.
	Die Zugangsdaten des Access Points findest du auf dem WLAN-Zugangsdaten-Aufkleber
	auf der Rückseite dieser Anleitung. Du kannst entweder den QR-Code des Aufklebers verwenden,
	der das WLAN automatisch konfiguriert, oder SSID und Passphrase abschreiben.

	\begin{minipage}{0.35\textwidth}
		Wenn die Verbindung mit dem Access Point der Wallbox hergestellt ist, kannst du das Webinterface
		unter \url{http://10.0.0.1} erreichen. Alternativ kannst du den nebenstehenden QR-Code scannen.
		Eventuell musst du deine mobile Datenverbindung deaktivieren.
	\end{minipage}\hfill
	\begin{minipage}{0.12\textwidth}
		\begin{flushright}
			\qrcode{http://10.0.0.1}
		\end{flushright}
	\end{minipage}

	Die Wallbox kann jetzt in ein vorhandenes WLAN eingebunden werden.
	\subsubsection{Einbinden in ein vorhandenes WLAN}
	\gfx{./img/resized/web_wifi_sta}
	Auf der Unterseite \enquote{WLAN-Verbindungseinstellungen} kannst du die Verbindung zu einem WLAN konfigurieren.
	Durch Drücken des Netzwerksuche-Buttons öffnet sich ein Menü, in dem das gewünschte WLAN ausgewählt werden kann.
	Es werden dann automatisch Netzwerkname (SSID) und BSSID eingetragen, sowie die Verbindung beim Neustart aktiviert.
	Gegebenenfalls musst du jetzt noch die Passphrase des gewählten Netzes eintragen.

	Falls die Wallbox sich zu einem versteckten Access Point verbinden soll, musst du den Netzwerknamen selbst eingeben,
	nachdem das Netzwerk mit der passenden BSSID ausgewählt wurde.

	Du kannst jetzt die Konfiguration mit dem Speichern-Button abspeichern.
	Das Webinterface startet dann neu und verbindet sich zum konfigurierten WLAN. Die Statusseite zeigt
	an, ob die Verbindung erfolgreich war. Der Access Point bleibt weiterhin
	geöffnet, sodass Konfigurationsfehler behoben werden können.
	Da der Access Point den selben Kanal wie ein eventuell verbundenes Netz verwendet,
	kann es sein, dass du dich jetzt neu zum Access Point verbinden musst.

	Bei einer erfolgreichen Verbindung sollte die Wallbox jetzt im konfigurierten Netz unter
	\url{http://[konfigurierter_hostname]}, z.B. \url{http://warp-ABC} erreichbar sein.

	\subsubsection{Konfiguration des Access Points}
	\gfx{./img/resized/web_wifi_ap}
	Der Access Point kann in einem von zwei Modi betrieben werden: Entweder kann er immer aktiv sein,
	oder nur dann, wenn die Verbindung zu einem anderen WLAN nicht konfiguriert oder fehlgeschlagen ist.
	Außerdem kann der Access Point komplett deaktiviert sein.
	\hint{Wir empfehlen, den Access Point nie komplett zu deaktivieren, da sonst bei einer
		fehlgeschlagenen Verbindung zu einem anderen Netz das Webinterface nicht mehr erreicht
		werden kann. Die einzige Möglichkeit das Interface wieder zu erreichen ist dann ein Zurücksetzen auf Werkszustand, siehe \ref{reset}}
	Der Modus des Access Points und weitere Einstellungen wie Netzwerkname, Passphrase usw. können
	auf der Unterseite \enquote{WLAN-Access-Point} konfiguriert werden.

	\subsection{Startseite / Status}
	\gfx{./img/resized/web_status}
	Die Startseite des Webinterfaces zeigt kompakt den aktuellen Ladestatus der
	Wallbox, sowie Ladezeit und -strom, und erlaubt es, die Ladung zu steuern.
	Du kannst hier sowohl das automatische Laden (de-)aktivieren, als auch
	manuell eine Ladung starten oder stoppen.

	In der Variante Pro mit verbautem Stromzähler wird zusätzlich der Ladeverlauf
	über die letzten 48 Stunden und die aktuelle Leistungsaufnahme gezeigt.

	Der \textbf{Ladestatus} gibt dir die Information, ob aktuell ein
	Fahrzeug mit der Wallbox verbunden ist und ob dieses geladen wird.

	Der \textbf{Ladestrom} zeigt dir an, mit welchem Strom das Fahrzeug geladen
	wird / würde. Über den Haken kannst du diesen Strom innerhalb der Min-
	und Max-Grenzen einstellen. Minimal können 6A eingestellt werden. Das
	Maximum hängt von der Hardwarekonfiguration deiner Wallbox ab.

	Die \textbf{Ladekontrolle} ermöglicht es dir, manuell einen Ladevorgang zu
	starten oder abzubrechen (Start / Stop). Wenn du \textbf{Autostart}
	einschaltest, startet der Ladevorgang automatisch, sobald ein Fahrzeug
	angeschlossen wird.

	\textbf{Ladeverlauf und Leistungsaufnahme} sind nur in der Variante Pro
	vorhanden. Hier werden dir die aktuelle Leistungsaufnahme und ein Chart über
	die letzten 48 Stunden angezeigt.

	\textbf{WLAN-Verbindung zu ...} zeigt dir im Namen an, zu welchem WLAN
	sich die Wallbox verbinden soll und wie der Status der Verbindung ist.

	Der \textbf{WLAN-Access-Point}-Status bildet den Status des Access Points ab.
	\enquote{Deaktiviert} beziehungsweise \enquote{Aktiviert} zeigt den Zustand, wenn der Access Point nicht
	nur als Fallback für die WLAN-Verbindung verwendet wird. Falls der Status \enquote{Fallback inaktiv} ist,
	war die WLAN-Verbindung erfolgreich und der Access Point wurde deshalb deaktiviert.
	Beim Status \enquote{Fallback aktiv} ist der Aufbau der WLAN-Verbindung fehlgeschlagen und der
	Access Point wurde deshalb aktiviert.

	\textbf{MQTT-Verbindung zu ...} zeigt den aktuellen Status der MQTT-Verbindung
	zum konfigurierten Broker.

	\subsection{Ladecontroller}
	\gfx{./img/resized/web_evse}
	Die Unterseite des Ladecontrollers gibt detaillierte Auskunft über den Zustand
	des Ladecontrollers (des EVSEs) und dessen Hardware-Konfiguration. Der
	erlaubte Ladestrom, mit dem ein Auto geladen werden kann, ist das Minimum des
	konfigurierten Ladestroms und der Maximalströme der Zuleitung und des Typ-2-Kabels.

	Probleme beim Laden kannst du mit den Informationen dieser
	Seite diagnostizieren.

	\subsection{Stromzähler}
	\gfx{./img/resized/web_meter}
	(Nur bei der Variante WARP Charger Pro). Auf der Seite
	siehst du ein Diagramm mit der Leistungsaufnahme für die letzten 24h und die
	Statistiken dazu.

	\subsection{System}
	Im System-Unterabschnitt kannst du das Ereignis-Log einsehen, das Webinterface durch Zugangsdaten schützen und Firmware-Updates einspielen.
	\subsubsection{Ereignis-Log}
	\gfx{./img/resized/web_event_log}
	Das Ereignis-Log zeichnet relevante Informationen des Systemstarts, sowie WLAN- und MQTT-Verbindungsabbrüche und Ladefehler auf.
	Falls Probleme mit der Wallbox auftreten, kannst du diese mit dem Log diagnostizieren.
	Das Log wird wenn die Unterseite geöffnet wird und danach periodisch aktualisiert.
	Falls du ein Problem mit der Wallbox an uns melden möchtest, kannst du das Ereignis-Log,
	sowie einen Debug-Report abrufen, die uns helfen das Problem zu verstehen und zu lösen.

	\subsubsection{Firmware-Update}
	\gfx{./img/resized/web_firmware_update}
	Hier kannst du die Firmware der Wallbox aktualisieren. Wir werden die Funktionalität
	laufend weiterentwickeln. Aktuelle Firmwares findest du unter \url{https://warp-charger.com}.
	Außerdem kannst du hier das Webinterface neustarten, ohne einen Ladevorgang zu unterbrechen.

	\subsubsection{Zugangsdaten}
	\gfx{./img/resized/web_authentication}
	Auf dieser Unterseite kannst du Zugangsdaten konfigurieren, mit denen man sich anmelden muss,
	damit das Webinterface und die HTTP-API verwendet werden können.
	\hint{Wenn du die Zugangsdaten des Webinterfaces vergisst, kannst du nur nach einem Zurücksetzen
	auf den Auslieferungszustand wieder darauf zugreifen.}

	\subsection{Zurücksetzen auf Auslieferungszustand}\label{reset}
	Falls das Webinterface nicht korrekt funktioniert, oder die Konfiguration defekt ist,
	kannst du auf der Firmware-Update-Unterseite alle Einstellungen auf den Auslieferungszustand zurücksetzen.
	Das Zurücksetzen dauert ungefähr eine Minute, danach startet das Webinterface wieder und öffnet
	den Access Point mit dem Netzwerknamen und Passwort, die auf dem Aufkleber vermerkt sind.

	Falls du das Webinterface nicht mehr erreichen kannst, kannst du das Zurücksetzen auch
	auslösen, indem du den Deckel abschraubst und den linken Button des ESP32 Bricks (markiert mit IO0)
	10 Sekunden lang gedrückt hältst. Die grüne LED fängt sofort an zu leuchten und blinkt nach 10 Sekunden
	dreimal, um anzuzeigen, dass das Zurücksetzen beginnt.

	\vfill

	\gfx{./img/resized/warp_esp_open}
	\newpage
	\section{MQTT- und HTTP-Schnittstelle zur Fernsteuerung der Wallbox}
	Die Wallbox kann per MQTT oder HTTP ferngesteuert werden. Über diese Schnittstellen ist eine
	Einbindung in Hausautomatisationssysteme wie openHAB, ioBroker, FHEM o.ä. möglich. Die aktuelle
	Dokumentation der Schnittstellen findet sich auf \url{https://warp-charger.com/api.html}

	Falls du die Zugangsdaten für das Webinterface gesetzt und die Anmeldung aktiviert hast, musst du
	für die HTTP-API die selben Zugangsdaten verwenden.

	\gfx{./img/resized/web_mqtt}
	Auf der MQTT-Unterseite kannst du die Verbindung zu einem MQTT-Broker konfigurieren. Folgende Einstellungen können vorgenommen werden:
	\begin{itemize}
		\item \textbf{Broker-Hostname oder IP-Adresse} Der Hostname oder die IP-Adresse des Brokers, zu dem sich die Wallbox verbinden soll.
		\item \textbf{Broker-Port} Der Port, unter dem der Broker erreichbar ist. Der typische MQTT-Port 1883 ist voreingestellt.
		\item \textbf{Broker-Benutzername und Passwort} Manche Broker unterstützen eine Authentifizierung mit Benutzername und Passwort.
		\item \textbf{Topic-Präfix} Dieses Präfix wird allen Topics vorangestellt, die die Wallbox verwendet.
		      Voreingestellt ist warp/ABC, wobei ABC eine eindeutige Kennung pro Wallbox ist,
		      es sind aber andere Präfixe wie z.B. garage\_links möglich.
		      Falls mehrere Wallboxen mit dem selben Broker kommunizieren,
		      müssen eindeutige Präfixe pro Box gewählt werden.
		\item \textbf{Client-ID} Mit dieser ID registriert sich die Wallbox beim Broker.
	\end{itemize}
	Nachdem die Konfiguration gesetzt und der \enquote{MQTT aktivieren}-Schalter aktiviert ist, kann die Konfiguration gespeichert werden.
	Das Webinterface startet dann neu und verbindet sich zum Broker.
	Auf der Status-Seite wird angezeigt, ob die Verbindung aufgebaut werden konnte.

	\newpage \section{Fehlerbehebung}\label{fehlerbehebung} \subsection{Fehlersuche}
	Fehlerzustände werden von der Wallbox durch die LED im Deckel
	dargestellt. Bei den Varianten WARP Charger Smart und WARP Charger Pro gibt die Statusseite des Ladecontrollers
	weitere Informationen.

	\subsubsection*{Wallbox-LED ist aus}
	Für diesen Fehlerzustand gibt es verschiedene mögliche Ursachen:
	\begin{itemize}
		\item Die Wallbox ist über den Schlüsselschalter deaktiviert. Siehe \fullref{lockswitch}.
		\item Die Wallbox-LED geht nach etwa 15 Minuten Inaktivität aus. Das Drücken des Tasters
		      oder das Anschließen eines Elektrofahrzeugs zur Ladung weckt die Wallbox wieder
		      und die LED sollte wieder dauerhaft leuchten.
		\item Die Wallbox ist nicht mit Strom versorgt. Mögliche Ursachen: Stromausfall,
		      Sicherung oder Fehlerstrom-schutzschalter haben ausgelöst
		\item Der interne Ladecontroller ist ohne Strom. Die Wallbox verfügt intern über zwei
		      Feinsicherungen, gegebenenfalls ist eine defekt.
		\item Das innere Anschlusskabel zum Deckel wurde am Taster 180\textdegree\ verdreht aufgesteckt.
	\end{itemize}

	\subsubsection*{Wallbox-LED blinkt sehr schnell}
	Nach dem Einschalten der Stromversorgung kalibriert die Wallbox die
	DC-Fehlerstromerkennung. Nach ca. 10 Sekunden sollte die Kalibrierung
	abgeschlossen sein und die Wallbox-LED sollte dauerhaft leuchten
	(betriebsbereit).

	\subsubsection*{Wallbox-LED blinkt 2x im Intervall \\ Webinterface zeigt Schalterfehler}
	Die Wallbox wurde nicht korrekt installiert. Die Schalter-Einstellung des Ladecontrollers ist
	noch auf dem Werkszustand. Siehe \fullref{ladestrom_schalter}.

	\begin{minipage}{\linewidth} %use minipage to control footnote placement
		\subsubsection*{Wallbox-LED blinkt 3x im Intervall \\ Webinterface zeigt Kalibrierungsfehler}
		Die DC-Kalibrierung konnte nicht abgeschlossen werden. \footnote{Bei einer Firmware älter
			als 1.1.0 darf beim Einschalten der Wallbox kein Fahrzeug angeschlossen sein, damit die
			Kalibrierung erfolgreich verläuft.} Die Stromversorgung der Wallbox
		trennen und nach 5 Sekunden wieder einschalten. Die Kalibrierung sollte nun
		erfolgreich verlaufen.
	\end{minipage}

	\subsubsection*{Wallbox-LED blinkt 4x im Intervall \\ Webinterface zeigt Schützfehler}
	Für diesen Fehlerzustand gibt es verschiedene mögliche Ursachen:
	\begin{itemize}
		\item Erdungsfehler der Wallbox → Erdung überprüfen
		\item Phase L1 ohne Spannung
		\item Schütz schaltet nicht korrekt ein (Keine Spannung für L1 nach dem Schütz), kein
		      Kontakt
		\item Schütz schaltet nicht korrekt ab (Spannung von L1 liegt trotz Abschalten noch
		      nach dem Schütz an), \enquote{Schütz klebt}
		\item Eine der internen Feinsicherungen ist defekt.
	\end{itemize}

	\subsubsection*{Wallbox-LED blinkt 5x im Intervall \\ Webinterface zeigt Kommunikationsfehler}
	Es besteht ein Kommunikationsfehler mit dem Elektrofahrzeug. Bei erstmaligem
	auftreten das Ladekabel vom Fahrzeug trennen, 10 Sekunden warten und das
	Ladekabel erneut mit dem Fahrzeug verbinden (erneuter Ladevorgang).

	Sollte das Problem bestehen bleiben, so kann es verschiedene Gründe dafür
	geben:
	\begin{itemize}
		\item Es liegt ein Fehler beim Ladekabel vor (Kurzschluss, verschmutze / feuchte
		      Kontakte o.ä.). Die Wallbox ist dann sofort außer Betrieb zu nehmen und
		      fachmännisch in Stand zu setzen.
		\item Es liegt ein technischer Defekt beim Fahrzeug vor.
		\item Es liegt ein technischer Defekt bei der Wallbox vor (Ladecontroller defekt o.ä.)
		\item Das Fahrzeug fordert den IEC 61851-1 Status \enquote{D – Laden mit Belüftung}
		      an. Dieser Modus wird von der Wallbox nicht unterstützt.
		\item Das Fahrzeug übermittelt den IEC 61851-1 Status E oder F. In beiden Fällen
		      handelt es sich um einen Fehler, den das Fahrzeug erkannt hat.
	\end{itemize}

	\subsubsection*{Die Wallbox ist nicht über das WLAN erreichbar, aber die LED leuchtet}
	In diesem Fall ist zu prüfen, ob die Wallbox gegebenenfalls in den Access-Point-Fallback
	gegangen ist. Wie im Auslieferungszustand eröffnet die Wallbox dann ein eigenes
	WLAN. Wenn die Zugangsdaten nicht geändert werden, entsprechen sie der Werkseinstellungen und sind dem
	Aufkleber auf der Rückseite der Anleitung zu entnehmen.

	\subsection{Ersatzteile}
	\begin{tabular}{ll}
		%\toprule
		\textbf{Artikelnummer}                                                                                                   & \textbf{Bauteil}                                              \\
		\cmidrule(lr{0.5em}){1-2}
		WARP-CASE                                                                                                                & WARP Charger Case                                             \\
		\cmidrule(lr{0.5em}){1-2}
		\href{https://www.tinkerforge.com/de/shop/warp/contactor-4-pole-din-rail-63a.html}{WARP-CON-4P-63A}                      & Schaltschütz 4 Pol,                                           \\
		                                                                                                                         & Hutschiene, \SI{63}{\ampere}                                  \\
		\cmidrule(lr{0.5em}){1-2}
		\href{https://www.tinkerforge.com/de/shop/warp/dc-residual-current-protection-module-6ma.html}{WARP-DC-PROTECT}          & DC Fehlerstrom-                                               \\
		                                                                                                                         & schutzmodul \SI{6}{\milli\ampere}                             \\
		\cmidrule(lr{0.5em}){1-2}
		\href{https://www.tinkerforge.com/de/shop/warp/bidirectional-polyphase-meter-3-phase-rs485-mid.html}{WARP-METER-3PH-MID} & Zweirichtungs-                                                \\
		                                                                                                                         & Drehstromzähler,                                              \\
		                                                                                                                         & 3 Phasen, RS485, MID                                          \\
		\cmidrule(lr{0.5em}){1-2}
		\href{https://www.tinkerforge.com/de/shop/warp/din-rail-power-supply-230vac-12vdc-1-25a.html}{WARP-PS-12V}               & Hutschienennetzteil                                           \\
		                                                                                                                         & \SI{230}{\volt}AC – \SI{12}{\volt} \SI{1,25}{\ampere}         \\
		\cmidrule(lr{0.5em}){1-2}
		WARP-T2-5M-16A                                                                                                           & Typ 2 Stecker mit                                             \\
		                                                                                                                         & \SI{5}{\meter} Kabel \SI{11}{\kilo\watt} / \SI{16}{\ampere}   \\
		\cmidrule(lr{0.5em}){1-2}
		WARP-T2-5M-32A                                                                                                           & Typ 2 Stecker mit                                             \\
		                                                                                                                         & \SI{5}{\meter} Kabel \SI{22}{\kilo\watt} / \SI{32}{\ampere}   \\
		\cmidrule(lr{0.5em}){1-2}
		WARP-T2-75M-16A                                                                                                          & Typ 2 Stecker mit                                             \\
		                                                                                                                         & \SI{7,5}{\meter} Kabel \SI{11}{\kilo\watt} / \SI{16}{\ampere} \\
		\cmidrule(lr{0.5em}){1-2}
		WARP-T2-75M-32A                                                                                                          & Typ 2 Stecker mit                                             \\
		                                                                                                                         & \SI{7,5}{\meter} Kabel \SI{22}{\kilo\watt} / \SI{32}{\ampere} \\
		\cmidrule(lr{0.5em}){1-2}
		113                                                                                                                      & ESP32 Brick                                                   \\
		\cmidrule(lr{0.5em}){1-2}
		\href{https://www.tinkerforge.com/de/shop/bricklets/rs485-bricklet.html}{277}                                            & RS485 Bricklet                                                \\
		\cmidrule(lr{0.5em}){1-2}
		2159                                                                                                                     & EVSE Bricklet                                                 \\
		\cmidrule(lr{0.5em}){1-2}
		\href{https://www.tinkerforge.com/de/shop/accessories/cable/bricklet-cable-15cm-7p-7p.html}{6150}                        & Bricklet Kabel                                                \\
		                                                                                                                         & \SI{15}{\centi\meter} (7p-7p)                                 \\
		\cmidrule(lr{0.5em}){1-2}
		6189                                                                                                                     & \SI{7} - \SI{28}{\volt} zu \SI{5}{\volt} Inline               \\
		                                                                                                                         & Stromversorgung                                               \\
		%\bottomrule
	\end{tabular}

	\subsection{Stromlaufplan}
	Ein Stromlaufplan ist in einem gesonderten Dokument verfügbar.

	\subsection{Sicherungswechsel}
	Die Wallbox ist intern über zwei 6,3x32mm Feinsicherungen (mittelträge (m), 500mA) abgesichert.
	Tinkerforge verbaut Sicherungen vom Typ \glqq ESKA  632.214\grqq.

	\section{Technische Daten}

	%use minipage here to control footnote placement
	\begin{minipage}{\linewidth}

		\begin{description}[leftmargin=!,labelwidth=\widthof{\textbf{Fehlerstromerkennung}}]
			\setlength{\itemsep}{3pt}
			\item[Ladestandard] DIN EN 61851‐1
			\item[Ladeleistung] einstellbar
			      bis \SI{11}{\kilo\watt} / \SI{22}{\kilo\watt}~\footnote[7]{\label{fn:1} je nach Variante}
			\item[Fahrzeugladestecker] Typ 2
			\item[Abmessungen] 280 × 215 × \SI{95}{\milli\meter} (B/H/T)
			\item[Nennspannung] \SI{230}{\volt} / \SI{400}{\volt} / 1/3
			      AC$\sim$~\footref{fn:1}
			\item[Nennfrequenz] \SI{50}{\hertz}
			\item[Nennstrom] \SI{16}{\ampere} / \SI{32}{\ampere}
			      \footref{fn:1}
			\item[Standby, WLAN an] Basic/Smart $\leq\SI{3}{\watt}$; Pro $\leq\SI{5}{\watt}$
			\item[Ladekabellänge] \SI{5}{\meter} / \SI{7,5}{\meter}~\footref{fn:1}
			\item[Zuleitungsquerschnitt] \SI{2,5}{\square\milli\meter} bis
			      \SI{10}{\square\milli\meter}
			\item[Zugangsverriegelung]
			      Schlüsselschalter\\Webinterface~\footref{fn:1}\\Konfigurierbare Ladezeiten~\footref{fn:1}
			\item[Betriebstemperatur] \SI{-25}{\celsius}
			      bis \SI{+50}{\celsius} (Durchschnitt in \SI{24}{\hour}: $\leq \SI{35}{\celsius}$)
			\item[Fehlerstromerkennung] DC \SI{6}{\milli\ampere} (integriert)
			\item[Schutzart] IP54
			      (spritzwassergeschützt, für
			      den Außenbereich geeignet)
			\item[Lieferumfang] Wallbox,
			      Bedienungsanleitung inklusive Installationsanleitung, Prüfprotokoll
		\end{description}
	\end{minipage}

	\section{Kontakt}
	Tinkerforge GmbH\\ Zur Brinke 7\\ 33758 Schloß Holte-Stukenbrock\\
	\begin{description}[leftmargin=!,labelwidth=\widthof{\textbf{Website}}]
		\item[E-Mail] \href{mailto:info@tinkerforge.com}{\texttt{info@tinkerforge.com}}
		\item[Website] \href{https://warp-charger.com}{\texttt{warp-charger.com}}
		\item[Shop] \href{https://tinkerforge.com/de/shop/warp.html}{\texttt{tinkerforge.com/de/shop/warp.html}}
	\end{description}

	\section{Konformitätserklärung}
	Die EU-Konformitätserklärung ist in einem gesonderten Dokument verfügbar.

	\section{Entsorgung}
	\begin{minipage}{0.35\textwidth}
		Die Wallbox und die Verpackung ist bei Gebrauchsende ordnungsgemäß zu
		entsorgen. Altgeräte dürfen nicht über den Hausmüll entsorgt werden.
	\end{minipage}\hfill
	\begin{minipage}{0.1\textwidth}
		\includegraphics[width=\linewidth]{./img/resized/weee.pdf}
	\end{minipage}

	\section{Dokumentversionen}
	\begin{tabular}{lll}
		\toprule
		Datum      & Version & Kommentar                   \\
		\midrule
		30.11.2020 & 0.1     & Initialversion              \\
		04.12.2020 & 0.2     & Fotos hinzugefügt           \\
		05.12.2020 & 0.3     & Portierung nach \LaTeX      \\
		28.12.2020 & 0.4     & Webinterface-Dokumentation  \\
		           &         & erweitert                   \\
		29.12.2020 & 0.5     & Beschreibungen erweitert    \\
		27.01.2021 & 0.6     & Installationsbeschreibung   \\
		           &         & ergänzt                     \\
		28.01.2021 & 0.7     & Ersteinrichtung ergänzt;    \\
		           &         & Layout überarbeitet         \\
		29.01.2021 & 1.0     & Druckversion                \\
		26.02.2021 & 1.1     & Anpassung an Firmware 1.1   \\
		11.03.2021 & 1.2     & Anpassung an Firmware 1.2   \\
		24.03.2021 & 1.2.1   & Beschreibung des Schlüssel- \\
		           &         & schalters hinzugefügt;      \\
		           &         & \fullref{tests} überarbeitet\\
		21.04.2021 & 1.2.2   & Herstellerangaben der Kabel-\\
		           &         & verschraubung eingefügt     \\
		27.04.2021 & 1.2.3   & Testhinweis für DC-Fehler-  \\
		           &         & stromschutzmodul ergänzt    \\
		28.04.2021 & 1.2.4   & Standby-Verbrauch           \\
		           &         & aktualisiert                \\
		02.06.2021 & 1.2.5   & Fehlerstromschutzschalter   \\
		           &         & aus Ersatzteilliste entfernt;\\
		           &         & 180\textdegree\ Verdrehung am Taster-\\
		           &         & anschluss als Fehlerursache \\
		           &         & für \glqq Wallbox-LED ist aus\grqq\\
		           &         & hinzugefügt                 \\
		\bottomrule
	\end{tabular}
	\newpage
	\pagestyle{empty}
	\null
	\vfill
	WLAN-Zugangsdaten
	\begin{tcolorbox}[width=4.2cm,height=2.7cm, boxrule=0.25mm]

	\end{tcolorbox}
	Dieser Aufkleber befindet sich\\ auch unter dem Deckel der Wallbox.
	\columnbreak

	\null
	\vfill
	Typenschild
	\begin{tcolorbox}[width=7.8cm,height=4.1cm, boxrule=0.25mm]

	\end{tcolorbox}
	Dieser Aufkleber befindet sich auch auf der Unterseite\\ der Wallbox.
\end{multicols*}
\end{document}
