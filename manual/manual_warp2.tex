\documentclass[a4paper,10pt]{article}
\ifdefined\forprint
    \usepackage[width=21.6cm,height=30.3cm,center]{crop}
\fi
\usepackage[utf8]{inputenc}
\usepackage[margin=2cm,headheight=26pt,includeheadfoot]{geometry}

\usepackage[german]{babel}
\usepackage[german=quotes]{csquotes}

\usepackage{nameref}
\usepackage{microtype}
\usepackage{float}
\usepackage{siunitx}
\sisetup{
    locale = DE,
    binary-units,
    detect-all,
    per-mode = symbol %enables m/s instead of ms^-1
}
\AtBeginDocument{\DeclareSIUnit{\kWh}{kWh}}

\usepackage{caption} %for \caption*
\usepackage{hhline}
\usepackage{tabularx}
\usepackage{array}
\usepackage{calc}
\usepackage{multicol}
\usepackage{multirow}
\usepackage{parskip}
\usepackage{booktabs}
\usepackage{textcomp}

\usepackage{fancyhdr}
\pagestyle{fancy}
\setlength{\headheight}{48pt}
\renewcommand{\headrulewidth}{0pt}

\usepackage{xcolor,colortbl}
\usepackage{makecell}

\usepackage[symbol*]{footmisc}
\renewcommand{\thefootnote}{\fnsymbol{footnote}}
\renewcommand{\thempfootnote}{\fnsymbol{mpfootnote}}

\usepackage{color}
\usepackage{enumitem}

\usepackage{pdfpages}

\usepackage{phonenumbers}

% Word-Stealth-Modus, kann aber kein Omega
%\usepackage[scaled]{helvet}

\usepackage[inline,nomargin]{fixme}
\fxsetup{
    author=,
    layout=inline,
    theme=color
}

\definecolor{fxnote}{rgb}{0.8000,0.0000,0.0000}
\colorlet{fxnotebg}{yellow}

\definecolor{boxgray}{rgb}{0.33,0.33,0.33}
\usepackage{tcolorbox}
\title{}
\author{}

\renewcommand{\familydefault}{\sfdefault}

\newcommand{\hint}[1]{\begin{tcolorbox}[colback=boxgray,colframe=black,coltext=
white,title=Hinweis,left*=2mm,right*=2mm,boxsep=1mm,bottom=1mm,top=1mm]#1\end{tcolorbox}}

\newcommand{\gfx}[1]{\includegraphics[width=\linewidth]{#1}}

\newcommand*{\fullref}[1]{\hyperref[{#1}]{\ref*{#1}~\nameref*{#1}}}

\fancyhf{}
\fancyhead{\colorbox{boxgray}{
    \makebox[\dimexpr\linewidth-2\fboxsep][l]{
        \includegraphics[height=1cm]{./img_warp2/resized/logo}\hfill\color{white}\Huge\raisebox{.5ex}{\thepage}
        }
    }
}

\usepackage{hyperref}
\usepackage{qrcode}

\appto\UrlNoBreaks{\do\.\do\:\do\/\do\_}

\newcommand\rurl[2]{%
  \href{#1}{\nolinkurl{#2}}%
}


\newcommand{\tdesc}[1]{\multicolumn{3}{l}{\footnotesize #1}}

\usepackage{hyphenat}

\hyphenation{Web-inter-face}

\begin{document}
\pagestyle{empty}
\begin{titlepage}
	\vspace*{-3.08cm}
	\colorbox{boxgray}{\makebox[\dimexpr\linewidth-2\fboxsep][c]{\includegraphics[width=0.6\textwidth]{./img_warp2/resized/logo}}}
	\vfill
	\begin{center}
		\Huge
		WARP2 Charger Betriebsanleitung\\\vspace{1cm}
		\large
		Version 2.0.5\\\vspace{0.25cm}
		09.12.2022
	\end{center}
	\vfill \gfx{./img_warp2/resized/warp_perspective_blue_ready}
\end{titlepage}
\newpage
\null
\newpage
\pagestyle{fancy}
\begin{multicols*}{2}
	\tableofcontents
	\newpage
	\section{Einführung}
	\subsection{Vorwort} Vielen Dank, dass du
	dich für einen WARP Charger von Tinkerforge entschieden hast!

	\enquote{WARP} steht
	für \textbf{W}all \textbf{A}ttached
	\textbf{R}echarge \textbf{P}oint. Mit dem WARP2 Charger
	erhältst du die zweite Generation der hochwertigen und langlebigen Wallbox,
	mit der du dein Elektrofahrzeug laden kannst.
	Die Wallbox ist modular aufgebaut, sodass
	einzelne Komponenten einfach ausgetauscht werden können. Sowohl Hardware als
	auch Software sind Open Source. Die nachfolgende Betriebsanleitung gibt dir
	alle notwendigen Informationen zu Sicherheit, Montage, Installation, Betrieb
	und Wartung der Wallbox.

	\gfx{./img_warp2/resized/type_2_connector_ready}\vspace{-0.3cm}

	\subsection{Funktionsweise}
	\vspace{-0.1cm}
	Den WARP2 Charger bieten wir aktuell in drei Varianten: Basic, Smart und Pro.
	Mit dem \textbf{WARP2 Chargers Basic} und den weiteren Varianten kannst du dein
	Elektrofahrzeug nach DIN EN 61851‐1 Mode 3 laden.
	Fahrzeuge können an der Wallbox ein-, zwei- oder dreiphasig laden
	(abhängig vom Fahrzeug). Jede Wallbox kann ein- oder dreiphasig
	angeschlossen werden und ist als \SI{11}{\kilo\watt}- und
	\SI{22}{\kilo\watt}-Variante erhältlich. Bei der \SI{11}{\kilo\watt}- und
	der \SI{22}{\kilo\watt}-Variante unterscheidet sich unter anderem der
	Leitungsquerschnitt des Typ-2-Ladekabels der Wallbox. Der maximale Ladestrom
	kann von \SI{6}{\ampere} bis \SI{16}{\ampere}
	(dreiphasig \SI{11}{\kilo\watt}) bzw. \SI{32}{\ampere} (dreiphasig \SI{22}{\kilo\watt}) über
	Schiebeschalter in der Wallbox eingestellt werden werden.

	\vspace{-0.1cm}
	Nach dem Einstecken des Typ-2-Ladesteckers in
	dein Fahrzeug zeigt dir die blaue LED auf der Frontblende der Wallbox den
	Ladezustand an. Innerhalb der LED befindet sich ein Taster, mit dem
	du sofort einen aktiven Ladevorgang abbrechen kannst.

	Die Variante \textbf{WARP2 Charger Smart} ist zusätzlich mit einem WLAN und
	LAN-fähigen Controller ausgestattet.
	Dieser kann als \nohyphens{Access} Point ein eigenes WLAN eröffnen oder in
	ein vorhandenes Netzwerk eingebunden werden. Alternativ ist ein Anschluss
	per LAN mittels einer spritzwassergeschützten RJ45-Buchse auf der
	Unterseite der Wallbox möglich.

	Per WLAN oder LAN kannst du auf das Webinterface des WARP2 Chargers Smart
	zugreifen. Auf diesem kannst du den aktuellen Ladezustand einsehen und
	Einstellungen an der Wallbox vornehmen. Du kannst über das Webinterface
	zum Beispiel das Ladeverhalten und die maximale Ladeleistung konfigurieren.
	Über verschiedene Schnittstellen (siehe \fullref{interfaces}) kannst du aus der Ferne den aktuellen Zustand
	der Wallbox kontrollieren. Die Einbindung der Wallbox in andere System ist somit möglich.

	Zusätzlich bietet dir der WARP2 Charger Smart die Möglichkeit Ladevorgänge
	per NFC (RFID) freizuschalten. Über die Webseite kannst du dazu NFC-Tags
	anlernen und verwalten. Mehrere WARP Charger können sich einen
	Stromanschluss durch da eingebauten Lastmanagement teilen optimal nutzen.

	Die Variante \textbf{WARP Charger Pro} bietet dir alle Funktionen des WARP Chargers Smart.
	Zusätzlich ist diese Wallbox mit einem MID-geeichten Zähler
	ausgestattet, der misst, wie viel Energie (\SI{}{\kWh}) geladen
	wurde. Außerdem bietet der Zähler dir Statistiken mit denen du einen Überblick über deine
	Stromkosten erhältst. Die Statistiken können pro NFC-Tag bzw. Benutzer aufgezeichnet,
	und im zu üblichen Tabellenkalkulationsprogrammen kompatiblen CSV-Format
	heruntergeladen werden.

	Alle Wallboxen werden mit einem fest angeschlossenen
	\SI{5}{\meter}- oder \SI{7,5}{\meter}-Ladekabel mit Typ-2-Stecker geliefert.
	In der Standardausführung werden alle WARP2 Charger ohne Anschlusskabel
	(Zuleitung zur Wallbox) ausgeliefert. In diesem Fall muss bei der Installation
	ein Anschlusskabel bereitgestellt und in der Wallbox angeschlossen werden.
	Die Einführung des Anschlusskabels kann entweder von der Unter- oder von
	der Rückseite der Wallbox erfolgen.

	Als Option können alle Wallboxen mit einem bereits ab Werk
	installierten Anschlusskabel bestellt werden. Es besteht zusätzlich die
	Möglichkeit dieses mit einem CEE-Stecker ausstatten zu lassen.
	Für die optionalen Anschluss\-kabel verwenden wir folgende Leitungen und CEE-Stecker:

	\begin{description}[leftmargin=!,labelwidth=\widthof{\textbf{\SI{22}{\kilo\watt}}}]
		\item[\SI{11}{\kilo\watt}]Gummianschlussleitung H07RN-F 5G4
		      (\SI{4}{\square\milli\meter}
		      Querschnitt) + \SI{16}{\ampere}-CEE-Stecker
		\item[\SI{22}{\kilo\watt}]Gummianschlussleitung H07RN-F 5G6
		      (\SI{6}{\square\milli\meter}
		      Querschnitt) + \SI{32}{\ampere}-CEE-Stecker
	\end{description}

	\newpage
	\section{Sicherheitshinweise}
	Die Wallbox ist so konstruiert, dass ein sicherer Betrieb gewährleistet ist,
	wenn sie korrekt installiert wurde, in einem einwandfreien technischen Zustand
	ist und diese Betriebsanleitung befolgt wird. \hint{Die Wallbox darf nur von einer ausgewiesenen Elektrofachkraft installiert
		werden.}

	\subsection{Bestimmungsgemäße Verwendung}
	Mit dem WARP Charger können Elektrofahrzeuge gemäß DIN EN 61851-1 geladen
	werden. Für andere Anwendungen ist die Wallbox nicht geeignet. Eine Verwendung
	an Orten, an denen explosionsfähige oder brennbare Substanzen lagern, ist nicht
	zulässig. Jegliche Modifikation des Ladesystems und auch der Betrieb mit
	Verlängerungskabeln, Mehrfach-Steckdosen oder Ähnlichem ist verboten. Der
	Ladestecker ist vor Beschädigungen, Feuchtigkeit und Verschmutzungen zu
	schützen und darf nicht genutzt werden, wenn kein sicherer Betrieb
	gewährleistet werden kann. \hint{Mit einem beschädigten, verschmutzten oder feuchten Ladestecker darf kein Ladevorgang durchgeführt
		werden.}

	\subsection{Gerätestörung / Technischer Defekt}
	Sollte es Anzeichen für einen technischen Defekt geben, ist sofort die
	Stromversorgung der Wallbox durch Abschalten der Wallbox-Sicherung im Verteilerkasten zu trennen.
	Die Sicherung ist mit dem Hinweis, dass sie nicht wieder eingeschaltet weden darf, zu markieren.
	Danach ist eine Elektrofachkraft zu informieren.

	\subsection{Schutzeinrichtungen der Wallbox}\label{dcerrorhint}
	Der AC-Fehlerstromschutz wird über den hausseitig verbauten
	Typ-A AC-Fehlerstromschutzschalter (RCCB) oder einem eigens dafür installierten
	Typ-A \SI{30}{\milli\ampere}-Fehlerstromschutzschalter gewährleistet. Die Wallbox ist
	mit einer integrierten DC-Fehlerstromüberwachung ausgestattet.
	Bei einem DC-Fehlerstrom $\geq \SI{6}{\milli\ampere}$ wird dieser
	Fehlerstrom von der Wallbox erkannt und die Verbindung zum Fahrzeug sofort
	unterbrochen (Schütz schaltet ab). Die Wallbox befindet sich ab sofort in einem
	Fehlerzustand und kann erst durch Aus- und Einschalten der
	Stromversorgung oder über das Webinterface wieder zurückgesetzt werden.
	\hint{Tritt ein DC-Fehlerstrom auf ist unbedingt die Ursache zu
	ermitteln! Ein DC-Fehlerstrom kann den vorgeschalteten Fehlerstromschutzschalter
	\enquote{erblinden} lassen, so dass dann auch Wechselspannungs
	(AC)-Fehlerströme nicht mehr korrekt erkannt werden!}

	Darüber hinaus bietet die Wallbox weitere Schutzeinrichtungen: Dazu zählt eine
	permanente Erdungsüberwachung (PE). Ist die Erdung unterbrochen, so geht die
	Wallbox in einen Fehlerzustand. Außerdem prüft die Wallbox bei jedem
	Schaltvorgang, ob das verbaute Schütz korrekt schaltet. Sollte das
	Schütz nicht mehr korrekt schalten, geht die Wallbox ebenfalls in einen Fehlerzustand.
	Fehler können, wie im Abschnitt \fullref{fehlerbehebung} beschrieben, diagnostiziert werden.

	\newpage
	\section{Montage und Installation}
	\subsection{Montage}
	\subsubsection{Lieferumfang}
	Im Lieferumfang der Wallbox befinden sich:
	\begin{itemize}
		\item Vormontierte Wallbox inkl. Deckel
		\item DIN A4 Umschlag mit:
		\begin{itemize}
			\item Dieser Betriebsanleitung
			\item Testprotokoll der Wallbox
			\item Bohrschablone
		\end{itemize}
		\item 3$\times$ NFC-Karte (nur Varianten Smart und Pro)
		\item Umschlag mit: % yes even the basic gets the envelope
		\begin{itemize}
			\item RJ45-Verschraubung für RJ45-Durchführung
			\item RJ45-Crimpstecker
		\end{itemize}
	\end{itemize}

	\subsubsection{Montageort}
	Nach Möglichkeit sollte die Wallbox vor Witterungseinflüssen geschützt
	installiert werden. Direkte Sonneneinstrahlung ist zu vermeiden, um ein
	unnötiges Aufheizen der Wallbox zu verhindern. Auf eine ausreichende Belüftung
	ist zu achten. Die Staubschutzkappe des Typ2 Steckers sollte nicht aufgesteckt
	werden, wenn diese durch Regen o.ä. mit Wasser voll laufen könnte. In diesem Fall
	droht ansonsten eine Korrosion der Kontakte des Typ2 Steckers.

	\subsubsection{Wandmontage}\label{wandmontage}
	Zur Montage der Wallbox muss der Deckel entfernt werden. Dazu müssen die
	vier Kreuzschlitzschrauben gelöst werden.

	\gfx{./img_warp2/resized/warp_screw_points_ready}
	Nach Lösen der Schrauben des Deckels kann dieser von der Wallbox herunter genommen
	werden.

	\hint{Der Taster im Deckel ist über ein Anschlusskabel verbunden und muss
		durch Drücken der Raste vom Kabel gelöst werden.}

	\gfx{./img_warp2/resized/warp2_button_and_gnd_600}

	Zusätzlich muss der Erdungsstecker von der Front\-blende abgesteckt werden.
	Erst danach kann der Deckel vollständig zur Seite gelegt werden.

	Nach Entfernen des Deckels kann das Gehäuse an die Wand montiert werden. Zum
	Bohren der Befestigungslöcher kann die mitgelieferte Bohrschablone genutzt
	werden. Bei der Montage ist auf einen ausreichend stabilen Untergrund zu
	achten.

	Wir empfehlen zur Montage den Einsatz von $\SI{5}{\milli\meter}$ oder
	$\SI{6}{\milli\meter}$ Schrauben. Die Schraubenlänge ist abhängig vom
	Untergrund. Der Schraubenkopfdurchmesser darf nicht mehr als
	$\SI{11}{\milli\meter}$ betragen, da ansonsten die Schraube nicht durch die
	entsprechende Öffnung im Gehäuse passt. Bei einer Montage auf einer Steinwand
	können beispielsweise 5$\times\SI{80}{\milli\meter}$ Holzschrauben
	mit 8$\times\SI{50}{\milli\meter}$ Dübeln  verwendet werden.

	\subsubsection{Anforderungen an die Elektroinstallation}
	Die Wahl des Leitungsquerschnitts und der Lei\-tungs\-ab\-sicher\-ung der
	Wallboxzuleitung muss in Übereinstimmung mit den nationalen Vorschriften
	erfolgen. Üblicherweise erfolgt der Anschluss der Wallbox dreiphasig.
	Dafür sollte ein dreiphasiger Leitungsschutzschalter mit C-Charakteristik
	verwendet werden. Bei einem einphasigen Betrieb der Wallbox ist
	dementsprechend ein einphasiger Leitungsschutzschalter einzusetzen.
	Die Wallbox verfügt über eine interne DC-Fehlerstromerkennung, welche
	bei einem DC-Fehlerstrom $\geq \SI{6}{\milli\ampere}$ den Ladevorgang
	unterbricht. Daher ist nur ein vorgeschalteter Typ-A \SI{30}{\milli\ampere}-Fehlerstromschutzschalter (RCCB)
	notwendig.
	Die Wallbox darf nur in einem TN~/~TT-Netz angeschlossen werden.

	\newpage
	\subsection{Elektrischer Anschluss}
	\hint{Die in diesem Kapitel beschriebenen Arbeiten dürfen nur von einer ausgewiesenen
		Elektrofachkraft durchgeführt werden.}

	Nachdem die Wallbox montiert wurde, kann sie nun angeschlossen werden. Dazu
	muss der Deckel (siehe \fullref{wandmontage}) entfernt werden.

	\gfx{./img_warp2/resized/warp_cable_cut_ready}

	Die Zuleitung muss für alle Varianten wie auf dem Foto oben abgebildet
	angefertigt werden. Wir empfehlen, das Kabel dafür auf einer Länge von
	\SI{23}{\centi\meter} abzumanteln. Für die Klemmen wird eine
	Abisolierlänge von 10 bis \SI{12}{\milli\meter} vorgegeben.

	Wie diese Zuleitung angeschlossen wird, unterscheidet sich bei
	den Varianten Basic / Smart (ohne Zähler) und Pro (mit Zähler) und ist
	nachfolgend beschrieben.

	\subsubsection{Variante Basic / Smart}

	% Force non-floating figure. Floating envs are not allowed in multicol.
	\begin{figure}[H]
		\gfx{./img_warp2/resized/warp2_basic_top_open}
		\caption*{WARP Charger Basic}
	\end{figure}

	Bei den Wallbox-Varianten Basic und Smart
	wird die Zuleitung an einen internen Klemmenblock
	angeschlossen. Um bei starren Leitern maximalen Bewegungsspielraum zu bieten,
	werden die Adern um den Klemmenblock geführt und mittels der freien
	Federklemmplätze angeschlossen. Die Adern werden anhand der Reihenfolge und
	Klemmenbeschriftungen in die Klemmen gesteckt.

	Als Letztes muss die Kabelverschraubung festgezogen werden. Die Verschraubung
	hat einen Klemmbereich von \SI{11}{\milli\meter} bis \SI{22}{\milli\meter} und soll laut Hersteller mit
	\SI{10}{\newton{}\meter} angezogen werden.

	Der korrekte Sitz der Adern und die Phasenzugehörigkeit ist nach der
	Installation zu prüfen! Alle Verschraubungen innerhalb der Wallbox sind nachzuziehen.
	Fortfahren mit \fullref{ladestrom_schalter}.


	\subsubsection{Variante Pro}

	\begin{figure}[H]
		\gfx{./img_warp2/resized/warp2_pro_top_open}
		\caption*{WARP Charger Pro}
	\end{figure}

	Die Variante Pro verfügt aus Platzgründen nur über einen Klemmenblock für PE. Die
	Zuleitungsadern außer PE müssen oben an den Zähler angeschlossen werden.

	Als Letztes muss die Kabelverschraubung festgezogen werden. Die Verschraubung
	hat einen Klemmbereich von \SI{11}{\milli\meter} bis \SI{22}{\milli\meter} und soll laut Hersteller mit
	\SI{10}{\newton{}\meter} angezogen werden.

	Der korrekte Sitz der Adern und die Phasenzugehörigkeit ist nach der
	Installation zu prüfen! Alle Verschraubungen innerhalb der Wallbox sind
	nachzuziehen. Fortfahren mit \fullref{ladestrom_schalter}.

	\subsubsection{Varianten mit werkseitig angeschlossener Zuleitung}
	Wird die Wallbox mit einer ab Werk vorinstallierten Zuleitung bestellt, so
	muss diese extern verbunden werden. Die Farben sind nach DIN belegt und wie
	folgt zugeordnet: L1 braun, L2 schwarz, L3 grau, N blau, PE gelb/grün.

	Der korrekte Sitz der Adern und die Phasenzugehörigkeit ist nach der
	Installation zu prüfen! Fortfahren mit \fullref{ladestrom_schalter}.

	\subsubsection{Kabeleinführung von der Rückseite}
	Ab Version 2.1 des WARP Chargers kann die Kabel\-einführung von der Unterseite
	(Auslieferungszustand) umgebaut werden, so dass eine Kabeleinführung von der
	Rückseite erfolgt.

	Dazu muss die Kabeleinführung für die Zuleitung (M32) und die
	Kabeleinführung für das Netzwerkkabel vom Wallboxgehäuse abgeschraubt
	werden. Die Bohrungen in der Rückseite der Wallbox sind im
	Auslieferungszustand mit Blindstopfen von innen verschlossen.
	Diese müssen entfernt und in die nun offenen Bohrungen an der Unterseite
	eingeschraubt werden. Die Kabeleinführungen werden anschließend von
	der Rückseite in das Wallboxgehäuse eingeschraubt werden.

	\begin{figure}[H]
		\gfx{./img_warp2/resized/warp2_1_back_wp_ready_1000.jpg}
	\end{figure}

	\subsubsection{Einphasiger Betrieb}
	Alle Wallboxen können auch einphasig angeschlossen und betrieben werden.
	Dazu ist unbedingt Phase L1 anzuschließen, da diese Phase ebenfalls zur
	Stromversorgung der Wallbox genutzt wird. L2 und L3 werden von der Wallbox
	nur durchgeschaltet und können dementsprechend unbeschaltet bleiben.

	\subsubsection{Einstellen des Ladestroms}\label{ladestrom_schalter}
	Der maximal erlaubte Ladestrom muss abhängig von der gebäudeseitigen
	Leitungsabsicherung eingestellt werden. Der Ladestrom darf nicht höher gewählt
	werden, als die Leitungsabsicherung es zulässt.

	Zum Einstellen des Ladestroms muss der Deckel (siehe \fullref{wandmontage})
	geöffnet werden. Über zwei Schiebeschalter auf dem internen Ladecontroller (EVSE) wird der
	maximale Ladestrom eingestellt.

	\gfx{./img_warp2/resized/warp2_current_configure_w_caption_600}

	Die verschiedenen Schalterstellungen sind im obigen Foto dokumentiert.
	Der schwarze Block stellt dabei jeweils die Position
	des Schalters dar. Im Werkszustand sind die Schalter so eingestellt,
	dass die Wallbox inaktiv ist. Im Foto ist exemplarisch der obere
	Schalter auf die linke und der untere auf die mittlere Position gestellt
	worden. Damit wird eine maximale Ladeleistung bei einem dreiphasigen
	Betrieb, von \SI{9}{\kilo\watt} (3$\times\SI{13}{\ampere}$) vorgegeben.
	Wird die Wallbox nur einphasig angeschlossen, können maximal
	\SI{3}{\kilo\watt} (1$\times\SI{13}{\ampere}$) über die Wallbox vom Hausanschluss
	bezogen werden.

	\hint{Die Schalterstellung und der damit verbundene maximale Ladestrom dürfen nach der
	      Installation nur von einer ausgewiesenen Elektrofachkraft unter
	      Berücksichtigung der genannten Bedingungen geändert werden!}

	\begin{tabular}{lp{0.09\textwidth}rrr}
		\toprule
		\multicolumn{2}{c}{Schalterstellung} & Strom            & \multicolumn{2}{c}{Ladeleistung}             \\
		\small{oben (1)} & \small{unten (2)} &                  & \small{einphasig}    & \small{dreiphasig}    \\
		\midrule
		links            & links             & \SI{0}{\ampere}  & \SI{0}{\kilo\watt}   & \SI{0}{\kilo\watt}    \\
		mitte            & links             & \SI{6}{\ampere}  & \SI{1.4}{\kilo\watt} & \SI{4.1}{\kilo\watt}  \\
		rechts           & links             & \SI{10}{\ampere} & \SI{2.3}{\kilo\watt} & \SI{6.0}{\kilo\watt}  \\
		links            & mitte             & \SI{13}{\ampere} & \SI{3.0}{\kilo\watt} & \SI{9.0}{\kilo\watt}  \\
		mitte            & mitte             & \SI{16}{\ampere} & \SI{3.7}{\kilo\watt} & \SI{11.0}{\kilo\watt} \\
		rechts           & mitte             & \SI{20}{\ampere} & \SI{4.6}{\kilo\watt} & \SI{13.8}{\kilo\watt} \\
		links            & rechts            & \SI{25}{\ampere} & \SI{5.6}{\kilo\watt} & \SI{17.3}{\kilo\watt} \\
		mitte            & rechts            & \SI{32}{\ampere} & \SI{7.4}{\kilo\watt} & \SI{22.0}{\kilo\watt} \\
		\bottomrule
	\end{tabular}

	\newpage
	\subsubsection{LAN- / RJ45-Kabel anfertigen}\label{ethernet}

	Um den WARP Charger mittels LAN anzubinden muss ein LAN-/ RJ45-Kabel
	angefertigt werden. Je nach Version der Wallbox unterscheidet sich hierbei
	die Ausführung.

	\paragraph{Warp 2.1}

	Ab Version 2.1 des WARP Chargers kann das RJ45-Kabel einfach mittels einer
	Kabeldurchführung in die Wallbox geführt werden. In der Wallbox befindet
	sich eine kabelgebundene RJ45-Buchse an der das eingeführte Kabel einfach
	mittels eines am Kabel anzubringenden RJ45-Steckers angeschlossen werden
	kann. Es können somit auch werkzeuglose oder über einen LSA Anschluss
	verfügende RJ45 Stecker genutzt werden.

	\gfx{./img_warp2/resized/warp2_1_top_ready_1000.jpg} % Ethernetkabel

	\paragraph{Warp 2.0}

	Bei dem WARP Charger 2.0 befindet sich eine spritzwassergeschützte
	RJ45-Durchführung auf der Unterseite an die der in der Wallbox verbaute Controller intern
	angeschlossen ist. Um ein LAN-Kabel anzuschließen muss der Blinddeckel
	abgeschraubt werden.

	\gfx{./img_warp2/resized/warp2_ethernet3_circle_600} % Blinddeckel

	Anschließend muss ein LAN-Kabel (z.B. Cat.~7) wie folgt
	angefertigt werden:

	\columnbreak

	\begin{enumerate}
		\item LAN-Kabel durch den Aufsatz ziehen
		\item Mitgelieferten RJ45-Stecker auf das LAN-Kabel crimpen. Die
		Kontaktierung erfolgt typischerweise nach TIA-568 Schema A oder B.
		Das verwendete Schema sollte auf beiden Kabelseiten identisch sein.

		\gfx{./img_warp2/resized/warp2_rj45_1_600}
		\hint{Es sollte der mitgelieferte RJ45-Stecker verwendet werden. Werkzeuglose RJ45-Stecker können auf Grund des begrenzten
		Platzes im Steckeraufsatz nicht verwendet werden.}

		\item Kabel im Aufsatz zurückziehen und die Zugentlastung handfest anziehen

		\gfx{./img_warp2/resized/warp2_rj45_2_600}
	\end{enumerate}

	Zum Schluss wird der RJ45-Stecker in die Wallbox eingesteckt und die Überwurfmutter
	handfest angezogen.

	\gfx{./img_warp2/resized/warp2_ethernet4_600} % Montiert

	\newpage
	\subsection{Prüfungen}\label{tests}
	Im Werk wurde jede Wallbox einzeln nach IEC 60364-6 sowie den entsprechenden gültigen
	nationalen Vorschriften geprüft, das jeweilige Messprotokoll liegt bei.
	Vor der ersten Inbetriebnahme ist dennoch eine Prüfung der Gesamtinstallation
	nach den selben Vorschriften notwendig.

	Die Wallbox führt in den ersten drei Sekunden nach dem Herstellen der Stromversorgung
	eine DC-Fehlerstromerkennungskalibrierung durch (siehe \ref{fast_blink} Blaue LED blinkt sehr schnell).
	Ein Ladevorgang kann erst nach dieser Kalibrierung beginnen.

	Bei der Messung des Isolationswiderstands wird für L1 ein niedrigerer Wert
	gemessen (ca. \SI{249}{\kilo\ohm}). Dies hat den Hintergrund, dass
	der verbaute Ladecontroller über je einen Optokoppler mit
	\SI{249}{\kilo\ohm} Vorwiderstand, vor und nach dem Schütz, zwischen L1 und
	PE verfügt (Erdungsüberwachung, Schützüberwachung). Wird während der Messung ein EVSE-Adapter verwendet,
	kann es aufgrund der genannten Überwachungsschaltung in Wechselwirkung mit dem EVSE-Adapter zu Fehlmessungen
	auf L2, L3 und N (gemessen gegen PE) kommen. Ist dies der Fall, so muss die Isolationsmessung
	ohne EVSE-Adapter direkt am Typ-2-Stecker durchgeführt werden.

	Die interne DC-Fehlerstromerkennung wird von der Wallbox automatisch getestet.

	Nachdem die Wallbox installiert
	und die korrekte elektrische Installation überprüft wurde, kann die Wallbox in
	Betrieb genommen werden.
	Im ersten Schritt wird die Stromversorgung zur Wallbox eingeschaltet. Die
	blaue LED blinkt anschließend sehr schnell. Die Wallbox führt
	für die ersten drei Sekunden eine Kalibrierung der
	DC-Fehlerstromerkennung durch. Nach Abschluss dieser Kalibrierung
	leuchtet die LED dauerhaft. Die Wallbox ist nun betriebsbereit. Sollte die LED jetzt
	nicht permanent leuchten wurde ein Fehler erkannt (siehe \fullref{fehlerbehebung}).

	Als Nächstes kann ein Elektrofahrzeug zum Laden mit der Wallbox verbunden
	werden. Dazu wird die Schutzkappe vom Ladestecker entfernt und den Stecker in die
	Ladebuchse des Elektrofahrzeugs gesteckt. Nach einer kurzen Zeit sollte hörbar
	das Schütz in der Wallbox schalten und das Fahrzeug sollte den Beginn
	der Ladung anzeigen. Die blaue LED \enquote{atmet} während des
	Ladevorgangs. Ist die Ladung beendet, so leuchtet die LED permanent. Nach ca.
	15 Minuten Inaktivität schaltet sich die LED aus.

	\subsection{Bedienelemente}\label{lockswitch}
	Das Drücken des Tasters auf der Frontseite unterbricht einen aktiven Ladevorgang
	sofort. Alternativ kann das Ladekabel vom Elektrofahrzeug entriegelt werden,
	wodurch der Ladevorgang ebenfalls unterbrochen wird. Um den Ladevorgang erneut
	zu starten, muss in beiden Fällen die Verbindung zum Fahrzeug getrennt und
	anschließend erneut hergestellt werden (Kabel aus- und wieder einstecken).

	Zusätzlich verfügen die Wallbox-Varianten Smart und Pro über ein NFC-Modul. Eine
	genaue Beschreibung befindet sich im Kapitel \fullref{NFC}.

	\newpage
	\section{Erste Schritte}\label{setup}

	Bei dem WARP Charger Basic sind nach der elektrischen Installation
	keine weiteren Einstellmöglichkeiten möglich. Die nachfolgende
	Beschreibung bezieht sich daher nur auf die Smart bzw. Pro Variante des WARP
	Chargers.

	Um weitere Einstellungen durchführen zu können, muss zuerst eine Verbindung
	zur Wallbox hergestellt werden, damit diese dann über den Browser
	konfiguriert werden kann.

	\subsection{Schritt 1: Verbindung zur Wallbox herstellen}

	\paragraph{Option 1: WLAN}
	Im Werkszustand öffnet die Wallbox einen WLAN-Access-Point. Über diesen kann
	die Konfiguration der Wallbox vorgenommen werden, indem auf das
	das Webinterface der Wallbox zugegriffen wird.

	Die Zugangsdaten des Access-Points findest du auf dem WLAN-Zugangsdaten-Aufkleber
	auf der Rückseite dieser Anleitung. Du kannst entweder den QR-Code des Aufklebers verwenden,
	der das WLAN automatisch konfiguriert, oder SSID und Passphrase abschreiben.
	Die meisten Kamera-Apps von Smartphones unterstützen das Auslesen des
	QR-Codes und das automatische Verbinden zu dem WLAN. Somit musst du die
	Zugangsdaten dann nicht abtippen. Wichtig ist, dass viele Smartphones
	erkennen, dass über das WLAN der Wallbox (Access-Point) kein Zugriff auf das
	Internet möglich ist. Dein Telefon fragt dann nach, ob du zu dem WLAN
	verbunden bleiben möchtest. Damit du weiter auf die Wallbox zugreifen
	kannst, darfst du das WLAN nicht wieder verlassen.

	\begin{minipage}{0.35\textwidth}
		Wenn die Verbindung mit dem Access-Point der Wallbox hergestellt ist, kannst du das Webinterface
		unter \url{http://10.0.0.1} über einen Browser deiner Wahl erreichen.
		Alternativ kannst du dazu den nebenstehenden QR-Code scannen.
		Eventuell musst du deine mobile Datenverbindung (z.B. LTE) deaktivieren.
	\end{minipage}\hfill
	\begin{minipage}{0.12\textwidth}
		\begin{flushright}
			\qrcode{http://10.0.0.1}
		\end{flushright}
	\end{minipage}

	\paragraph{Option 2: LAN}
	Als Alternative zum Zugriff über den WLAN-Accesspoint verbindet sich die
	Wallbox in den Werkseinstellungen automatisch zu einem
	kabelgebundenen Netzwerk (LAN), wenn ein LAN-Kabel eingesteckt ist (IP Bezug
	mittels DHCP). Die Wallbox kann dann entweder über die zugewiesene IP
	Adresse (\url{http://[IP-der-Wallbox]}, z.B. \url{http://192.168.0.42})
	oder den Hostnamen der Wallbox (\url{http://[hostname]}, z.B.
	\url{http://warp2-ABC}) erreicht werden.

	Der Hostname der Wallbox ist identisch zur SSID des WLANs. Den Hostnamen findest du
	auf dem WLAN-Zugangsdaten-Aufkleber auf der Rückseite dieser Anleitung.

	Kann die per DHCP vergebene IP der Wallbox nicht ermittelt werden, so kann der
	zuvor genannte Zugriff auf die Wallbox mittels WLAN-Access-Point genutzt
	werden um die IP Adresse der LAN Schnittstelle zu ermitteln (\glqq
	Status-Seite\grqq, Abschnitt \glqq LAN-Verbindung\grqq).

	\paragraph{Konfiguration mittels Webinterface}
	Ist die Wallbox nun per WLAN (Accesspoint) oder LAN mittels Browser erreichbar,
	können alle weiteren Einstellungen darüber durchgeführt werden.
	Das Webinterface ist unter \fullref{webinterface} vollständig beschrieben.

	\subsection{Schritt 2: Integration in das eigene Netzwerk}
	In den allermeisten Fällen soll die Wallbox in das eigene WLAN/LAN
	integriert werden. Dazu müssen die Netzwerkeinstellungen der Wallbox
	angepasst werden. Wie dies funktioniert ist im Abschnitt
	\fullref{network} beschrieben.

	\subsection{Schritt 3: Weitere Optionen}
	Generell empfehlen wir nach der Installation ein Update der Firmware der
	Wallbox. Somit erhältst du die neusten Funktionen und ggf. Bugfixes. Wie ein
	Firmware-Update durchgeführt wird, ist unter \fullref{firmware-update}
	beschrieben.

	Weitere Einstellungen hängen vom Verwendungszweck der Wallbox ab. Teilen
	sich mehrere Wallboxen einen Stromanschluss kann die Konfiguration eines
	Lastmanagements gewünscht sein (siehe Abschnitt \fullref{charge_manager}).

	Soll eine Ladefreigabe
	nur mittels NFC möglich sein, Ladevorgänge Nutzern oder Fahrzeugen
	zugeordnet werden (Ladelogbuch) oder das Webinterface mit einem Passwort
	geschützt werden so ist dies im Abschnitt \fullref{user_management} beschrieben.
	Bei einer Ersteinrichtung empfehlen wir zuerst die Benutzer anzulegen und
	anschließend den Benutzern NFC Tags zuzuordnen (Abschnitt \fullref{NFC}).

	Am besten du schaust die diversen Möglichkeiten im Webinterface an und
	entscheidest selbst, welche Optionen du nutzen möchtest.

	\section{Webinterface}\label{webinterface}
	Das Webinterface der Wallbox ist nur bei den Varianten Smart und Pro verfügbar.

	Über das Webinterface kannst du unter anderem das Laden steuern und überwachen.
	Es können diverse Einstellungen vorgenommen werden, die nachfolgend
	dokumentiert sind.

	Wenn du auf das Webinterface der Wallbox mit einem Browser zugreifst
	gelangst du auf die Start-/ Statusseite. Auf der linken Seite befindet sich
	die Menüleiste, über die du zu weiteren Einstellungsmöglichkeiten kommst.

	Auf mobilen Endgeräten wird
	diese Menüleiste stattdessen versteckt unter einem Menü-Symbol oben rechts
	im grauen Balken neben dem WARP Logo angezeigt (\glqq drei Striche untereinander\grqq).
	Hier kannst du das Menü durch einen Klick auf das Symbol ausklappen.

	\vspace{-0.2cm}
	\subsection{Status (Startseite)}
	Die Startseite des Webinterfaces zeigt kompakt den aktuellen Ladestatus der
	Wallbox, sowie Ladezeit und -strom, und erlaubt es, die Ladung zu steuern.
	Du kannst hier sowohl das automatische Laden (de-)aktivieren, als auch
	manuell eine Ladung starten oder stoppen.

	Außerdem wird der aktuell laufende Ladevorgang, die letzten drei
	Ladevorgänge, sowie der Status weiterer Features angezeigt.
	In der Variante Pro mit verbautem Stromzähler wird zusätzlich der Ladeverlauf
	über die letzten 48 Stunden und die aktuelle Leistungsaufnahme gezeigt.

	Der \textbf{Fahrzeugstatus} gibt dir die Information, ob aktuell ein
	Fahrzeug mit der Wallbox verbunden ist und ob dieses geladen wird.

	Die \textbf{Ladesteuerung} ermöglicht es dir, manuell einen Ladevorgang zu
	starten oder zu stoppen. Wenn \textbf{Autostart} deaktiviert ist, wird die
	Wallbox niemals einen Ladevorgang automatisch starten. In diesem Fall hast du
	manuell die Kontrolle mittels Start/Stop. Ist Autostart aktiviert,
	startet der Ladevorgang automatisch, sobald ein Fahrzeug
	angeschlossen wird und keine weiteren Freigabemechanismen (z.B. NFC) den
	direkten Start verhindern.

	Der \textbf{konfigurierte Ladestrom} bietet eine einfache Möglichkeit, den Ladestrom, der
	einem Fahrzeug erlaubt wird, einzustellen. Minimal können \SI{6}{\ampere} eingestellt werden.
	Das Maximum hängt vom Anschluss, sowie der Konfiguration deiner Wallbox ab.

	Der \textbf{erlaubte Ladestrom} gibt an, welcher Ladestrom derzeit einem Fahrzeug erlaubt
	wird beziehungsweise würde. Der Ladestrom ist das Minimum aller begrenzenden Faktoren wie
	beispielsweise dem Anschluss der Wallbox, eventuellen Grenzen pro konfiguriertem Benutzer,
	dem Lastmanagement usw.

	\gfx{./img_warp2/resized/web_status}

	\textbf{Lastmanager} zeigt den aktuellen Zustand des Lastmanagers an, falls diese Wallbox
	andere Wallboxen steuert. Hier kannst du den \textbf{verfügbare Strom} des Lastmanagement-Verbunds
	einstellen und der Zustand der \textbf{kontrollierten Wallboxen} wird angezeigt.

	\textbf{Ladeverlauf} und \textbf{Leistungsaufnahme} sind nur in der Variante Pro
	vorhanden. Hier werden dir die aktuelle Leistungsaufnahme und ein Diagramm über
	die letzten 48 Stunden angezeigt.

	\textbf{Letzte Ladevorgänge} zeigt einen Verlauf der zuletzt durchgeführten Ladevorgänge an.
	Je nach Variante der Wallbox und Konfiguration können Ladevorgänge Benutzern zugeordnet sein
	und der geladene Strom kann aufgezeichnet werden.

	\textbf{WLAN-Verbindung} zeigt an, ob eine Verbindung konfiguriert ist, ob sie erfolgreich aufgebaut wurde und
	unter welcher IP-Adresse die Wallbox per WLAN erreichbar ist.

	\textbf{LAN-Verbindung} zeigt analog an, ob eine LAN-Verbindung besteht und unter welcher IP-Adresse die Wallbox erreichbar ist.

	Der \textbf{WLAN-Access-Point}-Status bildet den Status des Access-Points ab.
	\enquote{Deaktiviert} beziehungsweise \enquote{Aktiviert} zeigt den Zustand, wenn der Access-Point nicht
	nur als Fallback für die WLAN-Verbindung verwendet wird. Falls der Status \enquote{Fallback inaktiv} ist,
	war die WLAN-Verbindung bzw. LAN-Verbindung erfolgreich und der Access-Point wurde deshalb deaktiviert.
	Beim Status \enquote{Fallback aktiv} ist der Aufbau der WLAN-Verbindung fehlgeschlagen und der
	Access-Point wurde deshalb aktiviert.

	\textbf{Zeitsynchronisierung} zeigt an, ob Datum und Uhrzeit per Netzwerk-Zeitsynchronisierung (NTP) aktualisiert werden konnten.

	\textbf{WireGuard-Verbindung} zeigt an, ob die konfigurierte WireGuard-VPN-Verbindung aufgebaut werden konnte. Hierfür ist eine bestehende Zeitsynchronisierung eine zwingende Voraussetzung.

	\textbf{MQTT-Verbindung} zeigt den aktuellen Status der MQTT-Verbindung
	zum konfigurierten Broker an.

	\subsection{Ladecontroller}\label{evse}
	Die Unterseite des Ladecontrollers gibt detaillierte Auskunft über den Zustand
	des Ladecontrollers (EVSE) und dessen Hardware-Konfiguration. Probleme beim Laden
	kannst du mit den Informationen dieser Seite diagnostizieren. Außerdem können hier verschiedene
	Einstellungen vorgenommen werden:
	\begin{description}
	 \item[Externe Steuerung] Wenn die externe Steuerung erlaubt ist, darf eine externe Steuerungssoftware, beispielsweise
	 EVCC (\rurl{https://evcc.io}{evcc.io}) den WARP Charger steuern. Eine Steuerungssoftware kann auch selbst entwickelt werden, hierzu stellen
	 wir unter \rurl{https://warp-charger.com/api.html}{warp-charger.com/api.html} eine detaillierte API-Dokumentation zur Verfügung.
	 \item[Tastereinstellung] Hiermit wird konfiguriert, welche Funktion der Taster an der Front
	 des WARP Chargers ausführen soll. Im halb-öffentlichen Raum kann es beispielsweise sinnvoll sein,
	 den Ladestop per Taster zu verbieten.
	 \item[Abschalteingang] Am Abschalteingang kann zum Beispiel ein Rundsteuerempfänger angeschlossen werden.
	 Hier kann eingestellt werden, wie auf Änderungen am Abschalteingang reagiert werden soll.
	 \item[Konfigurierbarer Eingang] Der konfigurierbare Eingang kann über die API abgefragt werden.
	 \item[Konfigurierbarer Ausgang] Der konfigurierbare Ausgang kann über die API gesteuert werden.
	\end{description}

	Auf dieser Unterseite werden außerdem die aktuellen Ladestromgrenzen angezeigt. Alle Grenzen, die
	derzeit aktiv sind, werden zur Entscheidung, ob ein Ladevorgang erlaubt ist und zur Berechnung des maximalen Ladestroms einbezogen:
	Nur wenn alle aktiven Ladestromgrenzen nicht blockieren, wird eine Ladung erlaubt.
	Der erlaubte Ladestrom ist dann das Minimum aller aktiven Grenzen. Folgende Grenzen können Teil der Berechnung sein:

	\begin{description}
		\item[Zuleitung] Der Maximalstrom der Zuleitung zum WARP Charger.
			Wird über die Schalter auf dem Ladecontroller konfiguriert. Siehe \fullref{ladestrom_schalter}.
		\item[Typ-2-Ladekabel] Der Maximalstrom des Typ-2-Ladekabels. Wird durch einen festen Widerstand im Kabel vorgegeben.
		\item[Abschalteingang] Je nach Konfiguration des Abschalteingangs kann diese Ladestromgrenze blockieren oder freigeben.
		\item[Konfigurierbarer Eingang] Wird noch nicht verwendet.
		\item[Autostart bzw. Taster] Die Autostart-Einstellung bzw. das Drücken des Tasters können diese Ladestromgrenze blockieren oder freigeben.
		\item[Konfiguration] Diese Ladestromgrenze wird durch das Eingabefeld auf der Statusseite eingestellt.
			Durch den \enquote{Freigeben}-Button wird eine eventuell eingetragene Ladestromgrenze komplett aufgehoben.
		\item[Benutzer] Falls die Benutzerautorisierung aktiviert ist blockiert diese Ladestromgrenze bis ein Benutzer die Ladung durch ein NFC-Tag freigibt.
			Danach wird die diesem Benutzer zugeordnete Ladestromgrenze eingetragen.
		\item[Lastmanagement] Empfangene Nachrichten des Lastmanagers steuern diese Ladestromgrenze, falls das Lastmanagement aktiviert ist.
		\item[Externe Steuerung] Diese Ladestromgrenze wird durch eine externe Steuerung über die API, beispielsweise EVCC gesteuert.
			Die Begrenzung kann durch den \enquote{Freigeben}-Button komplett aufgehoben werden.
	\end{description}

	\gfx{./img_warp2/resized/web_evse2}

	% Colors match those in the web interface
	\definecolor{mygray}{RGB}{108, 117, 125}
	\definecolor{mygreen}{RGB}{40, 167, 69}
	\definecolor{myblue}{RGB}{0, 123, 255}
	\definecolor{myorange}{RGB}{255, 193, 7}
	\definecolor{myred}{RGB}{220, 53, 69}
	Die Farbmarkierung neben einer Grenze haben folgende Bedeutung:
	\begin{itemize}
	 \item[\textbf{\textcolor{mygray}{Grau}}] Diese Ladestromgrenze ist nicht aktiv. Sie kann die Ladung nicht blockieren und geht nicht in Berechnung des erlaubten Ladestroms ein.
	 \item[\textbf{\textcolor{mygreen}{Grün}}] Diese Ladestromgrenze ist aktiv, beschränkt den erlaubten Ladestrom aber nicht.
	 \item[\textbf{\textcolor{myblue}{Blau}}] Diese Ladestromgrenze ist aktiv und gibt ein Ladestromlimit vor. Es gibt aber andere aktive Grenzen, die den Ladestrom stärker limitieren.
	 \item[\textbf{\textcolor{myorange}{Gelb}}] Diese Ladestromgrenze ist aktiv, blockiert die Ladung nicht, gibt aber die aktuell stärkste Limitierung des Ladestroms vor.
	 \item[\textbf{\textcolor{myred}{Rot}}] Diese Ladestromgrenze ist aktiv und blockiert die Ladung.
	\end{itemize}

	\subsection{Stromzähler (Nur Pro)}
	Auf der Seite siehst du ein Diagramm der Leistungsaufnahme der letzten 48 Stunden und Statistiken dazu.
	Außerdem wird angezeigt, auf welchen Phasen gerade geladen wird.
	Die Detail-Ansicht zeigt eine Vielzahl an Messwerten, beispielsweise Leistungswerte und -faktoren,
	Phasenverschiebungen, THD-Werte und Energiemessungen.

	\gfx{./img_warp2/resized/web_meter}

	\newpage

	\subsection{Netzwerk}\label{network}
	Die Wallbox kann in dein Netzwerk per WLAN oder LAN eingebunden werden.
	In diesem Unterabschnitt können alle dazugehörigen Einstellungen vorgenommen werden.

	\subsubsection{Allgemein}
	Hier kannst du den Hostnamen des WARP Chargers in allen verbundenen Netzwerken konfigurieren. Außerdem kann mDNS aktiviert oder deaktiviert werden.
	Über mDNS können andere Geräte im Netzwerk den WARP Charger finden. Damit wird zum Beispiel das Einrichten eines Lastmanagementverbunds vereinfacht

	\gfx{./img_warp2/resized/web_network}


	\subsubsection{WLAN-Verbindung}
	\gfx{./img_warp2/resized/web_wifi_sta}
	Eine Möglichkeit um die Wallbox in dein Netzwerk zu integrieren ist eine
	Anbindung mittels WLAN. Dieses kannst du hier konfigurieren.
	Durch Drücken des \enquote{Netzwerksuche}-Buttons öffnet sich ein Menü, in dem das gewünschte WLAN ausgewählt werden kann.
	Es werden dann automatisch Netzwerkname (SSID) und BSSID eingetragen, sowie die Verbindung beim Neustart aktiviert.
	Gegebenenfalls musst du jetzt noch die Passphrase des gewählten Netzes eintragen.

	Du kannst jetzt die Konfiguration mit dem Speichern-Button abspeichern.
	Das Webinterface startet dann neu und verbindet sich zum konfigurierten WLAN. Die Statusseite zeigt
	an, ob die Verbindung erfolgreich war. Der Access-Point bleibt weiterhin
	geöffnet, sodass Konfigurationsfehler behoben werden können.
	Da der Access-Point den selben Kanal wie ein eventuell verbundenes Netz verwendet,
	kann es sein, dass du dich jetzt neu zum Access-Point verbinden musst.

	Bei einer erfolgreichen Verbindung sollte die Wallbox jetzt im konfigurierten Netzwerk unter
	\url{http://[konfigurierter_hostname]}, z.B. \url{http://warp2-ABC} erreichbar sein.

	\subsubsection{WLAN-Access-Point}
	\gfx{./img_warp2/resized/web_wifi_ap}
	Der Access-Point kann in einem von zwei Modi betrieben werden: Entweder kann er immer aktiv sein,
	oder nur dann, wenn die Verbindung zu einem anderen WLAN bzw. zu einem LAN nicht konfiguriert oder fehlgeschlagen ist.
	Außerdem kann der Access-Point komplett deaktiviert werden.
	\hint{Wir empfehlen, den Access-Point nie komplett zu deaktivieren, da sonst bei einer
		fehlgeschlagenen Verbindung zu einem anderen Netzwerk das Webinterface nicht mehr erreicht
		werden kann. Die Wallbox kann dann nur über den \fullref{recovery} oder ein Zurücksetzen auf Werkszustand, siehe \ref{reset}, erreicht werden.}
	Die notwendigen Einstellungen, wie der Modus des Access-Points,
	Netzwerkname, Passphrase usw. müssen dazu hier festgelegt werden.

	\subsubsection{LAN-Verbindung}
	\gfx{./img_warp2/resized/web_ethernet}
	Alternativ zur WLAN-Verbindung kann die Wallbox auch per LAN kabelgebungen
	ins Netzwerk integriert werden. In den meisten Fällen wird eine
	LAN-Verbindung automatisch hergestellt, falls ein Kabel eingesteckt ist
	(IP Adresse wird per DHCP bezogen). Es ist aber auch möglich,
	eine statische IP-Konfiguration	einzutragen, oder, falls gewünscht, die LAN-Verbindung
	komplett zu deaktivieren.

	Bei einer erfolgreichen Verbindung sollte die Wallbox jetzt im LAN unter
	\url{http://[konfigurierter_hostname]}, z.B. \url{http://warp2-ABC} erreichbar sein.

	\subsubsection{Zeitsynchronisierung}\label{ntp}
	Um für den Ladetracker und das Ereignis-Log die aktuelle Uhrzeit zur Verfügung zu haben, kann der WARP Charger diese per NTP über
	eine Netzwerkverbindung synchronisieren. Auf dieser Unterseite kannst du NTP aktivieren oder deaktivieren und die Zeitzone, in der sich
	der WARP Charger befindet konfigurieren.

	Außerdem ist es möglich, zusätzlich zum konfigurierten Zeitserver einen Zeitserver zu verwenden, der von deinem Router per DHCP gesetzt wird. Dies funktioniert allerdings nur,
	wenn in der Netzwerkkonfiguration keine statische IP-Konfiguration verwendet wurde.

	\gfx{./img_warp2/resized/web_ntp}

	\subsubsection{WireGuard}

	WireGuard ist eine Möglichkeit die Wallbox in ein virtuelles privates Netzwerk (VPN)
	mittels einer verschlüsselten Verbindung einzubinden. WireGuard wird von
	verschiedenen Routern direkt unterstützt. Dies kann zum Beispiel genutzt
	werden um aus der Ferne auf die Wallboxen zuzugreifen und das
	Wallbox-Netzwerk vor einem Zugriff zu schützen. Zusätzlich kann das
	Lastmanagement zwischen den Wallboxen per WireGuard abgesichert werden.

	Die notwendigen Parameter sind WireGuard-typisch und werden an dieser Stelle
	nicht gesondert erläutert. Weitere Informationen finden sich auf
	\url{https://www.wireguard.com/}.

	\gfx{./img_warp2/resized/web_wireguard}

	\subsection{System}
	Im System-Unterabschnitt kannst du das Ereignis-Log einsehen und Firmware-Aktualisierungen einspielen.
	Außerdem können hier die Benutzer der WARP Chargers verwaltet werden (Siehe \ref{user_management}).

	\subsubsection{Ereignis-Log}
	\gfx{./img_warp2/resized/web_event_log}

	\newpage

	Das Ereignis-Log zeichnet relevante Informationen des Systemstarts, sowie WLAN- und MQTT-Verbindungsabbrüche und Ladefehler auf.
	Falls Probleme mit der Wallbox auftreten, kannst du diese mit dem Log diagnostizieren.
	Falls du ein Problem mit der Wallbox an uns melden möchtest, kannst du das Ereignis-Log,
	sowie einen Debug-Report abrufen, die uns helfen das Problem zu verstehen und zu lösen.

	\subsubsection{Firmware-Aktualisierung}\label{firmware-update}
	\gfx{./img_warp2/resized/web_firmware_update}
	Hier kannst du die Firmware der Wallbox aktualisieren. Wir entwickeln die Funktionalität
	der Wallbox laufend weiter. Bitte beachte, dass daher ggf. auch eine neue
	Version dieser Betriebsanleitung bereitgestellt wird.
	Die aktuelle Firmware und die neuste Betriebsanleitung findest du unter
	\rurl{https://warp-charger.com}{warp-charger.com} zum Download.

	Außerdem kannst du hier das Webinterface neustarten, ohne einen Ladevorgang zu unterbrechen.

	\subsection{Wiederherstellungsmodus}\label{recovery}
	Falls die Wallbox weder ihren Access Point öffnet, noch über ein konfiguriertes Netzwerk auf das Webinterface zugegriffen werden kann,
	kannst du wie folgt den Wiederherstellungsmodus starten:
	\begin{enumerate}
	 \item Mache die Wallbox stromlos
	 \item Halte den Taster im Deckel gedrückt
	 \item Versorge die Wallbox wieder mit Strom (ggfalls. mit einer zweiten Person)
	 \item Halte den Taster mindestens 10, aber maximal 30 Sekunden gedrückt
	\end{enumerate}
	Die Wallbox startet dann im Wiederherstellungsmodus. Zunächst werden die Netzwerkeinstellungen gelöscht, sowie die Anmeldung deaktiviert.
	Bei Erfolg sollte es jetzt möglich sein, über den Access Point wieder auf die Wallbox zuzugreifen.

	Durch erneutes Trennen und Verbinden der Stromversorgung innerhalb der ersten Minute im Wiederherstellungsmodus kann ein Zurücksetzen auf Werkszustand ausgelöst werden.

	\subsection{Zurücksetzen auf Werkszustand}\label{reset}
	Falls das Webinterface nicht korrekt funktioniert, oder die Konfiguration defekt ist,
	kannst du auf der Firmware-Aktualisierungs-Unterseite alle Einstellungen auf den Werkszustand zurücksetzen.
	\hint{Durch das Zurücksetzen auf Werkszustand gehen \mbox{\textbf{alle}} Konfigurationen, angelegte Benutzer, angelernte NFC-Tags und getrackte Ladungen verloren.}
	Nach dem Zurücksetzen startet das Webinterface wieder und öffnet
	den Access-Point mit der SSID und Passphrase, die auf dem Aufkleber vermerkt sind. Die Wallbox kann jetzt wieder nach \fullref{setup} konfiguriert werden.

	Damit getrackte Ladungen nicht verloren gehen kann alternativ nur die Konfiguration zurückgesetzt werden.
	Angelegte Benutzer (aber nicht der Benutzerverlauf des Ladetrackers) und NFC-Tags werden dennoch gelöscht.

	Falls du das Webinterface nicht mehr erreichen kannst, bestehen folgende Optionen:

	Falls eine Netzwerkverbindung aufgebaut werden kann, aber das Webinterface selbst nicht mehr funktioniert, kannst du versuchen, die Recovery-Seite zu öffnen.
	Falls du über den Access Point der Wallbox verbunden bist, erreichst du diese unter \url{http://10.0.0.1/recovery},
	bei einer bestehenden Verbindung zu einem LAN oder WLAN über \url{http://[konfigurierter_hostname]/recovery}, also z.B. \url{http://warp2-ABC/recovery}.
	Über die Recovery-Seite kannst du die Wallbox neustarten, Firmware-Updates einspielen,
	die Wallbox auf den Werkszustand zurücksetzen (Factory Reset), Debug-Reports
	herunterladen und die HTTP-API verwenden (siehe \fullref{http-interface})

	Alternativ kannst du die Wallbox (genauer: den verbauten ESP32 Ethernet
	Brick) neu flashen.
	Du benötigst dazu einen PC mit installiertem Brick Viewer 2.4.20 oder neuer. Diesen findest du unter
	\rurl{https://www.tinkerforge.com/de/doc/Software/Brickv.html}{tinkerforge.com/de/doc/Software/Brickv.html}.
	Außerdem benötigst du ein USB-C-Kabel um den Brick an deinen PC anzuschließen. Brick Daemon wird nicht benötigt.
	Gehe zum Neuflashen wie folgt vor:

	\columnbreak
	\begin{enumerate}
		\item Mache die Wallbox stromlos.
		\item Nimm den Deckel ab wie in \fullref{wandmontage} beschrieben.
		\item Rechts findest du den ESP32 Ethernet Brick. Schraube von diesem die vier schwarzen Kunststoffschrauben los (1) und ziehe das LAN-Kabel, sowie die zwei Bricklet-Kabel mit weißem Stecker
		ab (2). Danach kannst du den ESP32 Ethernet Brick aus der Wallbox nehmen.
	\end{enumerate}
	\gfx{./img_warp2/resized/warp_factory_reset_cropped}
	\begin{enumerate}
		\setcounter{enumi}{3}
		\item Den Brick musst du dann per USB-C an deinen PC anschließen und Brick Viewer starten
		\item Klicke links unten auf Updates / Flashing, dann oben auf Brick
	\end{enumerate}
	\gfx{./img_warp2/resized/warp_factory_flash}
	\begin{enumerate}
		\setcounter{enumi}{5}
		\item Bei Serial Port musst du den Port auswählen an dem der Brick angeschlossen ist.
		      Typischerweise sollte nur ein Port in der Liste auftauchen.
		      Der richtige Port ist einer, an dem ein \enquote{CP2102N USB to UART Bridge Controller} bzw. ein \enquote{ESP32 Ethernet Brick} aufgeführt wird.
		\item Wähle dann unter Firmware \enquote{WARP2 Charger} aus. Die aktuelle Firmware-Version wird automatisch ausgewählt.
		\item Klicke auf \enquote{Flash}
		\item Der Flash-Vorgang ist erst dann beendet, wenn die Status-LED des ESP32 Ethernet Brick beginnt blau zu blinken. Dies kann bis zu eine Minute länger dauern als das eigentliche Flashen.
		\hint{Der ESP32 Ethernet Brick darf nicht abgezogen werden, bevor die Status-LED blau blinkt. Andernfalls muss der Flash-Vorgang wiederholt werden.}
		\item Nachdem der Brick neu geflasht wurde, kannst du ihn vom PC abziehen und wie folgt in die Wallbox einbauen
		\item Stecke zuerst die weißen Bricklet-Stecker, sowie das LAN-Kabel wieder ein
		\item Schraube dann den Brick mit den Kunststoffschrauben auf die entsprechenden Abstandshalter.
		\item Schließe jetzt die Wallbox, indem du zuerst den Taster im Deckel, sowie den Erdungsstecker anschließt, den Deckel aufsetzt und die vier Schrauben festziehst
		\item Die Wallbox kann jetzt wieder mit Strom versorgt werden. Wenn der Flash-Vorgang erfolgreich war, sollte die Wallbox jetzt wieder den WLAN-Access-Point eröffnen und kann eingerichtet werden.
	\end{enumerate}

	\newpage
	\null
	\newpage

	\section{Schnittstellen zur Fernsteuerung der Wallbox}\label{interfaces}
	Die Wallbox kann per HTTP, MQTT, Modbus/TCP und OCPP ferngesteuert werden. Über diese Schnittstellen ist eine
	Einbindung in Hausautomatisationssysteme wie openHAB, ioBroker, FHEM o.ä.
	möglich. Auch eine Verwendung mit Lastmanagern oder Energiemanagern von Fremdanbietern
	ist darüber ebenfalls möglich.

	\subsection{HTTP}\label{http-interface}
	Eine Möglichkeit die Wallbox fernzusteuern ist HTTP. Dazu ist keine
	spezielle Konfiguration notwendig. Falls du die Zugangsdaten für das Webinterface gesetzt und die Anmeldung aktiviert hast, musst du
	für die HTTP-API die selben Zugangsdaten verwenden.
	Weitere Informationen über die HTTP-API der Wallbox befinden sich auf \rurl{https://warp-charger.com/api.html}{warp-charger.com/api.html}


	\subsection{MQTT}\label{mqtt-interface}

	\gfx{./img_warp2/resized/web_mqtt}
	Auf der MQTT-Unterseite kannst du die Verbindung zu einem MQTT-Broker konfigurieren. Folgende Einstellungen können vorgenommen werden:
	\begin{itemize}
		\item \textbf{Broker-Hostname oder -IP-Adresse} Der Host\-name oder die IP-Adresse des Brokers, zu dem sich die Wallbox verbinden soll.
		\item \textbf{Broker-Port} Der Port, unter dem der Broker erreichbar ist. Der typische MQTT-Port 1883 ist voreingestellt.
		\item \textbf{Broker-Benutzername} und \textbf{-Passwort} Manche Broker unterstützen eine Authentifizierung mit Benutzername und Passwort.
		\item \textbf{Topic-Präfix} Dieses Präfix wird allen Topics vorangestellt, die die Wallbox verwendet.
		      Voreingestellt ist warp/ABC, wobei ABC eine eindeutige Kennung pro Wallbox ist,
		      es sind aber andere Präfixe wie z.B. garage\_links möglich.
		      Falls mehrere Wallboxen mit dem selben Broker kommunizieren,
		      müssen eindeutige Präfixe pro Wallbox gewählt werden.
		\item \textbf{Client-ID} Mit dieser ID registriert sich die Wallbox beim Broker.
		\item \textbf{Sendeintervall} Der WARP Charger verschickt MQTT-Nachrichten nur, wenn sich die beinhalteten Daten geändert haben.
			Es gibt aber Teile der API, deren Daten sich sekündlich ändern. Das Sendeintervall kann hier reduziert werden, wenn weniger Netzwerktraffic
			erzeugt werden soll.
	\end{itemize}
	Nachdem die Konfiguration gesetzt und der \enquote{MQTT aktivieren}-Schalter aktiviert ist, kann die Konfiguration gespeichert werden.
	Das Webinterface startet dann neu und verbindet sich zum Broker.
	Auf der Status-Seite wird angezeigt, ob die Verbindung aufgebaut werden konnte.

	Weitere Informationen über die MQTT-API der Wallbox findest du auf \rurl{https://warp-charger.com/api.html}{warp-charger.com/api.html}

	\subsection{Modbus/TCP}

	\gfx{./img_warp2/resized/web_modbus_tcp}

	Mittels Modbus/TCP kann auf Funktionen der Wallbox zugegriffen werden.
	Als erstes muss mittels \textbf{Modbus/TCP-Modus} die Funktion aktiviert
	werden. Dazu kann entweder ein reiner Lesezugriff, d.h. ohne eine
	Steuerungsmöglickeit von außen oder ein Lese-/Schreibzugriff
	konfiguriert werden, mit dem z.B. Ladevorgänge gesteuert werden können.
	Der \textbf{Port} über dem die Funktion bereit gestellt
	wird, kann ebenfalls konfiguriert werden. Abschließend muss eine
	Registertabelle gewählt werden. Diese definiert, welche Funktionen unter
	welchen Registern bereit gestellt werden. Leider gibt es hier keinen
	allgemein nutzbaren Standard. Daher werden drei Möglichkeiten geboten:

	\begin{itemize}
		\item \textbf{WARP Charger} Diese Registertabelle bietet einen nahezu vollständigen Zugriff auf die Wallbox.
				Du findest sie im Abschnitt \fullref{modbus_tcp_registertabelle}, auf \rurl{https://warp-charger.com/api.html}{warp-charger.com/api.html} oder
				jeweils passend zur ausgeführten Firmware auf der Modbus/TCP-Unterseite des Webinterfaces.
		\item \textbf{Kompatibilität zu Bender CC613} Mit dieser Registertabelle emuliert der WARP Charger einen Bender CC613 Ladecontroller. Dieser wird in vielen Wallboxen verschiedener Hersteller verbaut.
		\item \textbf{Kompatibilität zu Keba C Series} Mit dieser Registertabelle emuliert der WARP Charger eine Wallbox der C-Series von Keba.
	\end{itemize}

	Sollen Fremdgeräte den WARP Charger fernsteuern, kann gegebenenfalls eine der
	kompatiblen Registertabellen verwendet werden.

	\subsection{OCPP}

	\gfx{./img_warp2/resized/web_ocpp}

	OCPP (Open Charge Point Protocol) ist ein standardisiertes Kommunikationsprotokoll zwischen
	Ladestationen und einem zentralen Managementsystem. Der WARP Charger
	unterstützt OCPPJ 1.6 Core Profile und Smart Charging Profile.

	\hint{08.12.2022: OCPP befindet sich zur Zeit in der Entwicklung und ist als
	Beta-Firmware verfügbar wird aber zeitnah Einzug in die offizielle Firmware
	halten. Bitte prüfe ob eine neue Firmware und eine neue Version dieser
	Betriebsanleitung verfügbar ist (siehe \fullref{firmware-update}).}

	Um OCPP zu nutzen muss auf der Konfigurationsseite nur OCPP aktiviert und die
	Endpoint-URL des Managementsystems eingetragen werden.


	\newpage
	\section{Lastmanagement zwischen mehreren WARP Chargern}\label{charge_manager}
	Mit dem Lastmanagement ist es möglich, einen verfügbaren Gesamt-Ladestrom zwischen bis zu 10 WARP Chargern aufzuteilen.
	Hierbei wird eine Wallbox als Lastmanager konfiguriert, die die weiteren bis zu 9 Wallboxen im Verbund steuert und ihnen Ladeströme
	zuweist. Es kann sowohl ein fester Gesamtstrom verteilt werden, um zum Beispiel den Hausanschluss nicht zu überlasten,
	als auch der Gesamtstrom über das Webinterface und die API dynamisch gesetzt werden
	um einen PV-Überschussstrom auf mehreren Wallboxen zu verteilen.

	\gfx{./img_warp2/resized/web_charge_manager}

	\subsection{Funktionsweise}
	Durch das Lastmanagement kontrollierte Wallboxen laden nur,
	wenn ihnen von außen ein erlaubter Ladestrom mitgeteilt wird. Wenn eine gewisse Zeit lang
	kein erlaubter Ladestrom empfangen wurde, stoppt die Wallbox die Ladung automatisch.
	Der Lastmanager stoppt seinerseits das Laden an allen kontrollierten Wallboxen,
	wenn eine Wallbox nicht mehr reagiert oder erreicht wird. Damit wird sichergestellt,
	dass der verfügbare Strom nicht überschritten wird.
	Der Lastmanager verteilt den verfügbaren Strom gleichmäßig zwischen allen Wallboxen, die laden bzw. ladebereit sind.
	Falls bereits eine Wallbox lädt, und an eine zweite ein Fahrzeug angeschlossen wird,
	wird der Ladestrom der ladenden Wallbox so beschränkt, dass für die zweite Ladung Strom verfügbar wird.

	\subsection{Konfiguration}
	Lastmanagement-Einstellungen werden für alle Wallboxen (egal ob Manager oder zu steuernde Wallbox) auf der
	Lastmanagement-Unterseite vorgenommen.

	Um das Lastmanagement zu verwenden, muss zunächst auf allen Wallboxen, die gesteuert werden sollen,
	der Lastmanagement-Modus auf \enquote{fremdgesteuert} konfiguriert werden.
	In diesem Modus lädt eine Wallbox nur noch, wenn die Ladung vom Lastmanager freigegeben wird.

	Auf der Wallbox, die die anderen Wallboxen steuern soll (dem Lastmanager) muss zunächst der Modus \enquote{Lastmanager} gewählt werden.
	Zusätzlich muss hier jede Wallbox, die gesteuert werden sollen, als \enquote{Kontrollierte Wallbox} hinzugefügt werden.
	Bei Klick auf \enquote{Wallbox hinzufügen} erscheinen nach wenigen Sekunden alle Wallboxen, die vom Lastmanager erreicht werden können.
	Durch Klicken auf eine gefundene Wallbox wird diese hinzugefügt. Wallboxen die nicht hinzugefügt werden können, werden grau hinterlegt.

	Im einfachsten Fall, in dem eine feste Menge Strom verteilt werden soll, muss nun nur noch dieser
	Strom als \enquote{Maximal verfügbarer Strom} konfiguriert werden.

	\subsection{Erweiterte Konfiguration}
	Je nach Einsatzzweck (z.B. PV-Überschussladen auf mehreren Wallboxen) können die folgenden Konfigurationen hilfreich sein.
	Diese werden für eine einfache Lastverteilung, z.B. \SI{16}{\ampere} auf zwei Wallboxen \textbf{nicht} benötigt.
	Die Konfigurationen finden sich unter den \enquote{Experteneinstellungen}.
	\vspace{-0.2cm}
	\paragraph{Voreingestellt verfügbarer Strom}
	Der voreingestellt verfügbare Strom ist der, der vom Lastmanagement verteilt werden darf, nachdem die steuernde Wallbox
	neugestartet wurde. Der verfügbare Strom kann über die API neu gesetzt werden, nach einem Neustart der Wallbox wird aber
	zunächst der voreingestellte Strom verwendet. Falls beispielsweise durch eine externe Steuerung der verfügbare PV-Überschussstrom
	gesetzt werden soll, kann der voreingestellte Strom auf \SI{0}{\ampere} konfiguriert werden, damit zwingend erst geladen wird,
	wenn die externe Steuerung mindestens einmal den verfügbaren Strom gesetzt hat
	\vspace{-0.2cm}
	\paragraph{Maximal verfügbarer Strom}
	Der maximal verfügbare Strom ist das Maximum, das über das Webinterface, bzw. die API als verfügbarer Strom gesetzt werden darf.
	Größere Ströme werden nicht akzeptiert. Falls eine externe Steuerung verwendet wird, empfehlen wir, den maximal verfügbaren Strom
	anhand der Kapazität der Zuleitungen und des Hausanschlusses so zu beschränken, dass durch die externe Steuerung nie zu große
	Ströme gesetzt werden können.

	\paragraph{Watchdog aktivieren}
	Der Watchdog erlaubt es der steuernden Wallbox, robust auf Ausfälle einer externen Steuerung zu reagieren. Falls über die API der Wallbox
	nicht mindestens alle 30 Sekunden der verfügbare Strom gesetzt wird, und der Watchdog aktiv ist, wird der verfügbare Strom wieder zurück auf den
	\enquote{Voreingestellt verfügbare Strom} gesetzt. Falls die externe Steuerung später wieder läuft, wird der Watchdog zurückgesetzt.
	\hint{Der Watchdog sollte nur dann aktiviert werden,
	wenn eine selbst programmierte Steuerung den für den Wallbox-Verbund verfügbaren Strom über die API dynamisch ändern soll.
	Für den normalen Lastmanagement-Betrieb ist der Watchdog nicht notwendig.}

	\paragraph{Stromverteilung protokollieren}
	Wenn aktiv, fügt der Lastmanager dem Ereignis-Log detaillierte Ausgaben hinzu, wann immer Strom umverteilt wird. Damit kann unerwartetes Verhalten des
	Lastmanagements untersucht werden.

	\paragraph{Minimaler Ladestrom}
	Der minimale Ladestrom ist der Strom, der für eine Wallbox zur Verfügung stehen muss, damit diese lädt. Dieser Strom muss mindestens \SI{6}{\ampere} betragen.
	Bestimmte Fahrzeuge laden aber erst bei höheren Strömen effizient. Mit einem WARP2 Charger Pro kann der Leistungsfaktor ermittelt werden.

	Mit dem minimalen Ladestrom kann zusätzlich gesteuert werden, wie viele Fahrzeuge gleichzeitig laden können.
	Maximal sind $\frac{\text{Verfügbarer Strom}}{\text{Minimaler Ladestrom}}$ Ladevorgänge gleichzeitig möglich. Falls beispielsweise nicht möglichst viele
	Fahrzeuge gleichzeitig, aber langsam, sondern ein Fahrzeug möglichst schnell nacheinander geladen werden soll, kann der minimale Ladestrom auf den selben Wert
	wie der verfügbare Strom gesetzt werden.

	\subsection{Fehlerbehebung}
	Bei der Verwendung des Lastmanagements können zwei Arten von Fehlern auftreten: Wallbox-Fehler, die nur eine spezifische Wallbox betreffen und Management-Fehler,
	bei deren Auftreten das Laden an \textbf{allen} gesteuerten Wallboxen gestoppt wird.

	Wallbox-Fehler müssen an der entsprechenden Wallbox behoben werden. Hier hilft Abschnitt \fullref{fehlerbehebung}. Im folgenden wird die Diagnose von Management-Fehlern erläutert:

	\paragraph{Kommunikationsfehler / Wallbox nicht erreichbar}
	Eine Wallbox kann nicht zuverlässig erreicht werden. Eventuell liegt ein Verbindungsproblem vor. Netzwerkverbindung bzw. Netzwerkkabel und IP-Konfiguration der Wallbox prüfen.

	\paragraph{Firmware inkompatibel}
	Das Lastmanagement benötigt kompatible Firmwares auf allen beteiligten Wallboxen. Die jeweils aktuellsten Firmwares sollten zueinander kompatibel sein,
	auch wenn WARP (1) und WARP 2 Charger in einem Lastmanagementverbund verwendet werden.

	\paragraph{Lastmanagement deaktiviert}
	Bei einer der zu steuernden Wallboxen ist das Lastmanagement deaktiviert. Damit ist keine Steuerung durch den Lastmanager möglich. Das Lastmanagement kann auf der Ladecontroller-Unterseite aktiviert werden.

	\paragraph{Ladecontroller nicht erreichbar}
	Der Ladecontroller einer Wallbox kann nicht erreicht werden, die Wallbox selbst aber schon. Ereignis-Log der betroffenen Wallbox prüfen.

	\paragraph{Ladecontroller reagiert nicht}
	Der Ladecontroller einer Wallbox reagiert nicht auf Stromzuweisungen. Eventuell ist auf diesem das Lastmanagement deaktiviert.

	\newpage
	\section{Ladetracker}
	\gfx{./img_warp2/resized/web_charge_tracker}
	Seit Firmware 2.0.0 zeichnet der WARP Charger alle durchgeführten Ladevorgänge auf. Pro Ladung werden die folgenden Informationen gespeichert:
	\begin{itemize}
	 \item Startdatum und Zeit der Ladung, falls Datum und Zeit bekannt sind. Siehe \fullref{ntp}.
	 \item Benutzer, der die Ladung gestartet hat, falls bekannt.
	\end{itemize}
	\hint{Damit Ladungen einem Benutzer zugeordnet werden können muss
		\begin{itemize}
			\item die Benutzerautorisierung des Ladecontrollers aktiviert sein (Siehe \fullref{evse})
			\item mindestens ein Benutzer angelegt sein (Siehe \fullref{user_management})
			\item dem Benutzer ein NFC-Tag zugeordnet sein (Siehe \fullref{NFC})
		\end{itemize}
		Im Werkszustand sind drei Benutzer mit jeweils einem NFC-Tag eingerichtet. Es muss dann nur die Benutzerautorisierung aktiviert werden.
	}
	\begin{itemize}
	 \item Zählerstand beim Start und Ende der Ladung (nur WARP Charger Pro). Hieraus wird die geladene Energie in \SI{}{\kWh} berechnet.
	 \item Dauer der Ladung
	\end{itemize}

	Diese Informationen werden \textbf{nur} auf dem WARP Charger gespeichert.
	Aufgezeichnete Ladevorgänge können im Webinterface auf der Ladetracker-Unterseite als ein CSV-Dokument,
	kompatibel zu üblichen Tabellenkalkulationsprogrammen heruntergeladen werden. Außerdem kann das erzeugte Dokument
	vorgefiltert werden, um beispielsweise nur Ladevorgänge eines bestimmten Benutzers in einem festgelegten Zeitraum zu erhalten.

	Zusätzlich kannst du einen Strompreis (ct/kWh) angeben. Wenn du über einen
	WARP Charger Pro verfügst (integrierter Stromzähler), dann kann die Wallbox
	damit und über die ermittelte gelandene Energie die Kosten des jeweiligen
	Ladevorgangs berechnen. Bitte beachte das eine Änderung des Strompreises
	immer die Kosten von allen aufgezeichneten Ladevorgängen neu berechnet.

	Der WARP Charger kann bis zu 7680 Ladevorgänge aufzeichnen.

	\section{Benutzerverwaltung} \label{user_management}

	Auf der Unterseite \enquote{Benutzerverwaltung} im System-Abschnitt des Webinterfaces können bis zu 16 Benutzer angelegt werden.
	Einem angelegten Benutzer, dem ein NFC-Tag zugeordnet wurde (siehe Abschnitt \fullref{NFC}) können vom Ladetracker Ladungen zugeordnet werden.

	In der Werkseinstellung sind exemplarisch drei Nutzer bereits angelegt,
	denen jeweils eine NFC Karte (mitgeliefert) zugeordnet wurde. Diese können
	umbenannt oder gelöscht werden.

	Ein neuer Nutzer kann mittels Klicken auf \glqq Benutzer hinzufügen\grqq~hinzugefügt werden.
	Anschließend öffnet sich ein kleines Fenster in dem der eigentliche Benutzername, der Anzeigename (für die Anzeige im Ladetracker)
	und der dem Nutzer erlaube maximale Ladestrom eingestellt werden kann.
	Zusätzlich kann dem Nutzer ein Passwort für die HTTP-Anmeldung (siehe
	folgenden Abschnitt) vergeben werden.

	Soll nur eine Ladefreigabe mittels NFC/Benutzerfreigabe möglich sein, so
	muss \glqq Benutzerautorisierung\grqq~aktiviert werden.

	Eine weitere Funktion der Benutzerverwaltung ist die HTTP-Anmeldung. Diese
	kann mittels \glqq Anmeldung aktiviert\grqq~aktiviert werden. Wenn diese aktiviert ist, muss zum Zugriff auf das Webinterface und zur Verwendung
	der HTTP-API eine Anmeldung als einer der Benutzer durchgeführt werden. Eine HTTP-Anmeldung als ein Benutzer ist nur möglich, wenn
	dem Benutzer ein Passwort konfiguriert wurde. Entsprechend können Benutzer erstellt werden, die nur für das Ladetracking per NFC-Tag
	verwendet werden, aber keinen Zugriff auf das Webinterface haben sollen, indem diesen kein Passwort konfiguriert wird.
	\hint{Wenn du die Zugangsdaten des HTTP-Anmeldung vergisst, kannst du nur über den Wiederherstellungsmodus \ref{recovery} oder nach einem Zurücksetzen auf den Werkszustand \ref{reset} wieder darauf zugreifen.}
	Die Funktion ist nur aktivierbar, wenn mindestens ein Nutzer mit einem
	aktiviertem Passwort existiert.

	Standardmäßig können sich Benutzer nicht am Webinterface anmelden. Dies wird
	im Passwort-Feld des Nutzers angezeigt indem das Verbotsschild aktiviert
	ist. Wird einem Nutzer ein Passwort vergeben, so ist das Verbotsschild
	deaktiviert. Der Nutzer kann sich mit seinem Nutzernamen und Passwort im
	Webinterface anmelden, wenn die Option \glqq
	Benutzerautorisierung\grqq~aktiviert wurde (siehe Abschnitt zuvor).
	Um einem Nutzer die Anmeldemöglichkeit wieder zu entziehen und sein Passwort
	zu löschen muss einfach das Verbottschild aktiviert werden. Das Passwort des
	Nutzers wird dann gelöscht und die Anmeldung deaktiviert. Der Nutzer kann
	aber nach wie vor eine Ladung per NFC Tag freigeben, wenn ihm ein Tag
	zugeordnet wurde. Um den Nutzer wieder eine Anmeldung zu ermöglichen, muss
	sein Passwort neu gesetzt werden.

	Sollen mehrere Nutzer angelegt werden, so empfehlen wir diese direkt
	nacheinander anzulegen. Anschließend müssen die Änderungen gespeichert und
	die Wallbox neugestartet werden, damit die Änderungen übernommen werden.

	\vfill
	\
	\columnbreak

	\gfx{./img_warp2/resized/web_users}
	\vfill
	\gfx{./img_warp2/resized/web_users_new}

	\newpage
	\section{Ladefreigabe per NFC}
	\label{NFC}
	Der WARP2 Charger unterstützt eine Ladefreigabe per NFC (siehe Abschnitt
	\fullref{user_management}). Wenn diese aktiviert ist,
	muss zum Starten und/oder zum Stoppen einer Ladung ein NFC-Tag, das einem Benutzer zugeordnet ist, an die rechte Seite
	der Wallbox gehalten werden. Im Lieferumfang des WARP2 Chargers sind drei NFC-Karten enthalten,
	die bereits angelernt sind. Es können aber beliebige andere NFC-Tags der Typen 1 bis 4,
	sowie Mifare Classic angelernt werden. Der WARP2 Charger unterstützt bis zu 16 angelernte Tags.

	\subsection{Konfiguration}
	\gfx{./img_warp2/resized/web_nfc}

	Auf der NFC-Unterseite des Webinterfaces kannst du die berechtigten Tags konfigurieren.
	Im Werkszustand sind die drei mitgelieferten NFC-Karten angelernt,
	das Starten und Stoppen einer Ladung ist aber so konfiguriert, dass eine
	Freigabe ohne Tag möglich ist.

	Durch Klicken auf den \enquote{Tag hinzufügen}-Button kann ein neues Tag angelernt werden.
	Es werden die zuletzt erkannten, aber noch nicht berechtigten Tags in einer
	Liste anzeigt, durch Klicken auf eines der Tags kann dieses übernommen werden. Ein Neustart der
	Wallbox leert die Liste. Sollen also mehrere Tags nacheinander hinzugefügt
	werden empfehlen wir, die Tags nacheinander an die Wallbox zu halten. Die
	Tags werden anschließend chronologisch in der Liste aufgeführt und können
	dann einfach nacheinander angelegt und existierenden Benutzern zugeordnet
	werden. Wurden alle NFC Tags angelernt können die Einstellungen gespeichert und die
	Wallbox neugestartet werden.

	\columnbreak

	Alternativ können Tag-ID und -Typ manuell eingegeben werden. Dies ist zum Beispiel sinnvoll,
	wenn Tag-ID und -Typ mittels externer Geräte (z.B. Smartphone mit passender
	App) ermittelt und eingetragen werden sollen.

	Auf der Benutzerverwaltungs-Unterseite \fullref{user_management} kann die Option \enquote{Ladefreigabe} aktiviert werden.
	Wenn diese aktiv ist, muss ein NFC-Tag verwendet werden um eine Ladung zu starten.
	Wenn zusätzlich die \enquote{Tastereinstellung} auf der Ladecontroller-Unterseite auf \enquote{keine Aktion} konfiguriert wird,
	muss auch zum Stoppen eines Ladevorgangs ein NFC-Tag an die Wallbox gehalten werden. Dies kann im
	halb-öffentlichen Raum sinnvoll sein.

	%\gfx{./img_warp2/resized/web_nfc_new}

	\subsection{Verwendung}
	Wenn die Benutzerautorisierung aktiviert ist und ein Fahrzeug angeschlossen wird,
	beginnt die Wallbox mit einem schnellen Auf- und Abblenden der blauen LED.
	Dies soll daran erinnern, dass ein Tag notwendig ist, um zu laden. Die
	nachfolgende Grafik illustriert diesen Blinkcode.

	\gfx{./img_warp2/resized/blink_nag}

	Wenn ein berechtigtes Tag erkannt wurde geht die LED
	dreimal aus und blendet danach wieder langsam auf. Danach folgt eine längere Pause.

	\gfx{./img_warp2/resized/blink_ack}

	Wenn ein unberechtigtes Tag erkannt wurde, wiederholt sich ein Muster von langsamen Abblenden
	und schnellem Aufleuchten sechsmal.

	\gfx{./img_warp2/resized/blink_nack}

	Wenn ein berechtigtes Tag erkannt wurde, sollte der Ladevorgang kurz danach
	freigeschaltet werden. Es kann sein, dass die Ladung nicht
	sofort beginnt, sondern erst nachdem eine Ladefreigabe z.B. vom Lastmanagement erhalten wurde,
	und das Fahrzeug eine Ladung anfordert. Die NFC-Freigabe bleibt aber erhalten,
	bis das Ladekabel vom Fahrzeug getrennt wird.

	\newpage \section{Fehlerbehebung}\label{fehlerbehebung} \subsection{Fehlersuche}
	Fehlerzustände werden von der Wallbox durch die blaue LED im Deckel
	dargestellt. Bei den Varianten WARP2 Charger Smart und WARP2 Charger Pro gibt die Statusseite des Ladecontrollers
	weitere Informationen.

	\subsubsection*{Blaue LED ist aus}
	Für diesen Fehlerzustand gibt es verschiedene mögliche Ursachen:
	\begin{itemize}
		\item Die blaue LED geht nach etwa 15 Minuten Inaktivität aus. Das Drücken des Tasters
		      oder das Anschließen eines Elektrofahrzeugs zur Ladung weckt die Wallbox wieder
		      und die LED sollte wieder dauerhaft leuchten.
		\item Die Wallbox ist nicht mit Strom versorgt. Mögliche Ursachen: Stromausfall,
		      Sicherung oder Fehlerstrom\-schutzschalter haben ausgelöst
		\item Der interne Ladecontroller ist ohne Strom. Die Wallbox verfügt intern über zwei
		      Feinsicherungen, gegebenenfalls ist eine defekt.
		\item Das innere Anschlusskabel zum Deckel wurde nicht korrekt aufgesteckt (zum Beispiel am Taster \SI{180}{\degree} verdreht).
	\end{itemize}

	\subsubsection*{Blaue LED blinkt sehr schnell}\label{fast_blink}
	Nach dem Einschalten der Stromversorgung kalibriert die Wallbox die
	DC-Fehlerstromerkennung. Nach drei Sekunden sollte die Kalibrierung
	abgeschlossen sein und die blaue LED sollte dauerhaft leuchten
	(betriebsbereit).

	\subsubsection*{Blaue LED blinkt 2$\times$ im Intervall \\ Webinterface zeigt Schalterfehler}
	Die Wallbox wurde nicht korrekt installiert. Die Schalter-Einstellung des Ladecontrollers ist
	noch auf dem Werkszustand. Siehe \fullref{ladestrom_schalter}.

	\begin{minipage}{\linewidth} %use minipage to control footnote placement
		\subsubsection*{Blaue LED blinkt 3$\times$ im Intervall \\ Webinterface zeigt DC-Fehler}
		Ein DC-Fehlerstrom wurde erkannt. Der Fehler kann entweder über die Webseite der Wallbox oder aber über
		ein kurzzeitiges Trennen der Stromversorgung der Wallbox zurückgesetzt
		werden. Achtung: den Hinweis in \fullref{dcerrorhint} beachten!
	\end{minipage}

	\subsubsection*{Blaue LED blinkt 4$\times$ im Intervall \\ Webinterface zeigt Schützfehler}
	Für diesen Fehlerzustand gibt es verschiedene mögliche Ursachen:
	\begin{itemize}
		\item Erdungsfehler der Wallbox (PE nicht korrekt angeschlossen?)
		\item Phase L1 ohne Spannung (L1/N vertauscht?)
		\item Schütz schaltet nicht korrekt ein (keine Spannung an L1 nach dem Schütz), kein
		      Kontakt
		\item Schütz schaltet nicht korrekt ab (Spannung von L1 liegt trotz Abschalten noch
		      nach dem Schütz an), \enquote{Schütz klebt}
		\item Eine der beiden internen Feinsicherungen ist defekt.
	\end{itemize}

	\subsubsection*{Blaue LED blinkt 5$\times$ im Intervall \\ Webinterface zeigt Kommunikationsfehler}
	Es besteht ein Kommunikationsfehler mit dem Elektrofahrzeug. Bei erstmaligem
	Auftreten das Ladekabel vom Fahrzeug trennen, 10 Sekunden warten und das
	Ladekabel erneut mit dem Fahrzeug verbinden (erneuter Ladevorgang).

	Sollte das Problem bestehen bleiben, so kann es verschiedene Gründe dafür
	geben:
	\begin{itemize}
		\item Es liegt ein Fehler beim Ladekabel vor (Kurzschluss, verschmutze / feuchte
		      Kontakte o.ä.). Die Wallbox ist dann sofort außer Betrieb zu nehmen und
		      in Stand zu setzen.
		\item Es liegt ein technischer Defekt beim Fahrzeug vor.
		\item Es liegt ein technischer Defekt bei der Wallbox vor (Ladecontroller defekt o.ä.)
		\item Das Fahrzeug fordert den IEC 61851-1 Status \enquote{D – Laden mit Belüftung}
		      an. Dieser Modus wird von der Wallbox nicht unterstützt.
		\item Das Fahrzeug übermittelt den IEC 61851-1 Status E oder F. In beiden Fällen
		      handelt es sich um einen Fehler, den das Fahrzeug erkannt hat.
	\end{itemize}

	\subsubsection*{Die Wallbox ist nicht über LAN / WLAN erreichbar, aber die blaue LED leuchtet}
	In diesem Fall ist zu prüfen, ob die Wallbox gegebenenfalls in den Access-Point-Fallback
	gegangen ist. Wie im Werkszustand eröffnet die Wallbox dann ein eigenes
	WLAN. Wenn die Zugangsdaten nicht geändert wurden, entsprechen sie den Werkseinstellungen und sind dem
	Aufkleber auf der Rückseite der Anleitung zu entnehmen.

	\subsection{Ersatzteile}
	\begin{tabular}{ll}
		\toprule
		\textbf{Artikelnummer}                                                                                                      & \textbf{Bauteil}                     \\
		\cmidrule(lr{0.5em}){1-2}
		\href{https://www.tinkerforge.com/de/shop/warp/contactor-4-pole-din-rail-63a.html}{WARP-CON-4P-63A}                         & Schaltschütz 4 Pol,                  \\
		                                                                                                                            & Hutschiene, \SI{63}{\ampere}         \\
		\cmidrule(lr{0.5em}){1-2}
		\href{https://www.tinkerforge.com/de/shop/warp/warp2-spare-parts/contactor-4-pole-din-rail-63a.html}{WARP2-METER-3PH-MID}   & Zweirichtungs-                       \\
		                                                                                                                            & drehstromzähler,                     \\
		                                                                                                                            & 3 Phasen, RS485, MID                 \\
		\cmidrule(lr{0.5em}){1-2}
		\href{https://www.tinkerforge.com/de/shop/warp/warp2-spare-parts/type-2-plug-with-5m-cable-11kw-16a.html}{WARP-T2-5M-16A}   & Typ-2-Stecker mit                    \\
		                                                                                                                            & \SI{5}{\meter} Kabel                 \\
		                                                                                                                            & \SI{11}{\kilo\watt}/\SI{16}{\ampere} \\
		\cmidrule(lr{0.5em}){1-2}
		\href{https://www.tinkerforge.com/de/shop/warp/warp2-spare-parts/type-2-plug-with-5m-cable-22kw-32a.html}{WARP-T2-5M-32A}   & Typ-2-Stecker mit                    \\
		                                                                                                                            & \SI{5}{\meter} Kabel                 \\
		                                                                                                                            & \SI{22}{\kilo\watt}/\SI{32}{\ampere} \\
		\cmidrule(lr{0.5em}){1-2}
		\href{https://www.tinkerforge.com/de/shop/warp/warp2-spare-parts/type-2-plug-with-75m-cable-11kw-16a.html}{WARP-T2-75M-16A} & Typ-2-Stecker mit                    \\
		                                                                                                                            & \SI{7,5}{\meter} Kabel               \\
		                                                                                                                            & \SI{11}{\kilo\watt}/\SI{16}{\ampere} \\
		\cmidrule(lr{0.5em}){1-2}
		\href{https://www.tinkerforge.com/de/shop/warp/warp2-spare-parts/type-2-plug-with-75m-cable-22kw-32a.html}{WARP-T2-75M-32A} & Typ-2-Stecker mit                    \\
		                                                                                                                            & \SI{7,5}{\meter} Kabel               \\
		                                                                                                                            & \SI{22}{\kilo\watt}/\SI{32}{\ampere} \\
		\cmidrule(lr{0.5em}){1-2}
		\href{https://www.tinkerforge.com/de/shop/warp/warp2-spare-parts/warp-fuse-05a.html}{WARP-FUSE-0.5A}                        & 2x Feinsicherung                     \\
		                                                                                                                            & 5x\SI{20}{\milli\meter}              \\
		                                                                                                                            & mittelträge \SI{0,5}{\ampere}        \\
		\cmidrule(lr{0.5em}){1-2}
		\href{https://www.tinkerforge.com/de/shop/warp/warp2-spare-parts/warp-eth-feed-through.html}{WARP-ETH-FEED-THR}             & Ethernet-                            \\
		                                                                                                                            & gehäusedurchführung                  \\
		\cmidrule(lr{0.5em}){1-2}
		\href{https://www.tinkerforge.com/de/shop/warp/warp2-spare-parts/warp-nfc-sticker.html}{WARP-NFC-STICKER}                   & NFC-Aufkleber                        \\
		\cmidrule(lr{0.5em}){1-2}
		\href{https://www.tinkerforge.com/de/shop/warp/warp2-spare-parts/warp2-dc-protect.html}{WARP2-DC-PROTECT}                   & DC-Fehlerstrom-                      \\
		                                                                                                                            & schutzmodul (6mA)                    \\
		\cmidrule(lr{0.5em}){1-2}
		\href{https://www.tinkerforge.com/de/shop/warp/warp2-spare-parts.html}{WARP2-CASE}                                          & WARP2 Gehäuse                        \\
		\cmidrule(lr{0.5em}){1-2}
		\small{\href{https://www.tinkerforge.com/de/shop/warp/warp2-spare-parts/warp2-cable-harness.html}{WARP2-CABLE-HARNESS}}     & WARP2 Kabelbaum                      \\
		\cmidrule(lr{0.5em}){1-2}
		\href{https://www.tinkerforge.com/de/shop/warp/warp2-spare-parts/warp2-terminal-blocks.html}{WARP2-TERMINAL-}               & WARP2 Klemmen-                       \\
		\href{https://www.tinkerforge.com/de/shop/warp/warp2-spare-parts/warp2-terminal-blocks.html}{BLOCKS}                        & Baugruppe                            \\
		\cmidrule(lr{0.5em}){1-2}
		\href{https://www.tinkerforge.com/de/shop/warp/warp2-spare-parts/warp2-nfc-karte.html}{WARP2-NFC-CARD}                      & 3$\times$WARP2 NFC-Karten            \\
		\cmidrule(lr{0.5em}){1-2}
		\href{https://www.tinkerforge.com/de/shop/warp/warp2-spare-parts/warp2-screws.html}{WARP2-SCREWS}                           & WARP2 Schraubenset                   \\
		\cmidrule(lr{0.5em}){1-2}
		\href{https://www.tinkerforge.com/de/shop/warp/warp2-spare-parts/warp2-pb-led-set.html}{WARP2-PB-LED}                       & WARP2 Taster/LED                     \\
		\cmidrule(lr{0.5em}){1-2}
		\href{https://www.tinkerforge.com/de/shop/warp/warp2-spare-parts/warp-res-220.html}{WARP-RES-220}                           & Widerstand \SI{220}{\ohm}            \\
		\cmidrule(lr{0.5em}){1-2}
		\href{https://www.tinkerforge.com/de/shop/warp/warp2-spare-parts/warp-res-680.html}{WARP-RES-680}                           & Widerstand \SI{680}{\ohm}            \\
		\cmidrule(lr{0.5em}){1-2}
		\href{https://www.tinkerforge.com/de/shop/warp/warp2-spare-parts/warp2-esp32-eth.html}{WARP2-ESP32-ETH}	                    & ESP32 Ethernet Brick                 \\
		                                                                                                                            & mit WARP2-Firmware                   \\
		\cmidrule(lr{0.5em}){1-2}
		\href{https://www.tinkerforge.com/de/shop/warp/warp2-spare-parts/evse-v2-bricklet.html}{2167}                               & EVSE Bricklet 2.0                    \\
		\cmidrule(lr{0.5em}){1-2}
		\href{https://www.tinkerforge.com/de/shop/warp/warp2-spare-parts/nfc-bricklet.html}{286}                                    & NFC Bricklet                         \\
		\cmidrule(lr{0.5em}){1-2}
		\href{https://www.tinkerforge.com/de/shop/accessories/cable/bricklet-cable-15cm-7p-7p.html}{6150}                           & Bricklet-Kabel                       \\
		                                                                                                                            & \SI{15}{\centi\meter} (7p-7p)        \\
		\cmidrule(lr{0.5em}){1-2}
		\href{https://www.tinkerforge.com/de/shop/accessories/cable/bricklet-cable-6cm-7p-7p.html}{6149}                            & Bricklet-Kabel                       \\
		                                                                                                                            & \SI{6}{\centi\meter} (7p-7p)         \\
		\bottomrule
	\end{tabular}

	\subsection{Sicherungswechsel}
	Die Wallbox ist intern über zwei 5$\times\SI{20}{\milli\meter}$ Feinsicherungen (mittelträge (m), \SI{500}{\milli\ampere}) abgesichert.
	Tinkerforge verbaut Sicherungen vom Typ \enquote{ESKA 521.014}.

	\section{Konformitätserklärung}
	Die EU-Konformitätserklärung zur Wallbox ist in einem gesonderten Dokument verfügbar.

	\section{Entsorgung}
	\begin{minipage}{0.43\textwidth}
		Wallbox und Verpackung sind bei Gebrauchsende ordnungsgemäß zu
		entsorgen. Altgeräte dürfen nicht über den Hausmüll entsorgt werden.
	\end{minipage}\hfill
	\begin{minipage}{0.045\textwidth}
		\includegraphics[width=\linewidth]{./img_warp2/resized/weee.pdf}
	\end{minipage}

	\section{Technische Daten}

	%use minipage here to control footnote placement
	\begin{minipage}{\linewidth}

		\begin{description}[leftmargin=!,labelwidth=\widthof{\textbf{Fehlerstromerkennung}}]
			\setlength{\itemsep}{3pt}
			\item[Ladestandard] DIN EN 61851‐1
			\item[Ladeleistung] einstellbar
			      bis \SI{11}{\kilo\watt} / \SI{22}{\kilo\watt}~\footnote[7]{\label{fn:1} je nach Variante}
			\item[Fahrzeugladestecker] Typ 2
			\item[Abmessungen] 280 × 215 × \SI{95}{\milli\meter} (B/H/T)
			\item[Nennspannung] \SI{230}{\volt} / \SI{400}{\volt} / 1/3
			      AC$\sim$~\footref{fn:1}
			\item[Nennfrequenz] \SI{50}{\hertz}
			\item[Nennstrom] \SI{16}{\ampere} / \SI{32}{\ampere}
			      \footref{fn:1}
			\item[Standby, WLAN an] Basic/Smart $\leq\SI{3}{\watt}$; Pro $\leq\SI{5}{\watt}$
			\item[Ladekabellänge] \SI{5}{\meter} / \SI{7,5}{\meter}~\footref{fn:1}
			\item[Zuleitungsquerschnitt] \SI{2,5}{\square\milli\meter} bis
			      \SI{10}{\square\milli\meter}
			\item[Zugangsverriegelung]
			      NFC~\footref{fn:1}\\Webinterface~\footref{fn:1}
			\item[Betriebstemperatur] \SI{-25}{\celsius}
			      bis \SI{+50}{\celsius} (Durchschnitt in \SI{24}{\hour}: $\leq \SI{35}{\celsius}$)
			\item[Fehlerstromerkennung] DC \SI{6}{\milli\ampere} (integriert)
			\item[Schutzart] IP54
			      (spritzwassergeschützt, für
			      den Außenbereich geeignet)
			\item[Lastmanagement] max. 10 Teilnehmer~\footref{fn:1}
			\item[NFC-Tags] 3 im Lieferumfang\\max. 16 anlernbar~\footref{fn:1}
			\item[Benutzer] max. 16 konfigurierbar~\footref{fn:1}
			\item[Schnittstellen] HTTP, MQTT, Modbus/TCP, OCPP~\footref{fn:1}
		\end{description}
	\end{minipage}

	\newpage

	\section{Kontakt}
	Tinkerforge GmbH\\ Zur Brinke 7\\ 33758 Schloß Holte-Stukenbrock
	\begin{description}[leftmargin=!,labelwidth=\widthof{\textbf{Website}}]
		\item[E-Mail] \href{mailto:info@tinkerforge.com}{\texttt{info@tinkerforge.com}}
		\item[Website] \href{https://warp-charger.com}{\texttt{warp-charger.com}}
		\item[Telefon] \phonenumber{052078998614}
		\item[Shop] \href{https://tinkerforge.com/de/shop/warp.html}{\texttt{tinkerforge.com/de/shop/warp.html}}
	\end{description}

	\section{Dokumentversionen}
	\begin{tabular}{lll}
		\toprule
		Datum      & Version & Kommentar                       \\
		\midrule
		09.08.2021 & 0.1     & Initialversion                  \\
		17.08.2021 & 0.2     & Neue Features hinzugefügt       \\
		23.08.2021 & 0.3     & Inhaltliche Verbesserungen      \\
		07.09.2021 & 1.0     & Druckversion                    \\
		12.10.2021 & 1.0.1   & Ersatzteilliste vervollständigt \\
		26.10.2021 & 1.1     & Montage-, Ladestrom- und        \\
		           &         & Lastmanagement-                 \\
		           &         & beschreibung verbessert;        \\
		           &         & Zurücksetzen auf Werks-         \\
		           &         & einstellung per Brick Viewer    \\
		           &         & hinzugefügt                     \\
		06.12.2021 & 1.1.1   & Zuleitungsaderfarben            \\
		           &         & verbessert                      \\
		31.03.2022 & 2.0.0   & Aktualisiert für Firmware 2.0.0 \\
		06.04.2022 & 2.0.1   & Watchdog-Hinweis hinzugefügt    \\
		23.06.2022 & 2.0.2   & Staubschutzkappen- und          \\
		           &         & Verschraubungshinweis           \\
		           &         & hinzugefügt.                    \\
		14.09.2022 & 2.0.3   & Aktualisiert für Firmware 2.0.8 \\
		           &         & Wiederherstellungsmodus         \\
		           &         & hinzugefügt                     \\
		11.11.2022 & 2.0.4   & Aktualisiert für WARP2.1        \\
		           &         & mit neuer Kabelführung          \\
		09.12.2022 & 2.0.5   & Modbus-TCP, OCPP und            \\
		           &         & Wireguard hinzugefügt           \\
		           &         & \enquote{Erste Schritte} überarbeitet \\
		\bottomrule
	\end{tabular}

	\vfill
	\null

	\columnbreak
\appendix

\section{Modbus/TCP Registertabelle}
\label{modbus_tcp_registertabelle}
Nachfolgend die Registertabelle für Modbus/TCP für die Einstellung \textbf{WARP
Charger}.

Input Registers können nur gelesen werden und liefern Informationen über den
Zustand der Wallbox. Gewisse Register sind nur verfügbar, wenn das dazu
angegebene \textbf{Feature} verfügbar ist. So sind zum Beispiel die
Informationen zur Ladeleistung, Energie usw. nur verfügbar, wenn die Wallbox
über ein \textbf{meter} verfügt. Das heißt ein WARP2 Charger Pro (Version mit
Stromzähler) liefert diese Werte. Ein WARP2 Charger Smart (Version ohne
Stromzähler) nicht.

Welche Features die Wallbox bietet kann über \textbf{Discrete Inputs} ausgelesen
werden. Eine Steuerung der Wallbox ist über die \textbf{Holding Registers}
möglich.

\end{multicols*}

\subsection{Input Registers}
\begin{tabularx}{\textwidth}{rXll} \toprule
    \textbf{Register-} & \textbf{Name}& \textbf{Typ} & \textbf{Benötigtes}                                                      \\
    \textbf{adresse}   &              &              & \textbf{Feature}                                                         \\ \midrule
0             & Version der Registertabelle             & uint32       & ---                                     \\
              & \tdesc{Aktuelle Version: 1}                                                                                     \\ \cmidrule{2-4}
2             & Firmware-Version                       & uint32 (x4)       & ---                                                    \\
              & \tdesc{Major, Minor, Patch, Zeitstempel jeweils uint32. Beispielsweise 2, 0, 8, 0x63218f23 für}                 \\
              & \tdesc{Firmware 2.0.8-63218f23. 0x63218f23 ist der Unix-Zeitstempel des 14. September 2022 08:21:55 UTC.}       \\ \cmidrule{2-4}
10            & Charger-ID                              & uint32       & ---                                                    \\
              & \tdesc{Dekodierte Form der Base58-UID, die für Standard-Hostnamen, -SSID usw. genutzt wird.}                    \\
              & \tdesc{Zum Beispiel 185460 für X8A}                                                                             \\ \cmidrule{2-4}
12            & Uptime (s)                              & uint32       & ---                                                    \\
              & \tdesc{Zeit in Sekunden seit dem Start der Wallbox-Firmware.}                                                   \\ \cmidrule{2-4}
1000          & IEC-61851-Zustand                       & uint32       & evse                                                   \\
              & \tdesc{0-A, 1-B, 2-C, 3-D, 4-E/F}                                                                               \\ \cmidrule{2-4}
1002          & Fahrzeugstatus                          & uint32       & evse                                                   \\
              & \tdesc{0-Nicht verbunden, 1-Warte auf Freigabe, 2-Ladebereit, 3-Lädt, 4-Fehler}                                 \\ \cmidrule{2-4}
1004          & User-ID                                 & uint32       & evse                                                   \\
              & \tdesc{ID des Benutzers der den Ladevorgang gestartet hat. 0 falls eine Freigabe}                               \\
              & \tdesc{ohne Nutzerzuordnung erfolgt ist. 0xFFFFFFFF falls gerade kein Ladevorgang läuft.}                       \\ \cmidrule{2-4}
1006          & Start-Zeitstempel (min)                 & uint32       & evse                                                   \\
              & \tdesc{Ein Unix-Timestamp in Minuten, der den Startzeitpunkt des Ladevorgangs angibt.}                          \\
              & \tdesc{0 falls zum Startzeitpunkt keine Zeitsynchronisierung verfügbar war.}                                    \\ \cmidrule{2-4}
1008          & Ladedauer (s)                           & uint32       & evse                                                   \\
              & \tdesc{Dauer des laufenden Ladevorgangs in Sekunden. Auch ohne Zeitsynchronisierung verfügbar}                  \\ \cmidrule{2-4}
1010          & Erlaubter Ladestrom                     & uint32       & evse                                                   \\
              & \tdesc{Maximal erlaubter Ladestrom, der dem Fahrzeug zur Verfügung gestellt wird.}                              \\
              & \tdesc{Dieser Strom ist das Minimum der Stromgrenzen aller Ladeslots.}                                          \\ \cmidrule{2-4}
1012          & Ladestromgrenzen (mA)                   & uint32 (x20) & evse                                                   \\
              & \tdesc{Aktueller Wert der Ladestromgrenzen in Milliampere. 0xFFFFFFFF falls eine Stromgrenze nicht aktiv ist.}  \\
              & \tdesc{0 falls eine Stromgrenze blockiert. Sonst zwischen 6000 (6A) und 32000 (32A).}                           \\ \cmidrule{2-4}
2000          & Stromzählertyp                          & uint32       & meter                                                  \\
              & \tdesc{0-Kein Stromzähler verfügbar, 1-SDM72 (nur WARP1), 2-SDM630, 3-SDM72 V2}                                 \\ \cmidrule{2-4}
2002          & Ladeleistung (W)                        & float        & meter                                                  \\
              & \tdesc{Die aktuelle Ladeleistung in Watt}                                                                       \\ \cmidrule{2-4}
2004          & absolute Energie (kWh)                  & float        & meter                                                  \\
              & \tdesc{Die geladene Energie seit der Herstellung des Stromzählers.}                                             \\ \cmidrule{2-4}
2006          & relative Energie (kWh)                  & float        & meter                                                  \\
              & \tdesc{Die geladene Energie seit dem letzten Reset. (siehe Holding Register 2000)}                              \\ \cmidrule{2-4}
2008          & Energie des Ladevorgangs                & float        & meter                                                  \\
              & \tdesc{Die während des laufenden Ladevorgangs geladene Energie}                                                 \\ \cmidrule{2-4}
2100          & weitere Stromzähler-Werte               & float (x85)  & all\_values                                            \\
              & \tdesc{Siehe \rurl{https://www.warp-charger.com/api.html\#meter\_all\_values}{warp-charger.com/api.html\#meter\_all\_values}} \\ \cmidrule{2-4}
3000          & CP-Unterbrechung                        & uint32       & cp\_disc                                               \\
              & \tdesc{Noch nicht implementiert!}                                                                               \\ \bottomrule
\end{tabularx}

\subsection{Holding Registers}
\begin{tabularx}{\textwidth}{rXll} \toprule
    \textbf{Register-} & \textbf{Name} & \textbf{Typ} & \textbf{Benötigtes}                                                     \\
    \textbf{adresse}   &      &     & \textbf{Feature}                                                                          \\ \midrule
0             & Reboot                                  & uint32       & ---                                                    \\
              & \tdesc{Startet die Wallbox (bzw. den ESP-Brick) neu, um beispielsweise Konfigurationsänderungen anzuwenden.}    \\
              & \tdesc{Password 0x012EB007}                                                                                     \\ \cmidrule{2-4}
1000          & Ladefreigabe                            & uint32       & evse                                                   \\
              & \tdesc{0 zum Blockieren des Ladevorgangs; ein Wert != 0 zum Freigeben}                                          \\ \cmidrule{2-4}
1002          & Erlaubter Ladestrom (mA)                & uint32       & evse                                                   \\
              & \tdesc{0mA oder 6000mA bis 32000 mA. Andere Ladestromgrenzen können den Strom weiter verringern!}               \\ \cmidrule{2-4}
2000          & Relative Energie zurücksetzen           & uint32       & meter                                                  \\
              & \tdesc{Setzt den relativen Energiewert zurück (Input Register 2006). Password 0x3E12E5E7}                       \\ \cmidrule{2-4}
3000          & CP-Trennung auslösen                    & uint32       & cp\_disc                                               \\
              & \tdesc{Noch nicht implementiert!}                                                                               \\ \bottomrule
\end{tabularx}

\subsection{Discrete Inputs}
\begin{tabularx}{\textwidth}{rXll} \toprule
    \textbf{Register-} & \textbf{Name} & \textbf{Typ} & \textbf{Benötigtes}                                                     \\
    \textbf{adresse}   &      &     & \textbf{Feature}                                                                          \\ \midrule
0             & Feature \enquote{evse} verfügbar        & bool         & ---                                                    \\
              & \tdesc{Ein Ladecontroller steht zur Verfügung. Dieses Feature sollte bei allen WARP Chargern,}                  \\
              & \tdesc{deren Hardware funktionsfähig ist, vorhanden sein.}                                                      \\ \cmidrule{2-4}
1             & Feature \enquote{meter} verfügbar       & bool         & ---                                                    \\
              & \tdesc{Ein Stromzähler und Hardware zum Auslesen desselben über RS485 ist verfügbar. Dieses Feature wird }      \\
              & \tdesc{erst gesetzt, wenn ein Stromzähler mindestens einmal erfolgreich über Modbus ausgelesen wurde.}          \\ \cmidrule{2-4}
2             & Feature \enquote{phases} verfügbar      & bool         & ---                                                    \\
              & \tdesc{Der verbaute Stromzähler kann Energie und weitere Messwerte einzelner Phasen messen.}                    \\ \cmidrule{2-4}
3             & Feature \enquote{all\_values} verfügbar & bool         & ---                                                    \\
              & \tdesc{Der verbaute Stromzähler kann weitere Messwerte auslesen.}                                               \\ \cmidrule{2-4}
4             & Feature \enquote{cp\_disc} verfügbar    & bool         & ---                                                    \\
              & \tdesc{Noch nicht implementiert!}                                                                               \\ \cmidrule{2-4}
2100          & Phase L1 angeschlossen                  & bool         & phases                                                 \\
              & \tdesc{}                                                                                                        \\ \cmidrule{2-4}
2101          & Phase L2 angeschlossen                  & bool         & phases                                                 \\
              & \tdesc{}                                                                                                        \\ \cmidrule{2-4}
2102          & Phase L3 angeschlossen                  & bool         & phases                                                 \\
              & \tdesc{}                                                                                                        \\ \cmidrule{2-4}
2103          & Phase L1 aktiv                          & bool         & phases                                                 \\
              & \tdesc{}                                                                                                        \\ \cmidrule{2-4}
2104          & Phase L2 aktiv                          & bool         & phases                                                 \\
              & \tdesc{}                                                                                                        \\ \cmidrule{2-4}
2105          & Phase L3 aktiv                          & bool         & phases                                                 \\
              & \tdesc{}                                                                                                        \\ \bottomrule
   \end{tabularx}
	\begin{multicols*}{2}




	\newpage
	\pagestyle{empty}
	\null
	\vfill
	WLAN-Zugangsdaten
	\begin{tcolorbox}[width=4.2cm,height=2.7cm, boxrule=0.25mm]

	\end{tcolorbox}
	Dieser Aufkleber befindet sich\\ auch im Inneren der Wallbox.
	\columnbreak

	\null
	\vfill
	Typenschild
	\begin{tcolorbox}[width=7.8cm,height=4.1cm, boxrule=0.25mm]

	\end{tcolorbox}
	Dieser Aufkleber befindet sich auch auf der Unterseite\\ der Wallbox.
\end{multicols*}
\end{document}
