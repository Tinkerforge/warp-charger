\documentclass[a4paper,10pt]{article}
\ifdefined\forprint
    \usepackage[width=21.6cm,height=30.3cm,center]{crop}
\fi
\usepackage[utf8]{inputenc}

\usepackage[margin=2cm,headheight=26pt,includeheadfoot]{geometry}

\usepackage[german]{babel}
\usepackage[german=quotes]{csquotes}

\usepackage{nameref}
\usepackage{microtype}
\usepackage{float}
\usepackage{siunitx}
\sisetup{
    locale = DE,
    binary-units,
    detect-all,
    per-mode = symbol %enables m/s instead of ms^-1
}
\AtBeginDocument{\DeclareSIUnit{\kWh}{kWh}}

\usepackage{caption} %for \caption*
\usepackage{hhline}
\usepackage{tabularx}
\usepackage{array}
\usepackage{calc}
\usepackage{multicol}
\usepackage{multirow}
\usepackage{parskip}
\usepackage{booktabs}

\usepackage{fancyhdr}
\pagestyle{fancy}
\setlength{\headheight}{48pt}
\renewcommand{\headrulewidth}{0pt}

\usepackage{xcolor,colortbl}
\usepackage{makecell}

\usepackage[symbol*]{footmisc}
\renewcommand{\thefootnote}{\fnsymbol{footnote}}
\renewcommand{\thempfootnote}{\fnsymbol{mpfootnote}}

\usepackage{color}
\usepackage{enumitem}

\usepackage{pdfpages}

% Word-Stealth-Modus, kann aber kein Omega
%\usepackage[scaled]{helvet}

\usepackage[inline,nomargin]{fixme}
\fxsetup{
    author=,
    layout=inline,
    theme=color
}

\definecolor{fxnote}{rgb}{0.8000,0.0000,0.0000}
\colorlet{fxnotebg}{yellow}

\definecolor{boxgray}{rgb}{0.33,0.33,0.33}
\definecolor{boxred}{rgb}{0.8,0,0}
\usepackage{tcolorbox}
\title{}
\author{}

\renewcommand{\familydefault}{\sfdefault}

\newcommand{\hint}[1]{\begin{tcolorbox}[colback=boxgray,colframe=black,coltext=
white,title=Hinweis]#1\end{tcolorbox}}

\newcommand{\warn}[1]{\begin{tcolorbox}[colback=boxred,colframe=red,coltext=
white,title=Warnung]#1\end{tcolorbox}}

\newcommand{\gfx}[1]{\includegraphics[width=\linewidth]{#1}}

\newcommand*{\fullref}[1]{\hyperref[{#1}]{\ref*{#1}~\nameref*{#1}}}

\fancyhf{}
\fancyhead{\colorbox{boxgray}{
    \makebox[\dimexpr\linewidth-2\fboxsep][l]{
        \includegraphics[height=1cm]{./img/logo}\hfill\color{white}\Huge\raisebox{.5ex}{\thepage}
        }
    }
}

\usepackage{hyperref}
\usepackage{qrcode}

\appto\UrlNoBreaks{\do\.\do\:\do\/\do\_}

\usepackage{hyphenat}

\begin{document}
\pagestyle{empty}
\begin{titlepage}
	\vspace*{-3.08cm}
	\colorbox{boxgray}{\makebox[\dimexpr\linewidth-2\fboxsep][c]{\includegraphics[width=0.6\textwidth]{./img/logo}}}
	\vfill
	\begin{center}
		\Huge
		WARP Ladesäule Betriebsanleitung\\\vspace{1cm}
		\large
		Version 1.0.1\\\vspace{0.25cm}
		\today
	\end{center}
	\vfill
	\begin{center}
		\includegraphics[width=0.6\linewidth]{./img/warp-charger-stands}
	\end{center}
\end{titlepage}
\newpage
\null
\newpage
\pagestyle{fancy}
\begin{multicols*}{2}
	\tableofcontents \section{Einführung}
	\subsection{Vorwort} Vielen Dank, dass du
	dich für eine WARP Ladesäule von Tinkerforge entschieden hast!

	\enquote{WARP} steht
	für \textbf{W}all \textbf{A}ttached
	\textbf{R}echarge \textbf{P}oint. Mit der WARP Ladesäule
	erhältst du eine hochqualitative und langlebige Ladesäule, mit der du dein
	Elektrofahrzeug laden kannst. Die Ladesäule ist so aufgebaut, dass
	einzelne Komponenten einfach ausgetauscht werden können. Sowohl Hardware als
	auch Software der verbauten WARP2 Charger sind Open Source. Die nachfolgende Betriebsanleitung gibt dir	alle notwendigen Informationen zu Sicherheit, Montage, Installation, Betrieb
	und Wartung der Ladesäule.

	\subsection{Beschreibung}

	Die WARP Ladesäule kann, je nach Modell, mit einem oder zwei WARP2 Charger Wallboxen
	ausgestattet werden. Die Ladesäule besteht abhängig von der Wahl aus
	\SI{1.5}{\milli\meter} starkem V4A Edelstahl oder \SI{2.0}{\milli\meter}
	verzinktem Stahl, dann in DB703 pulverbeschichtet. Unabhängig von der Variante ist
	die Säule mit Schutzklasse IP54 für den
	Außenbereich geeignet. Die mitgelieferte Montagehilfe, die die Befestigungsschrauben im
	richtigen Abstand positioniert, ermöglicht es dir das Betonfundament
	vorzubereiten.
	Nach der Fertigstellung des Fundaments und gegebenenfalls weiteren
	notwendigen Erdarbeiten wird die Ladesäule montiert.
	Über die geteilte Rückwand kannst du
	die Zuleitungen erreichen und die Ladesäule bequem anschließen. 
\vspace{-0.2cm}
	\section{Sicherheitshinweise}
\vspace{-0.2cm}
	\subsection{Allgemein}
	Die WARP Ladesäule ist so konstruiert, dass ein sicherer Betrieb gewährleistet ist,
	wenn sie korrekt installiert wurde, in einem einwandfreien technischen Zustand
	ist und diese Betriebsanleitung befolgt wird. \hint{Die Ladesäule darf nur von einer ausgewiesenen Elektrofachkraft installiert
		werden.}
\vspace{-0.2cm}
	\subsection{Bestimmungsgemäße Verwendung}
	Die Ladesäule dient zur Aufnahme von WARP2 Charger Wallboxen auf einer freien
	Fläche, bei der eine Wandmontage nicht möglich oder nicht gewünscht ist. 
	Es dürfen ausschließlich WARP2 Charger
	an der Ladesäule montiert werden. Den Installations- und
	Verwendungshinweisen der verwendeten WARP2 Charger ist Folge zu leisten.
	Diese sind in der Betriebsanleitung der WARP2 Charger zu finden.

	Die Ladesäule ist in zwei Produktvarianten erhältlich:
	\begin{itemize}
		\item Aufnahme für eine WARP2 Charger Wallbox.
		\item Aufnahme für zwei WARP2 Charger Wallboxen.
	\end{itemize}

	Bei Variante 2 ist es nicht zulässig, nur eine Wallbox zu montieren und den anderen Platz frei zu lassen.
\vspace{-0.2cm}
	\subsection{Allgemeine Sicherheitshinweise}

	Die WARP Ladesäule wurde gemäß den relevanten Sicherheits- und Umweltvorschriften und -bestimmungen
	entwickelt, hergestellt, geprüft und dokumentiert. Sie darf nur in technisch einwandfreiem Zustand verwendet werden.

	Störungen, die die Sicherheit von Personen oder des Geräts beeinträchtigen,
	sind sofort von einer autorisierten Elektrofachkraft nach den national geltenden Regeln beheben zu lassen.

	\warn{Nichtbeachtung der Sicherheitshinweise kann zu Lebensgefahr,
	Verletzungen und Schäden am Gerät führen! Der Hersteller lehnt jede Haftung
	für daraus resultierende Ansprüche ab!
	\\
	\\
	\textbf{Elektrische Gefahr!}\\
	Die Montage, erste Inbetriebnahme und Wartung der Ladestation darf nur von
	einer ausgebildeten, qualifizierten und befugten Elektrofachkraft
	durchgeführt werden, die dabei für die Beachtung der bestehenden Normen und
	Installationsvorschriften verantwortlich ist. Halte die angeführten Vorgaben 
	für die Standortauswahl und die baulichen Voraussetzungen ein! Abweichungen 
	zu den Standortvorgaben können zu Tod, schweren Körperverletzungen oder 
	Sachschäden führen, wenn die entsprechenden Vorsichtsmaßnahmen nicht 
	getroffen werden!
	}

	\section{Montage und Installation}
	\subsection{Montage}
	\subsubsection{Lieferumfang}
	Im Lieferumfang der Ladesäule befinden sich:
	\begin{itemize}
		\item Edelstahlstele
		\item gewählte WARP2 Charger Wallboxen inkl. Zubehör
		\item Befestigungsmaterial
			\begin{itemize}
				\item 1x Montagehilfe
				\item 4x M8 Schrauben
				\item 4x Unterlegscheiben
				\item 8x M8 Muttern
			\end{itemize}
		\item diese Betriebsanleitung, Betriebsanleitungen und Testprotokolle der Wallboxen
	\end{itemize}
\vspace{-0.1cm}
	\subsubsection{Montageort}
	Folgende Anforderungen müssen zur Montage der WARP Ladesäule erfüllt sein:
	\begin{itemize}
		\item Der Einbauort muss alle Anforderungen erfüllen, die in der
		WARP2 Charger Bedienungsanleitung aufgeführt sind.
		\item Bei Installation der Ladesäule an einer Straße oder einem
		öffentlichen Parkplatz, offen oder überdacht, muss ein entsprechender
		Anfahr-/Rammschutz angebaut werden.
		\item Sollen mehrere Ladesäulen nebeneinander installiert werden, muss
		der Abstand zwischen den einzelnen Ladesäulen mindestens \SI{200}{\milli\meter} betragen.
		\item Die Oberfläche muss vollständig plan sein.
	\end{itemize}

    \hint{Die Ladesäule nicht auf Asphalt installieren! Auf Asphalt ist die
	Standsicherheit nicht gewährleistet.}


	\subsubsection{Herstellung des Fundaments}
    Zur sicheren Installation der Ladesäule wird ein Betonfundament empfohlen.
	Auslegung, Konstruktion und Ausführung des Fundaments sind Aufgabe des
	Herstellers des Betonfundaments und gegebenenfalls den örtlichen Gegebenheiten
	anzupassen.

	Auf Seite~\pageref{appendix_base} findet sich eine
	Abbildung zur Herstellung des Fundaments. Die Größe des Betonfundaments ist als
	Mindestgröße anzusehen um einen sicheren Stand zu gewährleisten.

	Die Montagehilfe muss wie angegeben mit den mitgelieferten Schrauben,
	Unterlegscheiben und Muttern aufgebaut werden. Die angegebenen Maße sind
	einzuhalten. Anschließend kann die so aufgebaute Montagehilfe im
	Betonfundament positioniert werden.

	Es darf sich kein Wasser am Fundament ansammeln, dieses muss abfließen können. Die
	Stromversorgungskabel, ein Erdungskabel und gegebenenfalls vorhandene Netzwerkkabel müssen in der Mitte
	des Betonfundaments austreten und eine überstehende Länge von mindestens
	\SI{1500}{\milli\meter} haben. Dazu ist in der Montagehilfe eine
	Ovale-Aussparung mit Außenabmessungen von 75 × \SI{35}{\milli\meter}.
	Der Fundamenthersteller muss
	für einen ausreichenden Schutz der Kabel sorgen. Schutzhüllen müssen mindestens
	\SI{300}{\milli\meter} aus dem Beton herausreichen. Ein Erdungsanschluss ist
	zwingend erforderlich.

	\subsubsection{Montage der Ladesäule}
	Nachdem das Betonfundament gegossen wurde und ausgehärtet ist, kann
	die Ladesäule montiert werden. Zuvor sollten die Rückseiten
	entfernt werden. Die bei der Montagehilfe
	eingesetzten oberen M8 Muttern (4 Stück) und die Unterlegscheiben müssen nun wieder losgeschraubt werden,
	damit die Gewinde offen liegen. Anschließend kann die Stele mittig über den
	Kabeln, die aus dem Fundament ragen, positioniert werden. Die
	einbetonierte Montagehilfe sorgt für die korrekte Position der
	Befestigungsschrauben. Nach dem Auftstellen der Stele auf die Schrauben kann diese
	mittels der zuvor entfernten Unterlegscheiben und Muttern auf das Fundament
	geschraubt werden.

	Auf Seite~\pageref{appendix_erection} findet sich eine
	Abbildung zu diesem Arbeitsschritt.

	\subsection{Elektrischer Anschluss}
	Der elektrische Anschluss der Wallboxen erfolgt für jede Wallbox separat.
	Die Hinweise der Elektroinstallation der WARP2 Charger aus deren
	Betriebsanleitung sind zu beachten.

	\subsubsection{Anforderungen an die Elektroinstallation}
	Die Anforderungen an die Elektroinstallation der gewählten WARP2
	Charger Wallboxen sind zu beachten. Diese sind der mitgelieferten
	Betriebsanleitung der Wallbox zu entnehmen.

	\subsubsection{Elektrische Installation}
	Werksseitig kann jede Wallbox mit bereits einer Anschlussleitung
	und verbundenen Verteilergehäuse bestellt werden. Nachfolgend die
	Beschreibung für diese Variante. Alternativ kann auch die Ladesäule ohne
	Vorinstallation bestellt werden. Die Verteilergehäuse fehlen dann und die
	Wallboxen müssen direkt angeschlossen werden.

	\begin{center}
		\includegraphics[width=0.5\linewidth]{./img/warp-charger-stand-back-opened}
	\end{center}

	Im Verteilergehäuse befinden sich Klemmen vom Typ SRK 10
	oder vergleichbar. Es können ein- und mehrdrähtige Leiter mit einem Querschnitt von bis zu
	\SI{16}{\square\milli\meter} und Leiter mit Aderendhülse mit einem
	Querschnitt von bis zu \SI{10}{\square\milli\meter} angeschlossen werden.
	Die Abisolierlänge beträgt \SI{10}{\milli\meter}. Der Anschluss erfolgt
	anhand der Beschriftung der Klemmen im Verteilergehäuse.

	\includegraphics[width=\linewidth]{./img/warp-charger-stand-clamps}

	Anschließend sind die in der Betriebsanleitung des WARP2 Chargers geforderten Prüfungen für die
	Wallboxen durchzuführen.

	\subsubsection{Erdung}
	Die Ladesäule ist unbedingt zu erden. Dazu befindet sich ein Erdungspunkt in
	der Säule auf Höhe der Wallboxen. Dieser ist anzuschließen und die Erdung zu
	überprüfen.

	\subsubsection{RJ45 - Ethernet}
	Zum Anschluss der Ethernetleitung dient ebenfalls ein Verteilergehäuse.
	Im Gehäuse können die bestehenden Ethernetleitungen der Wallbox
	angeschlossen werden.

	\begin{center}
		\includegraphics[width=\linewidth]{./img/warp-charger-stand-eth}
	\end{center}

	\section{Inbetriebnahme}
	Eine Inbetriebnahme der eigentlichen Ladesäule erfolgt nicht, sondern eine 
	Inbetriebnahme der verbauten WARP2 Charger. Dazu finden sich weitere Informationen in der
	Betriebsanleitung der Wallboxen.

	\section{Wartung und Reinigung}
	\subsection{Wartung}
	Eine Wartung der Ladesäule ist nicht notwendig.

	\subsection{Reinigung}
	\warn{\textbf{Gefahr! Hochspannung}\\Gefahr von tödlichen elektrischen
	Stromschlägen. Die Ladesäule und die Wallboxen niemals mit einem Hochdruckreiniger oder einem ähnlichen Gerät reinigen.}
	\begin{itemize}
		\item Anlage nur mit einem feuchten Tuch oder einem Edelstahlreiniger abwischen (Anwendungshinweise des Herstellers beachten). Keine aggressiven Reinigungsmittel, Wachs oder Lösungsmittel verwenden.
		\item Teste das Reinigungsmittel immer erst an einer unaufälligen Stelle auf Verträglichkeit.
	\end{itemize}

	\section{Technische Daten}

	%use minipage here to control footnote placement
	\begin{minipage}{\linewidth}
		\begin{description}[leftmargin=!,labelwidth=\widthof{\textbf{WARP2 Charger}}]
			\setlength{\itemsep}{3pt}
			\item[Material] V4A Edelstahl oder verzinkter Stahl DB703 pulverbeschichtet
			\item[Materialstärke] \SI{1.5}{\milli\meter} (Edelstahl) oder \SI{2.0}{\milli\meter} (DB703)
			\item[WARP2 Charger] 1 bis 2 montierbar
			\item[Abmessungen] ca. 344 × 1406 × \SI{102}{\milli\meter} (B/H/T)
			\item[Gewicht] ca. \SI{22}{\kilo\gram} (Edelstahl), ca.
			\SI{25}{\kilo\gram} (DB703), jeweils ohne WARP2 Charger
			\item[Verriegelung]
			      Vierkant, wahlweise mit Schlössern\\ lieferbar
			\item[Lieferumfang] Ladesäule mit gewählten WARP2 Chargern,
			      Bedienungsanleitungen,\\ Befestigungsmaterial
		\end{description}
	\end{minipage}

	\section{Kontakt}
	Tinkerforge GmbH\\ Zur Brinke 7\\ 33758 Schloß Holte-Stukenbrock\\
	\begin{description}[leftmargin=!,labelwidth=\widthof{\textbf{Website}}]
		\item[E-Mail] \href{mailto:info@tinkerforge.com}{\texttt{info@tinkerforge.com}}
		\item[Website] \href{https://warp-charger.com}{\texttt{warp-charger.com}}
		\item[Shop] \href{https://tinkerforge.com/de/shop/warp.html}{\texttt{tinkerforge.com/de/shop/warp.html}}
	\end{description}

	\section{Konformitätserklärung}
	Die EU-Konformitätserklärung ist in einem gesonderten Dokument verfügbar.

	\section{Entsorgung}
	\begin{minipage}{0.40\textwidth}
		Die Ladesäule und die Verpackung ist bei Gebrauchsende ordnungsgemäß zu
		entsorgen. Altgeräte dürfen nicht über den Hausmüll entsorgt werden.
	\end{minipage}\hfill
	\begin{minipage}{0.07\textwidth}
		\includegraphics[width=\linewidth]{./img/weee.pdf}
	\end{minipage}

	\section{Dokumentversionen}
	\begin{tabular}{lll}
		\toprule
		Datum      & Version & Kommentar                   \\
		\midrule
		16.08.2022 & 1.0     & Initialversion              \\
		17.08.2023 & 1.1     & DB703 Variante hinzugefügt    \\
		\bottomrule
	\end{tabular}

	\end{multicols*}
	\pagestyle{empty}
	\newpage
	\null
	\newpage
	\newpage
	\null
	\newpage
	\pagestyle{fancy}
	\section*{Anhang}

	\subsection*{Herstellung des Fundaments inkl. Zuleitung}
	\label{appendix_base}
	\begin{center}
		\includegraphics[width=0.9\linewidth]{./img/stand_overview}
	\end{center}

	\subsection*{Montage der Stele}
	\label{appendix_erection}
	\begin{center}
		\includegraphics[width=0.9\linewidth]{./img/stand_erection}
	\end{center}

	\subsection*{Maße Ladesäule mit einem WARP2 Charger}
	\label{appendix_stand1}
	\begin{center}
		\includegraphics[width=0.9\linewidth]{./img/stand_1}
	\end{center}

	\subsection*{Maße Ladesäule mit zwei WARP2 Chargern}
	\label{appendix_stand2}
	\begin{center}
		\includegraphics[width=0.9\linewidth]{./img/stand_2}
	\end{center}


\end{document}
