\documentclass[a4paper,10pt]{article}
\ifdefined\forprint
    \usepackage[width=21.6cm,height=30.3cm,center]{crop}
\fi
\usepackage[utf8]{inputenc}
\usepackage[margin=2cm,headheight=26pt,includeheadfoot]{geometry}

\usepackage[german]{babel}
\usepackage[german=quotes]{csquotes}

\usepackage{nameref}
\usepackage{microtype}
\usepackage{float}
\usepackage{siunitx}
\sisetup{
    locale = DE,
    binary-units,
    detect-all,
    per-mode = symbol %enables m/s instead of ms^-1
}
\AtBeginDocument{\DeclareSIUnit{\kWh}{kWh}}

\usepackage{caption} %for \caption*
\usepackage{hhline}
\usepackage{tabularx}
\usepackage{array}
\usepackage{calc}
\usepackage{multicol}
\usepackage{multirow}
\usepackage{parskip}
\usepackage{booktabs}
\usepackage{textcomp}

\usepackage{fancyhdr}
\pagestyle{fancy}
\setlength{\headheight}{48pt}
\renewcommand{\headrulewidth}{0pt}

\usepackage{xcolor,colortbl}
\usepackage{makecell}

\usepackage[symbol*]{footmisc}
\renewcommand{\thefootnote}{\fnsymbol{footnote}}
\renewcommand{\thempfootnote}{\fnsymbol{mpfootnote}}

\usepackage{color}
\usepackage{enumitem}

\usepackage{pdfpages}

\usepackage{phonenumbers}

% Word-Stealth-Modus, kann aber kein Omega
%\usepackage[scaled]{helvet}

\usepackage[inline,nomargin]{fixme}
\fxsetup{
    author=,
    layout=inline,
    theme=color
}

\definecolor{fxnote}{rgb}{0.8000,0.0000,0.0000}
\colorlet{fxnotebg}{yellow}

\definecolor{boxgray}{rgb}{0.33,0.33,0.33}
\usepackage{tcolorbox}
\title{}
\author{}

\renewcommand{\familydefault}{\sfdefault}

\newcommand{\hint}[1]{\begin{tcolorbox}[colback=boxgray,colframe=black,coltext=
white,title=Hinweis,left*=2mm,right*=2mm,boxsep=1mm,bottom=1mm,top=1mm]#1\end{tcolorbox}}

\newcommand{\gfx}[1]{\includegraphics[width=\linewidth]{#1}}

\newcommand*{\fullref}[1]{\hyperref[{#1}]{\ref*{#1}~\nameref*{#1}}}

\fancyhf{}
\fancyhead{\colorbox{boxgray}{
    \makebox[\dimexpr\linewidth-2\fboxsep][l]{
        \includegraphics[height=1cm]{./img/resized/logo}\hfill\color{white}\Huge\raisebox{.5ex}{\thepage}
        }
    }
}

\usepackage{hyperref}
\usepackage{qrcode}

\appto\UrlNoBreaks{\do\.\do\:\do\/\do\_}

\newcommand\rurl[2]{%
  \href{#1}{\nolinkurl{#2}}%
}


\newcommand{\tdesc}[1]{\multicolumn{3}{l}{\footnotesize #1}}

\usepackage{hyphenat}

\hyphenation{Web-inter-face}
\hyphenation{Werks-ein-stel-lungen}
\hyphenation{Firm-ware-up-dates}

\begin{document}
\pagestyle{empty}
\begin{titlepage}
	\vspace*{-3.08cm}
	\colorbox{boxgray}{\makebox[\dimexpr\linewidth-2\fboxsep][c]{\includegraphics[width=0.6\textwidth]{./img/resized/logo}}}
	\vfill
	\begin{center}
		\Huge
		WARP Energy Manager Betriebsanleitung\\\vspace{1cm}
		\large
		Version 1.0.0\\\vspace{0.25cm}
		03.03.2023
	\end{center}
	\vfill \gfx{./img/resized/warp-energy-manager.png}
\end{titlepage}
\newpage
\null
\newpage
\pagestyle{fancy}
\begin{multicols*}{2}
	\tableofcontents
	\newpage
	\section{Einführung}
	Vielen Dank, dass du dich für einen WARP Energy Manager von Tinkerforge entschieden hast!
	Mit dem WARP Energy Manager
	erhältst du unseren Energiemanager zur Schaltschrankmontage, mit dem du den
	Energieverbrauch zu Hause überwachen, steuern und optimieren kannst.

	\gfx{./img/resized/warp-energy-manager.png}

	In Verbindung mit unseren WARP Charger Wallboxen kannst du das Laden von
	Elektrofahrzeugen abhängig von deinem Strombezug und deiner Stromeinspeisung steuern.
	Unter anderem ist damit ein PV-Überschussladen oder ein dynamisches
	Lastmanagement möglich.

	\hint{Der WARP Energy Manager ist mit einer Basisfirmware veröffentlicht
	worden. Mittels kostenloser Firmwareupdates wird die Funktionalität Schritt
	für Schritt erweitert. Der hier dokumentierte Stand bezieht sich auf die
	Funktionen der Firmware 1.0.0. Informationen zum dynamischen Lastmanagement
	werden gegeben, auch wenn diese Funktion noch nicht in der Basisfirmware
	enthalten ist.}

	\subsection{Features}
	Der WARP Energy Manager kann mit einem dreiphasigen bidirektionalen
	Stromzähler die Leistung am Stromnetzanschluss (z.\,B. Hausanschluss) kontinuierlich
	messen. Es werden verschiedene Zählertypen und Anschlussarten unterstützt.

	\subsubsection{Energiemonitoring}
	Die Messwerte des Stromzählers stellt der WARP Energy Manager in seinem
	Webinterface dar. Dort wird angezeigt, wie groß die Leistung ist, die aus dem Stromnetz
	bezogen bzw., falls du eine Photovoltaik-Anlage besitzt, einspeist wird.
	Leistungs- und weitere Messwerte werden dir live auf dem Webinterface dargestellt.

	Alle fünf Minuten werden die Messwerte lokal auf der microSD-Karte des
	Energiemanagers gespeichert. Damit ist der WARP Energy Manager unabhängig
	von Datenaufzeichnungen auf Cloud-Servern. Diese Daten kannst du dir für jeden Tag
	graphisch anzeigen lassen.

	Zusätzlich werden auf Tagesebene dein Energiebezug und -einspeisung
	aufgezeichnet. Damit kannst du deinen Energieverbrauch auf Tages-, Monats- und
	Jahresbasis analysieren.

	\subsubsection{Steuerung von Wallboxen}
	Der WARP Energy Manager kann WARP Charger Wallbox verbrauchsabhängig steuern.
	Bis zu zehn Wallboxen vom Typ WARP Charger Smart,
	WARP Charger Pro, WARP2 Charger Smart und WARP2 Charger Pro werden unterstützt. Die
	Steuerung erfolgt über eine Netzwerkverbindung (LAN, WLAN) zwischen den Wallboxen und dem WARP Energy Manager.

	Mit verschiedenen Einstellungen kannst du definieren,
	unter welchen Bedingungen und mit wie viel Leistung Fahrzeuge geladen werden.

	\subsubsection{Phasenumschaltung}
	Mittels eines externen Schützes kann der WARP Energy Manager
	angeschlossene Wallboxen zwischen einem ein- und dreiphasigen Betrieb
	umschalten.

	\hint{WARP Charger (1. Generation) Wallboxen unterstützen leider keine Phasenumschaltung.}

	Durch die Phasenumschaltung kann die minimale Ladeleistung von
	ca.~\SI{4.1}{\kilo\watt} bei einem dreiphasigen Betrieb (minimaler Ladestrom
	\SI{6}{\ampere}) auf ca.~\SI{1.4}{\kilo\watt} reduziert werden. Somit
	kann auch ein geringer Leistungsüberschuss zum Laden eines Fahrzeugs verwendet werden.
	Ohne Phasenumschaltung ist bei kleinem Leistungsüberschuss ein Ladevorgang nicht möglich
	und der Überschuss wird ins Netz
	eingespeist. Alternativ müsste zusätzliche Leistung aus dem Netz bezogen werden,
	damit ein Ladevorgang beginnen kann.

	\subsubsection{Eingänge für potentialfreie Kontakte}
	Der WARP Energy Manager verfügt über zwei Eingänge für potentialfreie
	Schaltkontakte. Wird die Phasenumschaltung genutzt, wird einer dieser
	Eingänge fest zur Schützüberwachung verwendet. Ansonsten kann die Reaktion
	des Energiemanagers auf die Eingänge konfiguriert werden. Es kann zum
	Beispiel eine generelle Ladefreigabe realisiert oder der
	Ladestrom der Wallboxen begrenzt werden.

	\subsubsection{Potentialfreier Relaisausgang}
	Der potentialfreie Relaisschaltausgang des WARP Energy Managers kann
	genutzt werden, um externe Verbraucher o.ä. zu schalten. Der Ausgang kann
	konfiguriert werden und zum Beispiel abhängig von der verfügbaren Leistung,
	des momentanen Netzbezuges oder einer erfolgten Phasenumschaltung geschaltet
	werden.

	\hint{Mit dem Relais kann keine Netzspannung (230V) geschaltet
	werden. Es können bis zu 30V/1A geschaltet werden.}

	Mit einem künftigen Firmware-Update können beispielsweise SG Ready-Steuereingänge von Wärmepumpen mit
	diesem Relaisausgang gesteuert werden.

	\subsubsection{Status-LED}
	Der WARP Energy Manager besitzt auf der Frontseite eine Status-LED. Ist PV-Überschussladen aktiviert (siehe \ref{pv_ueberschussladen}),
	visualisiert diese LED den Zustand am Netzanschluss durch ein langsames Pulsieren bzw \enquote{atmen}. Die LED-Farben sind wie folgt:
	\begin{description}
		\item[Grün] Leistung wird ins Netz eingespeist
		\item[Gelb] Leistung wird aus dem Netz bezogen
		\item[Blau] \enquote{Keine} Leistung am Netzanschluss ($< \pm\SI{200}{\watt}$)
	\end{description}

	Wenn PV-Überschussladen nicht aktiviert ist, atmet die LED grün. In Fehlerfällen blinkt die Status-LED (siehe \ref{fehlerbehebung}).

	\subsection{Typische Anwendungen}

	\subsubsection{PV-Überschussladen}
	\label{pv_ueberschussladen}

	Besitzt du eine Photovoltaik-Anlage, möchtest du vermutlich möglichst viel
	von deinem produzierten Strom selbst nutzen. Der WARP Energy Manager kann
	dir dabei helfen, indem er ein reines PV-Überschussladen ermöglicht, bei
	dem nur überschüssige Energie ins Fahrzeug geladen wird. Alternativ kannst
	du auch erlaubten anteiligen Netzbezug definieren. Das ist sinnvoll,
	wenn die selbst produzierte Leistung nicht ausreicht, um einen Ladevorgang
	zu starten, du aber dennoch laden möchtest.

	Für das PV-Überschussladen benötigt der WARP Energy Manager einen Stromzähler
	an deinem Stromnetzanschluss, um den Überschuss, d.h. die Einspeisung von
	elektrischer Leistung ins Stromnetz, zu ermitteln. Der WARP Energy Manager
	steuert dann die Wallboxen so, dass keine Leistung ins Netz eingespeist wird
	(Netzbezug = 0) oder aber ein definierter Netzbezug eingehalten wird. Dies
	ist abhängig von deinen Einstellungen.

	Entscheidend ist hier, dass nur eine Leistungsregelung stattfindet, die einzelnen Phasenströme werden nicht geregelt. Da der Netzbetreiber-Stromzähler,
	der die Stromkosten ermittelt, saldierend arbeitet, ist eine
	Phasenstromregelung nicht notwendig.

	\vspace{-0.05cm}
	\subsubsection{Statisches Lastmanagement}
	\label{statisches_lastmanagement}

	Teilen sich mehrere Wallboxen eine gemeinsame Zuleitung, ist oft der
	Maximalstrom durch diese Zuleitung begrenzt. Als Beispiel könnten sich mehrere
	Wallboxen eine \SI{32}{\ampere} Leitung teilen. Zwei Wallboxen könnten jeweils als \SI{11}{\kilo\watt}
	Wallboxen ($2\cdot\SI{16}{\ampere}$) betrieben werden. Es wäre aber auch möglich, eine
	Wallbox mit \SI{22}{\kilo\watt} (\SI{32}{\ampere}) zu betreiben, wenn die zweite Wallbox nicht genutzt
	wird. Für diese Anwendungen kommt das statische Lastmanagement zum Einsatz.

	Der WARP Energy Manager kann das statische Lastmanagement für die Wallboxen
	übernehmen. Hierbei ist kein Stromzähler notwendig, es ist nur der
	Maximalstrom der Zuleitung zu definieren. Dieser Strom muss jederzeit zur
	Verfügung stehen. Der Energiemanager verteilt den Strom
	je nach Anforderung an die kontrollierten Wallboxen.

	\vspace{-0.05cm}
	\subsubsection{Dynamisches Lastmanagement}
	\label{dynamisches_lastmanagement}

	In manchen Fällen ist ein dynamisches Lastmanagement auf Phasenstromebene erforderlich.
	Ein typisches Beispiel dafür sind Mietobjekte, bei denen der Stromnetzanschluss der
	Immobilie nicht ausreicht, um mehrere Wallboxen gleichzeitig zu betreiben.
	Die Absicherung des Netzanschlusses beschränkt den zulässigen Phasenstrom.

	Im einfachsten Fall kann für alle Wallboxen ein bestimmter Phasenstrom garantiert werden.
	In diesem Fall können die Wallboxen ein statisches Lastmanagement durchführen,
	bei dem der verfügbare Phasenstrom zwischen den WARP Chargern aufgeteilt wird. (siehe \fullref{statisches_lastmanagement}).

	Oftmals kann jedoch nicht garantiert werden, dass ein bestimmter Phasenstrom jederzeit
	für Ladevorgänge zur Verfügung steht, da sich die Wallboxen den Netzanschluss mit anderen Verbrauchern teilen.
	Wenn diese Verbraucher ein- und ausgeschaltet werden,
	ändert sich der für die Wallboxen zur Verfügung stehende Phasenstrom
	ständig. In diesem Fall ist ein dynamisches Lastmanagement notwendig, um
	sicherzustellen, dass der maximale Phasenstrom nicht überschritten wird und
	keine Sicherung auslöst.

	Der WARP Energy Manager ermöglicht ein dynamisches Lastmanagement auf Phasenstromebene.
	Dazu ist ein Stromzähler am Stromnetzanschluss erforderlich, der vom Energiemanager
	ausgewertet werden kann. Der Energiemanager überwacht den zur Verfügung stehenden
	Phasenstrom vom Netzanschluss und regelt die Leistung der Wallboxen entsprechend.
	Dadurch wird sichergestellt, dass der maximale Phasenstrom nicht überschritten wird
	und keine Sicherung auslöst. Wenn eine Photovoltaik-Anlage vorhanden ist und Energie
	produziert, erhöht sie automatisch die zur Verfügung stehende Leistung für den
	Energiemanager, um das Laden der Elektrofahrzeuge zu optimieren.

	\subsubsection{Kombination PV + Lastmanagement}
	PV-Überschussladen und ein statisches oder dynamisches Lastmanagement können
	kombiniert werden. Der WARP Energy Manager betreibt dann die
	Leistungsregelung für das PV-Überschussladen und stellt parallel sicher, dass die
	Phasenstrom-Begrenzungen durch das Lastmanagement eingehalten werden.

	\newpage
	\section{Sicherheitshinweise}
	Der WARP Energy Manager ist so konstruiert, dass ein sicherer Betrieb gewährleistet ist,
	wenn er korrekt installiert wurde, in einem einwandfreien technischen Zustand
	ist und diese Betriebsanleitung befolgt wird. \hint{Der WARP Energy Manager darf nur von einer ausgewiesenen Elektrofachkraft installiert
		werden.}

	\subsection{Bestimmungsgemäße Verwendung}
	Mit dem WARP Energy Manager kann in Verbindung mit einem externen
	Stromzähler ein Energie-Monitoring realisiert werden. In Verbindung mit WARP
	Charger Wallboxen kann somit eine leistungsbezogene Ladevorgangsteuerung von
	Elektrofahrzeugen realisiert werden. Für andere Anwendungen ist der
	Energiemanager nicht geeignet. Eine Verwendung
	an Orten, an denen explosionsfähige oder brennbare Substanzen lagern, ist nicht
	zulässig. Jegliche Modifikation des Energiemanagers oder unsachgemäßer Betrieb ist verboten.
	Der Energiemanager ist in einem geeigneten Verteilerschrank zu installieren
	und vor Beschädigungen, Feuchtigkeit/Verschmutzungen und unsachgemäßem
	Zugriff zu schützen. Er darf nicht genutzt werden, wenn kein sicherer Betrieb
	gewährleistet werden kann.

	\subsection{Gerätestörung / Technischer Defekt}
	Sollte es Anzeichen für einen technischen Defekt geben, ist sofort die
	Stromversorgung des Energiemanagers zu trennen und gegen erneutes Einschalten zu
	sichern. Danach ist eine Elektrofachkraft zu informieren.

	\newpage
	\section{Montage und Installation}
	\subsection{Montage}
	\subsubsection{Lieferumfang}
	Im Lieferumfang des WARP Energy Managers befinden sich:
	\begin{itemize}
		\item WARP Energy Manager (Hutschienenmodul)
		\item Steckbare Schraubklemmen
		\begin{itemize}
			\item 2-pol Schraubklemme \SI{5}{\milli\meter} (\SI{230}{\volt} Stromversorgung (L+N))
			\item 2-pol Schraubklemme \SI{5}{\milli\meter} (Schütz)
			\item 4-pol Schraubklemme \SI{3.5}{\milli\meter} (Eingänge)
			\item 2-pol Schraubklemme \SI{3.5}{\milli\meter} (Relaisausgang)
			\item 4-pol Schraubklemme \SI{3.5}{\milli\meter} (RS485 Modbus-RTU)
		\end{itemize}
		\item Diese Betriebsanleitung inkl. individueller WLAN-Zugangsdaten
		\item RJ45-LAN-Winkeladapter
	\end{itemize}

	\subsubsection{Montageort}
	Der WARP Energy Manager darf nur in einem geeigneten Verteilerschrank im
	Innenbereich installiert werden. Er ist vor Staub, Nässe und unsachgemäßem
	Zugriff zu schützen. Es sollte
	eine LAN-Verbindung zum WARP Energy Manager gelegt werden, da in vielen
	Fällen eine Anbindung des WARP Energy Managers mittels WLAN aufgrund
	der Metallabschirmung der Verteilung nicht zuverlässig möglich ist.

	Es muss ausreichend Platz vorhanden sein. Es darf kein Druck auf die Kabel
	ausgeübt werden, insbesondere nicht auf die LAN-Verbindung. Aus diesem Grund
	empfehlen wir die Verwendung des mitgelieferten LAN-Winkeladapters.

	\subsubsection{Montage}
	Zur Montage des WARP Energy Managers muss dieser auf die Hutschiene
	gesetzt werden. Das Gehäuse muss so installiert werden, dass die Anschlüsse
	nach unten zeigen.

	\gfx{./img/wem_mounting.jpg}

	Zuerst wird die obere Halterung auf die Hutschiene aufgesetzt und anschließend
	die Untere. Der Energiemanager sollte sich selbstständig verriegeln, falls dies
	nicht der Fall ist, kann mit einem Schraubendreher an der schwarzen Verriegelung
	auf der Unterseite nachgeholfen werden.

	Soll der WARP Energy Manager wieder von der Hutschiene entfernt werden, so
	müssen zuerst alle Zuleitungen entfernt werden (\textbf{Achtung: Spannungsfreiheit
	sicherstellen!}). Anschließend kann mittels Schlitz-Schraubendreher die schwarze
	Federverriegelung gezogen und der Energiemanager von der Hutschiene
	gehoben werden. Dabei sollte zuerst die untere Halterung angehoben werden,
	gefolgt von der oberen Halterung.

	\subsection{Elektrischer Anschluss}
	\hint{Die in diesem Kapitel beschriebenen Arbeiten dürfen nur von einer ausgewiesenen
		Elektrofachkraft durchgeführt werden!}

	\gfx{./img/wem_connections.jpg}


	\subsubsection{Stromversorgung}
	Nachdem der WARP Energy Manager montiert wurde, kann dieser angeschlossen werden.
	Die Schraubklemmen sind steckbar, sodass der elektrische Anschluss
	außerhalb erfolgen kann. Anschließend können die Schraubklemmen wieder in
	den WARP Energy Manager gesteckt werden.

	Die Stromversorgung des WARP Energy Managers erfolgt über eine zweipolige
	Schraubklemme (\textbf{L}+\textbf{N}). Die Zuleitung ist mit einem
	max.~\SI{16}{\ampere}~Leitungsschutzschalter mit B-Charakteristik abzusichern.

	Die Stromversorgung des Energiemanagers ist zusätzlich intern über eine Glassicherung
	(mittelträge (m), \SI{500}{\milli\ampere}) abgesichert.

	\subsubsection{Schütz zur Phasenumschaltung}
	\hint{Es ist nicht notwendig, ein Schütz zu installieren. Dieser Schritt ist optional,
	wenn keine Phasenumschaltung erfolgen soll.}

	Ein externes Schütz kann zur Phasenumschaltung, das heißt, der Umschaltung
	zwischen einphasigem und dreiphasigem Fahrzeug-Ladevorgang, installiert werden. Das
	Schütz wird mittels \SI{230}{\volt} Schaltausgang vom WARP Energy Manager gesteuert
	(\textbf{Lsw}).
	Der minimale Phasenstrom für das Typ-2 Laden beträgt \SI{6}{\ampere}. Somit
	kann die Minimale Ladeleistung von \SI{4.1}{\kilo\watt} auf
	\SI{1.4}{\kilo\watt} reduziert werden.

	Zu Ansteuerung werden \textbf{N} und \textbf{Lsw} nach außen geführt. Der
	\textbf{Lsw}-Schaltausgang ist intern über eine Glassicherung
	(mittelträge (m), \SI{500}{\milli\ampere}) abgesichert.


	\subsubsection{Eingänge}
	Der WARP Energy Manager besitzt zwei Eingänge für potentialfreie Kontakte.
	An diesen können Schließer und Öffner angeschlossen werden. Das Verhalten des
	Energiemanagers in Bezug auf diese Eingänge kann im Webinterface konfiguriert werden.

	Wird ein Schütz zur Phasenumschaltung installiert, so ist der Eingang~\textbf{4}
	fest zur Schützüberwachung konfiguriert. Es ist erforderlich, einen Schließer zwischen
	\textbf{12V} und \textbf{4} zu installieren, der vom zu überwachenden Schütz geschaltet wird.

	Wird kein Schütz zur Phasenumschaltung verwendet, kann Eingang~\textbf{4} für
	andere Zwecke verwendet werden. Eingang~\textbf{3} steht immer
	für eigene Zwecke zur Verfügung. Die Eingänge sind so ausgelegt, dass ein potentialfreier
	Kontakt extern angeschlossen werden kann (Schalter als Öffner/Schließer, Relais etc.).
	Die \textbf{12V}~Anschlüsse der Eingänge sind hochohmig ausgelegt, liefern
	keine Leistung und sind daher nicht zur Stromversorgung anderer Verbraucher
	geeignet.

	\subsubsection{Relais-Ausgang}
	Mit dem potentialfreien Relaisschaltausgang können bis zu \SI{30}{\volt}/\SI{1}{\ampere} geschaltet
	werden. Das Schalten von Netzspannung ist nicht möglich!

	\subsubsection{RS485 Modbus Stromzähler}
	\hint{Es ist nicht notwendig, einen RS485-Modbus-Stromzähler zu installieren. Dieser
	Schritt ist optional, wenn ein anderer unterstützter Stromzähler konfiguriert
	wird.}

	Der WARP Energy Manager benötigt einen Stromzähler, um den Leistungsbezug regeln zu
	können. Eine Möglichkeit dafür ist die Installation eines RS485-Modbus-Stromzählers vom Typ Eastron SDM72DMV2, SDM630MCT oder SDM630Modbus.

	Die Steckerbelegung ist \textbf{12V, A, B, GND}. Der Anschluss \textbf{12V}
	darf nicht belegt werden. \textbf{A~(+), B~(-), GND} sind entsprechend
	am jeweiligen Stromzähler anzuschließen.

	\subsubsection{LAN-Anschluss}
	Die Steuerung der Wallboxen erfolgt über ein Netzwerk. Wir empfehlen den
	Anschluss des WARP Energy Managers per LAN. Der dafür notwendige LAN-Anschluss
	befindet sich im eingebauten Zustand vor den anderen Anschlüssen. Um Beschädigungen
	zu vermeiden, ist die LAN-Buchse flexibel befestigt. Wir empfehlen es dennoch, ein LAN-Kabel
	nicht direkt an den Energiemanager anzuschließen, sondern den mitgelieferten RJ45-Winkeladapters zwischen Energiemanager und LAN-Kabel zu verwenden.

	\hint{Der Energiemanager ist noch nicht betriebsbereit! Er muss jetzt über das Webinterface konfiguriert werden. Siehe \fullref{setup}}

	\vfill
	\null
	\newpage
	\section{Erste Schritte}
	\label{setup}

	Nach der elektrischen Installation kann der WARP Energy Manager konfiguriert
	werden. Dazu muss zuerst eine Verbindung zum Energiemanager hergestellt werden,
	damit dieser dann über den Browser konfiguriert werden kann.

	\subsection{Schritt 1: Verbindung herstellen}


	\hint{Wir empfehlen unbedingt eine Anbindung des WARP Energy Managers per
	LAN. Auch wenn technisch eine Anbindung mittels WLAN möglich ist, so muss
	sichergestellt werden, dass diese Verbindung dauerhaft stabil ist. Gerade in
	Schaltschränken gestaltet sich dies meist schwierig.}

	\paragraph{Option 1: WLAN}\ \\
	Im Werkszustand öffnet der WARP Energy Manager einen WLAN-Access-Point. Über diesen kann
	die Konfiguration vorgenommen werden, indem auf das das Webinterface des
	Energiemanagers zugegriffen wird.

	Die Zugangsdaten des Access-Points findest du auf dem WLAN-Zugangsdaten-Aufkleber
	auf der Rückseite dieser Anleitung. Ein weiterer identischer Aufkleber
	befindet sich auf der Rückseite der Frontplatte des WARP Energy Managers.
	Du kannst entweder den QR-Code des Aufklebers verwenden,
	der das WLAN automatisch konfiguriert, oder SSID und Passphrase abschreiben.
	Die meisten Kamera-Apps von Smartphones unterstützen das Auslesen des
	QR-Codes und das automatische Verbinden zum WLAN. Viele Smartphones
	erkennen, dass über das WLAN des Energiemanagers (Access-Point) kein Zugriff auf das
	Internet möglich ist. Dein Telefon fragt dann nach, ob du zu dem WLAN
	verbunden bleiben möchtest. Damit du weiter auf den Energiemanager zugreifen
	kannst, darfst du das WLAN nicht wieder verlassen.

	\begin{minipage}{0.35\textwidth}
		Wenn die Verbindung mit dem Access-Point des Energiemanagers hergestellt ist, kannst du das Webinterface
		unter \url{http://10.0.0.1} über einen Browser deiner Wahl erreichen.
		Alternativ kannst du dazu den nebenstehenden QR-Code scannen.
		Eventuell musst du deine mobile Datenverbindung (z.\,B. LTE) deaktivieren.
	\end{minipage}\hfill
	\begin{minipage}{0.12\textwidth}
		\begin{flushright}
			\qrcode{http://10.0.0.1}
		\end{flushright}
	\end{minipage}

	\paragraph{Option 2: LAN}\ \\
	Als Alternative zum Zugriff über den WLAN-Accesspoint verbindet sich der
	Energiemanager in den Werkseinstellungen automatisch zu einem
	kabelgebundenen Netzwerk (LAN), wenn ein LAN-Kabel eingesteckt ist, und bezieht eine IP-Adresse
	mittels DHCP. Der Energiemanager kann dann entweder über die zugewiesene IP-Adresse
	(\url{http://[IP-des-Energy-Managers]}, z.\,B. \url{http://192.168.0.42})
	oder den Hostnamen (\url{http://[hostname]}, z.\,B. \url{http://wem-ABC}) erreicht werden.

	Der Hostname des Energiemanagers ist identisch zur SSID des WLANs. Den Hostnamen findest du
	auf dem WLAN-Zugangsdaten-Aufkleber auf der Rückseite dieser Anleitung.

	Kann die per DHCP vergebene IP des Energiemanagers nicht ermittelt werden, so kann der
	zuvor genannte Zugriff auf den Energiemanager mittels WLAN-Access-Point genutzt
	werden, um die IP-Adresse der LAN-Schnittstelle zu ermitteln
	(\enquote{Status-Seite}, Abschnitt \enquote{LAN-Verbindung}).


	\subsection{Schritt 2: Konfiguration mittels Webinterface}
	Generell empfehlen wir nach der Installation ein Update der Firmware des
	Energiemanagers, um die neusten Funktionen und Bugfixes zu erhalten. Wie ein
	Firmware-Update durchgeführt wird, ist unter \fullref{firmware-update}
	beschrieben.

	Anschließend kann der WARP Energy Manager über das Webinterface konfiguriert
	werden. Die Einstellungen hängen vom Anwendungsfall ab.
	Das Webinterface ist unter \fullref{webinterface} vollständig beschrieben.

	Folgende Einstellungen müssen \textbf{in jedem Fall} vorgenommen werden:
	\begin{itemize}
	 \item Auf der Wallboxen-Einstellungsseite (siehe \ref{chargers}):
	 \begin{itemize}
		\item Den maximalen Gesamtstrom konfigurieren.
		\item Mindestens eine Wallbox hinzufügen.
	 \end{itemize}
	 \item Auf der Energy Manager-Einstellungsseite (siehe \ref{energy_manager_settings}):
	 \begin{itemize}
		\item Konfigurieren, ob ein Schütz zur Phasenumschaltung angeschlossen ist.
		\item Phasenumschaltungs-Modus konfigurieren.
	 \end{itemize}
	\end{itemize}

	Für das PV-Überschussladen müssen zusätzlich mindestens folgende Einstellungen vorgenommen werden:
	\begin{itemize}
	 \item Auf der Energy Manager-Einstellungsseite (siehe \ref{energy_manager_settings}):
	 \begin{itemize}
		\item Überschussladen aktivieren.
	 \end{itemize}
	 \item Auf der Stromzähler-Einstellungsseite (siehe \ref{stromzaehler})
	 \begin{itemize}
		\item Stromzähler konfigurieren.
	 \end{itemize}
	\end{itemize}

	\newpage
	\section{Webinterface}
	\label{webinterface}
	\vspace{-0.2cm}

	Über das Webinterface kannst du den Energieverbrauch überwachen und
	unter anderem das Laden der kontrollierten Wallboxen steuern.
	Es können diverse Einstellungen vorgenommen werden, die nachfolgend
	dokumentiert sind.

	Wenn du auf das Webinterface der Wallbox mit einem Browser zugreifst,
	gelangst du auf die Start-/ Statusseite. Auf der linken Seite befindet sich
	die Menüleiste, über die du zu weiteren Einstellungen kommst.

	Auf mobilen Endgeräten wird
	diese Menüleiste stattdessen versteckt unter einem Menü-Symbol oben rechts
	im grauen Balken neben dem WARP Logo angezeigt (\enquote{drei Striche untereinander}).
	Hier kannst du das Menü durch antippen des Symbols ausklappen.
	\gfx{./img/resized/web_status}

	\vspace{-0.4cm}
	\subsection{Status (Startseite)}
	\label{status}
	Die Startseite des Webinterfaces bietet Schnelleinstellungen und zeigt Statusinformationen an.

	Mittels Schaltflächen kann der Lademodus gesteuerter
	Wallboxen gewählt werden:
	\begin{description}
	\item[PV] \enquote{100\% Eigener Strom}. Ob ein
	Ladevorgang startet, ist davon abhängig, ob die minimale Ladeleistung
	als Überschuss zur Verfügung steht. Ist dies nicht der Fall, so
	wird kein Ladevorgang gestartet.
	\item[Min~+~PV] Es wird die minimal notwendige Ladeleistung sichergestellt, damit immer ein Ladevorgang begonnen werden kann. Diese Leistung kann (anteilig) aus dem Netz bezogen werden. Wird genügend Leistung produziert (Netzeinspeisung), so wird
	der Ladestrom so weit erhöht, bis keine Einspeisung ins Stromnetz mehr
	erfolgt, oder aber die maximale Ladeleistung erreicht wird.
	\item[Schnell] Alle Wallboxen laden mit der maximal möglichen
	Ladeleistung ohne Beachtung einer Netzeinspeisung bzw. eines Netzbezugs. Die konfigurierte Ladestromgrenze wird weiterhin eingehalten, damit die Zuleitung der Wallboxen nicht überlastet wird.
	\item[Aus] Die kontrollierten Wallboxen sind deaktiviert. Es kann
	nicht geladen werden.
	\end{description}
	Die Optionen \textbf{PV} und \textbf{Min~+~PV} sind nur verfügbar, wenn PV-Überschussladen aktiviert wurde.

	\textbf{Energy Manager} zeigt den Zustand des Energy Managers an. Wenn der Zustand nicht OK ist, wird das Laden an allen gesteuerten Wallboxen deaktiviert.

	\textbf{Verbrauchsverlauf} und \textbf{Leistungsaufnahme} sind nur vorhanden, wenn ein Stromzähler konfiguriert ist.
	Hier werden dir der aktuelle Netzbezug und ein Diagramm über
	die letzten 48~Stunden angezeigt.

	\textbf{Kontrollierte Wallboxen} zeigt den aktuellen Zustand des Lastmanagers und der vom Energiemanager gesteuerten
	Wallboxen an.

	\textbf{WLAN-Verbindung} zeigt an, ob eine Verbindung konfiguriert ist, ob sie erfolgreich aufgebaut wurde und
	unter welcher IP-Adresse die Wallbox per WLAN erreichbar ist.

	\textbf{LAN-Verbindung} zeigt analog dazu an, ob eine LAN-Verbindung besteht und unter welcher IP-Adresse die Wallbox erreichbar ist.

	Der \textbf{WLAN-Access-Point}-Status bildet den Status des Access-Points ab.
	\enquote{Deaktiviert} beziehungsweise \enquote{Aktiviert} zeigt den Zustand, wenn der Access-Point nicht
	nur als Fallback für die WLAN-Verbindung verwendet wird. Falls der Status \enquote{Fallback inaktiv} ist,
	war die WLAN-Verbindung bzw. LAN-Verbindung erfolgreich und der Access-Point wurde deshalb deaktiviert.
	Beim Status \enquote{Fallback aktiv} ist der Aufbau der WLAN-Verbindung fehlgeschlagen und der
	Access-Point wurde deshalb aktiviert.

	\textbf{Zeitsynchronisierung} zeigt an, ob Datum und Uhrzeit per Netzwerk-Zeitsynchronisierung (NTP) aktualisiert werden konnten.

	\textbf{WireGuard-Verbindung} zeigt an, ob die konfigurierte WireGuard-VPN-Verbindung aufgebaut werden konnte. Hierfür ist eine bestehende Zeitsynchronisierung notwendig.

	\textbf{MQTT-Verbindung} zeigt den aktuellen Status der MQTT-Verbindung
	zum konfigurierten Broker an.

	\subsection{Energiebilanz}

	\gfx{./img/resized/web_em_energy_analysis}
	\vspace{-0.2cm}

	Die Seite Energiebilanz stellt Informationen zum Energiebezug zur Verfügung.
	Die Daten werden lokal auf dem WARP Energy Manager gespeichert und
	können als Tages- und Monatsverlauf dargestellt werden.
	Der Energy Manager zeichnet Daten aller konfigurierten Stromzähler und kontrollierten Wallboxen auf.

	\vspace{-0.3cm}
	\subsection{Energiemanager}
	\subsubsection{Einstellungen}
	\label{energy_manager_settings}

	Alle Einstellungen bezüglich des Energiemanagements werden hier vorgenommen.

	Als erstes kann der \textbf{Standard-Lademodus} definiert werden. Die
	verschiedenen Modi werden in \fullref{status} erläutert.
	Wird der Modus auf der Statusseite geändert, so bleibt dieser Modus gesetzt, bis ein anderer gewählt wird,
	oder der Energiemanager neustartet.
	Mittels \textbf{Täglich rücksetzen} kann die Einstellung aber auch
	automatisch täglich wieder auf den Standard-Lademodus zurückgesetzt werden.

	\vspace{-0.2cm}
	\paragraph{PV-Überschussladen}\ \\
	PV-Überschussladen kann im entsprechenden Abschnitt mittels
	Schieberegler aktiviert werden. Nach der Aktivierung werden die Lademodi
	\enquote{PV} und \enquote{Min~+~PV} (siehe \fullref{status}) angeboten. Für das PV-Überschussladen muss ein	Stromzähler konfiguriert werden, wie unter \ref{stromzaehler} beschrieben.

	Soll eine Phasenumschaltung zwischen einem einphasigen und dreiphasigen Betrieb
	der Wallboxen erfolgen, so muss ein externes Schütz entsprechend installiert
	werden und die Option \textbf{Schütz angeschlossen} aktiviert werden. Bei
	Konfiguration der Option \textbf{Phasenumschaltung} auf \textbf{automatisch}
	schaltet der WARP Energy Manager dann selbstständig auf einen einphasigen
	Betrieb, sollte die PV-Leistung unterhalb von \SI{4,1}{\kilo\watt} liegen ($3\cdot\SI{230}{\volt}\cdot\SI{6}{\ampere}$), um
	eine minimale Ladeleistung von \SI{1,4}{\kilo\watt} zu ermöglichen ($1\cdot\SI{230}{\volt}\cdot\SI{6}{\ampere}$).
	Entsprechend schaltet der WARP Energy Manager wieder automatisch zurück,
	sobald die Mindestladeleistung für ein dreiphasiges Laden erreicht wird.

	Über die Einstellungen \textbf{Immer einphasig/Immer dreiphasig} kann das
	Schütz auch fest konfiguriert werden.

	\gfx{./img/resized/web_em_settings}

	Der Energiemanager unterbricht alle Ladevorgänge, bevor eine
	Phasenumschaltung stattfindet.

	Ist kein externes Schütz installiert, muss eingestellt werden, ob die
	vorhandenen Wallboxen fest einphasig oder dreiphasig angeschlossen sind.

	Das \textbf{Regelverhalten} bestimmt, wieviel Leistung den an­ge­schlos­senen Wallboxen in Abhängigkeit vom ge­messen­en PV-Über­schuss zur Verfügung gestellt wird. Dabei ist relevant, ob ein Batteriespeicher vorhanden ist oder nicht. Wenn kein Batteriespeicher vorhanden ist, beeinflusst das Regelverhalten, ob tendenziell Strom bezogen oder eingespeist wird. Die konservativen Modi versuchen, Bezug zu reduzieren. Dadurch wird entsprechend mehr PV-Über­schuss nicht verbraucht und stattdessen eingespeist. Die aggressiven Modi versuchen, PV-Über­schuss selbst zu verwenden und Einspeisung zu verhindern. Dadurch muss häufiger Strom bezogen werden. Im ausgeglichenen Modus halten sich Bezug und Einspeisung ungefähr die Waage. \textbf{Wenn ein Batteriespeicher vorhanden ist}, ist dies dem Energy Manager aktuell nicht bekannt. In den konservativen Modi wird dem Batteriespeicher Vorrang gegeben, sodass dieser tagsüber erst geladen wird und nachts nicht zum Laden des Fahr­zeuges verwendet wird. In den agressiven Modi wird der Wallbox Vorrang gegeben, sodass der Batteriespeicher sowohl tagsüber als auch nachts verwendet wird, um das Fahrzeug zu laden. Im ausgeglichenen Modus hängt das Verhalten vom Batteriespeicher ab und kann nicht vorhergesagt werden. Dieser Modus wird nicht empfohlen. Der leicht konservative oder leicht aggressive Modus ist üb­licher­weise aus­reich­end, um eine gewünschte Tendenz vorzugeben.

	\paragraph{Dynamisches Lastmanagement}\ \\
	\hint{Die Funktion \textbf{Dynamisches Lastmanagement} wird mit einem künftigen Firmware-Update zur Verfügung gestellt.}

	Beim dynamischen Lastmanagement misst der WARP Energy Manager laufend mittels eines Stromzählers die
	Ströme aller Phasen am Stromnetzanschluss. Der noch rechnerisch zur
	Verfügung stehende Strom kann für jede Phase unterschiedlich sein und ändert
	sich laufend aufgrund des Zu- und Abschaltens von Verbrauchern. Auch eine
	parallel angeschlossene PV-Anlage beeinflusst die Phasenströme. Der WARP
	Energy Manager ermittelt rechnerisch den noch zur Verfügung stehenden
	Phasenstrom und gibt diesen den gesteuerten Wallboxen frei.
	Dabei wird sichergestellt, dass der Maximalstrom jeder Phase nicht überschritten wird und keine Sicherung ausgelöst wird.

	\paragraph{Relais}\ \\
	Der WARP Energy Manager verfügt über einen potentialfreien Schaltausgang
	(Relais). Dessen Funktion kann hier definiert werden.

	Im Modus
	\textbf{Regelbasiert} können mittels Drop-Down-Boxen verschiedene Bedingungen definiert werden, in
	denen der Relais-Ausgang geschlossen wird und geschlossen bleibt. Ist die
	Bedingung nicht mehr erfüllt, dann wird das Relais wieder geöffnet.
	Im Modus \textbf{Manuell gesteuert oder nicht verwendet} wird das Relais nicht automatisch vom Energy Manager geschaltet. Es kann mittels der API gesteuert werden.

	\paragraph{Eingänge 3 und 4}\ \\
	Die Eingänge 3 und 4 können genutzt werden, um potentialfreie Kontakte auszulesen,
	z.\,B. Schalter oder Relaisausgänge. Die Reaktion des WARP Energy Managers auf diese Eingänge
	kann kann hier definiert werden. Wird ein Schütz zur
	Phasenumschaltung angeschlossen und genutzt, dann steht Eingang 4 nicht mehr
	zur Verfügung, da mit diesem das Schütz überwacht wird.

	Als Optionen stehen zur Verfügung
	\begin{description}
		\item[Nicht verwendet] Der Eingang wird nicht genutzt.
		\item[Laden blockieren] Wenn der Eingang geschlossen bzw. geöffnet ist, sind
		Ladevorgänge bei allen Wallboxen nicht möglich bzw. werden gestoppt.
		\item[Ladestrom begrenzen] Wenn der Eingang geschlossen bzw. geöffnet ist,
		wird der Ladestrom aller Wallboxen auf den eingestellten Wert begrenzt.
		\item[Moduswechsel] Wenn der Eingang geschlossen bzw. geöffnet wird, wird
		der Lademodus auf den konfigurierten gewechselt.
	\end{description}

	\subsubsection{Stromzähler}
	\label{stromzaehler}

	\gfx{./img/resized/web_em_meter_config}

	Als Stromzähler am Netzanschluss können verschiedene Stromzähler-Typen
	konfiguriert werden. Hier muss ein Stromzähler konfiguriert werden, wenn
	der WARP Energy Manager die Funktionen \fullref{pv_ueberschussladen} oder
	\fullref{dynamisches_lastmanagement} ausführen soll.

	Mit der Einstellung \textbf{SDM630*/SDM72*} werden folgende
	RS485-(Modbus RTU-)Stromzähler unterstützt:
	\begin{itemize}
		\item Eastron SDM630
		\item Eastron SDM72DM V2
		\item Eastron SDM630MCT V2
	\end{itemize}

	Mit der Einstellung \textbf{Benutzerdefinierter Zähler MQTT/HTTP}
	werden Stromzählerwerte verwendet, die dem WARP Energy Manager per API übergeben
	werden.

	\subsubsection{Wallboxen}
	\label{chargers}

	\gfx{./img/resized/web_charge_manager}

	Hier werden die vom Energiemanager kontrollierten Wallboxen konfiguriert.
	Die hier vorgenommenen Einstellungen beeinflussen das Lastmanagement
	zwischen den Wallboxen.

	Typ-2-Wallboxen kommunizieren den angeschlossenen Fahrzeugen den maximal zur
	Verfügung stehenden Ladestrom. Das Fahrzeug entscheidet, ob dieser Ladestrom
	voll ausgenutzt wird und ob ein Ladevorgang ein-, zwei- oder dreiphasig durchgeführt
	wird.

	Als erste Einstellung muss der \textbf{Maximale Gesamtstrom}
	der Zuleitung zu den Wallboxen konfiguriert werden.
	Der Energiemanager stellt sicher, dass dieser
	Strom auf keiner Phase überschritten wird, indem niemals mehr als
	dieser Strom an die Wallboxen verteilt wird. Besitzen alle Wallboxen
	ausreichend dimensionierte getrennte Zuleitungen kann dieser Strom so
	hoch eingestellt werden, dass alle Wallboxen sicher ihren Maximalstrom
	erhalten. Alle andere Komponenten, wie zum Beispiel der Netzanschluss,
	müssen dann den konfigurierten maximalen Gesamtstrom liefern können.
	Der individuelle Maximalstrom jeder Wallbox bleibt hiervon unberührt
	(Zuleitung der Wallbox - Schiebeschaltereinstellung innerhalb der Wallbox).

	\hint{Hierbei handelt es sich um ein statisches Lastmanagement, bei dem
	davon ausgegangen wird, dass der eingestellte Strom auf jeder Phase
	zu jeder Zeit zur Verfügung steht. Andere Verbraucher als WARP Charger,
	welche vom Energiemanager nicht gesteuert werden können, werden nicht
	berücksichtigt!}

	Mit der Einstellung \textbf{Minimaler Ladestrom} kann der minimale Ladestrom
	angehoben werden. Der Typ-2-Ladestandard setzt als Minimum \SI{6}{\ampere} voraus. Eine
	Einstellung darunter ist nicht möglich. Allerdings gibt es Fahrzeuge, welche
	bei einem verfügbaren Strom von \SI{6}{\ampere} nicht mit einem Ladevorgang beginnen
	oder nur sehr ineffizient laden. Falls notwendig, kann hier ein höherer Ladestrom definiert werden.

	Am Ende der Seite werden die \textbf{Kontrollierten
	Wallboxen} dargestellt. Weitere Wallboxen können mittels Klick auf
	\textbf{Wallbox hinzufügen} der Steuerung durch den WARP Energy Manager
	hinzugefügt werden. Dazu muss der Anzeigename und die IP-Adresse oder der
	Hostname der Wallbox eingetragen werden und mittels Klick auf \enquote{hinzufügen} übernommen werden.

	Automatisch ermittelte Wallboxen, die noch nicht vom Energiemanager
	gesteuert werden, werden als Liste dargestellt.

	\subsection{Netzwerk}
	\label{network}
	Die Wallbox kann in dein Netzwerk per WLAN oder LAN eingebunden werden.
	In diesem Unterabschnitt können alle dazugehörigen Einstellungen vorgenommen werden.

	\subsubsection{Allgemein}

	\gfx{./img/resized/web_network}

	Hier kannst du den Hostnamen des WARP Energy Managers in allen verbundenen Netzwerken konfigurieren. Außerdem kann mDNS aktiviert oder deaktiviert werden.
	Über mDNS können andere Geräte im Netzwerk den WARP Energy Manager finden.


	\subsubsection{WLAN-Verbindung}
	\gfx{./img/resized/web_wifi_sta}

	Es besteht die Möglichkeit, den WARP Energy Manager mittels WLAN in dein Netzwerk
	zu integrieren. \textbf{Diese Option empfehlen wir aber ausdrücklich
	nicht!}
	Durch Drücken des \enquote{Netzwerksuche}-Buttons öffnet sich ein Menü, in dem das gewünschte WLAN ausgewählt werden kann.
	Es werden dann automatisch Netzwerkname (SSID) und BSSID eingetragen, sowie die Verbindung beim Neustart aktiviert.
	Gegebenenfalls musst du jetzt noch die Passphrase des gewählten Netzes eintragen.

	Du kannst jetzt die Konfiguration mit dem Speichern-Button abspeichern.
	Das Webinterface startet dann neu und verbindet sich zum konfigurierten WLAN. Die Statusseite zeigt
	an, ob die Verbindung erfolgreich war. Der Access-Point bleibt weiterhin
	geöffnet, sodass Konfigurationsfehler behoben werden können.
	Da der Access-Point den selben Kanal wie ein eventuell verbundenes Netz verwendet,
	kann es sein, dass du dich jetzt neu zum Access-Point verbinden musst. Bei einer erfolgreichen Verbindung sollte den Energiemanager jetzt im konfigurierten Netzwerk unter
	\url{http://[konfigurierter_hostname]}, z.\,B. \url{http://wem-ABC} erreichbar sein.

	\subsubsection{WLAN-Access-Point}

	Der Access-Point kann in einem von zwei Modi betrieben werden: Entweder kann er immer aktiv sein,
	oder nur dann, wenn die Verbindung zu einem Netzwerk nicht konfiguriert oder fehlgeschlagen ist.
	Außerdem kann der Access-Point komplett deaktiviert werden.

	\hint{Wir empfehlen, den Access-Point nie komplett zu deaktivieren, da sonst bei einer
		fehlgeschlagenen Verbindung zu einem anderen Netzwerk das Webinterface nicht mehr erreicht
		werden kann. Der WARP Energy Manager kann dann nur über ein Zurücksetzen auf Werkszustand, siehe \ref{reset}, erreicht werden.}

	\gfx{./img/resized/web_wifi_ap}


	Weitere Einstellungen, wie der Modus des Access-Points,
	Netzwerkname, Passphrase usw. können hier festgelegt werden.

	\subsubsection{LAN-Verbindung}
	\gfx{./img/resized/web_ethernet}
	In den meisten Fällen wird eine
	LAN-Verbindung automatisch hergestellt, wenn ein Kabel eingesteckt ist.
	Eine IP-Adresse wird per DHCP bezogen. Es ist aber auch möglich,
	eine statische IP-Konfiguration	einzutragen, oder, falls gewünscht, die LAN-Verbindung
	komplett zu deaktivieren.
	Bei einer erfolgreichen Verbindung sollte der WARP Energy Manager jetzt im LAN unter
	\url{http://[konfigurierter_hostname]}, z.\,B. \url{http://wem-ABC} erreichbar sein.

	\subsubsection{WireGuard}

	WireGuard ist eine Möglichkeit, den WARP Energy Manager mittels einer verschlüsselten
	Verbindung in ein virtuelles privates Netzwerk (VPN) einzubinden. WireGuard wird von
	verschiedenen Routern direkt unterstützt. Dies kann zum Beispiel genutzt
	werden, um aus der Ferne auf den Energiemanager zuzugreifen oder das
	Wallbox-Netzwerk vor fremdem Zugriff zu schützen. Zusätzlich kann das
	Lastmanagement zwischen Energy Manager und den Wallboxen per WireGuard abgesichert werden.

	Die notwendigen Parameter sind WireGuard-typisch und werden an dieser Stelle
	nicht gesondert erläutert. Weitere Informationen finden sich auf
	\url{https://www.wireguard.com/}.

	\gfx{./img/resized/web_wireguard}

	\subsection{Schnittstellen}
	\subsubsection{MQTT}
	\label{mqtt-interface}

	Auf der MQTT-Unterseite kannst du die Verbindung zu einem MQTT-Broker konfigurieren. Folgende Einstellungen können vorgenommen werden:
	\begin{itemize}
		\item \textbf{Broker-Hostname oder -IP-Adresse} Der Host\-name oder die
		IP-Adresse des Brokers, zu dem sich der WARP Energy Manager verbinden soll.
		\item \textbf{Broker-Port} Der Port, unter dem der Broker erreichbar ist. Der typische MQTT-Port 1883 ist voreingestellt.
		\item \textbf{Broker-Benutzername} und \textbf{-Passwort} Manche Broker unterstützen eine Authentifizierung mit Benutzername und Passwort.
		\item \textbf{Topic-Präfix} Dieses Präfix wird allen Topics vorangestellt, die der Energy Manager verwendet.
		      Voreingestellt ist wem/ABC, wobei ABC die eindeutige Kennung des
			  WARP Energy Managers ist,
		      es sind aber andere Präfixe möglich.
		      Falls mehrere Energiemanager mit dem selben Broker kommunizieren,
		      müssen eindeutige Präfixe gewählt werden.
		\item \textbf{Client-ID} Mit dieser ID registriert sich der WARP Energy Manager beim Broker.
		\item \textbf{Sendeintervall} Der WARP Energy Manager verschickt MQTT-Nachrichten nur, wenn sich die beinhalteten Daten geändert haben.
			Es gibt aber Teile der API, deren Daten sich sekündlich ändern. Das Sendeintervall kann hier reduziert werden, wenn weniger Netzwerktraffic
			erzeugt werden soll.
	\end{itemize}

	Nachdem die Konfiguration gesetzt und der \enquote{MQTT aktivieren}-Schalter aktiviert ist, kann die Konfiguration gespeichert werden.
	Das Webinterface startet dann neu und der Energy Manager verbindet sich zum Broker.
	Auf der Status-Seite wird angezeigt, ob die Verbindung aufgebaut werden konnte.

	Weitere Informationen über die MQTT-API des WARP Energy Managers findest du auf \rurl{https://warp-charger.com/api.html}{warp-charger.com/api.html}

	\gfx{./img/resized/web_mqtt}

	\subsection{System}
	Im System-Unterabschnitt kannst du Einstellungen zur Zeitsynchronisation
	vornehmen, die interne microSD-Karte formatieren und diverse Informationen zur Fehlerbehebung
	bekommen. Auch das Aktualisieren der Firmware ist hier möglich.

	\subsubsection{Zeitsynchronisierung}\label{ntp}
	Um für die Aufzeichnung der Energiebilanz und das Ereignis-Log die aktuelle Uhrzeit zur
	Verfügung zu haben, kann der WARP Energy Manager diese per NTP über
	eine Netzwerkverbindung synchronisieren. Auf dieser Unterseite kannst du NTP aktivieren oder deaktivieren und die Zeitzone, in der sich der WARP Energy Manager befindet, konfigurieren.

	Außerdem ist es möglich, zusätzlich zum konfigurierten Zeitserver einen Zeitserver zu verwenden, der von deinem Router per DHCP gesetzt wird. Dies funktioniert allerdings nur,
	wenn in der Netzwerkkonfiguration keine statische IP-Konfiguration verwendet wurde.

	\gfx{./img/resized/web_ntp}

	\subsubsection{SD-Karte}
	Die Daten des WARP Energy Managers werden intern auf einer microSD-Karte
	aufgezeichnet. Hier werden Informationen über die eingelegte Karte ausgegeben. Die microSD-Karte kann
	hier formatiert werden. Dadurch werden alle aufgezeichneten Informationen gelöscht!

	\gfx{./img/resized/web_em_sdcard}

	\subsubsection{Debug}
	Auf der Debug-Seite kann ein Energiemanager-Protokoll erstellt werden. Dies
	ist hilfreich, um etwaige Probleme bei der Energieverteilung zu diagnostizieren. Um
	ein Protokoll zu erzeugen, muss einfach nur auf \textbf{Start} geklickt
	werden. Der Energy Manager beginnt dann hochfrequent alle Zustände
	aufzuzeichnen. Mit \textbf{Stop+Download} kann die Aufzeichnung gestoppt und
	das erstellte Protokoll heruntergeladen werden.

	Unter \textbf{Interner Zustand} und \textbf{Low-Level-Zustand} werden interne Zustände vom Energy Manager
	dargestellt, die zur Fehlerbehebung hilfreich sein können.

	\subsubsection{Zugangsdaten}

	\gfx{./img/resized/web_authentication.png}

	Auf dieser Unterseite kannst du einen Benutzernamen und ein Passwort konfigurieren, mit denen du den Zugriff auf das Web Interface
	des WARP Energy Managers schützt. Zugriffe auf das Webinterface und die HTTP-API sind bei aktivierter Anmeldung nur möglich, wenn
	die korrekten Zugangsdaten angegeben werden.
	\hint{Falls du die Zugangsdaten vergisst ist ein Zugriff auf das Webinterface nur noch nach einem Zurücksetzen auf Werkszustand möglich. Siehe \fullref{reset}}

	\subsubsection{Ereignis-Log}
	\gfx{./img/resized/web_event_log}

	Das Ereignis-Log zeichnet relevante Informationen des Systemstarts, sowie WLAN- und MQTT-Verbindungsabbrüche und Regelungsinformationen auf.
	Falls Probleme mit dem WARP Energy Manager auftreten, kannst du diese mit dem Log diagnostizieren.
	Falls du ein Problem mit dem WARP Energy Manager an uns melden möchtest, kannst du einen Debug-Report abrufen,
	der uns helfen kann, das Problem zu verstehen und zu lösen. Diese beinhaltet neben dem Ereignis-Log die vollständige
	Konfiguration des Energiemanagers, mit Ausnahme von Passwörtern o.\,Ä.

	\subsubsection{Firmware-Aktualisierung}
	\label{firmware-update}
	\gfx{./img/resized/web_firmware_update}
	Hier kannst du die Firmware des Energy Managers aktualisieren und das Webinterface neustarten.
	Wir entwickeln die Funktionalität
	des Energy Managers laufend weiter. Bitte beachte, dass daher ggf. auch eine neue
	Version dieser Betriebsanleitung bereitgestellt wird.
	Die aktuelle Firmware und die neuste Betriebsanleitung findest du unter
	\rurl{https://warp-charger.com}{warp-charger.com} zum Download.

	\subsection{Zurücksetzen auf Werkszustand}\label{reset}
	Falls das Webinterface nicht korrekt funktioniert, oder die Konfiguration defekt ist,
	kannst du auf der Firmware-Aktualisierungs-Unterseite alle Einstellungen auf den Werkszustand zurücksetzen.
	\hint{Durch das Zurücksetzen auf Werkszustand gehen \mbox{\textbf{alle}} Konfigurationen verloren.}
	Nach dem Zurücksetzen startet das Webinterface wieder und öffnet
	den Access-Point mit der SSID und Passphrase, die auf dem Aufkleber vermerkt
	sind. Der WARP Energy Manager kann jetzt wieder nach \fullref{setup} konfiguriert werden.

	Falls du das Webinterface nicht mehr erreichen kannst, kannst du versuchen, die Recovery-Seite zu öffnen.
	Falls du über den Access Point der Wallbox verbunden bist, erreichst du diese unter \url{http://10.0.0.1/recovery},
	bei einer bestehenden Verbindung zu einem LAN oder WLAN über
	\url{http://[konfigurierter_hostname]/recovery}, also z.\,B. \url{http://wem-ABC/recovery}.
	Über die Recovery-Seite kannst du den WARP Energy Manager neustarten, Firmware-Updates einspielen,
	den Energy Manager auf den Werkszustand zurücksetzen (Factory Reset) und Debug-Reports
	herunterladen.

	Falls der WARP Energy Manager weder seinen Access Point öffnet, noch über ein konfiguriertes Netzwerk auf das Webinterface zugegriffen werden kann,
	kannst du wie folgt das Zurücksetzen auf Werkseinstellungen starten:
	\begin{enumerate}
	 \item Suche dir einen elektrisch nicht leitenden Stift (Kugelschreiber o.\,Ä.) und einen kleinen Schlitz-Schraubendreher (z.\,B. Phasenprüfer o.\,Ä.).
	 \item Öffne den Energy Manager, indem du die bedruckte Frontplatte mit dem Schraubendreher entfernst. Setze dazu in einem der seitlichen Schlitze an.
	 \item Drücke mit dem Stift einmal kurz auf \textbf{EN} (1). Die blaue LED fängt an zu blinken.
	 \item Drücke anschließend mit dem Stift \textbf{IO0} (2) und halte diesen gedrückt. Die blaue LED (3) fängt an schneller zu blinken.
	 \item Halte \textbf{IO0} (2) ca. 8\,Sekunden gedrückt, bis die LED (3) dauerhaft leuchtet.
	 \item Sobald die blaue LED (3) dauerhaft leuchtet, ist der Vorgang abgeschlossen. Sollte die LED (3) währenddessen ausgehen, so war der Vorgang nicht erfolgreich und muss wiederholt werden.
	\end{enumerate}
	Der WARP Energy Manager setzt jetzt alle Einstellungen auf den Werkszustand zurück. Bei Erfolg sollte es jetzt möglich sein, über den Access Point wieder auf den Energiemanager zuzugreifen.

	\gfx{./img/resized/factory_reset_2}

	\newpage
	\section{Fehlerbehebung}
	\label{fehlerbehebung}
	Die Status-LED des WARP Energy Manager blinkt in Fehlerfällen. Die Farbe gibt dir die Art des Fehlers an.
	\subsection{Status-LED blinkt gelb}
	Ist PV-Überschussladen aktiviert, atmet die Status-LED gelb, sobald Strom
	aus dem Netz bezogen wird. Dies ist kein Fehlerzustand.
	Sollte PV-Überschussladen nicht aktiv sein und die Status-LED blinkt gelb,
	dann ist der WARP Energy Manager nicht zum konfigurierten WLAN verbunden und es ist kein LAN-Kabel angeschlossen.

	\subsection{Status-LED blinkt rot}
	Blinkt die Status-LED rot, so ist der WARP Energy Manager in einem
	Fehlerzustand. Gründe können eine fehlgeschlagene Schützüberwachung oder ein
	interner Fehler sein. Das Webinterface gibt genauere Auskunft.

	\subsection{Status-LED blinkt violett}
	Blinkt die Status-LED violett, so ist die Konfiguration nicht vollständig. Es muss mindestens die in \fullref{setup} beschriebene Minimalkonfiguration vorgenommen werden, damit der Energy Manager korrekt funktioniert.

	\subsection{Sicherungswechsel}
	Der WARP Energy Manager ist intern über zwei 5$\times\SI{20}{\milli\meter}$ Feinsicherungen (mittelträge (m), \SI{500}{\milli\ampere}) abgesichert.
	Tinkerforge verbaut Sicherungen vom Typ \enquote{ESKA 521.014}. Die eine
	Sicherung befindet sich im Eingangspfad der 230V Stromversorgung (L). Die
	andere Sicherung befindet sich im Schaltausgang der Schützsteuerung.

	\section{Konformitätserklärung}
	Die EU-Konformitätserklärung zum WARP Energy Manager ist in einem gesonderten Dokument verfügbar.

	\section{Entsorgung}
	\begin{minipage}{0.43\textwidth}
		WARP Energy Manager und Verpackung sind bei Gebrauchsende ordnungsgemäß zu
		entsorgen. Altgeräte dürfen nicht über den Hausmüll entsorgt werden.
	\end{minipage}\hfill
	\begin{minipage}{0.045\textwidth}
		\includegraphics[width=\linewidth]{./img/resized/weee.pdf}
	\end{minipage}

	\section{Technische Daten}

	%use minipage here to control footnote placement
	\begin{minipage}{\linewidth}

		\begin{description}[leftmargin=!,labelwidth=\widthof{\textbf{PV-Überschussladen}}]
			\setlength{\itemsep}{3pt}
			\item[Abmessungen] 70 × 90 × \SI{63}{\milli\meter} (B/H/T)
			\item[Montageort] Schaltschrank
			\item[Montageart] Tragschiene
			\item[Nennspannung] \SI{230}{\volt} AC
			\item[Nennfrequenz] \SI{50}{\hertz}
			\item[Eigenverbrauch min.] \SI{1.1}{\watt}\footnote[1]{LAN aktiv, WLAN
			Fallback, Relais aus, LED aus}
			\item[Eigenverbrauch max.] $\sim$\SI{2}{\watt}\footnote[7]{LAN aktiv, WLAN
			ein, Relais ein, LED ein}
			\item[Betriebstemperatur] \SI{0}{\celsius}
			      bis \SI{+30}{\celsius}
			\item[Schutzklasse] II
			\item[PV-Überschussladen] max. 10 WARP Charger\footnote[12]{\label{fn:1}WARP
			Charger/WARP Charger 2 in Varianten Smart/Pro}
			\item[Lastmanagement] max. 10 WARP Charger\footref{fn:1}
			\item[Netzwerk] LAN, WLAN
			\item[Schnittstellen] HTTP, MQTT
		\end{description}
	\end{minipage}


	\section{Kontakt}
	Tinkerforge GmbH\\ Zur Brinke 7\\ 33758 Schloß Holte-Stukenbrock
	\begin{description}[leftmargin=!,labelwidth=\widthof{\textbf{Website}}]
		\item[E-Mail] \href{mailto:info@tinkerforge.com}{\texttt{info@tinkerforge.com}}
		\item[Website] \href{https://warp-charger.com}{\texttt{warp-charger.com}}
		\item[Telefon] \phonenumber{052078998614}
		\item[Shop] \href{https://tinkerforge.com/de/shop/warp.html}{\texttt{tinkerforge.com/de/shop/warp.html}}
	\end{description}

	\section{Dokumentversionen}
	\begin{tabular}{lll}
		\toprule
		Datum      & Version & Kommentar                       \\
		\midrule
		03.03.2023 & 1.0     & Initialversion                  \\
		\bottomrule
	\end{tabular}

	\vfill
	\null
	\newpage

	\columnbreak
\appendix



	\newpage
	\pagestyle{empty}
	\null
	\vfill
	WLAN-Zugangsdaten
	\begin{tcolorbox}[width=4.2cm,height=2.7cm, boxrule=0.25mm]

	\end{tcolorbox}
	Dieser Aufkleber befindet sich\\ auch unter der Frontplatte des WARP Energy
	Managers.
	\columnbreak

	\null
	\vfill
	Typenschild
	\begin{tcolorbox}[width=7.8cm,height=4.1cm, boxrule=0.25mm]

	\end{tcolorbox}
	Dieser Aufkleber befindet sich auch an der Seite\\ des WARP Energy Managers.
\end{multicols*}
\end{document}
