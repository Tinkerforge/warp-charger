\documentclass[a4paper,10pt]{article}
\ifdefined\forprint
    \usepackage[width=21.6cm,height=30.3cm,center]{crop}
\fi
\usepackage[utf8]{inputenc}
\usepackage[margin=2cm,headheight=26pt,includeheadfoot]{geometry}

\usepackage[german]{babel}
\usepackage[german=quotes]{csquotes}

\usepackage{nameref}
\usepackage{microtype}
\usepackage{float}
\usepackage{siunitx}
\sisetup{
    locale = DE,
    binary-units,
    detect-all,
    per-mode = symbol %enables m/s instead of ms^-1
}
\AtBeginDocument{\DeclareSIUnit{\kWh}{kWh}}

\usepackage{caption} %for \caption*
\usepackage{hhline}
\usepackage{tabularx}
\usepackage{array}
\usepackage{calc}
\usepackage{multicol}
\usepackage{multirow}
\usepackage{parskip}
\usepackage{booktabs}
\usepackage{textcomp}

\usepackage{fancyhdr}
\pagestyle{fancy}
\setlength{\headheight}{48pt}
\renewcommand{\headrulewidth}{0pt}

\usepackage{xcolor,colortbl}
\usepackage{makecell}

\usepackage[symbol*]{footmisc}
\renewcommand{\thefootnote}{\fnsymbol{footnote}}
\renewcommand{\thempfootnote}{\fnsymbol{mpfootnote}}

\usepackage{color}
\usepackage{enumitem}

\usepackage{pdfpages}

\usepackage{phonenumbers}

% Word-Stealth-Modus, kann aber kein Omega
%\usepackage[scaled]{helvet}

\usepackage[inline,nomargin]{fixme}
\fxsetup{
    author=,
    layout=inline,
    theme=color
}

\definecolor{fxnote}{rgb}{0.8000,0.0000,0.0000}
\colorlet{fxnotebg}{yellow}

\definecolor{boxgray}{rgb}{0.33,0.33,0.33}
\usepackage{tcolorbox}
\title{}
\author{}

\renewcommand{\familydefault}{\sfdefault}

\newcommand{\hint}[1]{\begin{tcolorbox}[colback=boxgray,colframe=black,coltext=
white,title=Hinweis,left*=2mm,right*=2mm,boxsep=1mm,bottom=1mm,top=1mm]#1\end{tcolorbox}}

\newcommand{\gfx}[1]{\includegraphics[width=\linewidth]{#1}}

\newcommand*{\fullref}[1]{\hyperref[{#1}]{\ref*{#1}~\nameref*{#1}}}

\fancyhf{}
\fancyhead{\colorbox{boxgray}{
    \makebox[\dimexpr\linewidth-2\fboxsep][l]{
        \includegraphics[height=1cm]{./img/resized/logo}\hfill\color{white}\Huge\raisebox{.5ex}{\thepage}
        }
    }
}

\usepackage{hyperref}
\usepackage{qrcode}

\appto\UrlNoBreaks{\do\.\do\:\do\/\do\_}

\newcommand\rurl[2]{%
  \href{#1}{\nolinkurl{#2}}%
}


\newcommand{\tdesc}[1]{\multicolumn{3}{l}{\footnotesize #1}}

\usepackage{hyphenat}

\hyphenation{Web-inter-face}

\begin{document}
\pagestyle{empty}
\begin{titlepage}
	\vspace*{-3.08cm}
	\colorbox{boxgray}{\makebox[\dimexpr\linewidth-2\fboxsep][c]{\includegraphics[width=0.6\textwidth]{./img/resized/logo}}}
	\vfill
	\begin{center}
		\Huge
		WARP Energy Manager Betriebsanleitung\\\vspace{1cm}
		\large
		Version 1.0.0\\\vspace{0.25cm}
		24.01.2023
	\end{center}
	\vfill \gfx{./img/resized/warp-energy-manager.png}
\end{titlepage}
\newpage
\null
\newpage
\pagestyle{fancy}
\begin{multicols*}{2}
	\tableofcontents
	\newpage
	\section{Einführung}
	\subsection{Vorwort} Vielen Dank, dass du
	dich für einen WARP Energy Manager von Tinkerforge entschieden hast!

	\enquote{WARP} steht
	für \textbf{W}all \textbf{A}ttached
	\textbf{R}echarge \textbf{P}oint. Mit dem WARP Energy Manager
	erhältst du unseren Energiemanager zur Schaltschrankmontage, mit dem du die
	Energieverbrauch zu Hause überwachen und steuern kannst. 

	In Verbindung mit unseren WARP Charger Wallboxen kannst du das Laden von
	Elektrofahrzeugen abhängig von deinem Strombezug oder -Einspeisung steuern.
	Unter anderem ist damit ein PV-Überschussladen oder ein dynamisches
	Lastmanagement möglich.

	\subsection{Features}
	\vspace{-0.1cm}
	Der WARP Energy Manager misst über einen dreiphasigen bidirektionalen 
	Stromzähler laufend die Leistung am Stromnetzanschluss (z.B. Hausanschluss).
	Es werden verschiedene Zählertypen und Anschlussarten unterstützt.

	\subsubsection{Energiemonitoring}
	Die Messwerte vom Stromzähler stellt dir der WARP Energy Manager in seinem
	Webinterface dar. Er zeigt dir an, wie groß die Leistung ist, die aus dem Stromnetz
	bezogen wird. Besitzt du eine Photovoltaik-Anlage kann es sein, dass du
	keine Leistung aus dem Netz beziehst, sondern Leistung einspeist. 
	Wie groß diese ist wird dir dann natürlich ebenfalls dargestellt. Die Werte
	werden dir live auf dem Webinterface dargestellt.

	Im fünf Minutentakt werden die Messwerte lokal auf dem 
	Energiemanager gespeichert. Damit iser der WARP Energy Manager unabhängig 
	von irgendwelchen Datenaufzeichnungen auf Cloud-Servern.

	Für jeden Tag kannst du dir den Verlauf deines Strombezugs oder die
	Einspeisung in einem Graphen anzeigen lassen.

	Zusätzlich werden auf Tagesebene dein Energiebezug und -einspeisung
	aufgezeichnet. Damit kannst du deinen Energieverbrauch auf Tages-, Monats- und
	Jahresebene analysieren.

	\subsubsection{Steuerung von Wallboxen}
	Verfügst du über eine WARP Charger Wallbox, so kann der WARP Energy Manager
	diese verbrauchsabhängig steuern. Bis zu XXX Wallboxen vom Typ WARP Charger Smart, 
	WARP Charger Pro, WARP2 Charger Smart und WARP2 Charger Pro werden unterstützt. Die
	Steuerung erfolgt über ein gemeinsames Netzwerk (LAN, WLAN) in dem die
	Wallboxen und der WARP Energy Manager sich befinden.

	\hint{WARP Charger Pro (1. Generation) Wallboxen unterstützen leider nicht die
	Phasenumschaltung, da der verbaute Stromzähler keinen einphasigen Betrieb
	unterstützt.}

	Du kannst verschiedene Einstellungen vornehmen, mit denen du definieren
	kannst unter welchen Bedingungen ein Fahrzeug geladen wird.

	\subsubsection{Phasenumschaltung}
	Mittels eines externen Schützes kann der WARP Energy Manager
	angeschlossene Wallboxen zwischen einem 1- und 3-phasigem Betrieb
	umschalten. 
	Dies hat den Vorteil, dass die minimale Ladeleistung von ca.
	\SI{4.1}{\kilo\watt}~bei einem dreiphasigem Betrieb (minimaler Ladestrom
	\SI{6}{\ampere}) auf ca. \SI{1.4}{\kilo\watt}~reduziert werden kann. Somit
	kann auch ein geringer Leistungsüberschuss in ein Fahrzeug geladen werden,
	anstatt dass dieser entweder ins Netz eingespeist wird, oder aber
	zusätzliche Leistung aus dem Netz bezogen werden muss.

	\subsubsection{Eingänge für potentialfreie Kontakte}
	Der WARP Energy Manager verfügt über zwei Eingänge für potentialfreie
	Schaltkontakte. Wird die Phasenumschaltung genutzt, so wird einer dieser
	Eingänge fest zur Schützüberwachung verwendet. Ansonsten kann das Verhalten
	des Energiemanagers auf die Eingänge konfiguriert werden. Es kann zum
	Beispiel darüber eine generelle Ladefreigabe realisiert werden, oder der
	Ladestrom der Wallboxen begrenzt werden.

	\subsubsection{Potentialfreier Relaisausgang}
	Ein Relaisschaltausgang (potentialfrei) auf dem WARP Energy Manager kann
	genutzt werden um externe Verbraucher o.ä. zu schalten. Der Ausgang kann
	konfiguriert werden und zum Beispiel abhängig von der verfügbaren Leistung,
	des momentanen Netzbezuges oder einer erfolgten Phasenumschaltung geschaltet
	werden.

	\textbf{Achtung mit dem Relais kann keine Netzspannung (230V) geschaltet
	werden. Es können bis zu 30V/1A geschaltet werden.}

	\section{Typische Anwendungen}

	\subsubsection{PV-Überschussladen}
	Besitzt du eine Photovoltaik-Anlage, so möchtest du vermutlich einen
	möglichst hohen Anteil deines produzierten Stroms selbst nutzen. Abhängig
	von deiner Konfiguration des WARP Energy Managers kannst du ein reines
	PV-Überschussladen realisieren, bei dem nur überschüssige Energie ins
	Fahrzeug geladen wird. Du kannst aber auch einen erlaubten anteiligen Netzbezug
	definieren. Dies ist dann sinnvoll, wenn die selbst produzierte Leistung
	nicht ausreichen würde um einen Ladevorgang zu beginnen, du aber dennoch
	laden möchtest.

	Todo hier beschrieben, Leistungsregelung, Saldierender Stromzähler

	\subsubsection{Statisches Lastmanagement}

	Todo hier beschrieben, Stromregelung

	\subsubsection{Dynamisches Lastmanagement}
	In manchen Situationen ist es notwendig ein Lastmanagement zu realisieren. 
	Als Beispiel kommt es bei Mietobjekten oftmals vor, dass der
	Stromnetzanschluss der Immobilie es nicht zulässt, dass mehrere Wallboxen
	zu jeder Zeit parallel betrieben werden können. Die Absicherung des
	Stromnetzanschlusses lässt nur einen bestimmten Phasenstrom zu.

	Im einfachsten Fall kann definiert werden, dass für alle Wallboxen ein
	gewisser Phasenstrom immer garantiert werden kann. In diesem Fall können
	die Wallboxen untereinander ein statisches Lastmanagement durchführen, bei
	dem der eingestellte Phasenstrom einfach zwischen allen WARP Chargern
	aufgeteilt wird.

	Oftmals kann aber ein bestimmter Phasenstrom nicht zu jeder Zeit garantiert
	werden, da sich die Wallboxen einen Stromnetzanschluss mit anderen
	Verbrauchern teilen. Da diese Verbraucher ein- und ausgeschaltet werden
	können ändert sich der für die Wallboxen zur Verfügung stehende Phasenstrom
	permanent. In diesem Fall ist ein dynamisches Lastmanagement notwendig um
	sicherzustellen, dass der maximale Phasenstrom nicht überschritten wird und
	keine Sicherung auslöst. 

	Der WARP Energy Manager kann ein dynamisches Lastmanagement durchführen,
	wenn ein Stromzähler am Stromnetzanschluss vorhanden ist, der vom Energy Manager
	ausgewertet werden kann. Dabei regelt er die Leistung der Wallboxen in Abhängigkeit 
	zu dem noch zur Verfügung stehenden Phasenstrom vom Netzanschluss. Er stellt somit sicher,
	dass die Sicherungen nicht auslösen. Ist eine Photovoltaik-Anlage vorhanden
	und produziert diese, so erhöht diese ganz automatisch die zur Verfügung stehende
	Leistung für den Energy Manager.

	Todo hier beschrieben, Stromregelung

	\newpage
	\section{Sicherheitshinweise}
	Der WARP Energy Manager ist so konstruiert, dass ein sicherer Betrieb gewährleistet ist,
	wenn er korrekt installiert wurde, in einem einwandfreien technischen Zustand
	ist und diese Betriebsanleitung befolgt wird. \hint{Der WARP Energy Manager darf nur von einer ausgewiesenen Elektrofachkraft installiert
		werden.}

	\subsection{Bestimmungsgemäße Verwendung}
	Mit dem WARP Energy Manager kann in Verbindung mit einem externen
	Stromzähler ein Energie-Monitoring realisiert werden. In Verbindung mit WARP
	Charger Wallboxen kann somit eine leistungsbezogene Ladevorgangsteuerung von
	Elektrofahrzeugen realisiert werden. Für andere Anwendungen ist der
	Energiemanagernicht nicht geeignet. Eine Verwendung
	an Orten, an denen explosionsfähige oder brennbare Substanzen lagern, ist nicht
	zulässig. Jegliche Modifikation des Managers oder unsachgemäßer Betrieb ist verboten. 
	Der Energy Manager ist in einem geeigneten Verteilerschrank zu installieren
	und vor Beschädigungen, Feuchtigkeit/Verschmutzungen und unsachgemäßigem
	Zugriff zu 	schützen. Er darf nicht genutzt werden, wenn kein sicherer Betrieb
	gewährleistet werden kann.

	\subsection{Gerätestörung / Technischer Defekt}
	Sollte es Anzeichen für einen technischen Defekt geben, ist sofort die
	Stromversorgung des Energiemanagers zu trennen und gegen erneutes Einschalten zu
	sichern. Danach ist eine Elektrofachkraft zu informieren.

	\newpage
	\section{Montage und Installation}
	\subsection{Montage}
	\subsubsection{Lieferumfang}
	Im Lieferumfang des WARP Energy Managers befinden sich:
	\begin{itemize}
		\item WARP Energy Manager (Hutschienenmodul)
		\item gesteckte Schraubklemmen
		\begin{itemize}
			\item 2 pol 5mm Schraubklemme (230V Stromversorgung (L+N))
			\item 2 pol 5mm Schraubklemme (Schütz)
			\item 4 pol 3.5mm Schraubklemme (Eingänge)
			\item 2 pol 3.5mm Schraubklemme (Relaisausgang)
			\item 4 pol 3.5mm Schraubklemme (RS485 Modbus-RTU)
		\end{itemize}
		\item DIN A4 Umschlag mit:
		\begin{itemize}
			\item Dieser Betriebsanleitung inkl. WLAN Zugangsdaten
			\item RJ45 LAN-Winkeladapter
		\end{itemize}
	\end{itemize}

	\subsubsection{Montageort}
	Der WARP Energy Manager darf nur in einem geeigneten Verteilerschrank im
	Innenbereich installiert werden. Er ist vor Staub, Nässe und unsachgemäßigem
	Zugriff zu schützen. Es sollte
	eine LAN-Verbindung zum WARP Energy Manager gelegt werden, da in vielen
	Fällen eine Anbindung des WARP Energy Managers mittels WLAN nicht stabil
	möglich ist (Metallabschirmung der Verteilung).

	Es muss ausreichend Platz vorhanden sein. Es darf kein Druck auf die Kabel
	ausgeübt werden, insbesondere nicht auf die LAN Verbindung. Aus diesem Grund
	empfehlen wir die Verwendung des mitgelieferten LAN-Winkeladapters.

	\subsubsection{Montage}
	Zur Montage des WARP Energy Managers muss dieser auf die Hutschiene
	gesetzt werden. Das Gehäuse muss so installiert werden, dass die Anschlüsse
	nach unten zeigen.

	Zuerst wird die obere Halterung auf die Hutschiene aufgesetzt werden und anschließend
	die Untere. Er sollte selbstständig verriegeln. Falls nicht kann mit einem
	Schraubendreher an der schwarzen Verriegelung auf der Unterseite
	nachgeholfen werden.
	\par
	Soll der WARP Energy Manager wieder von der Hutschiene entfernt werden, so
	müssen zuerst alle Zuleitungen entfernt werden (\textbf{Achtung: Spannungsfreiheit
	sicherstellen!}). Anschließend kann mittels Schlitz-Schraubendreher die schwarze
	Federverriegelung gezogen werden und der Energy Manager von der Hutschiene
	gehoben werden. Dabei wird zuerst die untere Halterung angehoben,
	anschließend die Obere.
	\newpage
	\subsection{Elektrischer Anschluss}
	\hint{Die in diesem Kapitel beschriebenen Arbeiten dürfen nur von einer ausgewiesenen
		Elektrofachkraft durchgeführt werden.}

	\gfx{./img/wem_connections.jpg}


	\subsubsection{230V Stromversorgung}
	Nachdem der WARP Energy Manager montiert wurde, kann dieser nun angeschlossen werden.
	Die Schraubklemmen sind steckbar, so dass der elektrische Anschluss
	außerhalb erfolgen kann. Anschließend können die Schraubklemmen wieder in
	den WARP Energy Manager gesteckt werden.

	Die Stromversorgung des WARP Energy Managers erfolgt über eine 2-polige
	Schraubklemme (\textbf{L}+\textbf{N}). Die Zuleitung ist mit einem max. 16A Leitungsschutzschalter mit
	B-Charakteristik abzusichern.

	Die Stromversorgung des Energy Managers ist zusätzlich intern über eine Glassicherung 
	abgesichert.

	\subsubsection{Schütz zur Phasenumschaltung}
	\hint{Es muss kein Schütz installiert werden. Dieser Schritt ist optional}

	Ein externes Schütz kann zur Phasenumschaltung, das heißt der Umschaltung
	zwischen 1-phasiger und 3-phasiger Fahrzeugladung, installiert werden. Das
	Schütz wird mittels 230V Schaltausgang vom WARP Energy Manager gesteuert
	(\textbf{Lsw}).
	Der minimale Phasenstrom für das Typ2 Laden beträgt $\SI{6}{\ampere}$. Somit
	kann die Minimale Ladeleistung von $\SI{4.1}{\kilo\watt}$ auf
	$\SI{1.4}{\kilo\watt}$ reduziert werden.

	Zu Ansteuerung wird \textbf{N} und \textbf{Lsw} nach außen geführt. Der
	\textbf{Lsw}-Schaltausgang ist intern über eine Glassicherung abgesichert.


	\subsubsection{Eingänge}
	Der WARP Energy Manager besitzt zwei Eingänge für potentialfreie Kontakte.
	An diesen können Schließer/Öffner angeschlossen werden. Das Verhalten des
	Energy Managers auf diese Eingänge kann im Webinterface konfiguriert werden.
	
	Wird ein Schütz zur Phasenumschaltung installiert, so ist der Eingang
	\textbf{3}~fest zur
	Schützüberwachung konfiguriert. Es muss ein Schließer, welcher vom
	dem zu überwachenden Schütz geschaltet wird, zwischen \textbf{12V}~und
	\textbf{3}~installiert werden.

	Wird kein Schütz zur Phasenumschaltung verwendet, kann Eingang \textbf{3}~für
	andere Zwecke verwendet werden (konfigurierbar). Eingang \textbf{4}~ steht immer
	für eigene Zwecke zur Verfügung. Die Eingänge sind so ausgelegt, dass ein potentialfreier
	Kontakt extern angeschlossen werden kann (Schalter als Öffner/Schließer, Relais etc.).
	Die \textbf{12V} Anschlüsse der Eingänge sind hochohmig ausgelegt, liefern
	keine Leistung und sind daher nicht zur Stromversorgung anderer Verbraucher
	geeignet.

	\subsubsection{RS485 Modbus Stromzähler}
	\hint{Es muss kein RS485 Modbus Stromzähler installiert werden. Dieser
	Schritt ist optional, wenn ein anderer unterstützer Stromzähler konfiguriert
	wird.}

	Der WARP Energy Manager benötigt einen Stromzähler um den Leistungsbezug regeln zu 
	können. Eine Möglichkeit dafür ist die Installation eines RS485 Modbus
	Stromzählers vom Typ Eastron SDM72DM, SDM630MCT oder SDM630Modbus.
	
	Die Steckerbelegung ist \textbf{12V, A, B, GND}. Der Anschluss \textbf{12V}
	darf nicht belegt werden. \textbf{A (+), B (-), GND} sind entsprechend 
	am jeweiligen Stromzähler anzuschließen.

	\section{Erste Schritte}

	Nach der elektrischen Installation kann der WARP Energy Manager konfiguriert
	werden. Dazu muss zuerst eine Verbindung zum Energy Manager hergestellt werden, 
	damit diese dann über den Browser konfiguriert werden kann.

	\subsection{Schritt 1: Verbindung herstellen}


	\hint{Wir empfehlen unbedingt eine Anbindung des WARP Energy Managers per
	LAN. Auch wenn technisch eine Anbindung mittels WLAN möglich ist, so muss
	sichergestellt werden, dass diese Verbindung dauerhaft stabil ist. Gerade in
	Schaltschränken gestaltet sich dies meist schwierig.}

	\paragraph{Option 1: WLAN}

	Im Werkszustand öffnet der WARP Energy Manager einen WLAN-Access-Point. Über diesen kann
	die Konfiguration vorgenommen werden, indem auf das das Webinterface des
	Energy Managers zugegriffen wird.

	Die Zugangsdaten des Access-Points findest du auf dem WLAN-Zugangsdaten-Aufkleber
	auf der Rückseite dieser Anleitung. Du kannst entweder den QR-Code des Aufklebers verwenden,
	der das WLAN automatisch konfiguriert, oder SSID und Passphrase abschreiben.
	Die meisten Kamera-Apps von Smartphones unterstützen das Auslesen des
	QR-Codes und das automatische Verbinden zu dem WLAN. Somit musst du die
	Zugangsdaten dann nicht abtippen. Wichtig ist, dass viele Smartphones
	erkennen, dass über das WLAN des Energy Managers (Access-Point) kein Zugriff auf das
	Internet möglich ist. Dein Telefon fragt dann nach, ob du zu dem WLAN
	verbunden bleiben möchtest. Damit du weiter auf den Energy Manager zugreifen
	kannst, darfst du das WLAN nicht wieder verlassen.

	\begin{minipage}{0.35\textwidth}
		Wenn die Verbindung mit dem Access-Point des Energy Managers hergestellt ist, kannst du das Webinterface
		unter \url{http://10.0.0.1} über einen Browser deiner Wahl erreichen.
		Alternativ kannst du dazu den nebenstehenden QR-Code scannen.
		Eventuell musst du deine mobile Datenverbindung (z.B. LTE) deaktivieren.
	\end{minipage}\hfill
	\begin{minipage}{0.12\textwidth}
		\begin{flushright}
			\qrcode{http://10.0.0.1}
		\end{flushright}
	\end{minipage}

	\paragraph{Option 2: LAN}
	Als Alternative zum Zugriff über den WLAN-Accesspoint verbindet sich der
	Energy Manager in den Werkseinstellungen automatisch zu einem
	kabelgebundenen Netzwerk (LAN), wenn ein LAN-Kabel eingesteckt ist (IP Bezug
	mittels DHCP). Der Energy Manager kann dann entweder über die zugewiesene IP
	Adresse (\url{http://[IP-des-Energy-Managers]}, z.B. \url{http://192.168.0.42})
	oder den Hostnamen (\url{http://[hostname]}, z.B. \url{http://warp2-ABC}) erreicht werden.

	Der Hostname des Energy Managers ist identisch zur SSID des WLANs. Den Hostnamen findest du
	auf dem WLAN-Zugangsdaten-Aufkleber auf der Rückseite dieser Anleitung.

	Kann die per DHCP vergebene IP des Energy Managers nicht ermittelt werden, so kann der
	zuvor genannte Zugriff auf den Energy Manager mittels WLAN-Access-Point genutzt
	werden um die IP Adresse der LAN Schnittstelle zu ermitteln (\glqq
	Status-Seite\grqq, Abschnitt \glqq LAN-Verbindung\grqq).


	\subsection{Schritt 2: Konfiguration mittels Webinterface}
	Generell empfehlen wir nach der Installation ein Update der Firmware des
	Energy Managers. Somit erhältst du die neusten Funktionen und ggf. Bugfixes. Wie ein
	Firmware-Update durchgeführt wird, ist unter \fullref{firmware-update}
	beschrieben.

	Anschließend kann der WARP Energy Manager über das Webinterface konfiguriert
	werden. Die Einstellungen etc. hängen vom Anwendungsfall ab. 
	Das Webinterface ist unter \fullref{webinterface} vollständig beschrieben.

	\section{Webinterface}

	Über das Webinterface kannst du den Energieverbrauch, Überwachen und 
	unter anderem das Laden der kontrollierten Wallboxen steuern und überwachen.
	Es können diverse Einstellungen vorgenommen werden, die nachfolgend
	dokumentiert sind.

	Wenn du auf das Webinterface der Wallbox mit einem Browser zugreifst
	gelangst du auf die Start-/ Statusseite. Auf der linken Seite befindet sich
	die Menüleiste, über die du zu weiteren Einstellungsmöglichkeiten kommst.

	Auf mobilen Endgeräten wird
	diese Menüleiste stattdessen versteckt unter einem Menü-Symbol oben rechts
	im grauen Balken neben dem WARP Logo angezeigt (\glqq drei Striche untereinander\grqq).
	Hier kannst du das Menü durch einen Klick auf das Symbol ausklappen.

	\vspace{-0.2cm}
	\subsection{Status (Startseite)}
	Die Startseite des Webinterfaces bietet Einstellmöglichkeiten und zeigt
	Statusinformationen.

	Mittels Schaltflächen kann zwischen verschiedenen Lademodi der angeschlossenen 
	Wallboxen gewechselt werden:
	\begin{description}
	\item[\textbf{PV}-Überschussladen] \glqq 100\% Eigener Strom\grqq. Ob ein
	Ladevorgang startet ist davon abhängig, ob die Minimale Ladeleistung
	als Überschuss zur Verfügung steht. Ist dies nicht der Fall, so
	wird keine Ladung gestartet.
	\item[\textbf{Min+PV}-Laden] Es wird sicher mit einer Ladung begonnen, indem 
	die minimale Ladeleistung sichergestellt wird. Zur Not erfolgt diese als
	Netzbezug. Wird genügend Leistung produziert (Netzeinspeisung), so wird
	der Ladestrom soweit erhöht bis keine Einspeisung ins Stromnetz mehr
	erfolgt, oder aber die maximale Ladeleistung erreicht wird.
	\item[\textbf{Schnell}-Laden] Alle Wallboxen laden mit der maximal möglichen
	Ladeleistung ohne Beachtung einer Netzeinspeisung bzw. eines Netzbezugs.
	\item[\textbf{Aus}] Die kontrollierten Wallboxen sind deaktiviert. Es kann
	nicht geladen werden.
	\end{description}
	Die PV-Optionen sind nur verfügbar, wenn PV-Überschussladen aktiviert wurde.

	Als nächstes wird eine Übersicht der \textbf{kontrollierten Wallboxen} und deren
	Zustand angezeigt. Zu diesem zählt unter anderem der zugeordnete Ladestrom.

	Am Ende wird der Zustand der verwendeten Schnittstellen angezeigt.

	%\gfx{./img_warp2/resized/web_status}


	\subsection{Energiebilanz}
	\subsection{Energiemanager}
	\subsubsection{Einstellungen}
	Alle Einstellungen bezüglich des Energiemanagements werden hier vorgenommen.

	Als erstes muss der \textbf{Standard-Lademodus} definiert werden. Die
	verschiedenen Modi wurden bereits hier TODO erklärt. Wird die Einstellung
	auf der Statusseite geändert, so ist diese Einstellung permanent.
	Mittels \textbf{Täglich zurücksetzen} kann die Einstellung aber auch 
	automatisch täglich wieder auf den Standard-Lademodus zurückgesetzt werden.

	Unter dem Abschnitt \textbf{Dynamisches Lastmanagement} werden zukünftig
	alle Einstellungen zum dynamischen Lastmanagement zu finden sein. Diese
	Funktion wird mittels Firmware-Update zur Verfügung gestellt.

	Hierbei misst der WARP Energy Manager laufend mittels eines Stromzählers die
	Ströme aller Phasen am Stromnetzanschluss. Der noch rechnerisch zur
	Verfügung stehende Strom kann für jede Phase unterschiedlich sein und ändert
	sich laufend auf Grund des Zu- und Abschaltens von Verbrauchern. Aber auch eine
	parallel angeschlosse PV-Anlage beeinflusst die Phasenströme. Der WARP
	Energy Manager kann somit laufend den noch rechnisch zur Verfügung stehenden
	Phasenstrom ermitteln und diesen den angeschlossenen Wallboxen bereit
	stellen. Dabei wird sicher gestellt, dass der Strom keiner Phase
	überschritten wird und keine Sicherung ausgelöst wird.


	\subsubsection{Stromzähler}
	\subsubsection{Wallboxen}
	Hier werden die vom Energy Manager kontrollierten Wallboxen konfiguriert.
	Die hier vorgenommenen Einstellungen beeinflussen das Lastmanagement
	zwischen den Wallboxen.

	Typ2 Wallboxen kommunizieren den angeschlossenen Fahrzeugen den maximal zur
	Verfügung stehenden Ladestrom. Das Fahrzeug entscheidet ob dieser Ladestrom
	voll ausgenutzt wird und ob eine Ladung 1- / 2- oder 3-phasig durchgeführt
	wird.

	Als erste Einstellung muss mittels \textbf{Maximaler Gesamtstrom} der 
	zulässige Maximalstrom der Zuleitung zu den Wallboxen konfiguriert werden. 
	Der Energy Manager stellt sicher, dass dieser
	Strom auf keiner Phase überschritten wird, indem in Summe niemals mehr als
	dieser Strom an die Wallboxen verteilt wird. Besitzen alle Wallboxen
	ausreichend dimensionierte getrennte Zuleitungen kann dieser Strom auch so
	hoch eingestellt werden, dass alle Wallboxen sicher ihren Maximalstrom
	erhalten. Damit wird diese Begrenzung außer Kraft gesetzt. Alle andere
	Komponenten, wie zum Beispiel der Netzanschluss, müssen dann aber den
	angeforderten Storm liefern können. 

	\hint{Hierbei handelt es sich um ein statisches Lastmanagement, bei dem
	davon ausgegangen wird, dass der eingestellte Strom auf jeder Phase
	zu jeder Zeit zur Verfügung steht. Andere Verbraucher als WARP Charger, 
	welche vom Energy Manager nicht gesteuert werden können, werden nicht
	berücksichtigt!}

	Der individuelle Maximalstrom jeder Wallbox bleibt hiervon unberührt
	(Zuleitung der Wallbox - Schiebeschaltereinstellung innerhalb der Wallbox).

	Eine typische Anwendung hierfür ist, die Strombegrenzung, 
	wenn die Wallboxen über eine gemeinsame Zuleitung verfügen.

	Mit der Einstellung \textbf{Minimaler Ladestrom} kann der minimale Ladestrom
	angehoben werden. Der Typ2-Ladestandard setzt als Minimum 6A voraus. Eine
	Einstellung darunter ist nicht möglich. Allerdings gibt es Fahrzeuge, welche 
	bei 6A nicht mit einer Ladung beginnen. Falls notwendig kann hier ein höherer 
	Ladestrom definiert werden. In den allermeisten Fällen kann die Einstellung
	bei 6A belassen werden.

	Am Ende der Seite werden die \textbf{Kontrollierte
	Wallboxen} dargestellt. Weitere Wallboxen können mittels Klick auf
	\textbf{Wallbox hinzufügen} der Steuerung durch den WARP Energy Manager
	hinzugefügt werden. Dazu muss der Anzeigename und die IP-Adresse oder der
	Hostname der Wallbox eingetragen werden und mittels Klick auf \glqq
	hinzufügen\grqq~übernommen werden.

	Automatisch ermittelte Wallboxen, die noch nicht dem Energy Manager
	angehören, werden als Liste dargestellt. Die Einstellungen dazu können
	mittels Klick übernommen werden und anschließend ebenfalls mittels \glqq
	hinzufügen\grqq~übernommen werden.

	Alle geänderte Einstellungen müssen mittels \glqq Speichern\grqq übernommen 
	werden.


	\subsection{Netzwerk}
	\subsection{MQTT}
	\subsection{System}


	\section{Konfiguration Energiemanagement}

	\section{Schnittstellen zur Fernsteuerung}

	\section{Fehlerbehebung}

	\subsection{Sicherungswechsel}
	Der WARP Energy Manager ist intern über zwei 5$\times\SI{20}{\milli\meter}$ Feinsicherungen (mittelträge (m), \SI{500}{\milli\ampere}) abgesichert.
	Tinkerforge verbaut Sicherungen vom Typ \enquote{ESKA 521.014}. Die eine
	Sicherung befindet sich im Eingangspfad der 230V Stromversorgung (L). Die
	andere Sicherung befindet sich im Schaltausgang der Schützsteuerung.

	\section{Konformitätserklärung}
	Die EU-Konformitätserklärung zum WARP Energy Manager ist in einem gesonderten Dokument verfügbar.

	\section{Entsorgung}
	\begin{minipage}{0.43\textwidth}
		WARP Energy Manager und Verpackung sind bei Gebrauchsende ordnungsgemäß zu
		entsorgen. Altgeräte dürfen nicht über den Hausmüll entsorgt werden.
	\end{minipage}\hfill
	\begin{minipage}{0.045\textwidth}
		\includegraphics[width=\linewidth]{./img/resized/weee.pdf}
	\end{minipage}

	\section{Technische Daten}

	%use minipage here to control footnote placement
	\begin{minipage}{\linewidth}

		\begin{description}[leftmargin=!,labelwidth=\widthof{\textbf{Fehlerstromerkennung}}]
			\setlength{\itemsep}{3pt}
			\item[Ladestandard] DIN EN 61851‐1
			\item[Ladeleistung] einstellbar
			      bis \SI{11}{\kilo\watt} / \SI{22}{\kilo\watt}~\footnote[7]{\label{fn:1} je nach Variante}
			\item[Fahrzeugladestecker] Typ 2
			\item[Abmessungen] 280 × 215 × \SI{95}{\milli\meter} (B/H/T)
			\item[Nennspannung] \SI{230}{\volt} / \SI{400}{\volt} / 1/3
			      AC$\sim$~\footref{fn:1}
			\item[Nennfrequenz] \SI{50}{\hertz}
			\item[Nennstrom] \SI{16}{\ampere} / \SI{32}{\ampere}
			      \footref{fn:1}
			\item[Standby, WLAN an] Basic/Smart $\leq\SI{3}{\watt}$; Pro $\leq\SI{5}{\watt}$
			\item[Ladekabellänge] \SI{5}{\meter} / \SI{7,5}{\meter}~\footref{fn:1}
			\item[Zuleitungsquerschnitt] \SI{2,5}{\square\milli\meter} bis
			      \SI{10}{\square\milli\meter}
			\item[Zugangsverriegelung]
			      NFC~\footref{fn:1}\\Webinterface~\footref{fn:1}
			\item[Betriebstemperatur] \SI{-25}{\celsius}
			      bis \SI{+50}{\celsius} (Durchschnitt in \SI{24}{\hour}: $\leq \SI{35}{\celsius}$)
			\item[Fehlerstromerkennung] DC \SI{6}{\milli\ampere} (integriert)
			\item[Schutzart] IP54
			      (spritzwassergeschützt, für
			      den Außenbereich geeignet)
			\item[Lastmanagement] max. 10 Teilnehmer~\footref{fn:1}
			\item[NFC-Tags] 3 im Lieferumfang\\max. 16 anlernbar~\footref{fn:1}
			\item[Benutzer] max. 16 konfigurierbar~\footref{fn:1}
			\item[Schnittstellen] HTTP, MQTT, Modbus/TCP, OCPP~\footref{fn:1}
		\end{description}
	\end{minipage}

	\newpage

	\section{Kontakt}
	Tinkerforge GmbH\\ Zur Brinke 7\\ 33758 Schloß Holte-Stukenbrock
	\begin{description}[leftmargin=!,labelwidth=\widthof{\textbf{Website}}]
		\item[E-Mail] \href{mailto:info@tinkerforge.com}{\texttt{info@tinkerforge.com}}
		\item[Website] \href{https://warp-charger.com}{\texttt{warp-charger.com}}
		\item[Telefon] \phonenumber{052078998614}
		\item[Shop] \href{https://tinkerforge.com/de/shop/warp.html}{\texttt{tinkerforge.com/de/shop/warp.html}}
	\end{description}

	\section{Dokumentversionen}
	\begin{tabular}{lll}
		\toprule
		Datum      & Version & Kommentar                       \\
		\midrule
		24.01.2023 & 1.0     & Initialversion                  \\
		\bottomrule
	\end{tabular}

	\vfill
	\null

	\columnbreak
\appendix

\section{Modbus/TCP Registertabelle}
\label{modbus_tcp_registertabelle}
TODO
Nachfolgend die Registertabelle für Modbus/TCP für die Einstellung \textbf{WARP
Energy Manager}.

Input Registers können nur gelesen werden und liefern Informationen über den
Zustand der Wallbox. Gewisse Register sind nur verfügbar, wenn das dazu
angegebene \textbf{Feature} verfügbar ist. So sind zum Beispiel die
Informationen zur Ladeleistung, Energie usw. nur verfügbar, wenn die Wallbox
über ein \textbf{meter} verfügt. Das heißt ein WARP2 Charger Pro (Version mit
Stromzähler) liefert diese Werte. Ein WARP2 Charger Smart (Version ohne
Stromzähler) nicht.

Welche Features die Wallbox bietet kann über \textbf{Discrete Inputs} ausgelesen
werden. Eine Steuerung der Wallbox ist über die \textbf{Holding Registers}
möglich.

\end{multicols*}

\subsection{Input Registers}
\begin{tabularx}{\textwidth}{rXll} \toprule
    \textbf{Register-} & \textbf{Name}& \textbf{Typ} & \textbf{Benötigtes}                                                      \\
    \textbf{adresse}   &              &              & \textbf{Feature}                                                         \\ \midrule
0             & Version der Registertabelle             & uint32       & ---                                     \\
              & \tdesc{Aktuelle Version: 1}                                                                                     \\ \cmidrule{2-4}
2             & Firmware-Version                       & uint32 (x4)       & ---                                                    \\
              & \tdesc{Major, Minor, Patch, Zeitstempel jeweils uint32. Beispielsweise 2, 0, 8, 0x63218f23 für}                 \\
              & \tdesc{Firmware 2.0.8-63218f23. 0x63218f23 ist der Unix-Zeitstempel des 14. September 2022 08:21:55 UTC.}       \\ \cmidrule{2-4}
10            & Charger-ID                              & uint32       & ---                                                    \\
              & \tdesc{Dekodierte Form der Base58-UID, die für Standard-Hostnamen, -SSID usw. genutzt wird.}                    \\
              & \tdesc{Zum Beispiel 185460 für X8A}                                                                             \\ \cmidrule{2-4}
12            & Uptime (s)                              & uint32       & ---                                                    \\
              & \tdesc{Zeit in Sekunden seit dem Start der Wallbox-Firmware.}                                                   \\ \cmidrule{2-4}
1000          & IEC-61851-Zustand                       & uint32       & evse                                                   \\
              & \tdesc{0-A, 1-B, 2-C, 3-D, 4-E/F}                                                                               \\ \cmidrule{2-4}
1002          & Fahrzeugstatus                          & uint32       & evse                                                   \\
              & \tdesc{0-Nicht verbunden, 1-Warte auf Freigabe, 2-Ladebereit, 3-Lädt, 4-Fehler}                                 \\ \cmidrule{2-4}
1004          & User-ID                                 & uint32       & evse                                                   \\
              & \tdesc{ID des Benutzers der den Ladevorgang gestartet hat. 0 falls eine Freigabe}                               \\
              & \tdesc{ohne Nutzerzuordnung erfolgt ist. 0xFFFFFFFF falls gerade kein Ladevorgang läuft.}                       \\ \cmidrule{2-4}
1006          & Start-Zeitstempel (min)                 & uint32       & evse                                                   \\
              & \tdesc{Ein Unix-Timestamp in Minuten, der den Startzeitpunkt des Ladevorgangs angibt.}                          \\
              & \tdesc{0 falls zum Startzeitpunkt keine Zeitsynchronisierung verfügbar war.}                                    \\ \cmidrule{2-4}
1008          & Ladedauer (s)                           & uint32       & evse                                                   \\
              & \tdesc{Dauer des laufenden Ladevorgangs in Sekunden. Auch ohne Zeitsynchronisierung verfügbar}                  \\ \cmidrule{2-4}
1010          & Erlaubter Ladestrom                     & uint32       & evse                                                   \\
              & \tdesc{Maximal erlaubter Ladestrom, der dem Fahrzeug zur Verfügung gestellt wird.}                              \\
              & \tdesc{Dieser Strom ist das Minimum der Stromgrenzen aller Ladeslots.}                                          \\ \cmidrule{2-4}
1012          & Ladestromgrenzen (mA)                   & uint32 (x20) & evse                                                   \\
              & \tdesc{Aktueller Wert der Ladestromgrenzen in Milliampere. 0xFFFFFFFF falls eine Stromgrenze nicht aktiv ist.}  \\
              & \tdesc{0 falls eine Stromgrenze blockiert. Sonst zwischen 6000 (6A) und 32000 (32A).}                           \\ \cmidrule{2-4}
2000          & Stromzählertyp                          & uint32       & meter                                                  \\
              & \tdesc{0-Kein Stromzähler verfügbar, 1-SDM72 (nur WARP1), 2-SDM630, 3-SDM72 V2}                                 \\ \cmidrule{2-4}
2002          & Ladeleistung (W)                        & float        & meter                                                  \\
              & \tdesc{Die aktuelle Ladeleistung in Watt}                                                                       \\ \cmidrule{2-4}
2004          & absolute Energie (kWh)                  & float        & meter                                                  \\
              & \tdesc{Die geladene Energie seit der Herstellung des Stromzählers.}                                             \\ \cmidrule{2-4}
2006          & relative Energie (kWh)                  & float        & meter                                                  \\
              & \tdesc{Die geladene Energie seit dem letzten Reset. (siehe Holding Register 2000)}                              \\ \cmidrule{2-4}
2008          & Energie des Ladevorgangs                & float        & meter                                                  \\
              & \tdesc{Die während des laufenden Ladevorgangs geladene Energie}                                                 \\ \cmidrule{2-4}
2100          & weitere Stromzähler-Werte               & float (x85)  & all\_values                                            \\
              & \tdesc{Siehe \rurl{https://www.warp-charger.com/api.html\#meter\_all\_values}{warp-charger.com/api.html\#meter\_all\_values}} \\ \cmidrule{2-4}
3000          & CP-Unterbrechung                        & uint32       & cp\_disc                                               \\
              & \tdesc{Noch nicht implementiert!}                                                                               \\ \bottomrule
\end{tabularx}

\subsection{Holding Registers}
\begin{tabularx}{\textwidth}{rXll} \toprule
    \textbf{Register-} & \textbf{Name} & \textbf{Typ} & \textbf{Benötigtes}                                                     \\
    \textbf{adresse}   &      &     & \textbf{Feature}                                                                          \\ \midrule
0             & Reboot                                  & uint32       & ---                                                    \\
              & \tdesc{Startet die Wallbox (bzw. den ESP-Brick) neu, um beispielsweise Konfigurationsänderungen anzuwenden.}    \\
              & \tdesc{Password 0x012EB007}                                                                                     \\ \cmidrule{2-4}
1000          & Ladefreigabe                            & uint32       & evse                                                   \\
              & \tdesc{0 zum Blockieren des Ladevorgangs; ein Wert != 0 zum Freigeben}                                          \\ \cmidrule{2-4}
1002          & Erlaubter Ladestrom (mA)                & uint32       & evse                                                   \\
              & \tdesc{0mA oder 6000mA bis 32000 mA. Andere Ladestromgrenzen können den Strom weiter verringern!}               \\ \cmidrule{2-4}
2000          & Relative Energie zurücksetzen           & uint32       & meter                                                  \\
              & \tdesc{Setzt den relativen Energiewert zurück (Input Register 2006). Password 0x3E12E5E7}                       \\ \cmidrule{2-4}
3000          & CP-Trennung auslösen                    & uint32       & cp\_disc                                               \\
              & \tdesc{Noch nicht implementiert!}                                                                               \\ \bottomrule
\end{tabularx}

\subsection{Discrete Inputs}
\begin{tabularx}{\textwidth}{rXll} \toprule
    \textbf{Register-} & \textbf{Name} & \textbf{Typ} & \textbf{Benötigtes}                                                     \\
    \textbf{adresse}   &      &     & \textbf{Feature}                                                                          \\ \midrule
0             & Feature \enquote{evse} verfügbar        & bool         & ---                                                    \\
              & \tdesc{Ein Ladecontroller steht zur Verfügung. Dieses Feature sollte bei allen WARP Chargern,}                  \\
              & \tdesc{deren Hardware funktionsfähig ist, vorhanden sein.}                                                      \\ \cmidrule{2-4}
1             & Feature \enquote{meter} verfügbar       & bool         & ---                                                    \\
              & \tdesc{Ein Stromzähler und Hardware zum Auslesen desselben über RS485 ist verfügbar. Dieses Feature wird }      \\
              & \tdesc{erst gesetzt, wenn ein Stromzähler mindestens einmal erfolgreich über Modbus ausgelesen wurde.}          \\ \cmidrule{2-4}
2             & Feature \enquote{phases} verfügbar      & bool         & ---                                                    \\
              & \tdesc{Der verbaute Stromzähler kann Energie und weitere Messwerte einzelner Phasen messen.}                    \\ \cmidrule{2-4}
3             & Feature \enquote{all\_values} verfügbar & bool         & ---                                                    \\
              & \tdesc{Der verbaute Stromzähler kann weitere Messwerte auslesen.}                                               \\ \cmidrule{2-4}
4             & Feature \enquote{cp\_disc} verfügbar    & bool         & ---                                                    \\
              & \tdesc{Noch nicht implementiert!}                                                                               \\ \cmidrule{2-4}
2100          & Phase L1 angeschlossen                  & bool         & phases                                                 \\
              & \tdesc{}                                                                                                        \\ \cmidrule{2-4}
2101          & Phase L2 angeschlossen                  & bool         & phases                                                 \\
              & \tdesc{}                                                                                                        \\ \cmidrule{2-4}
2102          & Phase L3 angeschlossen                  & bool         & phases                                                 \\
              & \tdesc{}                                                                                                        \\ \cmidrule{2-4}
2103          & Phase L1 aktiv                          & bool         & phases                                                 \\
              & \tdesc{}                                                                                                        \\ \cmidrule{2-4}
2104          & Phase L2 aktiv                          & bool         & phases                                                 \\
              & \tdesc{}                                                                                                        \\ \cmidrule{2-4}
2105          & Phase L3 aktiv                          & bool         & phases                                                 \\
              & \tdesc{}                                                                                                        \\ \bottomrule
   \end{tabularx}
	\begin{multicols*}{2}




	\newpage
	\pagestyle{empty}
	\null
	\vfill
	WLAN-Zugangsdaten
	\begin{tcolorbox}[width=4.2cm,height=2.7cm, boxrule=0.25mm]

	\end{tcolorbox}
	Dieser Aufkleber befindet sich\\ auch unter dem Deckel des WARP Energy
	Managers.
	\columnbreak

	\null
	\vfill
	Typenschild
	\begin{tcolorbox}[width=7.8cm,height=4.1cm, boxrule=0.25mm]

	\end{tcolorbox}
	Dieser Aufkleber befindet sich auch an der Seite\\ des WARP Energy Managers.
\end{multicols*}
\end{document}
