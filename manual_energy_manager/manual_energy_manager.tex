\documentclass[a4paper,10pt]{article}
\ifdefined\forprint
    \usepackage[width=21.6cm,height=30.3cm,center]{crop}
\fi
\usepackage[utf8]{inputenc}
\usepackage[margin=2cm,headheight=26pt,includeheadfoot]{geometry}

\usepackage[german]{babel}
\usepackage[german=quotes]{csquotes}

\usepackage{nameref}
\usepackage{microtype}
\usepackage{float}
\usepackage{siunitx}
\sisetup{
    locale = DE,
    binary-units,
    detect-all,
    per-mode = symbol %enables m/s instead of ms^-1
}
\AtBeginDocument{\DeclareSIUnit{\kWh}{kWh}}

\usepackage{caption} %for \caption*
\usepackage{hhline}
\usepackage{tabularx}
\usepackage{array}
\usepackage{calc}
\usepackage{multicol}
\usepackage{multirow}
\usepackage{parskip}
\usepackage{booktabs}
\usepackage{textcomp}

\usepackage{fancyhdr}
\pagestyle{fancy}
\setlength{\headheight}{48pt}
\renewcommand{\headrulewidth}{0pt}

\usepackage{xcolor,colortbl}
\usepackage{makecell}

\usepackage[symbol*]{footmisc}
\renewcommand{\thefootnote}{\fnsymbol{footnote}}
\renewcommand{\thempfootnote}{\fnsymbol{mpfootnote}}

\usepackage{color}
\usepackage{enumitem}

\usepackage{pdfpages}

\usepackage{phonenumbers}

% Word-Stealth-Modus, kann aber kein Omega
%\usepackage[scaled]{helvet}

\usepackage[inline,nomargin]{fixme}
\fxsetup{
    author=,
    layout=inline,
    theme=color
}

\definecolor{fxnote}{rgb}{0.8000,0.0000,0.0000}
\colorlet{fxnotebg}{yellow}

\definecolor{boxgray}{rgb}{0.33,0.33,0.33}
\usepackage{tcolorbox}
\title{}
\author{}

\renewcommand{\familydefault}{\sfdefault}

\newcommand{\hint}[1]{\begin{tcolorbox}[colback=boxgray,colframe=black,coltext=
white,title=Hinweis,left*=2mm,right*=2mm,boxsep=1mm,bottom=1mm,top=1mm]#1\end{tcolorbox}}

\newcommand{\gfx}[1]{\includegraphics[width=\linewidth]{#1}}

\newcommand*{\fullref}[1]{\hyperref[{#1}]{\ref*{#1}~\nameref*{#1}}}

\fancyhf{}
\fancyhead{\colorbox{boxgray}{
    \makebox[\dimexpr\linewidth-2\fboxsep][l]{
        \includegraphics[height=1cm]{./img/resized/logo}\hfill\color{white}\Huge\raisebox{.5ex}{\thepage}
        }
    }
}

\usepackage{hyperref}
\usepackage{qrcode}

\appto\UrlNoBreaks{\do\.\do\:\do\/\do\_}

\newcommand\rurl[2]{%
  \href{#1}{\nolinkurl{#2}}%
}


\newcommand{\tdesc}[1]{\multicolumn{3}{l}{\footnotesize #1}}

\usepackage{hyphenat}

\hyphenation{Web-inter-face}

\begin{document}
\pagestyle{empty}
\begin{titlepage}
	\vspace*{-3.08cm}
	\colorbox{boxgray}{\makebox[\dimexpr\linewidth-2\fboxsep][c]{\includegraphics[width=0.6\textwidth]{./img/resized/logo}}}
	\vfill
	\begin{center}
		\Huge
		WARP Energy Manager Betriebsanleitung\\\vspace{1cm}
		\large
		Version 1.0.0\\\vspace{0.25cm}
		24.01.2023
	\end{center}
	\vfill \gfx{./img/resized/warp-energy-manager.png}
\end{titlepage}
\newpage
\null
\newpage
\pagestyle{fancy}
\begin{multicols*}{2}
	\tableofcontents
	\newpage
	\section{Einführung}
	\subsection{Vorwort} Vielen Dank, dass du
	dich für einen WARP Energy Manager von Tinkerforge entschieden hast!

	\enquote{WARP} steht
	für \textbf{W}all \textbf{A}ttached
	\textbf{R}echarge \textbf{P}oint. Mit dem WARP Energy Manager
	erhältst du unseren Energiemanager zur Schaltschrankmontage, mit dem du den
	Energieverbrauch zu Hause überwachen, steuern und optimieren kannst. 

	\gfx{./img/resized/warp-energy-manager.png}\vspace{-0.3cm}

	In Verbindung mit unseren WARP Charger Wallboxen kannst du das Laden von
	Elektrofahrzeugen abhängig von deinem Strombezug und deiner Stromeinspeisung steuern.
	Unter anderem ist damit ein PV-Überschussladen oder ein dynamisches
	Lastmanagement möglich.

	\textbf{Der WARP Energy Manager ist mit einer Basisfirmware veröffentlicht
	worden. Mittels kostenloser Firmwareupdates wird die Funktionalität Schritt
	für Schritt erweitert. Der hier dokumentierte Stand bezieht sich auf die
	Funktionen der Basisfirmware. Informationen zum dynamischen Lastmanagement
	werden gegeben auch wenn diese Funktion noch nicht in der Basisfirmware
	enthalten ist.}

	\subsection{Features}
	\vspace{-0.1cm}
	Der WARP Energy Manager kann über einen dreiphasigen bidirektionalen 
	Stromzähler laufend die Leistung am Stromnetzanschluss (z.B. Hausanschluss)
	messen. Es werden verschiedene Zählertypen und Anschlussarten unterstützt.

	\subsubsection{Energiemonitoring}
	Die Messwerte vom Stromzähler stellt dir der WARP Energy Manager in seinem
	Webinterface dar. Er zeigt dir an, wie groß die Leistung ist, die aus dem Stromnetz
	bezogen wird. Besitzt du eine Photovoltaik-Anlage kann es sein, dass du
	keine Leistung aus dem Netz beziehst, sondern Leistung einspeist. 
	Wie groß diese ist wird dir dann natürlich ebenfalls dargestellt. Die Werte
	werden dir live auf dem Webinterface dargestellt.

	Im fünf Minutentakt werden die Messwerte lokal auf dem 
	Energiemanager gespeichert. Damit ist der WARP Energy Manager unabhängig 
	von irgendwelchen Datenaufzeichnungen auf Cloud-Servern.

	Für jeden Tag kannst du dir den Verlauf deines Strombezugs oder die
	Einspeisung in einem Graphen anzeigen lassen.

	Zusätzlich werden auf Tagesebene dein Energiebezug und -einspeisung
	aufgezeichnet. Damit kannst du deinen Energieverbrauch auf Tages-, Monats- und
	Jahresbasis analysieren.

	\subsubsection{Steuerung von Wallboxen}
	Verfügst du über eine WARP Charger Wallbox, so kann der WARP Energy Manager
	diese verbrauchsabhängig steuern. Bis zu zehn Wallboxen vom Typ WARP Charger Smart, 
	WARP Charger Pro, WARP2 Charger Smart und WARP2 Charger Pro werden unterstützt. Die
	Steuerung erfolgt über ein gemeinsames Netzwerk (LAN, WLAN) in dem die
	Wallboxen und der WARP Energy Manager sich befinden.

	\hint{WARP Charger Pro (1. Generation) Wallboxen unterstützen leider nicht die
	Phasenumschaltung, da der verbaute Stromzähler keinen einphasigen Betrieb
	unterstützt.}

	Du kannst verschiedene Einstellungen vornehmen, mit denen du definieren
	kannst unter welchen Bedingungen ein Fahrzeug geladen wird.

	\subsubsection{Phasenumschaltung}
	Mittels eines externen Schützes kann der WARP Energy Manager
	angeschlossene Wallboxen zwischen einem 1- und 3-phasigem Betrieb
	umschalten. 
	Dies hat den Vorteil, dass die minimale Ladeleistung von ca.
	\SI{4.1}{\kilo\watt}~bei einem dreiphasigem Betrieb (minimaler Ladestrom
	\SI{6}{\ampere}) auf ca. \SI{1.4}{\kilo\watt}~reduziert werden kann. Somit
	kann auch ein geringer Leistungsüberschuss in ein Fahrzeug geladen werden,
	anstatt dass eine Ladung gar nicht möglich ist und die Energie ins Netz 
	eingespeist wird, oder aber zusätzliche Leistung aus dem Netz bezogen werden 
	muss um eine Ladung zu beginnen.

	\subsubsection{Eingänge für potentialfreie Kontakte}
	Der WARP Energy Manager verfügt über zwei Eingänge für potentialfreie
	Schaltkontakte. Wird die Phasenumschaltung genutzt, wird einer dieser
	Eingänge fest zur Schützüberwachung verwendet. Ansonsten kann das Verhalten
	des Energiemanagers auf die Eingänge konfiguriert werden. Es kann zum
	Beispiel darüber eine generelle Ladefreigabe realisiert werden, oder der
	Ladestrom der Wallboxen begrenzt werden.

	\subsubsection{Potentialfreier Relaisausgang}
	Ein Relaisschaltausgang (potentialfrei) auf dem WARP Energy Manager kann
	genutzt werden um externe Verbraucher o.ä. zu schalten. Der Ausgang kann
	konfiguriert werden und zum Beispiel abhängig von der verfügbaren Leistung,
	des momentanen Netzbezuges oder einer erfolgten Phasenumschaltung geschaltet
	werden.

	\textbf{Achtung mit dem Relais kann keine Netzspannung (230V) geschaltet
	werden. Es können bis zu 30V/1A geschaltet werden.}

	Als Beispiel können oftmals SG Ready-Steuereingänge von Wärmepumpen mit
	diesem Relaisausgang gesteuert werden.

	
	\subsubsection{Status LED}
	Der WARP Energy Manager besitzt auf der Frontseite eine Status-LED.

	Ist PV-Überschussladen aktiviert (siehe \ref{pv_ueberschussladen}), dann visualisiert diese LED 
	den Zustand am Netzanschluss. Die LED Farben sind wie folgt:
	\begin{description}
		\item[Grün:] Leistung wird ins Netz eingespeist
		\item[Gelb:] Leistung wird aus dem Netz bezogen
		\item[Blau:] \glqq Keine\grqq~Leistung am Netzanschluss (+- $\SI{200}{\watt}$)
	\end{description}

	Wenn PV-Überschussladen nicht aktiviert ist, dann leuchtet die LED grün.

	In Fehlerfällen leuchtet die Status-LED gelb oder rot (siehe
	\ref{fehlerbehebung}).


	\section{Typische Anwendungen}

	\subsubsection{PV-Überschussladen}
	\label{pv_ueberschussladen}

	Besitzt du eine Photovoltaik-Anlage, möchtest du vermutlich möglichst viel
	von deinem produzierten Strom selbst nutzen. Der WARP Energy Manager kann
	dir dabei helfen, indem er ein reines PV-Überschussladen ermöglicht, bei
	dem nur überschüssige Energie ins Fahrzeug geladen wird. Alternativ kannst
	du auch einen erlaubten anteiligen Netzbezug definieren. Dies ist sinnvoll,
	wenn die selbst produzierte Leistung nicht ausreicht, um einen Ladevorgang
	zu starten, du aber dennoch laden möchtest.

	Für das PV-Überschussladen benötigt der WARP Energy Manager einen Stromzähler
	an deinem Stromnetzanschluss um einen Überschuss, d.h. die Einspeisung von
	elektrischer Leistung ins Stromnetz, zu ermitteln. Der WARP Energy Manager
	steuert dann die Wallboxen so, dass keine Leistung ins Netz eingespeist wird
	(Netzbezug = 0) oder aber ein definierter Netzbezug eingehalten wird. Dies
	ist abhängig von deinen Einstellungen.

	Entscheidend ist hier, dass nur eine Leistungsregelung stattfindet. Es werden
	nicht die einzelnen Phasenströme geregelt. Da der Netzbetreiber-Stromzähler,
	der die Stromkosten ermittelt, saldierend arbeitet, ist eine
	Phasenstromregelung nicht notwendig.

	\subsubsection{Statisches Lastmanagement}
	\label{statisches_lastmanagement}

	Teilen sich mehrere Wallboxen eine gemeinsame Zuleitung, so ist oftmals der
	Maximalstrom durch diese Leitung begrenzt. Als Beispiel könnten sich mehrere
	Wallboxen eine 32A Leitung teilen. Zwei Wallboxen könnten jeweils als 11kW
	Wallboxen (2x16A) betrieben werden. Es wäre aber natürlich auch möglich eine
	Wallbox mit 22kW (32A) zu betreiben, wenn die zweite Wallbox nicht genutzt
	wird. Für diese Anwendungen kommt das statische Lastmanagement zum Einsatz.

	Der WARP Energy Manager kann das statische Lastmanagement für die Wallboxen
	übernehmen. Hierbei ist kein Stromzähler notwendig. Es ist einfach der
	Maximalstrom der Zuleitung zu definierieren. Dieser Strom muss jederzeit zur
	Verfügung stehen. Der Energy Manager verteilt diesen Strom dynamisch
	je nach Anforderung an die kontrollierten Wallboxen.

	\subsubsection{Dynamisches Lastmanagement}
	\label{dynamisches_lastmanagement}

	In manchen Fällen ist ein dynamisches Lastmanagement auf Phasenstromebene erforderlich.
	Ein typisches Beispiel dafür sind Mietobjekte, bei denen der Stromnetzanschluss der
	Immobilie nicht ausreicht, um mehrere Wallboxen gleichzeitig zu betreiben.
	Die Absicherung des Stromnetzanschlusses beschränkt den zulässigen Phasenstrom.

	Im einfachsten Fall kann für alle Wallboxen ein bestimmter Phasenstrom garantiert werden.
	In diesem Fall können die Wallboxen ein statisches Lastmanagement durchführen,
	bei dem der verfügbare Phasenstrom zwischen den WARP Chargern aufgeteilt wird. (siehe \fullref{statisches_lastmanagement}).

	Oftmals kann jedoch nicht garantiert werden, dass ein bestimmter Phasenstrom jederzeit
	zur Verfügung steht, da sich die Wallboxen den Stromanschluss mit anderen Verbrauchern teilen.
	Wenn diese Verbraucher ein- und ausgeschaltet werden
	ändert sich der für die Wallboxen zur Verfügung stehende Phasenstrom
	ständig. In diesem Fall ist ein dynamisches Lastmanagement notwendig um
	sicherzustellen, dass der maximale Phasenstrom nicht überschritten wird und
	keine Sicherung auslöst. 

	Der WARP Energy Manager ermöglicht ein dynamisches Lastmanagement auf Phasenstromebene.
	Dazu ist ein Stromzähler am Stromnetzanschluss erforderlich, der vom Energy Manager
	ausgewertet werden kann. Der Energy Manager überwacht den zur Verfügung stehenden
	Phasenstrom vom Netzanschluss und regelt die Leistung der Wallboxen entsprechend.
	Dadurch wird sichergestellt, dass der maximale Phasenstrom nicht überschritten wird
	und keine Sicherung auslöst. Wenn eine Photovoltaik-Anlage vorhanden ist und Energie
	produziert, erhöht sie automatisch die zur Verfügung stehende Leistung für den
	Energy Manager, um die Ladung der Elektrofahrzeuge zu optimieren.

	\subsubsection{Kombination PV + Lastmanagement}
	PV-Überschussladen und ein statisches/dynamisches Lastmanagement können
	kombiniert werden. Rein technisch betreibt der WARP Energy Manager dann die
	Leistungs-Regelung für das PV-Überschussladen, stellt aber parallel sicher, dass die
	Phasenstrom-Begrenzungen durch das Lastmanagement eingehalten werden.

	\section{Sicherheitshinweise}
	Der WARP Energy Manager ist so konstruiert, dass ein sicherer Betrieb gewährleistet ist,
	wenn er korrekt installiert wurde, in einem einwandfreien technischen Zustand
	ist und diese Betriebsanleitung befolgt wird. \hint{Der WARP Energy Manager darf nur von einer ausgewiesenen Elektrofachkraft installiert
		werden.}

	\subsection{Bestimmungsgemäße Verwendung}
	Mit dem WARP Energy Manager kann in Verbindung mit einem externen
	Stromzähler ein Energie-Monitoring realisiert werden. In Verbindung mit WARP
	Charger Wallboxen kann somit eine leistungsbezogene Ladevorgangsteuerung von
	Elektrofahrzeugen realisiert werden. Für andere Anwendungen ist der
	Energiemanagernicht nicht geeignet. Eine Verwendung
	an Orten, an denen explosionsfähige oder brennbare Substanzen lagern, ist nicht
	zulässig. Jegliche Modifikation des Managers oder unsachgemäßer Betrieb ist verboten. 
	Der Energy Manager ist in einem geeigneten Verteilerschrank zu installieren
	und vor Beschädigungen, Feuchtigkeit/Verschmutzungen und unsachgemäßigem
	Zugriff zu 	schützen. Er darf nicht genutzt werden, wenn kein sicherer Betrieb
	gewährleistet werden kann.

	\subsection{Gerätestörung / Technischer Defekt}
	Sollte es Anzeichen für einen technischen Defekt geben, ist sofort die
	Stromversorgung des Energiemanagers zu trennen und gegen erneutes Einschalten zu
	sichern. Danach ist eine Elektrofachkraft zu informieren.

	\newpage
	\section{Montage und Installation}
	\subsection{Montage}
	\subsubsection{Lieferumfang}
	Im Lieferumfang des WARP Energy Managers befinden sich:
	\begin{itemize}
		\item WARP Energy Manager (Hutschienenmodul)
		\item Steckbare Schraubklemmen
		\begin{itemize}
			\item 2 pol 5mm Schraubklemme (230V Stromversorgung (L+N))
			\item 2 pol 5mm Schraubklemme (Schütz)
			\item 4 pol 3.5mm Schraubklemme (Eingänge)
			\item 2 pol 3.5mm Schraubklemme (Relaisausgang)
			\item 4 pol 3.5mm Schraubklemme (RS485 Modbus-RTU)
		\end{itemize}
		\item DIN A4 Umschlag mit:
		\begin{itemize}
			\item Dieser Betriebsanleitung inkl. individueller WLAN Zugangsdaten
			\item RJ45 LAN-Winkeladapter
		\end{itemize}
	\end{itemize}

	\subsubsection{Montageort}
	Der WARP Energy Manager darf nur in einem geeigneten Verteilerschrank im
	Innenbereich installiert werden. Er ist vor Staub, Nässe und unsachgemäßigem
	Zugriff zu schützen. Es sollte
	eine LAN-Verbindung zum WARP Energy Manager gelegt werden, da in vielen
	Fällen eine Anbindung des WARP Energy Managers mittels WLAN nicht stabil
	möglich ist (Metallabschirmung der Verteilung).

	Es muss ausreichend Platz vorhanden sein. Es darf kein Druck auf die Kabel
	ausgeübt werden, insbesondere nicht auf die LAN Verbindung. Aus diesem Grund
	empfehlen wir die Verwendung des mitgelieferten LAN-Winkeladapters.

	\subsubsection{Montage}
	Zur Montage des WARP Energy Managers muss dieser auf die Hutschiene
	gesetzt werden. Das Gehäuse muss so installiert werden, dass die Anschlüsse
	nach unten zeigen.

	\gfx{./img/wem_mounting.jpg}

	Zuerst wird die obere Halterung auf die Hutschiene aufgesetzt und anschließend
	die Untere. Der Energiemanager sollte sich selbstständig verriegeln, falls dies
	nicht der Fall ist, kann mit einem Schraubendreher an der schwarzen Verriegelung
	auf der Unterseite nachgeholfen werden.
	\par
	Soll der WARP Energy Manager wieder von der Hutschiene entfernt werden, so
	müssen zuerst alle Zuleitungen entfernt werden (\textbf{Achtung: Spannungsfreiheit
	sicherstellen!}). Anschließend kann mittels Schlitz-Schraubendreher die schwarze
	Federverriegelung gezogen werden und der Energy Manager von der Hutschiene
	gehoben werden. Dabei sollte zuerst die untere Halterung angehoben werden, 
	gefolgt von der oberen Halterung.

	\newpage
	\subsection{Elektrischer Anschluss}
	\hint{Die in diesem Kapitel beschriebenen Arbeiten dürfen nur von einer ausgewiesenen
		Elektrofachkraft durchgeführt werden!}

	\gfx{./img/wem_connections.jpg}


	\subsubsection{230V Stromversorgung}
	Nachdem der WARP Energy Manager montiert wurde, kann dieser nun angeschlossen werden.
	Die Schraubklemmen sind steckbar, so dass der elektrische Anschluss
	außerhalb erfolgen kann. Anschließend können die Schraubklemmen wieder in
	den WARP Energy Manager gesteckt werden.

	Die Stromversorgung des WARP Energy Managers erfolgt über eine 2-polige
	Schraubklemme (\textbf{L}+\textbf{N}). Die Zuleitung ist mit einem max. 16A Leitungsschutzschalter mit
	B-Charakteristik abzusichern.

	Die Stromversorgung des Energy Managers ist zusätzlich intern über eine Glassicherung 
	(mittelträge (m), \SI{500}{\milli\ampere}) abgesichert.

	\subsubsection{Schütz zur Phasenumschaltung}
	\hint{Es muss kein Schütz installiert werden. Dieser Schritt ist optional,
	wenn keine Phasenumschaltung erfolgen soll.}

	Ein externes Schütz kann zur Phasenumschaltung, das heißt der Umschaltung
	zwischen 1-phasiger und 3-phasiger Fahrzeugladung, installiert werden. Das
	Schütz wird mittels 230V Schaltausgang vom WARP Energy Manager gesteuert
	(\textbf{Lsw}).
	Der minimale Phasenstrom für das Typ2 Laden beträgt $\SI{6}{\ampere}$. Somit
	kann die Minimale Ladeleistung von $\SI{4.1}{\kilo\watt}$ auf
	$\SI{1.4}{\kilo\watt}$ reduziert werden.

	Zu Ansteuerung wird \textbf{N} und \textbf{Lsw} nach außen geführt. Der
	\textbf{Lsw}-Schaltausgang ist intern über eine Glassicherung 
	(mittelträge (m), \SI{500}{\milli\ampere}) abgesichert.


	\subsubsection{Eingänge}
	Der WARP Energy Manager besitzt zwei Eingänge für potentialfreie Kontakte.
	An diesen können Schließer/Öffner angeschlossen werden. Das Verhalten des
	Energy Managers in Bezug auf diese Eingänge kann im Webinterface konfiguriert werden.
	
	Wird ein Schütz zur Phasenumschaltung installiert, so ist der Eingang~\textbf{3} 
	fest zur Schützüberwachung konfiguriert. Es ist erforderlich, einen Schließer zwischen 
	\textbf{12V} und \textbf{3} zu installieren, welcher vom zu überwachenden Schütz geschaltet wird.

	Wird kein Schütz zur Phasenumschaltung verwendet, kann Eingang~\textbf{3} für
	andere Zwecke verwendet werden (konfigurierbar). Eingang~\textbf{4} steht immer
	für eigene Zwecke zur Verfügung. Die Eingänge sind so ausgelegt, dass ein potentialfreier
	Kontakt extern angeschlossen werden kann (Schalter als Öffner/Schließer, Relais etc.).
	Die \textbf{12V}~Anschlüsse der Eingänge sind hochohmig ausgelegt, liefern
	keine Leistung und sind daher nicht zur Stromversorgung anderer Verbraucher
	geeignet.

	\subsubsection{Relais-Ausgang}
	Mit dem Relaisschaltausgang (potentialfrei) können bis zu 30V/1A geschaltet
	werden. Das Schalten von Netzspannung ist also nicht direkt möglich!

	\subsubsection{RS485 Modbus Stromzähler}
	\hint{Es muss kein RS485 Modbus Stromzähler installiert werden. Dieser
	Schritt ist optional, wenn ein anderer unterstützer Stromzähler konfiguriert
	wird.}

	Der WARP Energy Manager benötigt einen Stromzähler um den Leistungsbezug regeln zu 
	können. Eine Möglichkeit dafür ist die Installation eines RS485 Modbus
	Stromzählers vom Typ Eastron SDM72DMV2, SDM630MCT oder SDM630Modbus.
	
	Die Steckerbelegung ist \textbf{12V, A, B, GND}. Der Anschluss \textbf{12V}
	darf nicht belegt werden. \textbf{A (+), B (-), GND} sind entsprechend 
	am jeweiligen Stromzähler anzuschließen.

	\subsubsection{LAN Anschluss}
	Die Steuerung der Wallboxen erfolgt über das Netzwerk. Wir empfehlen den
	Anschluss des WARP Energy Managers per LAN. Der dafür notwendige LAN
	Anschluss befindet sich oberhalb der anderen Anschlüsse. Um Beschädigungen
	zu vermeiden ist die LAN Buchse flexibel befestigt. Wir empfehlen es ein LAN
	Kabel nicht direkt an den Energy Manager anzuschließen, sondern die Nutzung 
	des mitgelieferten RJ45-Winkeladapters zwischen Energy Manager und
	LAN-Kabel.

	\section{Erste Schritte}
	\label{setup}

	Nach der elektrischen Installation kann der WARP Energy Manager konfiguriert
	werden. Dazu muss zuerst eine Verbindung zum Energy Manager hergestellt werden, 
	damit diese dann über den Browser konfiguriert werden kann.

	\subsection{Schritt 1: Verbindung herstellen}


	\hint{Wir empfehlen unbedingt eine Anbindung des WARP Energy Managers per
	LAN. Auch wenn technisch eine Anbindung mittels WLAN möglich ist, so muss
	sichergestellt werden, dass diese Verbindung dauerhaft stabil ist. Gerade in
	Schaltschränken gestaltet sich dies meist schwierig.}

	\paragraph{Option 1: WLAN}\ \\

	Im Werkszustand öffnet der WARP Energy Manager einen WLAN-Access-Point. Über diesen kann
	die Konfiguration vorgenommen werden, indem auf das das Webinterface des
	Energy Managers zugegriffen wird.

	Die Zugangsdaten des Access-Points findest du auf dem WLAN-Zugangsdaten-Aufkleber
	auf der Rückseite dieser Anleitung. Ein weiterer identischer Aufkleber
	befindet sich auf der Rückseite der Frontplatte des WARP Energy Managers.
	Du kannst entweder den QR-Code des Aufklebers verwenden,
	der das WLAN automatisch konfiguriert, oder SSID und Passphrase abschreiben.
	Die meisten Kamera-Apps von Smartphones unterstützen das Auslesen des
	QR-Codes und das automatische Verbinden zu dem WLAN. Somit musst du die
	Zugangsdaten dann nicht abtippen. Wichtig ist, dass viele Smartphones
	erkennen, dass über das WLAN des Energy Managers (Access-Point) kein Zugriff auf das
	Internet möglich ist. Dein Telefon fragt dann nach, ob du zu dem WLAN
	verbunden bleiben möchtest. Damit du weiter auf den Energy Manager zugreifen
	kannst, darfst du das WLAN nicht wieder verlassen.

	\begin{minipage}{0.35\textwidth}
		Wenn die Verbindung mit dem Access-Point des Energy Managers hergestellt ist, kannst du das Webinterface
		unter \url{http://10.0.0.1} über einen Browser deiner Wahl erreichen.
		Alternativ kannst du dazu den nebenstehenden QR-Code scannen.
		Eventuell musst du deine mobile Datenverbindung (z.B. LTE) deaktivieren.
	\end{minipage}\hfill
	\begin{minipage}{0.12\textwidth}
		\begin{flushright}
			\qrcode{http://10.0.0.1}
		\end{flushright}
	\end{minipage}

	\paragraph{Option 2: LAN}\ \\
	Als Alternative zum Zugriff über den WLAN-Accesspoint verbindet sich der
	Energy Manager in den Werkseinstellungen automatisch zu einem
	kabelgebundenen Netzwerk (LAN), wenn ein LAN-Kabel eingesteckt ist (IP Bezug
	mittels DHCP). Der Energy Manager kann dann entweder über die zugewiesene IP
	Adresse (\url{http://[IP-des-Energy-Managers]}, z.B. \url{http://192.168.0.42})
	oder den Hostnamen (\url{http://[hostname]}, z.B. \url{http://wem-ABC}) erreicht werden.

	Der Hostname des Energy Managers ist identisch zur SSID des WLANs. Den Hostnamen findest du
	auf dem WLAN-Zugangsdaten-Aufkleber auf der Rückseite dieser Anleitung.

	Kann die per DHCP vergebene IP des Energy Managers nicht ermittelt werden, so kann der
	zuvor genannte Zugriff auf den Energy Manager mittels WLAN-Access-Point genutzt
	werden um die IP Adresse der LAN Schnittstelle zu ermitteln (\glqq
	Status-Seite\grqq, Abschnitt \glqq LAN-Verbindung\grqq).


	\subsection{Schritt 2: Konfiguration mittels Webinterface}
	Generell empfehlen wir nach der Installation ein Update der Firmware des
	Energy Managers. Somit erhältst du die neusten Funktionen und ggf. Bugfixes. Wie ein
	Firmware-Update durchgeführt wird, ist unter \fullref{firmware-update}
	beschrieben.

	Anschließend kann der WARP Energy Manager über das Webinterface konfiguriert
	werden. Die Einstellungen etc. hängen vom Anwendungsfall ab. 
	Das Webinterface ist unter \fullref{webinterface} vollständig beschrieben.

	\newpage
	\section{Webinterface}
	\label{webinterface}

	Über das Webinterface kannst du den Energieverbrauch, Überwachen und 
	unter anderem das Laden der kontrollierten Wallboxen steuern und überwachen.
	Es können diverse Einstellungen vorgenommen werden, die nachfolgend
	dokumentiert sind.

	Wenn du auf das Webinterface der Wallbox mit einem Browser zugreifst
	gelangst du auf die Start-/ Statusseite. Auf der linken Seite befindet sich
	die Menüleiste, über die du zu weiteren Einstellungen kommst.

	Auf mobilen Endgeräten wird
	diese Menüleiste stattdessen versteckt unter einem Menü-Symbol oben rechts
	im grauen Balken neben dem WARP Logo angezeigt (\glqq drei Striche untereinander\grqq).
	Hier kannst du das Menü durch einen Klick auf das Symbol ausklappen.
	\gfx{./img/resized/web_status}

	\subsection{Status (Startseite)}
	\label{status}
	Die Startseite des Webinterfaces bietet Einstellen und zeigt Statusinformationen.

	Mittels Schaltflächen kann zwischen verschiedenen Lademodi der angeschlossenen 
	Wallboxen gewechselt werden:
	\begin{description}
	\item[\textbf{PV}-Überschussladen] \glqq 100\% Eigener Strom\grqq. Ob ein
	Ladevorgang startet ist davon abhängig, ob die Minimale Ladeleistung
	als Überschuss zur Verfügung steht. Ist dies nicht der Fall, so
	wird keine Ladung gestartet.
	\item[\textbf{Min+PV}-Laden] Es wird sicher mit einer Ladung begonnen, indem 
	die minimale Ladeleistung sichergestellt wird. Zur Not erfolgt diese als
	Netzbezug. Wird genügend Leistung produziert (Netzeinspeisung), so wird
	der Ladestrom soweit erhöht bis keine Einspeisung ins Stromnetz mehr
	erfolgt, oder aber die maximale Ladeleistung erreicht wird.
	\item[\textbf{Schnell}-Laden] Alle Wallboxen laden mit der maximal möglichen
	Ladeleistung ohne Beachtung einer Netzeinspeisung bzw. eines Netzbezugs.
	\item[\textbf{Aus}] Die kontrollierten Wallboxen sind deaktiviert. Es kann
	nicht geladen werden.
	\end{description}
	Die PV-Optionen sind nur verfügbar, wenn PV-Überschussladen aktiviert wurde.

	Als nächstes wird eine Übersicht der \textbf{kontrollierten Wallboxen} und deren
	Zustand angezeigt. Zu diesem zählt unter anderem der zugeordnete Ladestrom.

	Am Ende wird der Zustand der verwendeten Schnittstellen angezeigt.

	\subsection{Energiebilanz}

	\gfx{./img/resized/web_em_energy_analysis}
	Die Seite Energiebilanz stellt Informationen zum Energiebezug zur Verfügung.
	Die Daten werden lokal auf dem WARP Energy Manager gespeichert. Die Daten
	können auf Stunden-, Tages-, Wochen- oder Monatsbasis betrachtet werden.

	Die Funktion wird erst mit dem kommenden Firmwareupdate veröffentlicht.

	\subsection{Energiemanager}
	\subsubsection{Einstellungen}

	\gfx{./img/resized/web_em_settings}
	Alle Einstellungen bezüglich des Energiemanagements werden hier vorgenommen.

	Als erstes muss der \textbf{Standard-Lademodus} definiert werden. Die
	verschiedenen Modi wurden bereits unter \fullref{status} erläutert. 
	Wird die Einstellung auf der Statusseite geändert, so ist diese Einstellung permanent.
	Mittels \textbf{Täglich zurücksetzen} kann die Einstellung aber auch 
	automatisch täglich wieder auf den Standard-Lademodus zurückgesetzt werden.

	\paragraph{Dynamisches Lastmanagement}\ \\
	Unter dem Abschnitt Dynamisches Lastmanagement werden zukünftig
	alle Einstellungen zum dynamischen Lastmanagement zu finden sein. Diese
	Funktion wird mittels Firmware-Update zur Verfügung gestellt.

	Hierbei misst der WARP Energy Manager laufend mittels eines Stromzählers die
	Ströme aller Phasen am Stromnetzanschluss. Der noch rechnerisch zur
	Verfügung stehende Strom kann für jede Phase unterschiedlich sein und ändert
	sich laufend auf Grund des Zu- und Abschaltens von Verbrauchern. Aber auch eine
	parallel angeschlosse PV-Anlage beeinflusst die Phasenströme. Der WARP
	Energy Manager kann somit laufend den noch rechnisch zur Verfügung stehenden
	Phasenstrom ermitteln und diesen den angeschlossenen Wallboxen bereit
	stellen. Dabei wird sicher gestellt, dass der Strom keiner Phase
	überschritten wird und keine Sicherung ausgelöst wird.

	\paragraph{PV-Überschussladen}\ \\
	PV-Überschussladen muss im entsprechenden Abschnitt mittels
	Schieberegler aktiviert werden. Nach der Aktivierung werden die Ladeoptionen
	\glqq PV\grqq~und \glqq Min+PV\grqq~ angeboten. Für diesen Modus ist ein
	Stromzähler, wie unter \ref{stromzaehler} beschrieben zu konfigurieren.

	Im \textbf{Modus \glqq PV\grqq}
	steuert der WARP Energy Manager alle angeschlossenen Wallboxen so, dass die
	überschüssige PV Leistung genutzt wird um angeschlossene Fahrzeuge zu laden.
	Der Überschuss bzw. der Netzbezug wird auf Null geregelt. Steht keine
	ausreichende PV Leistung zur Verfügung wird nicht geladen.

	Im \textbf{Modus \glqq Min+PV\grqq}
	steuert der WARP Energy Manager alle angeschlossenen Wallboxen so, dass die
	überschüssige PV Leistung genutzt wird um angeschlossene Fahrzeuge zu laden.
	Zusätzlich wird die im Feld \textbf{\glqq Min+PV:
	Mindest-Ladeleistung\grqq} definierte Leistung für alle angeschlossenen
	Wallboxen in Summe garantiert, falls nicht anders möglich als Netzbezug.

	Soll eine Phasenumschaltung zwischen einem 1-phasigen und 3-phasigen Betrieb
	der Wallboxen erfolgen, so muss ein externes Schütz entsprechend installiert
	werden und die Option \textbf{Schütz angeschlossen} aktiviert werden. Bei
	Konfiguration der Option \textbf{Phasenumschaltung} auf \textbf{automatisch}
	schaltet der WARP Energy Manager dann selbstständig auf einem 1-phasigen
	Betrieb, sollte die PV Leistung unterhalb von 4.1kW liegen (3*230V*6A) um
	eine minimale Ladeleistung von 1.4kW zu ermöglichen (1*230V*6A).
	Entsprechend schaltet der WARP Energy Manager wieder automatisch zurück,
	sobald die Mindestladeleistung für ein 3-phasiges Laden erreicht wird.

	Über die Einstellungen \textbf{Immer einphasig/Immer dreiphasig} kann das
	Schütz auch fest konfiguriert werden.

	Der Energy Manager unterbricht alle Ladevorgänge bevor eine
	Phasenumschaltung stattfindet.

	\paragraph{Relais}\ \\
	Der WARP Energy Manager verfügt über einen potentialfreien Schaltausgang
	(Relais). Die Funktion kann hier definiert werden. Im Modus
	\textbf{Regelbasiert} können mittels Drop-Down-Boxen verschiedene Bedingungen definiert werden, in
	denen der Relais-Ausgang geschlossen wird und geschlossen bleibt. Ist die
	Bedingung nicht mehr erfüllt, dann wird das Relais wieder geöffnet.
	Im Modus \textbf{Extern gesteuert} kann das Relais mittels API gesteuert
	werden.
	\paragraph{Eingänge 3+4}\ \\
	Die Eingänge 3+4 können genutzt werden um potentialfreie Kontakte auszulesen
	(z.B. Schalter oder Relaisausgänge). Die Reaktion des WARP Energy Managers
	kann auf diese Eingänge kann hier definiert werden. Wird ein Schütz zur
	Phasenumschaltung angeschlossen und genutzt, dann steht Eingang 3 nicht mehr
	zur Verfügung da mit diesem dieses Schütz überwacht wird.

	Als Optionen stehen zur Verfügung
	\begin{description}
		\item[Nicht verwendet] Der Eingang wird nicht genutzt.
		\item[Laden blockieren] Wenn der Eingang geschlossen/geöffnet ist, ist
		eine Ladung bei allen Wallboxen nicht möglich.
		\item[Ladestrom begrenzen] Wenn der Eingang geschlossen/geöffnet ist,
		wird der Ladestrom jeder Wallbox auf die eingestellten Ampere begrenzt.
		\item[Moduswechsel] Hier kann konfiguriert werden in welchen Lademodus
		beim Schließen/Öffnen des Eingangs gewechselt werden soll.
	\end{description}

	\subsubsection{Stromzähler}
	\label{stromzaehler}

	\gfx{./img/resized/web_em_meter_config}

	Als Stromzähler am Netzanschluss können verschiedene Stromzähler-Typen
	konfiguriert werden. Hier muss ein Stromzähler konfiguriert werden, wenn 
	der WARP Energy Manager die Funktionen \fullref{pv_ueberschussladen} oder 
	\fullref{dynamisches_lastmanagement} ausführen soll.

	Mit der Einstellung \textbf{SDM630*/SDM72*} werden folgende
	RS485 (Modbus RTU) Stromzähler unterstützt:
	\begin{itemize}
		\item Eastron SDM630
		\item Eastron SDM72DM V2
		\item Eastron SDM630MCT V2
	\end{itemize}
	
	\par
	Mit der Einstellung \textbf{Benutzerdefinierter Zähler MQTT/HTTP}
	können anstatt eines direkt per RS485 (Modbus RTU) angeschlossenen
	Stromzählers, Stromzählerwerte dem WARP Energy Manager per API übergeben
	werden.


	\subsubsection{Wallboxen}

	\gfx{./img/resized/web_charge_manager}
	Hier werden die vom Energy Manager kontrollierten Wallboxen konfiguriert.
	Die hier vorgenommenen Einstellungen beeinflussen das Lastmanagement
	zwischen den Wallboxen.

	Typ2 Wallboxen kommunizieren den angeschlossenen Fahrzeugen den maximal zur
	Verfügung stehenden Ladestrom. Das Fahrzeug entscheidet ob dieser Ladestrom
	voll ausgenutzt wird und ob eine Ladung 1- / 2- oder 3-phasig durchgeführt
	wird.

	Als erste Einstellung muss mittels \textbf{Maximaler Gesamtstrom} der 
	zulässige Maximalstrom der Zuleitung zu den Wallboxen konfiguriert werden. 
	Der Energy Manager stellt sicher, dass dieser
	Strom auf keiner Phase überschritten wird, indem in Summe niemals mehr als
	dieser Strom an die Wallboxen verteilt wird. Besitzen alle Wallboxen
	ausreichend dimensionierte getrennte Zuleitungen kann dieser Strom auch so
	hoch eingestellt werden, dass alle Wallboxen sicher ihren Maximalstrom
	erhalten. Damit wird diese Begrenzung außer Kraft gesetzt. Alle andere
	Komponenten, wie zum Beispiel der Netzanschluss, müssen dann aber den
	angeforderten Storm liefern können. 

	\hint{Hierbei handelt es sich um ein statisches Lastmanagement, bei dem
	davon ausgegangen wird, dass der eingestellte Strom auf jeder Phase
	zu jeder Zeit zur Verfügung steht. Andere Verbraucher als WARP Charger, 
	welche vom Energy Manager nicht gesteuert werden können, werden nicht
	berücksichtigt!}

	Der individuelle Maximalstrom jeder Wallbox bleibt hiervon unberührt
	(Zuleitung der Wallbox - Schiebeschaltereinstellung innerhalb der Wallbox).

	Eine typische Anwendung hierfür ist, die Strombegrenzung, 
	wenn die Wallboxen über eine gemeinsame Zuleitung verfügen.

	Mit der Einstellung \textbf{Minimaler Ladestrom} kann der minimale Ladestrom
	angehoben werden. Der Typ2-Ladestandard setzt als Minimum 6A voraus. Eine
	Einstellung darunter ist nicht möglich. Allerdings gibt es Fahrzeuge, welche 
	bei 6A nicht mit einer Ladung beginnen. Falls notwendig kann hier ein höherer 
	Ladestrom definiert werden. In den allermeisten Fällen kann die Einstellung
	bei 6A belassen werden.

	Am Ende der Seite werden die \textbf{Kontrollierte
	Wallboxen} dargestellt. Weitere Wallboxen können mittels Klick auf
	\textbf{Wallbox hinzufügen} der Steuerung durch den WARP Energy Manager
	hinzugefügt werden. Dazu muss der Anzeigename und die IP-Adresse oder der
	Hostname der Wallbox eingetragen werden und mittels Klick auf \glqq
	hinzufügen\grqq~übernommen werden.

	Automatisch ermittelte Wallboxen, die noch nicht dem Energy Manager
	angehören, werden als Liste dargestellt. Die Einstellungen dazu können
	mittels Klick übernommen werden und anschließend ebenfalls mittels \glqq
	hinzufügen\grqq~übernommen werden.

	Alle geänderte Einstellungen müssen mittels \glqq Speichern\grqq~übernommen 
	werden.

	\newpage
	\subsection{Netzwerk}
	\label{network}
	Die Wallbox kann in dein Netzwerk per WLAN oder LAN eingebunden werden.
	In diesem Unterabschnitt können alle dazugehörigen Einstellungen vorgenommen werden.

	\subsubsection{Allgemein}
	Hier kannst du den Hostnamen des WARP Energy Managers in allen verbundenen Netzwerken konfigurieren. Außerdem kann mDNS aktiviert oder deaktiviert werden.
	Über mDNS können andere Geräte im Netzwerk den WARP Energy Manager finden.

	\gfx{./img/resized/web_network}


	\subsubsection{WLAN-Verbindung}
	\gfx{./img/resized/web_wifi_sta}

	Es Besteht die Möglichkeit den WARP Energy Manager in dein Netzwerk mittels
	WLAN zu integrieren. \textbf{Diese Option empfehlen wir aber ausdrücklich
	nicht!} Das WLAN kannst du hier konfigurieren.
	Durch Drücken des \enquote{Netzwerksuche}-Buttons öffnet sich ein Menü, in dem das gewünschte WLAN ausgewählt werden kann.
	Es werden dann automatisch Netzwerkname (SSID) und BSSID eingetragen, sowie die Verbindung beim Neustart aktiviert.
	Gegebenenfalls musst du jetzt noch die Passphrase des gewählten Netzes eintragen.

	Du kannst jetzt die Konfiguration mit dem Speichern-Button abspeichern.
	Das Webinterface startet dann neu und verbindet sich zum konfigurierten WLAN. Die Statusseite zeigt
	an, ob die Verbindung erfolgreich war. Der Access-Point bleibt weiterhin
	geöffnet, sodass Konfigurationsfehler behoben werden können.
	Da der Access-Point den selben Kanal wie ein eventuell verbundenes Netz verwendet,
	kann es sein, dass du dich jetzt neu zum Access-Point verbinden musst.

	Bei einer erfolgreichen Verbindung sollte den Energy Manager jetzt im konfigurierten Netzwerk unter
	\url{http://[konfigurierter_hostname]}, z.B. \url{http://wem-ABC} erreichbar sein.

	\subsubsection{WLAN-Access-Point}
	\gfx{./img/resized/web_wifi_ap}
	Der Access-Point kann in einem von zwei Modi betrieben werden: Entweder kann er immer aktiv sein,
	oder nur dann, wenn die Verbindung zu einem anderen WLAN bzw. zu einem LAN nicht konfiguriert oder fehlgeschlagen ist.
	Außerdem kann der Access-Point komplett deaktiviert werden.
	\hint{Wir empfehlen, den Access-Point nie komplett zu deaktivieren, da sonst bei einer
		fehlgeschlagenen Verbindung zu einem anderen Netzwerk das Webinterface nicht mehr erreicht
		werden kann. Der WARP Energy Manager kann dann nur über den \fullref{recovery} oder ein Zurücksetzen auf Werkszustand, siehe \ref{reset}, erreicht werden.}
	Die notwendigen Einstellungen, wie der Modus des Access-Points,
	Netzwerkname, Passphrase usw. müssen dazu hier festgelegt werden.

	\subsubsection{LAN-Verbindung}
	\gfx{./img/resized/web_ethernet}
	Wir empfehlen die Anbindung mittels kabelgebundenen LAN ins Netzwerk.
	In den meisten Fällen wird eine
	LAN-Verbindung automatisch hergestellt, falls ein Kabel eingesteckt ist
	(IP Adresse wird per DHCP bezogen). Es ist aber auch möglich,
	eine statische IP-Konfiguration	einzutragen, oder, falls gewünscht, die LAN-Verbindung
	komplett zu deaktivieren.

	Bei einer erfolgreichen Verbindung sollte der WARP Energy Manager jetzt im LAN unter
	\url{http://[konfigurierter_hostname]}, z.B. \url{http://wem-ABC} erreichbar sein.


	\subsubsection{WireGuard}

	WireGuard ist eine Möglichkeit den WARP Energy Manager in ein virtuelles privates Netzwerk (VPN)
	mittels einer verschlüsselten Verbindung einzubinden. WireGuard wird von
	verschiedenen Routern direkt unterstützt. Dies kann zum Beispiel genutzt
	werden um aus der Ferne auf den Energy Manager zuzugreifen und das
	Wallbox-Netzwerk vor einem Zugriff zu schützen. Zusätzlich kann das
	Lastmanagement zwischen Energy Manager und den Wallboxen per WireGuard abgesichert werden.

	Die notwendigen Parameter sind WireGuard-typisch und werden an dieser Stelle
	nicht gesondert erläutert. Weitere Informationen finden sich auf
	\url{https://www.wireguard.com/}.

	\gfx{./img/resized/web_wireguard}

	\subsection{Schnittstellen}
	\subsubsection{MQTT}
	\label{mqtt-interface}

	\gfx{./img/resized/web_mqtt}
	Auf der MQTT-Unterseite kannst du die Verbindung zu einem MQTT-Broker konfigurieren. Folgende Einstellungen können vorgenommen werden:
	\begin{itemize}
		\item \textbf{Broker-Hostname oder -IP-Adresse} Der Host\-name oder die
		IP-Adresse des Brokers, zu dem sich der WARP Energy Manager verbinden soll.
		\item \textbf{Broker-Port} Der Port, unter dem der Broker erreichbar ist. Der typische MQTT-Port 1883 ist voreingestellt.
		\item \textbf{Broker-Benutzername} und \textbf{-Passwort} Manche Broker unterstützen eine Authentifizierung mit Benutzername und Passwort.
		\item \textbf{Topic-Präfix} Dieses Präfix wird allen Topics vorangestellt, die die Wallbox verwendet.
		      Voreingestellt ist wem/ABC, wobei ABC eine eindeutige Kennung des
			  WARP Energy Managers ist,
		      es sind aber andere Präfixe wie z.B. energie\_manager möglich.
		      Falls mehrere Energy Manager mit dem selben Broker kommunizieren,
		      müssen eindeutige Präfixe gewählt werden.
		\item \textbf{Client-ID} Mit dieser ID registriert sich der WARP Energy Manager beim Broker.
		\item \textbf{Sendeintervall} Der WARP Energy Manager verschickt MQTT-Nachrichten nur, wenn sich die beinhalteten Daten geändert haben.
			Es gibt aber Teile der API, deren Daten sich sekündlich ändern. Das Sendeintervall kann hier reduziert werden, wenn weniger Netzwerktraffic
			erzeugt werden soll.
	\end{itemize}
	Nachdem die Konfiguration gesetzt und der \enquote{MQTT aktivieren}-Schalter aktiviert ist, kann die Konfiguration gespeichert werden.
	Das Webinterface startet dann neu und verbindet sich zum Broker.
	Auf der Status-Seite wird angezeigt, ob die Verbindung aufgebaut werden konnte.

	Weitere Informationen über die MQTT-API des WARP Energy Managers findest du auf \rurl{https://warp-charger.com/api.html}{warp-charger.com/api.html}

	\subsection{System}
	Im System-Unterabschnitt kannst du Einstellungen zur Zeitsynchronisation
	vornehmen, die interne SD Karte formatieren und diverse Debug informationen
	bekommen. Auch das Aktualisieren der Firmware ist hier möglich.

	\subsubsection{Zeitsynchronisierung}\label{ntp}
	Um für den Ladetracker und das Ereignis-Log die aktuelle Uhrzeit zur
	Verfügung zu haben, kann der WARP Energy Manager diese per NTP über
	eine Netzwerkverbindung synchronisieren. Auf dieser Unterseite kannst du NTP aktivieren oder deaktivieren und die Zeitzone, in der sich
	der WARP Energy Manager befindet konfigurieren.

	Außerdem ist es möglich, zusätzlich zum konfigurierten Zeitserver einen Zeitserver zu verwenden, der von deinem Router per DHCP gesetzt wird. Dies funktioniert allerdings nur,
	wenn in der Netzwerkkonfiguration keine statische IP-Konfiguration verwendet wurde.

	\gfx{./img/resized/web_ntp}

	\subsubsection{SD-Karte}
	Die Daten des WARP Energy Managers werden intern auf eine SD Karte
	aufgezeichnet. Hier werden Informationen dazu ausgegeben. Die SD Karte kann
	hier auch formatiert werden. Damit werden alle aufgezeichneten Informationen
	gelöscht!

	\subsubsection{Debug}
	Auf der Debug Seite kann ein Energy Manager-Protokoll erstellt werden. Dies
	ist hilfreich um etwaige Probleme bei der Energieverteilung aufzudecken. Um
	ein Protokoll zu erzeugen muss einfach nur auf \textbf{Start} geklickt
	werden. Der Energy Manager beginnt dann hochfrequent alle Zustände
	aufzuzeichnen. Mit \textbf{Stop+Download} kann die Aufzeichnung gestoppt und
	das erstellte Protokoll heruntergeladen werden.

	Im \textbf{Low-Level-Zustand} werden alle Zustände vom Energy Manager
	dargestellt.

	\subsubsection{Ereignis-Log}
	\gfx{./img/resized/web_event_log}

	Das Ereignis-Log zeichnet relevante Informationen des Systemstarts, sowie WLAN- und MQTT-Verbindungsabbrüche und Ladefehler auf.
	Falls Probleme mit der Energy Manager auftreten, kannst du diese mit dem Log diagnostizieren.
	Falls du ein Problem mit der WARP Energy Manager an uns melden möchtest, kannst du das Ereignis-Log,
	sowie einen Debug-Report abrufen, die uns helfen das Problem zu verstehen und zu lösen.

	\subsubsection{Firmware-Aktualisierung}
	\label{firmware-update}
	\gfx{./img/resized/web_firmware_update}
	Hier kannst du die Firmware des Energy Managers aktualisieren. Wir entwickeln die Funktionalität
	des Energy Managers laufend weiter. Bitte beachte, dass daher ggf. auch eine neue
	Version dieser Betriebsanleitung bereitgestellt wird.
	Die aktuelle Firmware und die neuste Betriebsanleitung findest du unter
	\rurl{https://warp-charger.com}{warp-charger.com} zum Download.

	Außerdem kannst du hier das Webinterface neustarten.

	\subsection{Wiederherstellungsmodus}
	\label{recovery}
	Falls der WARP Energy Manager weder seinen Access Point öffnet, noch über ein konfiguriertes Netzwerk auf das Webinterface zugegriffen werden kann,
	kannst du wie folgt den Wiederherstellungsmodus starten:
	\begin{enumerate}
	 \item Suche dir einen elektrisch nicht leitenden Stift (Kugelschreiber o.ä.) und einen kleinen Schlitz-Schraubendreher (z.B. Phasenprüfer o.ä.)
	 \item Entferne die Frontplatte (bedruckt) mit dem Schraubendreher (Öffne den Energy Manager)
	 \item Lokalisiere die zwei kleinen Taster auf der vorderen Platine (Beschriftet mit \textbf{EN} und \textbf{IO0})
	 \item Drücke mit dem Stift einmal kurz auf \textbf{EN}. Die blaue LED fängt an zu blitzen. 
	 \item Drücke anschließend mit dem Stift \textbf{IO0} und halte diesen gedrückt. Die blaue LED fängt an schnell zu blinken. 
	 \item Halte \textbf{IO0} ca. 8 Sekunden gedrückt, bis die LED dauerhaft leuchtet.
	 \item Sobald die blaue LED dauerhaft leuchtet ist der Vorgang abgeschlossen.
	 \item Sollte die LED währenddessen ausgehen, so war der Vorgang nicht erfolgreich und muss wiederholt werden.
	\end{enumerate}
	Der WARP Energy Manager startet dann im Wiederherstellungsmodus. Zunächst werden die Netzwerkeinstellungen gelöscht, sowie die Anmeldung deaktiviert.
	Bei Erfolg sollte es jetzt möglich sein, über den Access Point wieder auf den Energy Manager zuzugreifen.

	\subsection{Zurücksetzen auf Werkszustand}\label{reset}
	Falls das Webinterface nicht korrekt funktioniert, oder die Konfiguration defekt ist,
	kannst du auf der Firmware-Aktualisierungs-Unterseite alle Einstellungen auf den Werkszustand zurücksetzen.
	\hint{Durch das Zurücksetzen auf Werkszustand gehen \mbox{\textbf{alle}} Konfigurationen verloren.}
	Nach dem Zurücksetzen startet das Webinterface wieder und öffnet
	den Access-Point mit der SSID und Passphrase, die auf dem Aufkleber vermerkt
	sind. Der WARP Energy Manager kann jetzt wieder nach \fullref{setup} konfiguriert werden.

	Falls du das Webinterface nicht mehr erreichen kannst, bestehen folgende Optionen:

	Falls eine Netzwerkverbindung aufgebaut werden kann, aber das Webinterface selbst nicht mehr funktioniert, kannst du versuchen, die Recovery-Seite zu öffnen.
	Falls du über den Access Point der Wallbox verbunden bist, erreichst du diese unter \url{http://10.0.0.1/recovery},
	bei einer bestehenden Verbindung zu einem LAN oder WLAN über
	\url{http://[konfigurierter_hostname]/recovery}, also z.B. \url{http://wem-ABC/recovery}.
	Über die Recovery-Seite kannst du den WARP Energy Manager neustarten, Firmware-Updates einspielen,
	den Energy Manager auf den Werkszustand zurücksetzen (Factory Reset) und Debug-Reports
	herunterladen.

	Alternativ kannst du den Energy Manager (genauer: den verbauten ESP32 Ethernet
	Brick) neu flashen.
	Du benötigst dazu einen PC mit installiertem Brick Viewer 2.4.20 oder neuer. Diesen findest du unter
	 \rurl{https://www.tinkerforge.com/de/doc/Software/Brickv.html}{tinkerforge.com/de/doc/Software/Brickv.html}.
	Außerdem benötigst du ein USB-C-Kabel um den Brick an deinen PC anzuschließen. Brick Daemon wird nicht benötigt.
	Dazu muss der ESP32 Ethernet Brick aus dem Gehäuse ausgebaut und das USB-C
	Kabel eingesteckt werden.



	\newpage
	\section{Fehlerbehebung}
	\label{fehlerbehebung}

	\subsection{Status LED gelb}
	Ist PV-Überschussladen aktiviert, leuchtet die Status LED gelb sobald Strom
	aus dem Netz bezogen wird. Dies ist kein Fehlerzustand.
	Sollte PV-Überschussladen nicht aktiv sein und leuchtet die Status LED gelb,
	dann verfügt der WARP Energy Manager über keine funktionale
	Netzwerkverbindung (kein WLAN verbunden und keine LAN Verbindung).

	\subsection{Status LED rot}
	Leuchtet die Status LED rot, so ist der WARP Energy Manager in einem
	Fehlerfall. Gründe können eine fehlgeschlagene Schützüberwachung oder ein
	interner Fehler sein.


	\subsection{Sicherungswechsel}
	Der WARP Energy Manager ist intern über zwei 5$\times\SI{20}{\milli\meter}$ Feinsicherungen (mittelträge (m), \SI{500}{\milli\ampere}) abgesichert.
	Tinkerforge verbaut Sicherungen vom Typ \enquote{ESKA 521.014}. Die eine
	Sicherung befindet sich im Eingangspfad der 230V Stromversorgung (L). Die
	andere Sicherung befindet sich im Schaltausgang der Schützsteuerung.



	\section{Konformitätserklärung}
	Die EU-Konformitätserklärung zum WARP Energy Manager ist in einem gesonderten Dokument verfügbar.

	\section{Entsorgung}
	\begin{minipage}{0.43\textwidth}
		WARP Energy Manager und Verpackung sind bei Gebrauchsende ordnungsgemäß zu
		entsorgen. Altgeräte dürfen nicht über den Hausmüll entsorgt werden.
	\end{minipage}\hfill
	\begin{minipage}{0.045\textwidth}
		\includegraphics[width=\linewidth]{./img/resized/weee.pdf}
	\end{minipage}

	\section{Technische Daten}

	%use minipage here to control footnote placement
	\begin{minipage}{\linewidth}

		\begin{description}[leftmargin=!,labelwidth=\widthof{\textbf{statisches Lastmanagement}}]
			\setlength{\itemsep}{3pt}
			\item[Abmessungen] 70 × 90 × \SI{63}{\milli\meter} (B/H/T)
			\item[Montageort] Schaltschrank
			\item[Montageart] Tragschiene
			\item[Nennspannung] \SI{230}{\volt} AC
			\item[Nennfrequenz] \SI{50}{\hertz}
			\item[Eigenverbrauch min.] \SI{1.1}{\watt}\footnote[1]{LAN aktiv, WLAN
			Fallback, Relais aus, LED aus}
			\item[Eigenverbrauch max.] $\sim$\SI{2}{\watt}\footnote[2]{LAN aktiv, WLAN
			ein, Relais ein, LED ein}
			\item[Betriebstemperatur] \SI{0}{\celsius}
			      bis \SI{+30}{\celsius}
			\item[Schutzklasse] II
			\item[PV-Überschussladen] max. 10 WARP Charger\footnote[3]{\label{fn:1} WARP
			Charger/WARP Charger 2 in Varianten Smart/Pro}
			\item[stat./dyn. Lastmanagement] max. 10 WARP Charger\footref{fn:1}
			\item[Netzwerk] LAN, WLAN
			\item[Schnittstellen] HTTP, MQTT
		\end{description}
	\end{minipage}


	\section{Kontakt}
	Tinkerforge GmbH\\ Zur Brinke 7\\ 33758 Schloß Holte-Stukenbrock
	\begin{description}[leftmargin=!,labelwidth=\widthof{\textbf{Website}}]
		\item[E-Mail] \href{mailto:info@tinkerforge.com}{\texttt{info@tinkerforge.com}}
		\item[Website] \href{https://warp-charger.com}{\texttt{warp-charger.com}}
		\item[Telefon] \phonenumber{052078998614}
		\item[Shop] \href{https://tinkerforge.com/de/shop/warp.html}{\texttt{tinkerforge.com/de/shop/warp.html}}
	\end{description}

	\section{Dokumentversionen}
	\begin{tabular}{lll}
		\toprule
		Datum      & Version & Kommentar                       \\
		\midrule
		21.02.2023 & 1.0     & Initialversion                  \\
		\bottomrule
	\end{tabular}

	\vfill
	\null

	\columnbreak
\appendix



	\newpage
	\pagestyle{empty}
	\null
	\vfill
	WLAN-Zugangsdaten
	\begin{tcolorbox}[width=4.2cm,height=2.7cm, boxrule=0.25mm]

	\end{tcolorbox}
	Dieser Aufkleber befindet sich\\ auch unter der Frontplatte des WARP Energy
	Managers.
	\columnbreak

	\null
	\vfill
	Typenschild
	\begin{tcolorbox}[width=7.8cm,height=4.1cm, boxrule=0.25mm]

	\end{tcolorbox}
	Dieser Aufkleber befindet sich auch an der Seite\\ des WARP Energy Managers.
\end{multicols*}
\end{document}
