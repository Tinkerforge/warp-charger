\documentclass[a4paper,10pt]{article}
\ifdefined\forprint
    \usepackage[width=21.6cm,height=30.3cm,center]{crop}
\fi
\usepackage[utf8]{inputenc}
\usepackage[margin=2cm,headheight=26pt,includeheadfoot]{geometry}

\usepackage[german]{babel}
\usepackage[german=quotes]{csquotes}

\usepackage{nameref}
\usepackage{microtype}
\usepackage{float}
\usepackage{siunitx}
\sisetup{
    locale = DE,
    binary-units,
    detect-all,
    per-mode = symbol %enables m/s instead of ms^-1
}
\AtBeginDocument{\DeclareSIUnit{\kWh}{kWh}}

\usepackage{hhline}
\usepackage{tabularx}
\usepackage{array}
\usepackage{calc}
\usepackage{multicol}
\usepackage{multirow}
\usepackage{parskip}
\usepackage{booktabs}
\usepackage{textcomp}
\usepackage{afterpage}
\usepackage{fancyhdr}
\pagestyle{fancy}
\setlength{\headheight}{48pt}
\renewcommand{\headrulewidth}{0pt}

\usepackage{xcolor,colortbl}
\usepackage{makecell}

\usepackage[symbol*]{footmisc}
\renewcommand{\thefootnote}{\fnsymbol{footnote}}
\renewcommand{\thempfootnote}{\fnsymbol{mpfootnote}}
\newcommand\todo[1]{\textcolor{red}{\huge TODO: #1}}
\usepackage{color}
\usepackage{enumitem}

\usepackage{pdfpages}

\usepackage{phonenumbers}

% Word-Stealth-Modus, kann aber kein Omega
%\usepackage[scaled]{helvet}

\usepackage[inline,nomargin]{fixme}
\fxsetup{
    author=,
    layout=inline,
    theme=color
}

\definecolor{fxnote}{rgb}{0.8000,0.0000,0.0000}
\colorlet{fxnotebg}{yellow}

\definecolor{boxgray}{rgb}{0.33,0.33,0.33}
\definecolor{covergray}{rgb}{0.333,0.333,0.325}
\usepackage{tcolorbox}
\title{}
\author{}

\renewcommand{\familydefault}{\sfdefault}

\newcommand{\hint}[1]{\begin{tcolorbox}[colback=boxgray,colframe=black,coltext=
white,title=Hinweis,left*=2mm,right*=2mm,boxsep=1mm,bottom=1mm,top=1mm]#1\end{tcolorbox}}

\newcommand{\gfx}[1]{\includegraphics[width=\linewidth]{#1}}

\newcommand*{\fullref}[1]{Abschnitt \hyperref[{#1}]{\ref*{#1}~\nameref*{#1}}}

\fancyhf{}
\fancyhead{\colorbox{boxgray}{
    \makebox[\dimexpr\linewidth-2\fboxsep - 2.17mm][l]{
        \includegraphics[height=1cm]{./img_v2/logo}\hfill\color{white}\Huge\raisebox{.5ex}{\thepage}
        }
    }
}

\usepackage{hyperref}
\usepackage{qrcode}

\appto\UrlNoBreaks{\do\.\do\:\do\/\do\_}

\newcommand\rurl[2]{%
  \href{#1}{\nolinkurl{#2}}%
}


\newcommand{\tdesc}[1]{\multicolumn{3}{l}{\footnotesize #1}}

\usepackage{hyphenat}

\hyphenation{Web-inter-face}
\hyphenation{Werks-ein-stel-lungen}
\hyphenation{Firm-ware-up-dates}

\begin{document}
\pagestyle{empty}
\begin{titlepage}
	\vspace*{-3.08cm}
	\colorbox{boxgray}{\makebox[\dimexpr\linewidth-2\fboxsep][c]{\includegraphics[width=0.6\textwidth]{./img_v2/logo}}}
	\vfill
	\begin{center}
		\Huge
        \color{white}
		WARP Energy Manager 2.0 Betriebsanleitung\\\vspace{1cm}
		\large
		Version 1.0.0\\\vspace{0.25cm}
		06.12.2024
	\end{center}
	\vspace*{1cm} \gfx{./img_v2/warp-energy-manager2.png}
    \vfill
   	\pagecolor{covergray}

\end{titlepage}
\newpage
\pagecolor{white}
\null
\newpage
\pagestyle{fancy}
\begin{multicols*}{2}
	\tableofcontents
	\newpage
	\section{Einführung}
	Vielen Dank, dass du dich für einen WARP Energy Manager 2.0 von Tinkerforge entschieden hast!
	Mit dem WARP Energy Manager
	erhältst du unseren Energiemanager zur Schaltschrankmontage, mit dem du den
	Energieverbrauch zu Hause überwachen, steuern und optimieren kannst.

	\gfx{./img_v2/warp-energy-manager2.png}

	Der WARP Energy Manager kann abhängig von dynamischen Strompreisen, einem Photovoltaik-Überschuss oder dem aktuellen Strom am Netzanschluss
    Wärmepumpen über die SG-Ready-Schnittstelle steuern, zusätzliche Verbraucher ein- bzw. ausschalten und in Verbindung mit WARP Charger Wallboxen
    Elektrofahrzeuge laden. Auch die Steuerung des Lade- und Entladevorgangs von Batteriespeichern ist zukünftig möglich.

	\subsection{Features}
	Der WARP Energy Manager verfügt über folgende Features:

    \begin{itemize}
        \item Integriertes lokales Energiemonitoring
        \item Steuerung von Wärmepumpen mittels SG-Ready-Schnittstelle
        \item Vier Eingänge für potentialfreie Kontake, z.B.:\\ \S114a EnwG - Steuerbare Verbrauchseinrichtung
        \item Zwei Relaisausgänge zur Ansteuerung weiterer Verbraucher
        \item Steuerung von WARP Charger Wallboxen
		\item Steuerung von Batteriespeichern (zukünftig)
        \item Zugriff auf Stromzähler am Netzanschluss, Photovoltaik-Wechselrichter, Batteriespeicher
        \item Photovoltaik-Überschussnutzung
        \item Statisches und dynamisches Lastmanagement
        \item Photovoltaik-Prognose
        \item Dynamische Strompreise
    \end{itemize}

	\subsubsection{Integriertes Energiemonitoring}
	Die Messwerte der Stromzähler stellt der WARP Energy Manager in seinem
	Webinterface dar. Dort wird angezeigt, wie groß die Leistung ist, die aus dem Stromnetz
	bezogen bzw., (falls du eine Photovoltaik-Anlage besitzt) eingespeist wird. Zusätzlich können
    weitere Daten, zum Beispiel von PV-Wechselrichtern und Batteriespeichern, oder der dynamische Strompreis
    angezeigt werden. Leistungs- und weitere Messwerte werden dir live auf dem Webinterface dargestellt.

	Alle fünf Minuten werden die Messwerte lokal auf der microSD-Karte des
	Energiemanagers gespeichert. Damit ist der WARP Energy Manager unabhängig
	von Datenaufzeichnungen auf Cloud-Servern. Diese Daten kannst du dir für jeden Tag
	grafisch anzeigen lassen. Damit kannst du deinen Energieverbrauch auf Tages-, Monats- und
	Jahresbasis analysieren.

	\subsubsection{Steuerung von Wallboxen}
	Der WARP Energy Manager kann bis zu 64 Wallboxen vom Typ WARP Charger Smart oder WARP Charger Pro verbrauchsabhängig steuern.
	Die Steuerung erfolgt über eine gemeinsames Netzwerk (LAN, WLAN) zwischen den Wallboxen und dem WARP Energy Manager.

    Es kann ein statisches Lastmanagement durchgeführt werden, bei dem ein fest eingestellter Strom je nach Anforderung
    unter den Wallboxen aufgeteilt wird.

    Alternativ ist auch der Aufbau eines dynamischen Lastmanagements möglich. Dabei wird der Strom am Netzanschluss
    von einem Stromzähler gemessen. Der WARP Energy Manager regelt dann die Wallboxen so, dass der eingestellte Maximalstrom
    am Netzanschluss nicht überschritten wird.

	Mit verschiedenen Einstellungen kannst du definieren,
	unter welchen Bedingungen und mit wie viel Leistung Fahrzeuge geladen werden.

	\subsubsection{SG-Ready-Schnittstelle zur Ansteuerung von Wärmepumpen}
    Der WARP Energy Manager besitzt eine SG-Ready-Schnittstelle mit der Wärmepumpen gesteuert werden können.
    Diese besteht laut Standard aus zwei Relais-Ausgängen. Mit einem Ausgang kann die Wärmepumpe vollständig blockiert werden,
    zum Beispiel wenn der Strompreis einen Schwellwert übersteigt. Die Wärmepumpe kann mit dem zweiten
    Ausgang in einen erweiterten Betrieb versetzt werden, bei dem die Heizkreis- und Warmwassertemperaturen etwas erhöht werden.
    Damit lassen sich zum Beispiel ein Photovoltaik-Überschuss oder günstige Strompreise effizient nutzen. Ist keiner
    der Ausgänge geschaltet verrichtet die Wärmepumpe ganz normal ihren Dienst.

	\subsubsection{Zwei Relaisausgänge zur Ansteuerung weiterer Verbraucher}
	Der WARP Energy Manager verfügt über zwei Relaisausgänge.
    Mit diesen Ausgängen können zum Beispiel 230V Verbraucher direkt oder mittels zwischengeschaltetem Schütz auch größere Lasten
    geschaltet werden. Als Beispiel können so Heizstäbe gesteuert werden.

	\subsubsection{Vier Eingänge, z.B. für \S14a EnWG - Steuerbare Verbrauchseinrichtung}
    An die vier Eingänge des WARP Energy Managers kann jeweils ein potentialfreier Kontakt (Schließer/Öffner) angeschlossen werden.
    Diese Eingänge können für vom Anwender konfigurierbare Regeln genutzt werden. Eine Anwendung ist die Nutzung eines Eingangs
    um nach \S14a EnWG vom Energy Manager gesteuerte Wallboxen und die Wärmepumpe in der Leistung zu reduzieren.

    \subsubsection{Fronttaster mit Display}
    Ein 320x240 Pixel TFT Display zeigt Systeminformationen an. Dazu können bis zu sechs Kacheln vom Nutzer auf dem Display konfiguriert werden.
    Der Anzeigewert jeder Kachel kann vom Nutzer festgelegt werden. Es können zum Beispiel die Leistung am Netzanschluss,
    Zustandsinformationen von Wallboxen, der Status der SG-Ready-Schnittstelle, der Wert des dyn. Strompreis etc. angezeigt werden.
    Das Display schaltet sich nach 5 Minuten inaktivität ab.

    Zusätzlich verfügt der WARP Energy Manager über einen Fronttaster. Durch Drücken dieses Tasters kann zwischen verschiedenen Screens
    gewechselt werden. Zusätzlich zeigt die Farbe des Tasters an ob es Probleme gibt (Farbe rot) oder nicht (Farbe grün).


    \newpage

	\subsection{Typische Anwendungen}

	\subsubsection{PV-Überschussnutzung}
	\label{pv_ueberschussnutzung}

	Besitzt du eine Photovoltaik-Anlage, möchtest du vermutlich möglichst viel
	von deinem produzierten Strom selbst nutzen. Der WARP Energy Manager kann
	dir dabei helfen, indem Elektrofahrzeuge mit diesem Strom geladen werden (PV-Überschussladen).
    Der Strom kann aber auch genutzt werden um zum Beispiel eine Wärmepumpe im erweiterten Betrieb
    zu betreiben. Zusätzlich können an den Relaisausgängen weitere Verbraucher, wie zum Beispiel Heizstäbe,
    angeschlossen werden. Der Energy Manager kann diese dann zum Beispiel einschalten, wenn trotz
    Elektrofahrzeug-Ladung und Ansteuern der Wärmepumpe noch PV-Überschuss übrig ist.

	Für die PV-Überschussnutzung benötigt der WARP Energy Manager einen Stromzähler
	an deinem Stromnetzanschluss, um den Überschuss, d.h. die Einspeisung von
	elektrischer Leistung ins Stromnetz, zu ermitteln. Der WARP Energy Manager
	steuert dann die Geräte so, dass keine Leistung ins Netz eingespeist wird
	(Netzbezug = 0) oder aber ein definierter Netzbezug eingehalten wird. Dies
	ist abhängig von deinen Einstellungen.

	Entscheidend ist hier, dass nur eine Leistungsregelung stattfindet,
    die einzelnen Phasenströme werden nicht geregelt. Da der Netzbetreiber-Stromzähler,
	der die Stromkosten ermittelt, saldierend arbeitet, ist eine Phasenstromregelung nicht notwendig.

	\subsubsection{Statisches Lastmanagement}
	\label{statisches_lastmanagement}

	Teilen sich mehrere Wallboxen eine gemeinsame Zuleitung, ist oft der
	Maximalstrom durch diese Zuleitung begrenzt. Als Beispiel könnten sich mehrere
	Wallboxen eine \SI{32}{\ampere} Leitung teilen. Zwei Wallboxen könnten jeweils als \SI{11}{\kilo\watt}
	Wallboxen ($2\cdot\SI{16}{\ampere}$) betrieben werden. Es wäre aber auch möglich, eine
	Wallbox mit \SI{22}{\kilo\watt} (\SI{32}{\ampere}) zu betreiben, wenn die zweite Wallbox nicht genutzt
	wird. Für diese Anwendungen kommt das statische Lastmanagement zum Einsatz.

	Der WARP Energy Manager kann das statische Lastmanagement für die Wallboxen
	übernehmen. Hierbei ist kein Stromzähler notwendig, es ist nur der
	Maximalstrom der Zuleitung zu definieren. Dieser Strom muss jederzeit zur
	Verfügung stehen. Der Energiemanager verteilt den Strom
	je nach Anforderung an die kontrollierten Wallboxen.

	\subsubsection{Dynamisches Lastmanagement}
	\label{dynamisches_lastmanagement}

	In manchen Fällen ist ein dynamisches Lastmanagement auf Phasenstromebene erforderlich.
	Ein typisches Beispiel dafür sind Mietobjekte, bei denen der Stromnetzanschluss der
	Immobilie nicht ausreicht, um mehrere Wallboxen gleichzeitig zu betreiben.
	Die Absicherung des Netzanschlusses beschränkt den zulässigen Phasenstrom.

	Im einfachsten Fall kann für alle Wallboxen ein bestimmter Phasenstrom garantiert werden.
	In diesem Fall können die Wallboxen ein statisches Lastmanagement durchführen,
	bei dem der verfügbare Phasenstrom zwischen den WARP Chargern aufgeteilt wird. (siehe \fullref{statisches_lastmanagement}).

	Oftmals kann jedoch nicht garantiert werden, dass ein bestimmter Phasenstrom jederzeit
	für Ladevorgänge zur Verfügung steht, da sich die Wallboxen den Netzanschluss mit anderen Verbrauchern teilen.
	Wenn diese Verbraucher ein- und ausgeschaltet werden,
	ändert sich der für die Wallboxen zur Verfügung stehende Phasenstrom
	ständig. In diesem Fall ist ein dynamisches Lastmanagement notwendig, um
	sicherzustellen, dass der maximale Phasenstrom nicht überschritten wird und
	keine Sicherung auslöst.

	Der WARP Energy Manager ermöglicht ein dynamisches Lastmanagement auf Phasenstromebene.
	Dazu ist ein Stromzähler am Stromnetzanschluss erforderlich, der vom Energiemanager
	ausgewertet werden kann. Der Energiemanager überwacht den zur Verfügung stehenden
	Phasenstrom vom Netzanschluss und regelt die Leistung der Wallboxen entsprechend.
	Dadurch wird sichergestellt, dass der maximale Phasenstrom nicht überschritten wird
	und keine Sicherung auslöst. Wenn eine Photovoltaik-Anlage vorhanden ist und Energie
	produziert, erhöht sie automatisch die zur Verfügung stehende Leistung für den
	Energiemanager, um das Laden der Elektrofahrzeuge zu optimieren.

	\subsubsection{Kombination PV + Lastmanagement}
	Die PV-Überschussnutzung und ein statisches bzw. dynamisches Lastmanagement können
	kombiniert werden. Der WARP Energy Manager betreibt dann die
	Leistungsregelung für das PV-Überschussladen und stellt parallel sicher, dass die
	Phasenstrom-Begrenzungen durch das Lastmanagement eingehalten werden.

	\newpage
	\section{Sicherheitshinweise}
	Der WARP Energy Manager ist so konstruiert, dass ein sicherer Betrieb gewährleistet ist,
	wenn er korrekt installiert wurde, in einem einwandfreien technischen Zustand
	ist und diese Betriebsanleitung befolgt wird. \hint{Der WARP Energy Manager darf nur von einer ausgewiesenen Elektrofachkraft installiert
		werden.}

	\subsection{Bestimmungsgemäße Verwendung}
    Der WARP Energy Manager wurde für die in diesem Dokument beschriebenen Anwendungen entwickelt.
    Für andere Anwendungen ist dieser nicht geeignet. Eine Verwendung
	an Orten, an denen explosionsfähige oder brennbare Substanzen lagern, ist nicht
	zulässig. Jegliche Modifikation des Energiemanagers oder unsachgemäßer Betrieb ist verboten.
	Der WARP Energy Manager ist in einem geeigneten Verteilerschrank zu installieren
	und vor Beschädigungen, Feuchtigkeit/Verschmutzungen und unsachgemäßem
	Zugriff zu schützen. Er darf nicht genutzt werden, wenn kein sicherer Betrieb
	gewährleistet werden kann.

	\subsection{Gerätestörung / Technischer Defekt}
	Sollte es Anzeichen für einen technischen Defekt geben, ist sofort die
	Stromversorgung des WARP Energy Manager zu trennen und gegen erneutes Einschalten zu
	sichern. Danach ist eine Elektrofachkraft zu informieren.

	\newpage
	\section{Montage und Installation}
	\subsection{Montage}
	\subsubsection{Lieferumfang}
	Im Lieferumfang des WARP Energy Managers befinden sich:
	\begin{itemize}
		\item WARP Energy Manager (Hutschienenmodul)
		\item Steckbare Schraubklemmen
		\begin{itemize}
			\item 2-pol Schraubklemme \SI{5}{\milli\meter} (\SI{230}{\volt} Stromversorgung (L+N))
			\item 4-pol Schraubklemme \SI{5}{\milli\meter} (2x Relais-Ausgang)
			\item 4-pol Schraubklemme \SI{3.5}{\milli\meter} (SG-Ready-Schnittstelle (2x Relais-Ausgang))
			\item 6-pol Schraubklemme \SI{3.5}{\milli\meter} (4x Eingang für potentialfreien Kontakt)
			\item 4-pol Schraubklemme \SI{3.5}{\milli\meter} (RS485 Modbus-RTU)
		\end{itemize}
		\item Diese Betriebsanleitung inkl. individueller WLAN-Zugangsdaten
		\item RJ45-LAN-Winkeladapter
	\end{itemize}

	\subsubsection{Montageort}
	Der WARP Energy Manager darf nur in einem geeigneten Verteilerschrank im
	Innenbereich installiert werden. Er ist vor Staub, Nässe und unsachgemäßem
	Zugriff zu schützen. Es sollte
	eine LAN-Verbindung zum WARP Energy Manager gelegt werden, da in vielen
	Fällen eine Anbindung des WARP Energy Managers mittels WLAN aufgrund
	der Metallabschirmung der Verteilung nicht zuverlässig möglich ist.

	Es muss ausreichend Platz vorhanden sein. Es darf kein Druck auf die Kabel
	ausgeübt werden, insbesondere nicht auf die LAN-Verbindung. Aus diesem Grund
	empfehlen wir die Verwendung des mitgelieferten LAN-Winkeladapters.

	\subsubsection{Montage}
	Zur Montage des WARP Energy Managers muss dieser auf die Hutschiene
	gesetzt werden. Das Gehäuse muss so installiert werden, dass die Anschlüsse
	nach unten zeigen.

	Zuerst wird die obere Halterung auf die Hutschiene aufgesetzt, anschließend
	die untere Halterung. Der Energiemanager sollte sich selbstständig verriegeln, falls dies
	nicht der Fall ist, kann mit einem Schraubendreher an der schwarzen Verriegelung
	auf der Unterseite nachgeholfen werden.

	Soll der WARP Energy Manager wieder von der Hutschiene entfernt werden, so
	müssen zuerst alle Zuleitungen entfernt werden (\textbf{Achtung: Spannungsfreiheit
	sicherstellen!}). Anschließend kann mittels Schlitz-Schraubendreher die schwarze
	Federverriegelung gezogen und der Energiemanager von der Hutschiene
	gehoben werden. Dabei sollte zuerst die untere Halterung angehoben werden,
	gefolgt von der oberen Halterung.

	\gfx{./img_v2/wem2_mounting.jpg}

	\subsection{Elektrischer Anschluss}
	\hint{Die in diesem Kapitel beschriebenen Arbeiten dürfen nur von einer ausgewiesenen
		Elektrofachkraft durchgeführt werden!}

    Auf der Oberseite des WARP Energy Managers sind folgende Anschlüsse zu finden:
	\begin{center}
        \includegraphics[width=0.5\linewidth]{./img_v2/wem2-connections-top.png}
    \end{center}

	\subsubsection{Stromversorgung (230V)}
	Nachdem der WARP Energy Manager montiert wurde, kann dieser angeschlossen werden.
	Die Schraubklemmen sind steckbar, sodass der elektrische Anschluss
	außerhalb erfolgen kann. Anschließend können die Schraubklemmen wieder in
	den WARP Energy Manager gesteckt werden.

	Die Stromversorgung des WARP Energy Managers erfolgt über eine zweipolige
	Schraubklemme (\textbf{L}+\textbf{N}). Die Zuleitung ist mit einem
	max.~\SI{16}{\ampere}~Leitungsschutzschalter mit B-Charakteristik abzusichern.

	Die Stromversorgung des Energiemanagers ist zusätzlich intern über eine Glassicherung
	(träge (T), \SI{2}{\ampere}) abgesichert.


    \subsubsection{RS485/Modbus-Stromzähler}

	Der WARP Energy Manager benötigt einen Stromzähler, um den Leistungsbezug regeln zu
	können. Eine Möglichkeit dafür ist die Installation eines RS485-Modbus-Stromzählers vom Typ Eastron SDM72DMV2, SDM630MCT oder SDM630Modbus.

	Die Steckerbelegung ist \textbf{12V, A, B, GND}. Der Anschluss \textbf{12V}
	darf nicht belegt werden. \textbf{A~(+), B~(-), GND} sind entsprechend
	am jeweiligen Stromzähler anzuschließen.

    Auf der Unterseite des WARP Energy Managers sind folgende Anschlüsse zu finden:
    \begin{center}
        \includegraphics[width=0.5\linewidth]{./img_v2/wem2-connections-bottom.png}
    \end{center}



    \subsubsection{Eingänge}
	Der WARP Energy Manager besitzt vier Eingänge für potentialfreie Kontakte.
	An diesen können Schließer oder Öffner angeschlossen werden. Das Verhalten des
	Energiemanagers in Bezug auf diese Eingänge kann im Webinterface konfiguriert werden.

    Jeweils zwei Eingänge besitzen einen \enquote{Common (C)}-Anschluss. Dieser ist mit einer Seite des Schließes/Öffners
    zu verbinden. Die andere Seite ist dann auf den jeweiligen Eingang zu legen (1,2,3,4).

	\subsubsection{SG-Ready-Schnittstelle}
    Wärmepumpen, die über eine SG-Ready-Schnittstelle gesteuert werden können, können hier angeschlossen werden.
    Dazu bietet der WARP Energy Manager zwei Relais-Ausgänge.
    Ausgang 1 wird betätigt, wenn der Betrieb blockiert werden soll (Betriebszustand 1).
    Ausgang 2 wird betätigt, wenn der Wärmepumpe eine Einschaltempfehlung für den erweiterten Betrieb
    für Raumheizung und Warmwasserbereitung gegeben werden soll (Betriebszustand 3).

	\subsubsection{Relais-Ausgänge}
	Mit den zwei potentialfreien Relaisschaltausgängen können bis zu \SI{230}{\volt AC}/\SI{3}{\ampere} geschaltet
	werden. Das Ansteuern dieser Relais erfolgt über vom Nutzer zu definierende Regeln.

	\subsubsection{LAN-Anschluss}
	Die Steuerung der Wallboxen erfolgt über ein Netzwerk. Wir empfehlen den
	Anschluss des WARP Energy Managers per LAN. Der dafür notwendige LAN-Anschluss
	befindet sich im eingebauten Zustand vor den anderen Anschlüssen. Um Beschädigungen
	zu vermeiden, ist die LAN-Buchse flexibel befestigt. Wir empfehlen es dennoch, ein LAN-Kabel
	nicht direkt an den Energiemanager anzuschließen, sondern den mitgelieferten RJ45-Winkeladapters zwischen Energiemanager und LAN-Kabel zu verwenden.

	\hint{Der Energiemanager ist noch nicht betriebsbereit! Er muss jetzt über das Webinterface konfiguriert werden. Siehe \fullref{setup}}

	\vfill
	\null
	\newpage
	\section{Erste Schritte}
	\label{setup}

	Nach der elektrischen Installation kann der WARP Energy Manager konfiguriert
	werden. Dazu muss zuerst eine Verbindung zum Energiemanager hergestellt werden,
	damit dieser dann über den Browser konfiguriert werden kann.

	\subsection{Schritt 1: Verbindung herstellen}


	\hint{Wir empfehlen unbedingt eine Anbindung des WARP Energy Managers per
	LAN. Auch wenn technisch eine Anbindung mittels WLAN möglich ist, so muss
	sichergestellt werden, dass diese Verbindung dauerhaft stabil ist. Gerade in
	Zählerschränken gestaltet sich dies meist schwierig.}

	\paragraph{Option 1: WLAN}\ \\
	Im Werkszustand öffnet der WARP Energy Manager einen WLAN-Access-Point. Über diesen kann
	die Konfiguration vorgenommen werden, indem auf das das Webinterface des
	Energiemanagers zugegriffen wird.

	Die Zugangsdaten des Access-Points findest du auf dem WLAN-Zugangsdaten-Aufkleber
	auf der Rückseite dieser Anleitung. Du kannst entweder den QR-Code des Aufklebers verwenden,
	der das WLAN automatisch konfiguriert, oder SSID und Passphrase abschreiben.
    Zusätzlich zeigt der WARP Energy Manager den QR Code mit den Zugangsdaten im 2. Screen auf dem
    Display an. Zum Anzeigen des Codes einfach drei mal auf den Taster neben dem Frontpanel drücken.

	Die meisten Kamera-Apps von Smartphones unterstützen das Auslesen des
	QR-Codes und das automatische Verbinden zum WLAN. Viele Smartphones
	erkennen, dass über das WLAN des Energiemanagers (Access-Point) kein Zugriff auf das
	Internet möglich ist. Dein Telefon fragt dann nach, ob du zu dem WLAN
	verbunden bleiben möchtest. Damit du weiter auf den Energiemanager zugreifen
	kannst, darfst du das WLAN nicht wieder verlassen.

	\begin{minipage}{0.35\textwidth}
		Wenn die Verbindung mit dem Access-Point des Energiemanagers hergestellt ist, kannst du das Webinterface
		unter \url{http://10.0.0.1} über einen Browser deiner Wahl erreichen.
		Alternativ kannst du dazu den nebenstehenden QR-Code scannen.
		Eventuell musst du deine mobile Datenverbindung (z.\,B. LTE) deaktivieren.
	\end{minipage}\hfill
	\begin{minipage}{0.12\textwidth}
		\begin{flushright}
			\qrcode{http://10.0.0.1}
		\end{flushright}
	\end{minipage}

	\columnbreak

	\paragraph{Option 2: LAN}\ \\
	Als Alternative zum Zugriff über den WLAN-Accesspoint verbindet sich der
	Energiemanager in den Werkseinstellungen automatisch zu einem
	kabelgebundenen Netzwerk (LAN), wenn ein LAN-Kabel eingesteckt ist, und bezieht eine IP-Adresse
	mittels DHCP. Der Energiemanager kann dann entweder über die zugewiesene IP-Adresse
	(\url{http://[IP-des-Energy-Managers]}, z.\,B. \url{http://192.168.0.42})
	oder den Hostnamen (\url{http://[hostname]}, z.\,B. \url{http://wem2-ABC}) erreicht werden.

	Der Hostname des Energiemanagers ist identisch zur SSID des WLANs. Den Hostnamen findest du
	auf dem WLAN-Zugangsdaten-Aufkleber auf der Rückseite dieser Anleitung.

	Kann die per DHCP vergebene IP des Energiemanagers nicht ermittelt werden, so kann der
	zuvor genannte Zugriff auf den Energiemanager mittels WLAN-Access-Point genutzt
	werden, um die IP-Adresse der LAN-Schnittstelle zu ermitteln
	(\enquote{Status-Seite}, Abschnitt \enquote{LAN-Verbindung}).


	\subsection{Schritt 2: Konfiguration mittels Webinterface}
	Generell empfehlen wir nach der Installation ein Update der Firmware des
	Energiemanagers, um die neusten Funktionen und Bugfixes zu erhalten. Wie ein
	Firmware-Update durchgeführt wird, ist unter \fullref{firmware-update}
	beschrieben.

	Anschließend kann der WARP Energy Manager über das Webinterface konfiguriert
	werden. Die Einstellungen hängen vom Anwendungsfall ab.
	Das Webinterface ist unter \fullref{webinterface} vollständig beschrieben.

	\newpage
	\section{Webinterface}
	\label{webinterface}

	Über das Webinterface kannst du den Energieverbrauch überwachen und
	unter anderem das Laden der kontrollierten Wallboxen steuern.
	Es können diverse Einstellungen vorgenommen werden, die nachfolgend
	dokumentiert sind.

	Wenn du auf das Webinterface der Wallbox mit einem Browser zugreifst,
	gelangst du auf die Start-/ Statusseite. Auf der linken Seite befindet sich
	die Menüleiste, über die du zu weiteren Einstellungen kommst.

	Auf mobilen Endgeräten wird
	diese Menüleiste stattdessen versteckt unter einem Menü-Symbol oben rechts
	im grauen Balken neben dem WARP Logo angezeigt (\enquote{drei Striche untereinander}).
	Hier kannst du das Menü durch antippen des Symbols ausklappen.
	\gfx{./img_v2/wem2-web-status}

	\subsection{Status (Startseite)}
	\label{status}
	Die Startseite des Webinterfaces bietet Schnelleinstellungen und zeigt Statusinformationen an.
    Je nach Konfiguration des Energy Managers stellt die Statusseite verschiedene Informationen dar und bietet andere Einstellungen.

	Mittels Schaltflächen kann der Lademodus gesteuerter
	Wallboxen gewählt werden:
	\begin{description}
	\item[Aus] Die kontrollierten Wallboxen sind deaktiviert. Es kann
	nicht geladen werden.
	\item[PV] \enquote{100\% Eigener Strom}. Ob ein
	Ladevorgang startet, ist davon abhängig, ob die minimale Ladeleistung
	als Überschuss zur Verfügung steht. Ist dies nicht der Fall, so
	wird kein Ladevorgang gestartet.
	\item[Min~+~PV] Es wird die minimal notwendige Ladeleistung sichergestellt, damit immer ein Ladevorgang begonnen werden kann. Diese Leistung kann (anteilig) aus dem Netz bezogen werden. Wird genügend Leistung produziert (Netzeinspeisung), so wird
	der Ladestrom so weit erhöht, bis keine Einspeisung ins Stromnetz mehr
	erfolgt, oder aber die maximale Ladeleistung erreicht wird.
	\item[Schnell] Alle Wallboxen laden mit der maximal möglichen
	Ladeleistung ohne Beachtung einer Netzeinspeisung bzw. eines Netzbezugs. Die konfigurierte Ladestromgrenze wird weiterhin eingehalten, damit die Zuleitung der Wallboxen nicht überlastet wird.
	\end{description}
	Die Optionen \textbf{PV} und \textbf{Min~+~PV} sind nur verfügbar, wenn PV-Überschussladen aktiviert wurde.

	\textbf{Energy Manager} zeigt den Zustand des Energy Managers an. Wenn der Zustand nicht OK ist, wird das Laden an allen gesteuerten Wallboxen deaktiviert.

	Der Chart des \textbf{Leistungsverlaufs} ist vorhanden, wenn mindestens ein Stromzähler konfiguriert ist.
	Es werden die Leistungsdaten der letzten 24~Stunden angezeigt.

    Ist der WARP Energy Manager als \textbf{Lastmanager} für WARP Charger Wallboxen konfiguriert, dann wird als nächstes der Status des Lastmanagers angezeigt, sowie
	der Status der \textbf{kontrollierte Wallboxen}.

    Ist die Heizungssteuerung mittels SG-Ready-Schnittstelle aktiv, so wird unter \textbf{SG-Ready} der Zustand vom blockierenden Ausgang und vom erweiterten Ausgang dargestellt.

    Wenn ein dynamischer Strompreis verwendet wird, wird der \textbf{Durchschnittspreis} von aktuellen Tag und vom morgigen Tag (sofern schon verfügbar) dargestellt.

    Wurde die \textbf{PV-Ertragsprognose} konfiguriert, so wird der erwartete PV-Ertrag ab jetzt und der erwartete PV-Ertrag für den aktuellen und den nächsten Tag dargestellt.

	Der \textbf{WLAN-Access-Point}-Status bildet den Status des Access-Points ab. \enquote{Deaktiviert} beziehungsweise \enquote{Aktiviert} zeigt den Zustand, wenn der Access-Point nicht
	nur als Fallback für die WLAN-Verbindung verwendet wird. Falls der Status \enquote{Fallback inaktiv} ist,
	war die WLAN-Verbindung bzw. LAN-Verbindung erfolgreich und der Access-Point wurde deshalb deaktiviert.
	Beim Status \enquote{Fallback aktiv} ist der Aufbau der WLAN-Verbindung fehlgeschlagen und der
	Access-Point wurde deshalb aktiviert.

	\textbf{LAN-Verbindung} zeigt analog dazu an, ob eine LAN-Verbindung besteht und unter welcher IP-Adresse die Wallbox erreichbar ist.

    \textbf{MQTT-Verbindung} zeigt den aktuellen Status der MQTT-Verbindung zum konfigurierten Broker an.

	\textbf{Systemzeit} zeigt an, ob Datum und Uhrzeit per Netzwerk-Zeitsynchronisierung (NTP) aktualisiert werden konnten.

	\textbf{WireGuard-Verbindung}, sofern konfiguriert, zeigt an ob die konfigurierte WireGuard-VPN-Verbindung aufgebaut werden konnte. Hierfür ist eine bestehende Zeitsynchronisierung notwendig.


    \subsection{Energy Manager}
    \subsubsection{Display}
    \gfx{./img_v2/wem2-web-display}
    Hier kann eingestellt werden, welche Informationen auf dem Display des WARP Energy Managers angezeigt werden. Für sechs Kacheln kann der anzuzeigende Inhalt hier definiert werden.



	\subsubsection{Energiebilanz}

	\gfx{./img_v2/wem2-web-energyanalysis}

	Die Seite Energiebilanz stellt Informationen zur Energieverteilung zur Verfügung.
	Die Daten werden lokal auf dem WARP Energy Manager gespeichert und
	können als Tages- und Monatsverlauf dargestellt werden.
	Der Energy Manager zeichnet Daten aller konfigurierten Stromzähler und kontrollierten Wallboxen auf. Zusätzlich werden, wenn konfiguriert,
    die Schaltausgänge der SG-Ready-Schnittstelle (Heizung) und der Verlauf des dynamischen Strompreises aufgezeichnet.

    \subsubsection{Automatisierung}

    \gfx{./img_v2/wem2-web-automation}

    Auf dieser Unterseite können Regeln definiert werden, die automatisch ausgeführt werden sollen. Jede Regel besteht aus einer Bedingung (z.B.: \glqq Eingang 1 geschlossen\grqq)
    und einer Aktion (z.B.: \glqq Schalte SG-Ready Ausgang 1 auf geschlossen\grqq).
    Es können bis zu 14 Regeln angelegt werden.

    \subsubsection{SD-Karte}

    \gfx{./img/resized/web_em_sdcard}

    Die Daten des WARP Energy Managers werden intern auf einer microSD-Karte
	aufgezeichnet. Hier werden Informationen über die eingelegte Karte ausgegeben. Die microSD-Karte kann
	hier formatiert werden. Dadurch werden alle aufgezeichneten Informationen gelöscht!

	\subsubsection{Debug}
	Auf der Debug-Seite kann ein Energiemanager-Protokoll erstellt werden. Dies
	ist hilfreich, um etwaige Probleme bei der Energieverteilung zu diagnostizieren. Um
	ein Protokoll zu erzeugen, muss einfach nur auf \textbf{Start} geklickt
	werden. Der Energy Manager beginnt dann hochfrequent alle Zustände
	aufzuzeichnen. Mit \textbf{Stop+Download} kann die Aufzeichnung gestoppt und
	das erstellte Protokoll heruntergeladen werden.

	Unter \textbf{Interner Zustand} und \textbf{Low-Level-Zustand} werden interne Zustände vom Energy Manager
	dargestellt, die zur Fehlerbehebung hilfreich sein können.

	\subsection{Energiemanagement}
    \subsubsection{Stromzähler}
	\label{stromzaehler}

	\gfx{./img_v2/wem2-web-meter}

	Auf dieser Seite können bis zu sieben Stromzähler konfiguriert werden.
    Dazu zählt ein Stromzähler am Netzanschluss aber auch als Stromzähler betrachtete PV-Wechselrichter und Batterien.

	Ein Stromzähler am Netzanschluss muss konfiguriert werden, wenn
	der WARP Energy Manager die Funktionen \fullref{pv_ueberschussnutzung} oder
	\fullref{dynamisches_lastmanagement} ausführen soll.

    Im Graph wird die gemessene Leistung aller konfigurierten Stromzähler
	angezeigt, entweder als Verlauf über die letzten \SI{48}{\hour} oder in Abstufungen bis herunter zu einer Live-Ansicht (6 Minuten). Die Ansicht jeders Zählers kann mit dem jeweiligen blauen Pfeil aufgeklappt werden, um weitere Statistiken und Messwerte anzuzeigen.

	\columnbreak

    Es können beispielsweise SunSpec-Zähler oder -Wechselrichter, SMA Speedwire-Geräte
    sowie virtuelle Stromzähler, die über die API befüllt werden können, konfiguriert werden. SunSpec-(Modbus TCP)-Geräte können nach Angabe des Hosts
    automatisch erkannt werden. Abhängig von den Fähigkeiten des SunSpec-Geräts werden verschiedene Messwerte abgerufen.

    Es können auch Modbus-TCP-Geräte ausgelesen werden, die SunSpec nicht unterstützen. Wir bieten vordefinierte Registertabellen für bestimmte Geräte, siehe \rurl{https://docs.warp-charger.com/docs/compatible\_meters}{docs.warp-charger.com/docs/compatible\_meters}. Von uns nicht unterstützte Modbus-TCP-Geräte können über eine benutzerdefinierte Registertabelle ausgelesen werden.

	\subsubsection{Wallboxen}
	\label{chargers}

	\gfx{./img_v2/wem2-web-wallbox}

	Hier werden die vom Energiemanager kontrollierten Wallboxen konfiguriert.
	Die hier vorgenommenen Einstellungen beeinflussen das Lastmanagement
	zwischen den Wallboxen.

	Typ-2-Wallboxen kommunizieren den angeschlossenen Fahrzeugen den maximal zur
	Verfügung stehenden Ladestrom. Das Fahrzeug entscheidet, ob dieser Ladestrom
	voll ausgenutzt wird und ob ein Ladevorgang ein-, zwei- oder dreiphasig durchgeführt
	wird.

	Als erste Einstellung muss der \textbf{Maximale Gesamtstrom}
	der Zuleitung zu den Wallboxen konfiguriert werden.
	Der Energiemanager stellt sicher, dass dieser
	Strom auf keiner Phase überschritten wird, indem niemals mehr als
	dieser Strom an die Wallboxen verteilt wird. Besitzen alle Wallboxen
	ausreichend dimensionierte getrennte Zuleitungen kann dieser Strom so
	hoch eingestellt werden, dass alle Wallboxen sicher ihren Maximalstrom
	erhalten. Alle andere Komponenten, wie zum Beispiel der Netzanschluss,
	müssen dann den konfigurierten maximalen Gesamtstrom liefern können.
	Der individuelle Maximalstrom jeder Wallbox bleibt hiervon unberührt.

	\hint{Hierbei handelt es sich um ein statisches Lastmanagement, bei dem
	davon ausgegangen wird, dass der eingestellte Strom auf jeder Phase
	zu jeder Zeit zur Verfügung steht. Andere Verbraucher als WARP Charger,
	welche vom Energiemanager nicht gesteuert werden können, werden nicht
	berücksichtigt!}

	Am Ende der Seite werden die \textbf{Kontrollierten
	Wallboxen} dargestellt. Weitere Wallboxen können mittels Klick auf
	\textbf{Wallbox hinzufügen} der Steuerung durch den WARP Energy Manager
	hinzugefügt werden. Dazu muss der Anzeigename und die IP-Adresse oder der
	Hostname der Wallbox eingetragen werden und mittels Klick auf \enquote{hinzufügen} übernommen werden.

	Automatisch ermittelte Wallboxen, die noch nicht vom Energiemanager
	gesteuert werden, werden als Liste dargestellt. Für jede Wallbox kann zusätzlich
    die Phasenrotation eingestellt werden.

    \subsubsection{Heizung}


    Auf dieser Seite kann die Ansteuerung einer Wärmepumpe mittels der SG-Ready-Schnittstelle konfiguriert werden. Im Normalbetrieb der Wärmepumpe wird diese nicht gesteuert.
    Mittels SG-Ready kann die Wärmepumpe aber auch in einen blockierenden Betrieb versetzt werden. Das heißt die Wärmepumpe läuft dann nicht.
    Alternativ kann die Heizung auch aufgefordert werden in einem erweiterten Betrieb zu laufen. Je nach Konfiguration der Wärmepumpe erhöht sie die Zieltemperaturen für Warmwasser und Heizungsvorlauf.

    Nachfolgend werden die Einstellungen kurz zusammengefasst. Weitere Informationen und Konfigurationsbeispiele gibt es in den Hilfetexten
    im Webinterface.

   	\begin{description}
    \item[Stromzähler] Hier muss ein zuvor konfigurierter Stromzähler am Netzanschluss ausgewählt werden, damit der PV-Überschuss gemessen werden kann.
    \item[Mindesthaltezeit] Mindestzeit für die der Zustand der SG-Ready Ausgänge beibehalten wird.
    \item[Resthaltezeit] Hier wird die aktuelle Resthaltezeit für die Ausgänge dargestellt. Dabei handelt es sich um die ablaufende Mindesthaltezeit. Für Tests kann die Resthaltezeit auch zurückgesetzt werden.
    \item[SG-Ready-Ausgang 1+2] Konfiguration der Ausgänge.
    \item[Regelzeit] Zeitraum, der für die nachfolgenden Regeln betrachtet werden soll.
    \item[Erweitertes Logging] Wenn aktiv werden im Ereignis-Log weitere Informationen ausgegeben.
   	\end{description}


	\gfx{./img_v2/wem2-web-heater}

    \paragraph{Erweiterter Betrieb}\ \\
    Der erweiterte Betrieb wird geschaltet,
    \begin{itemize}
    \item immer wenn der Schwellwert zum \textbf{PV-Überschuss} überschritten wird (PV-Überschussnutzung).
    \item oder in einer zeitbasierten Steuerung:
    \begin{itemize}
    \item \textbf{für die günstigsten X Stunden} innerhalb der \textbf{Regelzeit}
    \item aber nur wenn die \textbf{PV-Ertragsprognose} für den betrachteten Zeitraum unter dem eingestellten Schwellwert liegt.
    \end{itemize}
   	\end{itemize}

	\vfill
	\null
	\columnbreak

    \paragraph{Blockierender Betrieb}\ \\
    Der blockierende Betrieb wird geschaltet,
    \begin{itemize}
    \item in einer zeitbasierten Steuerung:
    \begin{itemize}
    \item \textbf{bei den teuersten X Stunden} innerhalb der \textbf{Regelzeit}
    \end{itemize}
   	\end{itemize}
    Wird der Schwellwert des zuvor konfigurierten PV-Überschuss am Netzanschluss überschritten, so wird der blockierende Betrieb aufgehoben und die Wärmepumpe in den erweiterten Betrieb versetzt.

    \hint{Maximal kann die Summe der konfigurierten Stunden für den erweiterten und dem blockierenden Betrieb der eingestellten Regelzeit entsprechen. In diesem Fall befindet sich die Wärmepumpe nie im Normalbetrieb.}

    \paragraph{Status}\ \\
    In diesem Abschnitt wird im Graph dargestellt, wann für den aktuellen Strompreisverlauf eine Einschaltempfehlung (grün) gegeben und wann die Heizung blockiert wird (rot). Werden Zeitblöcke nicht markiert, dann befindet sich
    die Wärmepumpe hier im Normalbetrieb. Im Graph wird nicht das Einschalten des erweiterten Betriebs für den Fall eines PV-Überschuss berücksichtigt, sondern nur die zeitbasierte Steuerung.

    \paragraph{S14a EnWG}\ \\
    Soll der SG-Ready Ausgang für den blockierenden Betrieb genutzt werden um auf Grund von \textbf{S14a EnWG} die Heizung zu sperren, so kann in diesem Abschnitt der entsprechende Eingang des WARP Energy Managers gewählt werden.

	\subsubsection{PV-Überschussladen}
    Beim Photovoltaik-Überschussladen ist das Ziel, die nicht selbst genutzte Leistung einer
	Photovoltaikanlage in ein
	Elektrofahrzeug zu laden, anstatt sie in das Stromnetz einzuspeisen. Die Maximierung der Eigenstromnutzung steht hier
	im Vordergrund.

	\paragraph{Funktionsweise}\ \\
	Steht ein entsprechender Stromzähler zur Verfügung, kann die Wallbox den
	Ladevorgang so steuern, dass auf einen Soll-Netzbezug geregelt wird.

	Typischerweise handelt es sich um einen Stromzähler am Hausanschluss, der auf
	einen Bezug von \SI{0}{\watt} geregelt werden soll. Das heißt, die gesamte
	PV-Leistung soll in das Fahrzeug geladen werden, ohne dass ein Netzbezug
	stattfindet (\enquote{PV-Überschuss}).

	WARP3 Charger Smart und Pro sind mit zwei getrennten Schützen
	ausgestattet und können somit intern zwischen einem einphasigen und dreiphasigen
	Ladevorgang umschalten. Das Umschalten auf eine einphasige Ladung bietet den Vorteil, dass auch geringe
	Leistungsüberschüsse in ein Fahrzeug geladen werden können (ab ca.
	\SI{1,4}{\kilo\watt}), wohingegen ein dreiphasiger Ladevorgang die jeweilige
	Maximalleistung der Wallbox ermöglicht (\SI{11}{\kilo\watt} oder
	\SI{22}{\kilo\watt}).

	\paragraph{Lademodi}\ \\

	Das Photovoltaik-Überschussladen kann Ladevorgänge in einem der folgenden vier Modi regeln:

	\begin{description}[labelindent=0.5cm, leftmargin=0.5cm]
	 \item[Aus] Alle Ladevorgänge werden gestoppt
	 \item[PV] Fahrzeuge werden nur aus dem PV-Überschuss geladen.
               Wenn nicht genügend Überschuss zur Verfügung steht, wird der Ladevorgang gestoppt.
	 \item[Min+PV] Fahrzeuge werden aus dem PV-Überschuss geladen.
                   Falls nicht genügend Überschuss zur Verfügung steht, wird Strom aus dem Netz bezogen,
                   damit Ladevorgänge nicht gestoppt werden.
                   Der erlaubte Netzbezug kann konfiguriert werden.
     \item[Schnell] Fahrzeuge werden so schnell wie möglich geladen, unabhängig davon, wie viel PV-Überschuss zur Verfügung steht.
	\end{description}


	\paragraph{Konfiguration}\ \\

	Folgende Einstellungen können vorgenommen werden:
	\begin{description}[labelindent=0.5cm, leftmargin=0.5cm]
		\item[Überschussladen aktiviert] Schaltet den PV-Über\-schuss\-regler ein oder aus.
		\item[Standardlademodus] Der Lademodus, der bei einem Neustart des WARP3 Chargers verwendet wird.
		\item[Stromzähler] Der Stromzähler, mit dem der PV-Überschuss gemessen wird. Dieser Stromzähler muss vorher entsprechend \fullref{stromzaehler} angelegt werden.
		\item[Min+PV: Mindestladeleistung] Legt fest, welche Leistung im \enquote{Min+PV}-Lademodus aus dem Netz bezogen werden darf.
		\item[Regelverhalten] Legt fest, auf welchen
		Netzbezug geregelt werden soll, damit beispielsweise ein Batteriespeicher höher oder niedriger priorisiert wird als das Laden von Fahrzeugen.
		\item[Wolkenfilter] Stellt die Trägheit der Regelung
		ein. Bei wechselnd bewölktem Wetter ist es sinnvoll, dass die Regelung
		träge reagiert, damit kurze Schwankungen der PV-Leistung geglättet werden.
	\end{description}

    \gfx{./img_v2/wem2-web-pv-excess}

    Mit den obigen Einstellungen kann der WARP Energy Manager nur die Leistung am Hausanschluss regeln, indem er den Ladevorgang steuert. Ein eventuell vorhandener
    Batteriespeicher versucht typischerweise seinerseits den Hausanschluss auszuregeln. Damit das Verhalten des Batteriespeichers vom System berücksichtigt werden kann,
    müssen folgende Einstellungen vorgenommen werden:

    \begin{description}[labelindent=0.5cm, leftmargin=0.5cm]
		\item[Stromzähler] Der Stromzähler, mit dem der Batteriespeicher gemessen wird. Dieser Stromzähler muss nach \fullref{stromzaehler} angelegt werden.
		\item[Speicherpriorität] Legt fest, ob ein PV-Überschuss zuerst den Batteriespeicher oder die Fahrzeuge laden soll.
		\item[Energieflussrichtung des Speichers] Legt fest, wie die Werte vom Batteriespeicher-Stromzähler interpretiert werden sollen.
		\item[Bezugs- und Einspeisetoleranz] Gibt an, welcher Bereich um \SI{0}{\watt} am Netzanschluss als ausgeglichen betrachtet wird.
	\end{description}


	\subsubsection{Lastmanagement}

	Beim dynamischen Lastmanagement misst der WARP Energy Manager laufend mittels eines Stromzählers die
	Ströme aller Phasen am Stromnetzanschluss. Der noch rechnerisch zur
	Verfügung stehende Strom kann für jede Phase unterschiedlich sein und ändert
	sich laufend aufgrund des Zu- und Abschaltens von Verbrauchern. Auch eine
	parallel angeschlossene PV-Anlage beeinflusst die Phasenströme. Der WARP
	Energy Manager ermittelt rechnerisch den noch zur Verfügung stehenden
	Phasenstrom und gibt diesen den gesteuerten Wallboxen frei.
	Dabei wird sichergestellt, dass der Maximalstrom jeder Phase nicht überschritten wird und keine Sicherung ausgelöst wird.

    \gfx{./img_v2/wem2-web-load-management}

    Damit das dynamische Lastmanagement verwendet werden kann, muss zunächst ein
    Stromzähler hinzugefügt werden, der die Phasenströme am Hausanschluss messen kann.
    Dieser kann, so gewünscht, auch für ein PV-Überschussladen verwendet werden,
    das dynamische Lastmanagement kann aber auch ohne PV-Anlage verwendet werden. Dieser Stromzähler muss vorher entsprechend \fullref{stromzaehler} angelegt werden.

    Nachdem ein Zähler hinzugefügt wurde, kann auf der \texttt{Energiemanagement} $\rightarrow$ \texttt{Lastmanagement}-Unterseite das dynamische Lastmanagement aktiviert und konfiguriert werden.

    Zunächst muss der eben konfigurierte Stromzähler ausgewählt werden.

    Danach muss der \textbf{Maximale Strom am Netzanschluss} konfiguriert werden. Dieser ist typischerweise der Nennwert der Absicherung.
    Das dynamische Lastmanagment stellt sicher, dass dieser Wert nicht überschritten wird.

    Als letztes muss der zu erwartende \textbf{Stromverbrauch des größten Verbrauchers} konfiguriert werden.
    Dieser kann beispielsweise ein Durchlauferhitzer oder eine Wärmepumpe sein, mindestens aber 16 Ampere aus einer Schuko-Dose.
    Der Wert gibt den größten zu er­war­ten­den plötz­li­chen Sprung des Strom­bezugs am Zähler an, den das dy­na­mi­sche Last­manage­ment kurz­fris­tig (in unter 30 Sekunden) kom­pen­sieren können muss.

    \hint{Die vom Last­manage­ment ge­steu­er­ten Wallboxen müssen nicht be­rück­sich­tigt werden!}


    \subsubsection{Dynamischer Strompreis}

	\gfx{./img_v2/wem2-web-dyn-price}

    Der WARP Energy Manager unterstützt dynamische Strompreise. Diese können zum Beispiel zur Steuerung von Wallboxen oder einer Heizung verwendet werden.
    Hierbei werden nicht die Preise von einem spezifischen Stromanbieter abgefragt, sondern der Börsenstrompreis der jeweiligen Strombörse.
    Diese kann über die Einstellung \textbf{Bereich} gewählt werden. Alle dynamischen Stromanbieter rechnen mit den gleichen Börsenstrompreis. Der Unterschied zwischen den verschiedenen Anbietern besteht
    in der zeitlichen Auflösung des Tarifs (60 Minuten oder 15 Minuten) und den Zusatzkosten (z.B.: Netznutzungsentgelt, Steuern, Abgaben und Umlagen). Diese Parameter können eingestellt werden um
    die Kosten des Anbieters direkt abzubilden. Die im System angezeigten Stromkosten werden auch lokal in der Energiebilanz aufgezeichnet. Es müssen keine Zusatzkosten eingetragen werden.
    Dann zeigt das System die reinen netto Strombörsenpreise an. Werden alle Zusatzkosten eingetragen, so zeigt das System die Netto-Stromkosten an. Wird zusätzlich der MwSt. Satz eingetragen so
    werden die Brutto-Stromkosten angezeigt und aufgezeichnet.



    \subsubsection{Solarprognose}

	\gfx{./img_v2/wem2-web-pv-prognosis}

    Die Solarprognose kann für die Heizungssteuerung verwendet werden. Dazu können bis zu sechs verschiedene PV-Flächen angelegt werden. Für jede dieser Flächen wird eine Prognose berechnet.
    Der Service begenzt die Prognoseabfragen auf 12 Abfragen pro 2 Stunden. Das heißt ist nur eine PV-Fläche konfiguriert, können die Daten für die Fläche 6 mal pro Stunde abgerufen werden.
    Sind zwei Flächen konfiguriert, so sind nur noch 3 Abfragen möglich. Sind keine Abfragen mehr möglich, so muss bis zum Ablauf des zwei Stundenfensters gewartet werden. In der Praxis kann dies eigentlich nur
    durch häufiges Umkonfigurieren oder Neustarten des WARP Energy Managers erzeugt werden.


	\subsection{Netzwerk}
	\label{network}
	Die Wallbox kann in dein Netzwerk per WLAN oder LAN eingebunden werden.
	In diesem Unterabschnitt können alle dazugehörigen Einstellungen vorgenommen werden.

	\subsubsection{Einstellungen}

	\gfx{./img/resized/web_network}

	Hier kannst du den Hostnamen des WARP Energy Managers in allen verbundenen Netzwerken konfigurieren. Außerdem kann mDNS aktiviert oder deaktiviert werden.
	Über mDNS können andere Geräte im Netzwerk den WARP Energy Manager finden.


	\subsubsection{WLAN-Verbindung}
	\gfx{./img/resized/web_wifi_sta}

	Es besteht die Möglichkeit, den WARP Energy Manager mittels WLAN in dein Netzwerk
	zu integrieren. \textbf{Diese Option empfehlen wir aber ausdrücklich
	nicht!}
	Durch Drücken des \enquote{Netzwerksuche}-Buttons öffnet sich ein Menü, in dem das gewünschte WLAN ausgewählt werden kann.
	Es werden dann automatisch Netzwerkname (SSID) und BSSID eingetragen, sowie die Verbindung beim Neustart aktiviert.
	Gegebenenfalls musst du jetzt noch die Passphrase des gewählten Netzes eintragen.

	Du kannst jetzt die Konfiguration mit dem Speichern-Button abspeichern.
	Das Webinterface startet dann neu und verbindet sich zum konfigurierten WLAN. Die Statusseite zeigt
	an, ob die Verbindung erfolgreich war. Der Access-Point bleibt weiterhin
	geöffnet, sodass Konfigurationsfehler behoben werden können.
	Da der Access-Point den selben Kanal wie ein eventuell verbundenes Netz verwendet,
	kann es sein, dass du dich jetzt neu zum Access-Point verbinden musst. Bei einer erfolgreichen Verbindung sollte den Energiemanager jetzt im konfigurierten Netzwerk unter
	\url{http://[konfigurierter_hostname]}, z.\,B. \url{http://wem-ABC} erreichbar sein.

	\subsubsection{WLAN-Access-Point}

	Der Access-Point kann in einem von zwei Modi betrieben werden: Entweder kann er immer aktiv sein,
	oder nur dann, wenn die Verbindung zu einem Netzwerk nicht konfiguriert oder fehlgeschlagen ist.
	Außerdem kann der Access-Point komplett deaktiviert werden.

	\hint{Wir empfehlen, den Access-Point nie komplett zu deaktivieren, da sonst bei einer
		fehlgeschlagenen Verbindung zu einem anderen Netzwerk das Webinterface nicht mehr erreicht
		werden kann. Der WARP Energy Manager kann dann nur über ein Zurücksetzen auf Werkszustand, siehe \ref{werkszustand}, erreicht werden.}

	\gfx{./img/resized/web_wifi_ap}


	Weitere Einstellungen, wie der Modus des Access-Points,
	Netzwerkname, Passphrase usw. können hier festgelegt werden.

	\subsubsection{LAN-Verbindung}
	\gfx{./img/resized/web_ethernet}
	In den meisten Fällen wird eine
	LAN-Verbindung automatisch hergestellt, wenn ein Kabel eingesteckt ist.
	Eine IP-Adresse wird per DHCP bezogen. Es ist aber auch möglich,
	eine statische IP-Konfiguration	einzutragen, oder, falls gewünscht, die LAN-Verbindung
	komplett zu deaktivieren.
	Bei einer erfolgreichen Verbindung sollte der WARP Energy Manager jetzt im LAN unter
	\url{http://[konfigurierter_hostname]}, z.\,B. \url{http://wem-ABC} erreichbar sein.

	\subsubsection{WireGuard}

	WireGuard ist eine Möglichkeit, den WARP Energy Manager mittels einer verschlüsselten
	Verbindung in ein virtuelles privates Netzwerk (VPN) einzubinden. WireGuard wird von
	verschiedenen Routern direkt unterstützt. Dies kann zum Beispiel genutzt
	werden, um aus der Ferne auf den Energiemanager zuzugreifen oder das
	Wallbox-Netzwerk vor fremdem Zugriff zu schützen. Zusätzlich kann das
	Lastmanagement zwischen Energy Manager und den Wallboxen per WireGuard abgesichert werden.

	Die notwendigen Parameter sind WireGuard-typisch und werden an dieser Stelle
	nicht gesondert erläutert. Weitere Informationen finden sich auf
	\url{https://www.wireguard.com/}.

	\gfx{./img/resized/web_wireguard}

	\subsection{Schnittstellen}
	\subsubsection{MQTT}
	\label{mqtt-interface}

    Auf der \texttt{MQTT}-Unterseite kann die Verbindung zu einem MQTT-Broker konfiguriert werden. Folgende Einstellungen können vorgenommen werden:
    \begin{description}[labelindent=0.5cm, leftmargin=0.5cm]
        \item[Protokoll] Die verwendete MQTT-Protokollvariante. MQTT kann unverschlüsselt, TLS-verschlüsselt, unverschlüsselt über WebSockets oder TLS-verschlüsselt über WebSockets verwendet werden.
        \item[Broker-Hostname oder -IP-Adresse] Definiert den Hostname oder die IP-Adresse des Brokers, zu dem sich die Wallbox verbinden soll.
        \columnbreak
        \item[Broker-Port] Definiert den Port, unter dem der Broker erreichbar ist. Der typische MQTT-Port 1883 ist voreingestellt.
        \item[Broker-Benutzername und -Passwort] Manche Broker unterstützen eine Authentifizierung mit Benutzername und Passwort.
        \item[Topic-Präfix] Dieses Präfix wird allen Topics vorangestellt, die die Wallbox verwendet.
              Voreingestellt ist warp3/ABC, wobei ABC eine eindeutige Kennung pro Wallbox ist,
              es sind aber andere Präfixe wie z.B. garage\_links möglich.
              Falls mehrere Wallboxen mit dem selben Broker kommunizieren,
              müssen eindeutige Präfixe pro Wallbox gewählt werden.
        \item[Client-ID] Mit dieser ID registriert sich die Wallbox beim Broker.
        \item[Sendeintervall] Der WARP3 Charger verschickt MQTT-Nachrichten nur, wenn sich die enthaltenen Daten geändert haben.
            Es gibt Teile der API, deren Daten sich sekündlich ändern. Das Sendeintervall kann hier reduziert werden, wenn weniger Netzwerktraffic
            erzeugt werden soll.
        \item[Discovery] Erlaubt die automatische Integration in manche Hausautomatisierungssysteme, z.B. HomeAssistant.
    \end{description}
    Wenn die MQTT-Verbindung konfiguriert und aktiviert ist, verbindet sich die Wallbox nach dem nächsten Neustart zum Broker. Auf der Statusseite wird angezeigt, ob die Verbindung aufgebaut werden konnte.

    Weitere Informationen über die MQTT-API der Wallbox finden sich unter
	\rurl{https://warp-charger.com/api.html}{warp-charger.com/api.html}.

	\gfx{./img/resized/web_mqtt}

	\subsection{System}
    Im System-Unterabschnitt können das Ereignis-Log eingesehen und Firmware-Aktualisierungen eingespielt werden. Zusätzlich sind hier Einstellungen zu TLS-Zertifikaten, die Konfiguration des Fernzugriffs und zur Systemzeit möglich.

    \subsubsection{Einstellungen}
    \label{einstellungen}

    Hier kann die Systemsprache konfiguriert werden. Alternativ ist es möglich, dass der WARP Energy Manager
    die Systemsprache mittels der Browsersprache festlegt.

    Es ist möglich den Energiemanager mittels Knopfdruck neu zu starten.

	Falls das Webinterface nicht korrekt funktioniert, oder die Konfiguration defekt ist,
	kannst du hier alle Einstellungen auf den Werkszustand zurücksetzen.
	\hint{Durch das Zurücksetzen auf Werkszustand gehen \mbox{\textbf{alle}} Konfigurationen verloren.}
	Nach dem Zurücksetzen startet das Webinterface wieder und öffnet
	den Access-Point mit der SSID und Passphrase, die auf dem Aufkleber vermerkt
	sind. Der WARP Energy Manager kann jetzt wieder nach \fullref{setup} konfiguriert werden.


    \subsubsection{TLS-Zertifikate}
    Hier können bis zu acht TLS-Zertifikate hochgeladen werden. Diese Zertifikate können
    für MQTT-Verbindungen sowie zum Aufbau einer WiFi-Enterprise-Verbindung genutzt werden.

    \subsubsection{Fernzugriff}

    Über die Fernzugriffs-Funktion kann auch von außerhalb des eigenen Netzwerks auf das Webinterface des WARP Energy Managers zugegriffen werden.

    Zur Verwendung des Fernzugriffs, bzw. der Apps muss zunächst ein Account auf \rurl{https://my.warp-charger.com}{my.warp-charger.com} erstellt werden. Die Account-Daten müssen dann auf dem Webinterface des WARP Energy Managers unter \texttt{System} $\rightarrow$ \texttt{Fernzugriff} eingetragen werden.

    %\gfx{./img_warp3/resized/web_remote_access}

    \hint{Fernzugriffsverbindungen sind Ende-zu-Ende verschlüsselt. Der \url{my.warp-charger.com}-Server überträgt die verschlüsselten Daten zwischen Energy Manager und Browser, bzw. App und hat weder Einsicht in die übertragenen Daten, noch Zugriff auf den Energy Manager.}


	\subsubsection{Systemzeit}\label{ntp}
	Um für die Aufzeichnung der Energiebilanz und das Ereignis-Log die aktuelle Uhrzeit zur
	Verfügung zu haben, kann der WARP Energy Manager diese per NTP über
	eine Netzwerkverbindung synchronisieren. Auf dieser Unterseite kannst du NTP aktivieren oder deaktivieren und die Zeitzone, in der sich der WARP Energy Manager befindet, konfigurieren.

	Außerdem ist es möglich, zusätzlich zum konfigurierten Zeitserver einen Zeitserver zu verwenden, der von deinem Router per DHCP gesetzt wird. Dies funktioniert allerdings nur,
	wenn in der Netzwerkkonfiguration keine statische IP-Konfiguration verwendet wurde.

	\gfx{./img/resized/web_ntp}


	\subsubsection{Zugangsdaten}

	\gfx{./img/resized/web_authentication.png}

	Auf dieser Unterseite kannst du einen Benutzernamen und ein Passwort konfigurieren, mit denen du den Zugriff auf das Web Interface
	des WARP Energy Managers schützt. Zugriffe auf das Webinterface und die HTTP-API sind bei aktivierter Anmeldung nur möglich, wenn
	die korrekten Zugangsdaten angegeben werden.
	\hint{Falls du die Zugangsdaten vergisst ist ein Zugriff auf das Webinterface nur noch nach einem Zurücksetzen auf Werkszustand möglich. Siehe \fullref{werkszustand}}

	\subsubsection{Ereignis-Log}
	\gfx{./img/resized/web_event_log}

	Das Ereignis-Log zeichnet relevante Informationen des Systemstarts, sowie WLAN- und MQTT-Verbindungsabbrüche und Regelungsinformationen auf.
	Falls Probleme mit dem WARP Energy Manager auftreten, kannst du diese mit dem Log diagnostizieren.
	Falls du ein Problem mit dem WARP Energy Manager an uns melden möchtest, kannst du einen Debug-Report abrufen,
	der uns helfen kann, das Problem zu verstehen und zu lösen. Diese beinhaltet neben dem Ereignis-Log die vollständige
	Konfiguration des Energiemanagers, mit Ausnahme von Passwörtern o.\,Ä.

	\subsubsection{Firmware-Aktualisierung}
	\label{firmware-update}
	\gfx{./img/resized/web_firmware_update}
	Hier kannst du die Firmware des Energy Managers aktualisieren.
	Wir entwickeln die Funktionalität
	des Energy Managers laufend weiter. Bitte beachte, dass daher ggf. auch eine neue
	Version dieser Betriebsanleitung bereitgestellt wird.
	Die aktuelle Firmware und die neuste Betriebsanleitung findest du auch unter
	\rurl{https://warp-charger.com}{warp-charger.com} zum Download.

	\newpage
	\section{Fehlerbehebung}
	\label{fehlerbehebung}
	Die Taster-LED des WARP Energy Manager blinkt in Fehlerfällen rot. Weitere Infos zu den Fehlern gibt das Display.

    \subsection{Zurücksetzen auf Werkszustand}
    \label{werkszustand}

    Im Webinterface kann der WARP Energy Manager mittels Knopfdruck auf Werkseinstellungen zurückgesetzt werden (siehe \ref{einstellungen})

    Falls das Webinterface nicht mehr erreichen kannst, kannst du versuchen, die Recovery-Seite zu öffnen.
	Falls du über den Access Point der Wallbox verbunden bist, erreichst du diese unter \url{http://10.0.0.1/recovery},
	bei einer bestehenden Verbindung zu einem LAN oder WLAN über
	\url{http://[konfigurierter_hostname]/recovery}, also z.\,B. \url{http://wem2-ABC/recovery}.
	Über die Recovery-Seite kannst du den WARP Energy Manager neustarten, Firmware-Updates einspielen,
	den Energy Manager auf den Werkszustand zurücksetzen (Factory Reset) und Debug-Reports
	herunterladen.

	Falls der WARP Energy Manager weder seinen Access Point öffnet, noch über ein konfiguriertes Netzwerk auf das Webinterface zugegriffen werden kann,
	kannst du wie folgt das Zurücksetzen auf Werkseinstellungen starten:
	\begin{enumerate}
	 \item Öffne den Energy Manager, indem du den Boden dem Schraubendreher entfernst. Der Boden besitzt dazu vier Harken.
     \item Baue den Platinenstapel aus WARP Energy Manager Bricklet 2.0 und ESP32 Ethernet Brick aus. Ggf. ist dazu das Verbindungskabel zum Frontpanel zu lösen.
     \item Trenne die Kabelverbindung zwischen den beiden Platinen, so dass der ESP32 Ethernet Brick mit nichts mehr verbunden ist. Schließe eine USB-C Stromversorgung an den ESP32 Ethernet Brick an.
	 \item Drücke mit dem Stift einmal kurz auf \textbf{EN} (1). Die blaue LED fängt an zu blinken.
	 \item Drücke anschließend mit dem Stift \textbf{IO0} (2) und halte diesen gedrückt. Die blaue LED (3) fängt an schneller zu blinken.
	 \item Halte \textbf{IO0} (2) ca. 8\,Sekunden gedrückt, bis die LED (3) dauerhaft leuchtet.
	 \item Sobald die blaue LED (3) dauerhaft leuchtet, ist der Vorgang abgeschlossen. Sollte die LED (3) währenddessen ausgehen, so war der Vorgang nicht erfolgreich und muss wiederholt werden.
	\end{enumerate}
	Die WARP Energy Manager Firmware setzt jetzt alle Einstellungen auf den Werkszustand zurück. Bei Erfolg sollte es jetzt möglich sein, über den Access Point wieder auf den Energiemanager zuzugreifen.

	\gfx{./img/resized/factory_reset_2}

	\subsection{Sicherungswechsel}
	Der WARP Energy Manager ist intern über eine 5$\times\SI{20}{\milli\meter}$ Feinsicherungen (träge (t), \SI{2}{\ampere}) abgesichert.
	Tinkerforge verbaut Sicherungen vom Typ \enquote{ESKA 522.520}. Die
	Sicherung befindet sich im Eingangspfad der 230V Stromversorgung (L).

	\section{Konformitätserklärung}
	Die EU-Konformitätserklärung zum WARP Energy Manager ist in einem gesonderten Dokument verfügbar.

	\section{Entsorgung}
	\begin{minipage}{0.43\textwidth}
		WARP Energy Manager und Verpackung sind bei Gebrauchsende ordnungsgemäß zu
		entsorgen. Altgeräte dürfen nicht über den Hausmüll entsorgt werden.
	\end{minipage}\hfill
	\begin{minipage}{0.045\textwidth}
		\includegraphics[width=\linewidth]{./img/resized/weee.pdf}
	\end{minipage}

	\section{Technische Daten}

	%use minipage here to control footnote placement
	\begin{minipage}{\linewidth}

		\begin{description}[leftmargin=!,labelwidth=\widthof{\textbf{PV-Überschussladen}}]
			\setlength{\itemsep}{3pt}
			\item[Abmessungen] 70 × 90 × \SI{63}{\milli\meter} (B/H/T)
			\item[Montageort] Schaltschrank
			\item[Montageart] Tragschiene
			\item[Nennspannung] \SI{230}{\volt} AC
			\item[Nennfrequenz] \SI{50}{\hertz}
			\item[Eigenverbrauch min.] \SI{1.1}{\watt}\footnote[1]{LAN aktiv, WLAN
			Fallback, Relais aus, LED aus}
			\item[Eigenverbrauch max.] $\sim$\SI{2}{\watt}\footnote[7]{LAN aktiv, WLAN
			ein, Relais ein, LED ein}
			\item[Betriebstemperatur] \SI{0}{\celsius}
			      bis \SI{+30}{\celsius}
			\item[Schutzklasse] II
			\item[PV-Überschussladen] Max. 64 WARP, WARP2 und/oder WARP3 Charger
			\item[Lastmanagement] max. 64 WARP Charger\footref{fn:1}
			\item[Netzwerk] LAN, WLAN
			\item[Schnittstellen] HTTP, MQTT, RS485
		\end{description}
	\end{minipage}


    \section{Kontakt}
    Tinkerforge GmbH\\ Helleforthstraße 22-28\\ 33758 Schloß Holte-Stukenbrock
    \begin{description}[leftmargin=!,labelwidth=\widthof{\textbf{Website}}]
        \item[E-Mail] \href{mailto:info@tinkerforge.com}{\texttt{info@tinkerforge.com}}
        \item[Website] \href{https://warp-charger.com}{\texttt{warp-charger.com}}
        \item[Telefon] \phonenumber{+49 5207 897300-0}
        \item[Shop] \href{https://tinkerforge.com/de/shop/warp.html}{\texttt{tinkerforge.com/de/shop/warp.html}}
    \end{description}

	\section{Dokumentversionen}
	\begin{tabular}{lll}
		\toprule
		Datum      & Version & Kommentar                       \\
		\midrule
		06.11.2024 & 1.0.0   & Initialversion                  \\
		\bottomrule
	\end{tabular}

	\vfill
	\null
	\newpage

	\columnbreak

    \end{multicols*}

    \appendix

	\newpage
	\pagecolor{covergray}\afterpage{\nopagecolor}

   \begin{multicols*}{2}
    \pagestyle{empty}
    \null
    \vfill
	\color{white}
    WLAN-Zugangsdaten
    \begin{tcolorbox}[width=4.2cm,height=2.7cm, boxrule=0.25mm]

    \end{tcolorbox}
	Dieser Aufkleber befindet sich\\ auch unter der Frontplatte\\des WARP Energy
	Managers.
    \columnbreak

    \null
    \vfill
    Typenschild
    \begin{tcolorbox}[width=7.8cm,height=4.1cm, boxrule=0.25mm]

    \end{tcolorbox}
    Dieser Aufkleber befindet sich auch an der Seite\\ des WARP Energy Managers.
\end{multicols*}
\end{document}
